\chapter{Research surroundings and methodology}
\section{Company description}
The case company in this study is a client organisation of MindExpe project; Service Foundation for People with Intellectual Disability (KVPS). The Service Foundation for People with an Intellectual Disability was founded by Inclusion Finland KVTL which is a non-governmental organisation aiming to promote equal opportunities in society for people with intellectual disabilities and their families. The aim of the foundation is to promote a good life for people with intellectual disabilities and their families by lobbying decision-makers and legislators. They co-operate in advocacy work with NGO?s and other parties involved in the field. \citep{kvps.fi}
 
KVPS promotes a person-centred approach to the lives of people with intellectual disabilities, promotes their full citizenship rights and carry out development projects and organize various kinds of training. In addition they offer wide variety of respite care services to cater for the different needs and situations of families and people with special support needs and acquire apartments for young people and adults with intellectual disabilities who wish to live on their own. \citep{kvps.fi}
 
KVPS Tukena Ltd (later will be referred as Tukena), as part of KVPS, focuses on providing diverse, person-centered support services in partnership with local authorities and other providers. KVPS Tukena provides different solutions and housing services for young people and adults with intellectual disabilities who wish to live on their own, one of them being group housing. 10 out of 14 interviewees in this study were employees in several Tukena housing units in Finland. Rest of the interviewees worked in the operations development unit in various projects. \citep{tukena.fi}

\section{Experimentation challenge description}

In order to study experimentation behaviour in an organisation, an experimentation challenge was designed and organised. 
The challenge was organised separately for two client organisations of MindExpe project: The K-Retailer's Association and 
Service Foundation for People with an Intellectual Disability (In Finnish, Kehitysvammaisten Palvelus\"a\"ati�, KVPS). The data 
analysed in this thesis is gathered from different service units of KVPS and KVPS Tukena Ltd, which is a part of KVPS, 
focusing on providing support services for people with an intellectual disability. 

The kick-off for the experimentation challenge for the management level was held in April 2013. As the approach for 
developing through experimenting is not yet widely studied nor recognised way of working in the client organisation, 
during the launching two researchers of Mind told briefly through examples about the approach. Furthermore, practicalities 
and frames for the competition were presented. The managers of the units were thus given the responsibility to bring the 
information of experimentation challenge to their units. Mind team offered posters where easy steps for ideating and 
experimenting were presented.

Time for experimenting was from 24th of April until 11th of June, meaning seven weeks in total. Participants were asked to 
perform quick and easy experiments, reflect the learnings of them and report the experimentations through either e-mail or 
paper formula. 

Each unit participating experimentation challenge was responsible for its own activity. After the kick-off for experimentation 
challenge project leader called once to immediate superior of some units in order to gain knowledge how the team is 
contributing to the challenge and whether experimentations are conducted or not. However, no additional support or advising 
was given to units, and teams were self-driven in their activity. 

During an experimentation challenge, participants, employees of the company units, were asked to ideate ways to improve the work life, from the perspective of an employee and especially from the perspective of a customer. In addition to only ideate, they were encouraged to plan as small and easy way to test the idea as possible, in order to perform it during the time frame. Intentionally, Mind team did not restrict the style, theme or ways of experimenting. This let participants participate in a way feasible for them, their unit and working pace.

Experimentations were then reported to the jury, which consisted of members from both the development and management team of KVPS and Mind researchers. Best experiments and best reflections (experiments that helped the team to reflect and learn more about the idea, whether or not the experiment itself was successful) were rewarded in the closing session of the experimentation challenge as well as the unit that performed most experiments. In the evaluation process, jury focused on how well the goal of an experiment was recognised and kept in mind, how useful the experiment was (for instance for work efficiency or customer satisfaction), and what was learnt from the experiment. 

During the experimentation challenge KVPS Tukena reported 33 experiments and 20 were reported by the foundation, so altogether 53 experiments were reported. Experiments themselves were not further analysed in this study, as in the focus and interest of this study is the experience of an individual of the experimentation process. 

Experimentation challenge was essential part of empirical study, which overall took place during the year 2013. The experimentation challenge was organised during the spring and summer, following the closing session with rewards and interviews of 14 employees during the autumn 2013. Detailed dates of the challenge are described in table \ref{tbl:schedule}

\begin{table}[htcb]
\begin{center}
\caption{Schedule of experimentation challenge}
\begin{tabular}{ | p{2.3cm} | p{4.5cm} | }
\hline
	23rd of April 2013 & The experimentation challenge was launched to the whole KVPS and Tukena Group  \\  \hline
	24rd April to 11th June 2013 & Experimentation challenge   \\ \hline
	20th of September 2013 & Closing session of the challenge and rewarding winners \\ \hline
\end{tabular}
\label{tbl:schedule}
\end{center}
\end{table}

In order to better understand the practicalities and structure of the challenge, experimentation challenge was first pivoted with two units of KVPS and two stores of K-Retailer's Association, before the actual challenge for the whole organisation was launched. However, the data gathered for this study does not consist of the interviews made from the pilot challenge, yet they gave the direction and frames for the actual experimentation challenge and assisted in framing the structure for the interviews.


\section{Research methods}
Considering the complex and uncertain characteristics of organisational and human behaviour under study, qualitative research was used. According to \citet {morgan1980case} qualitative research should be seen as an approach rather than a set of techniques, and especially when exploring social phenomenon, as in this study, qualitative research serves relevant approach.

To enhance the credibility of this research and address the concerns associated with interpretive research, systematic and transparent description of the research methodology, data gathering and analysis process is offered. In addition, quotations from the data are presented widely in the results and in thematic and interpretive analysis process two researchers and supervisor of this thesis have been involved. 

\subsection{Case study}
The most recognised case study research is based on the work of \citet{yin1989case} and \citet{eisenhardt1989building}. According to their perspective, case study refers to a method dealing with contemporary phenomenon in real life context. Case study method is ideal in studies where boundaries between the phenomenon and its context cannot be clearly delimited. \citep{yin2014case} This is rather accurate when considering organisational change and behaviour, learning and applying novel approaches for developing. 

In case studies research question and boundaries of the study are expected and allowed to change during the study. Thus, in the beginning of the study too strict premises might create biases and limit the results. \citep{eisenhardt1989building}
According to \citet{eisenhardt1989building} no theory is necessarily needed in the beginning of the study, yet some basic theoretical assumptions are required to use as a guidance in the empirical world. 

When in the field of qualitative research, case study method can be used both in theory building and theory testing. Furthermore, it can also serve as a method for interpretive research design, which allows the constructs of interest emerging from the data and not to be defined and known in advance. In interpretive research social reality is seen as embedded within their social settings, as well as it is impossible to abstract it from them. Researchers then focus on interpreting the reality using sense-making process in comparison to hypothesis testing process. \citep{bhattacherjee2012social} 

While case studies are allowed to bring forth theories from empirical data, the empirical findings need to emerge through theory. Theory and findings are mutually dependent, the empirical data affects theories and theories need the verification from empirical findings. Thus, oftentimes case studies are conducted iteratively, the whole process including retesting and redefining of theories. \citep{dubois2004research}

\citet{yin2014case} concludes ideal usage for case study: when aiming to answer 'how' or 'why' -questions, when participants cannot be manipulated and when real life context is significant for the study as well as when the case and the context boundaries remain complex or unclear. In these occasions, case study is an appropriate approach. These aspects resonate well with this thesis, as the focus is on answering "How experimentation-driven developing can be fostered in organisation?", participants were only given an introduction to experimentation-driven development, yet their behaviour during the challenge depended solely on themselves and the work team, and researchers could not manipulate participants. When studying factors affecting experimentation behaviour in organisations, real life context is significant, yet boundaries of the work and the context of developing remain complex and unclear. \citep{yin2014case}

In this thesis, initial research questions were formed based on previous empirical and theoretical findings of experimentation-driven approach on organisational innovation. Urge to study how experimentation-driven process could be fostered in organisations served as an inspiration for the case study setting and literature review. Deeper study on literature review was conducted after gathering empirical data and it rising relevant themes to study further: learning and creativity in organisations, experimentation-driven approach as a tool for learning and factors affecting experimentation behaviour. 

\section{Data gathering}
The empirical data comprises of 14 semi-structured interviews and meeting notes. Interviewees were employees from KVPS Tukena housing service units and KVPS foundation. Writer of this thesis carried out all the interviews face-to-face with the interviewees. Interviews lasted from half an hour to an hour. The interviews were recorded and transcribed. All interviews were held in the interviewee's mother tongue, Finnish, therefore all the quotes presented in the thesis have been translated into English. Interviews concentrated on finding advantages and challenges concerning experimentation behaviour and experimentation-driven development. The interviewees' roles and work experience are summarised in table \ref{tbl:interviewees}. 

\begin{table}[htcb]
\begin{center}
\caption{Work experience and work description of interviewees}
\begin{tabular}{ | p{2.3cm} | p{2.5cm} | p{2.5cm} | p{3.7cm} | }
\hline
	\textbf{Interviewee} & \textbf{Years in current unit} & \textbf{Years of work experience} & \textbf{Work description}   \\ \hline
	1 & over 10  & over 10 &   \\  \cline{1-3}
	2 & 1,5  & over 10 & Development of    \\ \cline{1-3}
	3 & under 1 & over 10 &  operations   \\ \cline{1-3}
	4 & over 2 & over 20 &   \\ \hline
	5 & over 1 & over 10 &    \\ \cline{1-3}
	6 & under 1 & under 1 &    \\ \cline{1-3}
	7 & under 1 & over 10 &    \\ \cline{1-3}
	8 & over  2 & over 15 & Daily routines in   \\ \cline{1-3}
	9 & over 1 & over 1 &  housing service unit   \\ \cline{1-3}
	10 & over 1 & over 25 &    \\ \cline{1-3}
	11 & round 2 & over 10 &    \\ \cline{1-3}
	12 & round 2 & over 5 &    \\ \cline{1-3}
	13 & over 1 & over 10 &    \\ \cline{1-3}
	14 & over 1 & over 10 &    \\ \hline
\end{tabular}
\label{tbl:interviewees}
\end{center}
\end{table}

As this thesis is written as a part of a MINDexpe project, and the data collected will be used as a part of other MIND researchers doctoral studies, collecting data from the perspective of factors affecting experimenting was not the only topic of concern. Thus, all of the data in interviews were not straightly relevant to interest of this thesis. This, in turn, was a fruitful position to conduct a case study with thematic analysis, as too strict hypothesis were not formed in the beginning of the study and more space were left to essential themes rise from the interviews. 

Interviews were carried out after the experimentation challenge. Interviewees were chosen from the units of KVPS and Tukena Group so that both active and less active units were heard. Interviews focused on identifying factors affecting experimentation behaviour of an individual in organisational context. The structure of the interview can be found in the appendix x. Even though factors affecting experimentation behaviour were mainly in focus, during the analysis process another theme was recognised from the data: effects experimenting has on an individual. 

If the interviewee had not taken part in the experimentation competition the interview focused on finding whether the routine work of an interviewee consisted of characteristics of experimentation behaviour. Discussions with immediate superiors of the interviewees as well as interview notes served mostly as a tool for gaining an overall understanding of the routine work and attitude towards experimentation behaviour. 

\subsection{Analysis process}
Thematic analysis was used to analyse the data collected. The focus was on recognising factors affecting experimentation behaviour of an individual. Analysis process followed the idea of \citet{braun2006using} step-by-step process, which is used to identify, analyse and report patterns within data without being tied to any pre-existing theoretical framework. Whereas closely related method grounded theory focuses on theory building about the social phenomenon being studied, thematic analysis can be used more flexibly without detailed knowledge of theoretical framework. \citep{bhattacherjee2012social} In this thesis, using thematic analysis assisted in recognising and identifying factors affecting experimentation behaviour in organisations. In this section analysis process is presented. 

1. Transcribing of interviews
The author transcribed four of the interviews; the transcribing of the rest was outsourced.  

2. Coding of the interviews
After all interviews were transcribed, they were read through in order to create a preliminary understanding of the data collected.  
To further analyse the material, transcriptions were coded into initial themes arising from the interviews. In this phase, large amount of themes rose. 

3. Initial categorisation
Together with two researchers initial categorising under themes that seemed most appropriate was made. Even though the main focus of the study was to find factors affecting experimentation behaviour, in the beginning of the coding various interesting aspects of experience of experimenting rose from the data. Thus, another perspective, how experimenting affects an individual was refined. 

4. Theme creation and category refinement 
Once the initial categorisation phase was finished the author examined the themes more closely aiming to create appropriate themes and refine categories by bringing together themes that were closely connected and eliminate those without many quotations. With assist of the supervisor categories and themes were refined in order to enhance the credibility of the study. 

In this phase, context dependent organisational variables were considered as one category, yet in order to keep the focus, this was further left out of the review and analysis. 

5. Refinement of final themes and categories 
Final step of the analysis process resulted in describing the classes and categories in detail. Results are presented in chapter \ref{results}. The analysis process resulted in two classes, which were divided into categories presented in table \ref{tbl:classes}: 

\begin{table}[htcb]
\caption{Final classes and categories}
\begin{tabular}{| p{6cm} | p{7cm} |}
  \hline
   & Role of the immediate superior \\ \cline{2-2}
   Class 1: & Role of the team \\ \cline{2-2}
  Factors affecting experimentation & Structures and practices of developing \\ \cline{2-2}
   & Characteristics and know-how of an employee\\ \cline{2-2}
   & The gap between an idea and experiment \\ 
    \hline
    Class 2:  & Emotional experience and engagement \\ \cline{2-2}
     How experimenting affects an individual & Learning\\
  \hline
\end{tabular}
\label{tbl:classes}
\end{table}




