\chapter{Research surroundings and methodology}
\section{Company description}
The case company in this study is a client organisation of MindExpe project; Service Foundation for People with Intellectual Disability (KVPS). The Service Foundation for People with an Intellectual Disability was founded by Inclusion Finland KVTL which is a non-governmental organisation aiming to promote equal opportunities in society for people with intellectual disabilities and their families. The aim of the foundation is to promote a good life for people with intellectual disabilities and their families by lobbying decision-makers and legislators. They co-operate in advocacy work with NGO?s and other parties involved in the field.
 
KVPS promotes a person-centred approach to the lives of people with intellectual disabilities, promotes their full citizenship rights and carry out development projects and organize various kinds of training. In addition they offer wide variety of respite care services to cater for the different needs and situations of families and people with special support needs and acquire apartments for young people and adults with intellectual disabilities who wish to live on their own. 
 
KVPS Tukena Ltd (later will be referred as Tukena), as part of KVPS, focuses on providing diverse, person-centered support services in partnership with local authorities and other providers. KVPS Tukena provides different solutions and housing services for young people and adults with intellectual disabilities who wish to live on their own, one of them being group housing. 10 out of 14 interviewees in this study were employees in several Tukena housing units in Finland. Rest of the interviewees worked in the operations development unit in various projects.

\section{Experimentation challenge description}

In order to study experimentation behaviour in an organisation, an experimentation challenge was designed and organised. 
The challenge was organised separately for two client organisations of MindExpe project: The K-Retailer's Association and 
Service Foundation for People with an Intellectual Disability (In Finnish, Kehitysvammaisten Palvelus\"a\"ati�, KVPS). The data 
analysed in this thesis is gathered from different service units of KVPS and KVPS Tukena Ltd, which is a part of KVPS, 
focusing on providing support services for people with an intellectual disability. 

The kick-off for the experimentation challenge for the management level was held in April 2013. As the approach for 
developing through experimenting is not yet widely studied nor recognised way of working in the client organisation, 
during the launching two researchers of Mind told briefly through examples about the approach. Furthermore, practicalities 
and frames for the competition were presented. The managers of the units were thus given the responsibility to bring the 
information of experimentation challenge to their units. Mind team offered posters where easy steps for ideating and 
experimenting were presented.

Time for experimenting was from 24th of April until 11th of June, meaning seven weeks in total. Participants were asked to 
perform quick and easy experiments, reflect the learnings of them and report the experimentations through either e-mail or 
paper formula. 

Each unit participating experimentation challenge was responsible for its own activity. After the kick-off for experimentation 
challenge project leader called once to immediate superior of some units in order to gain knowledge how the team is 
contributing to the challenge and whether experimentations are conducted or not. However, no additional support or advising 
was given to units, and teams were self-driven in their activity. 

During an experimentation challenge, participants, employees of the company units, were asked to ideate ways to improve the 
work life, from the perspective of an employee and especially from the perspective of a customer. In addition to only ideate, 
they were encouraged to plan as small and easy way to test the idea as possible, in order to perform it during the time frame. 
Intentionally, Mind team did not restrict the style, theme or ways of experimenting. This let participants participate in a way 
feasible for them, their unit and working pace.

Experimentations were then reported to the jury, which consisted of members from both the development and management team of 
KVPS and Mind researchers. Best experiments and best reflections (experiments that helped the team to reflect and learn more 
about the idea, whether or not the experiment itself was successful) were rewarded in the closing session of the experimentation 
challenge as well as the unit that performed most experiments. In the evaluation process, jury focused on how well the goal of 
an experiment was recognised and kept in mind, how useful the experiment was (for instance for work efficiency or customer 
satisfaction), and what was learnt from the experiment. 

During the experimentation challenge KVPS Tukena reported 33 experiments and 20 were reported by the foundation, so altogether 
53 experiments were reported. Experiments themselves were not further analysed in this study, as in the focus and interest of 
this study is the experience of an individual of the experimentation process. 

Schedule of the experimentation challenge

Experimentation challenge was essential part of empirical study, which overall took place during the year 2013. The experimentation 
challenge was organised during the spring and summer, following the closing session with rewards and interviews of 14 employees 
during the autumn 2013. Below are the detailed dates of the challenge. 

23rd of April 2013: The experimentation challenge was launched to the whole KVPS and Tukena Group
24rd April to 11th June 2013: Experimentation challenge
20th of September 2013: Closing session of the challenge and rewarding winners

In order to better understand the practicalities and structure of the challenge, experimentation challenge was first pivoted with 
two units of KVPS and two stores of K-Retailer's Association, before the actual challenge for the whole organisation was launched. 
However, the data gathered for this study does not consist of the interviews made from the pilot challenge, yet they gave the direction 
and frames for the actual experimentation challenge and assisted in framing the structure for the interviews.


\section{Research methods}

*T\"a\"ah\"an on viel\"a ihan kesken, voi muna* 
According to Morgan and Smircich (1980) qualitative research should be seen as an approach rather than a set of techniques, and 
especially when exploring social phenomenon, as in this study, qualitative research serves as relevant approach. 

When in the field of qualitative research, case study method can be used both in theory building and theory testing. Furthermore, it can 
also serve as a method for interpretive research design, which allows the constructs of interest emerging from the data and not to be 
defined and known in advance. In interpretive research social reality is seen as embedded within their social settings, as well as it is 
impossible to abstract it from them. Researchers then focus on interpreting the reality using sense-making process in comparison to 
hypothesis testing process. (Bhattacherjee, 2012). In this study, case study together with action research is used. 

Action research together with case study allows researching in real-life setting and furthermore supports the use of interpretive data analysis. 

Thematic analysis was used to analyze the data collected. The focus was on recognising factors affecting experimentation behaviour of an 
individual. Analysis process followed the idea of Braun and Clarke?s (2006) step-by-step process, which is used to identify, analyze and 
report patterns within data without being tied to any pre-existing theoretical framework. Whereas closely related method grounded theory 
focuses on theory building about the social phenomenon being studied, thematic analysis can be used more flexibly as detailed knowledge 
of theoretical framework. 

However, action research has been critized on its similarities to consulting instead of proper scientific approach. Action research have 
been claimed of producing information unqualified for generalisation. To enhance the credibility of this research and address the concerns 
associated with interpretive and action research, systematic and transparent description of the research methodology, data gathering and 
analysis process is offered. In addition, quotations from the data are presented widely in the results and in thematic and interpretive 
analysis process two to three researchers and supervisor of this thesis have been involved. 

\section{Data gathering}
The empirical data comprises of 14 semi-structured interviews, meeting notes and altogether 53 reported experimentations. Interviewees 
were employees from KVPS Tukena housing service units and KVPS foundation. Writer of this thesis carried out all the interviews face-to-face 
with the interviewees. Interviews lasted from half an hour to an hour. The interviews were recorded and transcribed. All interviews were held 
in the interviewee?s mother tongue, Finnish, therefore all the quotes presented in the thesis have been translated into English. Interviews 
concentrated on finding advantages and challenges concerning experimentation behavior and experimentation-driven development. Brief summary 
of the interviews and roles of the interviewees are illustrated in table x. 

As this thesis is written as a part of a MINDexpe project, and the data collected will be used as a part of other MIND researchers doctoral 
studies, collecting data from the perspective of factors affecting experimenting was not the only topic of concern. Thus, all of the data in 
interviews were not straightly relevant to interest of this thesis.

Interviews were carried out after the experimentation challenge. Interviewees were chosen widely from the units of KVPS and Tukena Group so 
that both active and less active units were heard. Interviews focused on identifying factors affecting experimentation behaviour of an individual 
in organisational context. The structure of the interview can be found in the appendix x. Even though factors affecting experimentation behaviour 
were mainly on focus, during the analysis process another theme was recognised from the data: effects experimenting has on an individual. 

If the interviewee had not taken part in the experimentation competition the interview focused on finding whether the routine work of an interviewee 
consisted of characteristics of experimentation behavior. Discussions with immediate superiors of the interviewees as well as interview notes served 
mostly as a tool for gaining an overall understanding of the routine work and attitude towards experimentation behavior. 

\subsection{Analysis process}
After all interviews being transcribed they were read through in order to create a preliminary understanding of the data collected. Then, to further 
analyze the material, transcriptions were coded into themes arising from the interviews. Two to three researchers conducted this initial categorizing 
under themes that seemed most appropriate. Once this phase was finished the writer of this thesis examined the themes more closely aiming to create 
appropriate themes and refine categories by bringing together themes that were closely connected and eliminate those without many quotations. The 
supervisor of this thesis together with other researchers assisted in refining the categories and themes in order to enhance the credibility of the 
study. The analysis process resulted in two classes, which were further divided into categories and subcategories presented below: 

Class 1: Factors affecting experimentation 

Category 1: Role of the immediate superior (Leading by example, Supporting ideation and experimentation, Giving licenses to do experiments)

Category 2: Role of the team (Democracy and low hierarchy, Supportive climate and team practices, Attitude towards failing, Team engagement) 

Category 3: Structures and practices of developing (Resources allocated for ideation and development, Collecting of ideas, Implementing new ways of working)

Category 4: Characteristics and know-how of an employee (Substance know-how, Tolerance for uncertainty, Self-criticism and confidence, Attitude and motivation towards developing)

Category 5: The gap between an idea and experiment (Characteristics of an idea and experiment, Static friction, Stakeholder distance and customer involvement)

Class 2: Effects of experimenting on individual

Category 6: Emotional experience and engagement (Positive emotions: happiness, excitement, inspiration, boost to self-esteem, Negative emotions: frustration, 
disappointment, fear of failure and fatigue, Engagement and motivation towards work)

Category 7: Learning (Reflection of work, Process know-how, Reflection towards work) 
