\chapter{Research methodology} \label{researchdesign}
This chapter presents the empirical research design. First, case study method used in the study is presented in detail. Second section presents data collection process following detailed description of the data analysis.

\section{Case study as a research method} \label{case}
Considering the complex and uncertain characteristics of organisational and human behaviour under study, qualitative research was used. According to \citet {morgan1980case} qualitative research serves as a great approach especially well when exploring social phenomenon in real life context. 

The most recognised case study research is based on the work of \citet{yin1989case} and \citet{eisenhardt1989building}. According to their perspective, case study refers to a method dealing with contemporary phenomenon in real life context. Case study method is ideal in studies where boundaries between the phenomenon and its context cannot be clearly delimited. \citep{yin2014case} According to \citet{eisenhardt1989building} no theory is necessarily needed in the beginning of the study, yet some basic theoretical assumptions are required to use as a guidance in the empirical world. Research questions and boundaries of the study are expected and allowed to change during the study. Thus, in the beginning of the study too strict premises might even create biases and limit the results. \citep{eisenhardt1989building}

When in the field of qualitative research, case study method can be used both in theory building and theory testing. Furthermore, it can also serve as a method for interpretive research design, which allows the constructs of interest emerging from the data and not to be defined and known in advance. In interpretive research social reality is seen as embedded within their social settings, as well as it is impossible to abstract it from them. Researchers then focus on interpreting the reality using sense-making process in comparison to hypothesis testing process. \citep{bhattacherjee2012social} 

While case studies are allowed to bring forth theories from empirical data, the empirical findings need to emerge through theory. Theory and empirical findings are mutually dependent: the empirical data affects theories and theories need the verification from empirical findings. Thus, oftentimes case studies are conducted iteratively, the whole process including retesting and redefining of theories. \citep{dubois2004research}

According to \citet{yin2014case} the ideal usage for case study is when aiming to answer 'how' or 'why' -questions, when participants cannot be manipulated and when real life context is significant for the study as well as when the case and the context boundaries remain complex or unclear. These aspects resonate well with this thesis, as the focus is on identifying factors affecting experimentation behaviour of employees and how it can be fostered in organisations?". Participants were only given an introduction to experimentation-driven development, and their behaviour during the challenge depended solely on themselves and the work team, and researchers could not manipulate participants. When studying factors affecting experimentation behaviour in organisations, real life context is significant, yet boundaries of the work and the context of developing remain complex and unclear. \citep{yin2014case}

In this thesis, initial research questions were formed based on previous empirical and theoretical findings of experimentation-driven approach on organisational innovation. Urge to study how experimentation-driven process could be fostered in organisations served as an inspiration for the case study setting and literature review. Deeper study on literature review was conducted after gathering empirical data and rising relevant themes to study further: learning and creativity in organisations, experimentation-driven approach as a tool for learning and factors affecting experimentation behaviour. 

To enhance the credibility of this research, systematic and transparent description of the research methodology, data gathering and analysis process is offered. In addition, quotations from the data are presented widely in the results and in the analysis process two researchers and supervisor of the thesis have been involved. 

\section{Data collection}
In this section, an overview of the client organisation and the description of the study setting, an experimentation challenge, are provided. Finally, the interview process for collecting the data is described in detail. 

\subsection{Company description}
The case company in this study is a client organisation of the MINDexpe project; Service Foundation for People with an Intellectual Disability (KVPS). The Service Foundation for People with an Intellectual Disability was founded by Inclusion Finland KVTL which is a non-governmental organisation aiming to promote equal opportunities in society for people with intellectual disabilities and their families. The aim of the foundation is to promote a good life for people with intellectual disabilities and their families by lobbying decision-makers and legislators. They co-operate in advocacy work with NGO?s and other parties involved in the field. \citep{kvps.fi}
 
KVPS promotes a person-centred approach to the lives of people with intellectual disabilities, promotes their full citizenship rights and carry out development projects and organise various kinds of trainings. In addition they offer wide variety of respite care services to cater for the different needs and situations of families and people with special support needs and acquire apartments for young people and adults with intellectual disabilities who wish to live on their own. \citep{kvps.fi}
 
KVPS Tukena Ltd (later will be referred as Tukena), as part of KVPS, focuses on providing diverse, person-centred support services in partnership with local authorities and other providers. KVPS Tukena provides different solutions and housing services for young people and adults with intellectual disabilities who wish to live on their own, one of them being group housing. 10 out of 14 interviewees in this study were employees in several Tukena housing units in Finland. Rest of the interviewees worked in the operations development unit in various projects. \citep{tukena.fi}
 
Client organisation KVPS was interested in applying novel approach towards developing in their organisation, and through participating in the MINDexpe research project KVPS experienced experimentation-driven approach in action while MINDexpe benefited from the real life research context. 

\subsection{Experimentation challenge description}
The aim of the experimentation challenge was to encourage employees to improve their work by generating novel ideas and test them in action, serve customer needs better and to introduce employees to experimentation-driven development. Through the experimentation challenge factors affecting experimentation behaviour were studied. The research team instructed a client organisation on using experimentation-driven approach by organising an experimentation challenge where the units of the client organisation were tasked to create, develop and report new ideas to develop their work during a six-week time period. 

The challenge was organised separately for two client organisations of MINDexpe project: The K-Retailer's Association and Service Foundation for People with an Intellectual Disability (In Finnish, Kehitysvammaisten Palvelus\"a\"ati�, KVPS). The data analysed in this thesis is gathered from different service units of KVPS and KVPS Tukena Ltd, which is a part of KVPS, focusing on providing support services for people with an intellectual disability. 

The kick-off for the experimentation challenge for the management level was held in April 2013. As the approach for developing through experimenting is not yet widely studied nor recognised way of working in the client organisation, during the launching two researchers of MIND told briefly through examples about the approach. Furthermore, practicalities and frames for the competition were presented. The managers of the units were thus given the responsibility to bring the information of the experimentation challenge to their units. Researchers gave very brief introduction poster on experimentation and instructions for the challenge, further instructions during the challenge were not provided. The introduction poster can be found in appendix \ref{introposter}. 

Time for experimenting was from 24th of April until 11th of June. However, as it took few days for the leaders to inform their employees about the challenge, actual time for experiments was six weeks. During the challenge participants, employees of the company units, were asked to ideate ways to improve thieir work especially from the customer perspective. In addition, they were encouraged perform quick and easy experiments, reflect the learnings of them and report the experimentations through either an online or paper formula. Idea formula is presented in appendix \ref{ideaformula}. In the formula employees were asked to describe the idea, experiment, how they conducted it, how it went and what they learnt; what was successful and what left something to improve. Intentionally, MIND team did not restrict the style, theme or ways of experimenting. This let participants participate in a way feasible for them, their unit and working pace.

Each unit participating experimentation challenge was responsible for its own activity. After the kick-off for experimentation challenge project leader called once to immediate superior of some units in order to gain knowledge how the team is contributing to the challenge and whether experimentations are conducted or not. However, no additional support or advising was given to units, and teams were self-driven in their activity. 

Experiments were reported to the jury, which consisted of members from both the development and management team of KVPS and MIND researchers. Best experiments and best reflections (experiments that helped the team to reflect and learn more about the idea, whether or not the experiment itself was successful) were rewarded in the closing session of the experimentation challenge as well as the unit that performed most experiments. In the evaluation process, jury focused on how well the goal of an experiment was recognised and kept in mind, how useful the experiment was (for instance for work efficiency or customer satisfaction), and what was learnt from the experiment. 

During the experimentation challenge KVPS Tukena reported 33 experiments and 20 were reported by the foundation, so altogether 53 experiments were reported through an online form or traditional paper form. Experiments themselves were not further analysed in this study, as in the focus and interest of this study is the experience of an employee of the experimentation process. 

Experimentation challenge was essential part of empirical study, which overall took place during the year 2013. The experimentation challenge was organised during the spring and summer, following the closing session with rewards and interviews of 14 employees during the autumn 2013. Detailed dates of the challenge are described in table \ref{tbl:schedule}

\begin{table}[htcb]
\begin{center}
\caption{Schedule of experimentation challenge}
\begin{tabular}{ | p{2.3cm} | p{4.5cm} | }
\hline
	23rd of April 2013 & The experimentation challenge was launched to the whole KVPS and Tukena Group  \\  \hline
	24rd April to 11th June 2013 & Experimentation challenge   \\ \hline
	20th of September 2013 & Closing session of the challenge and rewarding winners \\ \hline
\end{tabular}
\label{tbl:schedule}
\end{center}
\end{table}

In order to better understand the practicalities and structure of the challenge, experimentation challenge was first pivoted with two units of KVPS and two stores of K-Retailer's Association, before the actual challenge for the whole organisation was launched. However, the data gathered for this study does not consist of the interviews made from the pilot challenge, yet they gave the direction and frames for the actual experimentation challenge and assisted in framing the structure for the interviews.

\subsection{The method used}
\textit{"I want to understand the world from your point of view. I want to know what you know in the way you know it. I want to understand the meaning of your experience, to walk in your shoes, to feel things as you feel them, to explain things as you explain them. Will you become my teacher and help me understand?"} \citep{spradley1979ethnographic}
\newline

As quoted above, interviews are an essential method for gathering information of interviewees thoughts and experiences, feelings and knowledge, ideas and preferences. Open-ended and semi-structured interviews leave space for all of the above mentioned, leading to highly qualitative data. \citep{monroe2001evaluation} In order to form understanding of employees perspective and experience of experimenting, semi-structured interviews were conducted. The structure of the interviews can be found in appendix \ref{haastisrunko}. 

The empirical data comprises of 14 semi-structured interviews. Interviewees were employees from five different KVPS Tukena housing service units and KVPS foundation. The author carried out all the interviews face-to-face with the interviewees. Interviews lasted from half an hour to an hour. The interviews were recorded and transcribed. All interviews were held in the interviewee's mother tongue, Finnish, therefore all the quotes presented in the thesis have been translated into English. Interviews concentrated on finding advantages and challenges concerning experimentation behaviour and experimentation-driven development. The interviewees' roles and work experience are summarised in table \ref{tbl:interviewees}. 

\begin{table}[htcb]
\begin{center}
\caption{Work experience and work description of interviewees}
\begin{tabular}{ | p{2.3cm} | p{2.5cm} | p{2.5cm} | p{3.7cm} | }
\hline
	\textbf{Interviewee} & \textbf{Years in current unit} & \textbf{Years of work experience} & \textbf{Work description}   \\ \hline
	1 & over 10  & over 10 &   \\  \cline{1-3}
	2 & 1,5  & over 10 & Development of    \\ \cline{1-3}
	3 & under 1 & over 10 &  operations   \\ \cline{1-3}
	4 & over 2 & over 20 &   \\ \hline
	5 & over 1 & over 10 &    \\ \cline{1-3}
	6 & under 1 & under 1 &    \\ \cline{1-3}
	7 & under 1 & over 10 &    \\ \cline{1-3}
	8 & over  2 & over 15 & Daily routines in   \\ \cline{1-3}
	9 & over 1 & over 1 &  housing service unit   \\ \cline{1-3}
	10 & over 1 & over 25 &    \\ \cline{1-3}
	11 & round 2 & over 10 &    \\ \cline{1-3}
	12 & round 2 & over 5 &    \\ \cline{1-3}
	13 & over 1 & over 10 &    \\ \cline{1-3}
	14 & over 1 & over 10 &    \\ \hline
\end{tabular}
\label{tbl:interviewees}
\end{center}
\end{table}

As this thesis is written as a part of a MINDexpe project, and the data collected will be used as a part of other MIND researchers doctoral studies, collecting data from the perspective of factors affecting experimenting was not the only topic of concern. Thus, all of the data in interviews were not straightly relevant to interest of this thesis. This, in turn, was a fruitful position to conduct a case study with thematic analysis, as too strict hypotheses were not formed in the beginning of the study and more space were left to essential themes to rise from the interviews. 

Interviews were carried out after the experimentation challenge. Interviewees were chosen from the units of KVPS and Tukena Group so that both units that reported many experiments and those who reported less or none were heard. Interviews focused on identifying factors affecting experimentation behaviour of an individual in organisational context and the effects experimenting has on an individual. The structure of the interview can be found in the appendix \ref{haastisrunko}. 

If the interviewee had not taken part in the experimentation challenge the interview focused on finding whether the routine work of an interviewee consisted of characteristics of experimentation behaviour, for instance ideating. Discussions with immediate superiors of the interviewees as well as interview notes served as a tool for gaining an overall understanding of the routine work and attitude towards experimentation behaviour, but were not included in the research data. 

\section{Data analysis}
Thematic analysis was used to analyse the data collected. It focuses on revealing themes rising from the data, emphasising the organisation and rich description of the data set. Instead of counting phrases of words in the data, thematic analysis aims at identifying both implicit and explicit ideas rising from the data. Coding is used as a primary process in the analysis of raw data. Through initial coding essentials from the data are recognised, and are further encoded into interpretations. These interpretations can further include recognising and comparing theme frequencies and co-occurrences. Thematic analysis is considered as a valuable method in revealing the complexity of meaning within a data set. \citep{braun2006using} 

Whereas closely related method grounded theory focuses on theory building about the social phenomenon being studied, thematic analysis can be used more flexibly without detailed knowledge of theoretical framework. \citep{bhattacherjee2012social} In this thesis, thematic analysis assisted in recognising and identifying factors affecting experimentation behaviour of an employee in organisations.

Analysis process followed the idea of \citet{braun2006using} step-by-step process, which is used to identify, analyse and report patterns within data without being tied to any pre-existing theoretical framework. The process consists of six phases: becoming familiar with the data, generating initial codes, searching for themes, reviewing themes, defining and naming themes and producing the final report. Next, analysis process of this study is presented. Along the analysis method, the author wrote research diary, which collected the phases of empirical analysis process.

\subsubsection*{1. Becoming familiar with the data}
In the first phase the aim is to become familiar with the data by reading and re-reading it noticing especially occurring patterns. \citep{braun2006using} The author transcribed four of the interviews; the transcribing of the rest was outsourced. In addition, interview notes were read in order to form understanding of highlights and to get into mindset of categorisation. After all interviews were transcribed, they were read through in order to create a preliminary understanding of the data collected.  

Transcribed interviews consisted altogether 140 pages, as each interview was approximately 10 pages. 

\subsubsection*{2. Generating initial codes}
Second phase of the analysis process consists of initial coding of the data. This is done through identifying patterns that occur. The data is collapsed into labels in order to define categories for further analysis. \citep{braun2006using}

Initial coding was conducted by the author labelling and commenting rising patterns and key words in the transcribed data. MS Office Word and its comment function was used to document the initial codes. For instance, if the quote from the interviewees included description of leader's support towards experimenting, the label for the code was "leadership behaviour" or if the interviewees described getting support from their colleagues towards telling ideas out loud, the label was "team support". 

\subsubsection*{3. Searching for themes}
In the initial coding phase various labels were created and in the next phase search for more patterned themes continued. In this phase initial codes were combined into overarching themes. 

After the second phase coded and labeled interviews were printed. Together with two researchers author read through all the labels and initial categorising under themes that seemed most appropriate was made. In order to be able to discuss about theme creation and make it transparent and clear, essential quotes from the printed papers were cut with scissors and collected into piles forming initial themes. 

Both factors affecting experimentation and the experience of experimentation were recognised and were taken into account in categorisation process. The initial categorisation can be seen in the table \ref{tbl:initialcategories}.

\begin{table}[htcb]
\caption{Initial categorisation}
\begin{tabular}{| p{4cm} | p{6.5cm} |}
  \hline
    & Structures and practices \\ 
    Factors affecting & Business field \\  
   experimentation& Job description \\ 
   & Process \\  
   & Idea \\   
   & Influence of an immediate superior  \\  
   & Understanding of experimentation  \\ 
   & Climate  \\  
   & Individual characteristics \\ 
   & Licence to conduct experiments \\  
   & Co-creation\\  
   & Expertise  \\  
   & Knowing of the customer  \\  \hline
    & Individual \\ 
    Experience   & Idea \\ 
    of experimentation & Process \\ 
    & Team \\ 
    & Making abstract ideas concrete\\ 
\hline
\end{tabular}
\label{tbl:initialcategories}
\end{table}

\subsubsection*{4. Reviewing themes}

In this stage, the researcher looks at how the themes support the data and the overarching theoretical perspective. If the analysis seems incomplete, the researcher needs to go back and find what is missing.

As can be seen in table \ref{tbl:initialcategories}, various categories were identified in the initial categorisation. However, the categories were rough and needed specification and further analysis. The analysis process continued with combining and examining the categories, leading to the second categorisation, which is described in table \ref{tbl:secondcategories}. More appropriate themes and refinement of categories was done by bringing together themes that were closely connected and eliminate those without many quotations. This phase was done with assist of the instructor in order to enhance the credibility of the study. 

In this phase, context dependent organisational and business variables (business field) were considered as one category, yet in order to keep the focus, this was further left out of the review and analysis. 

\begin{table}[htcb]
\caption{Second categorisation}
\begin{tabular}{| p{4cm} | p{6.5cm} |}
  \hline
    & Structures and practices \\ 
    Factors affecting & Business field \\  
   experimentation& Idea \\ 
   & Leadership behaviour  \\   
   & Climate  \\  
   & Individual know-how \\ 
   & Individual characteristics \\ \hline 
    & Emotional level \\ 
    Experience & Perspective towards work \\ 
    of experimentation & Personal development \\ 
\hline
\end{tabular}
\label{tbl:secondcategories}
\end{table}

\subsubsection*{5. Defining and naming themes}
The aim of this phase is to define each theme, the aspects of the data being captured through it as well as the essence and highlights of the themes. 

This step focused on describing categories and subcategories better to response to research objectives, as many of them were described in rather abstract level. The analysis process resulted in two classes, which were divided into categories presented in table \ref{tbl:classes}. The name of the second class was changed from 'Experience of experimentation' to more detailed 'How experimenting affects an individual'. 

\begin{table}[htcb]
\caption{Final classes and categories}
\begin{tabular}{| p{6cm} | p{7cm} |}
  \hline
   & Role of the immediate superior \\ 
   Class 1: & Role of the team \\ 
  Factors affecting experimentation & Structures and practices of developing \\ 
   & Characteristics and know-how of an employee\\ \
   & The gap between an idea and experiment \\ 
    \hline
    Class 2:  & Emotional experience and engagement \\ 
     How experimenting affects an individual & Learning\\
  \hline
\end{tabular}
\label{tbl:classes}
\end{table}

\subsubsection*{6. Producing the final report}
The final step of the analysis process consists of writing the description of categories and themes in detail. Results are presented in chapter \ref{results}. 