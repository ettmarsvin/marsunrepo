\chapter{Change and learning in an organization}
Theoretical part of the thesis forms a synthesis of organizational and individual learning and presents experimenting as a method for developing and learning. Furthermore, thesis presents factors that affect experimentation behaviour in an organisation. 

This chapter approaches developing in an organisation from the perspective of both organisational and individual learning. According to \citet{edmondson1999psychological}, learning behaviour is an essential concern in the fast-paced working environment, where organisational change and complexity are increasing. Reflection and learning are critical in order to understand what is going on and what actions to take. Defining concepts of today's business and working environment have increased uncertainty and change as well as decreased job security in future organisations. Thus, in order teams to function under ambiguous circumstances, they need to feel safe and are actually in a position to provide required safety to each others, with the help and support of managers and leaders. Today's environment require that team members do not fear asking questions, seek help and tolerate mistakes. \citep{edmondson1999psychological}

First, change in organisational and business environment is presented, forming background for the need of new methods, such as experimenting, to deal with change. 
Then, learning in organisational level, following individual perspective are presented. Learning being essential for experimenting, this chapter provides introduction to learning from both organisational and individual perspectives. 

Many studies relate experimenting to innovation and creative abilities of an individual. Thus, in the chapter three, where experimentation is presented, innovation, creativity and intrinsic motivation are outlined. In this thesis, innovation is outlined as experimenting has been related as essential tool to foster innovation. Even though in the data and case innovation was not in focus but developing and learning through experimenting, understanding experimenting in broader perspective may assist in seeing its advantages. 


\section{Change in organisational and business environment}
\citet{hammer1993reengineering} have summarised aspects of change in organisational environment, beginning from the change in organisational structure; from functional departments to process teams. Work tasks change from simple and detailed tasks to multi-dimensional knowledge work while employees are becoming more autonomous instead of strict control. Furthermore, instead of educating, focus is in the learning of an employee, and evaluation of work will change from operations to outcomes. Knowledge and capability are preferred over single performance and values change to more productive behaviour than over-protective. Superiors turn from leaders of the work to coaches and hierarchical organisational structures turn lower while managers focus on leadership instead of task management. \citep{hammer1993reengineering} 

Understanding change analytically and from systems perspective in the turbulent world appears challenging. Change being hectic and fast calls for different skills and strategy than before. Only when change is understood can it be managed, and in order to survive new perspective and understanding towards change is required from an organisation. 
In the changing environment tolerance for uncertainty is needed, and while future can not be predicted, forecasting is a usable method in order to cope with the anxiousness resulting from the uncertainty. Furthermore, even more emphasis should be put on ability to learn and adapt to changes. \citep{senge1990fifth}

Current and future business environment requires continuous learning from organisations, meaning deploying the collective knowledge, skills and creative efforts of their employees \citep{dess2001changing}. Wide access to the information has put tremendous pressure on today's business and companies to increase their efficiency and effectiveness. Simultaneously, budgets are squeezed and margins of profit grow smaller. Concurrently, value of great ideas and demand for creative endeavours have rose in order to improve and develop products and processes. Employees who are able to produce those competitive ideas are precious for the current competitive business environment and organisations which strive for innovativeness. \citep{andriopoulos2000enhancing,oldham1996employee}

Furthermore, technological change, as hectic and rapid it has been, cannot be ignored. Fierce competition for market share and urge for technological innovations have increased the pace of change leading organisations in high pressure to adapt new business environment, rearrange resources, understand and meet new customer and business environment demands. \citep{andriopoulos2000enhancing}

Today, economy is driven by innovation and innovativeness, requiring new understanding and abilities to generate great ideas in order to survive in every business and business level. innovativeness calls for creativity, which again calls for new managing skills of leaders. In contrast to many leaders beliefs, creativity and creative individuals can be managed and encouraged. \citep{amabile2008creativity} 

Hand in hand with innovation comes creativity and employees' ability to be creative in their work \citep{hennessey19881,shalley2004leaders}. Organizations have realised the value of creativity across a variety of tasks, occupations and industries. Dynamic and quickly changing work environment requires new skills and approaches from managers. For instance, they need to motivate and involve employees in various ways in order to foster creativity and innovation that may lead to competitive advantage in the business field. Individual creativity forms the base for organisational creativity and innovation \citep{hennessey19881}, which has been realised to have influence on performance and survival of the company \citep{nystrom1990organizational}. However, all jobs do not require same amount of creativity, and in all jobs the weight of creativity is not as important, yet all organisations benefit from understanding where creativity is really needed and how it can be fostered and managed. \citep{shalley2004leaders} 

Although currently creativity and creative processes of an individual at work are rather well recognised and essential, even more focus should be put on organisations' ability to mobilise creative actions of employees to create novel, socially valued products or services and more efficient ways of working \citep{mumford1988creativity}. In other words, creative actions of an employee are not worthwhile to an organisation when not coordinated or harnessed to yield organisational-level outcomes \citep{jung2003role}.

\section{Learning as competitive advantage and way to survive}
Company's ability to learn is related to its competitive advantage, development and way to survive in hectic business environment. For instance, according to \citet{geus1997living} the only one can maintain company's competitive advantage is to make sure the company is able to learn faster than rivals. Generally organisations are considered as machines, yet recently more emphasis has been put on organisations as living organisms. When considered as machine, organisational model is mechanic and simple, which purpose is to gain profit. Whereas organisation as a living organism is a whole-systemic model, and organisations are considered as place which has deeper, permanent meaning offering people the opportunity to grow and fulfil themselves while earning money. Liable vision of the future focuses on the latter perspective of organisations, where learning and renewal form the essence of being. \citep{geus1997living}

Learning is often related to organisation's ability to be innovative and develop; according to \citet{buijs2007innovation} all innovation processes are processes for organisational learning. Also \citep{quinn1985managing} argues how especially from the management perspective major innovations should be considered as incremental and interactive learning processes that is driven by certain goal. 


\section{Organizational learning}

Learning is essential part of developing and innovation; according to \citet{buijs2007innovation} all innovation processes are processes for organisational learning. Through the understanding of organisational learning real growth and support for learning can be offered. Learning process needs to be understood all from organisational, team and individual perspective \citep{buijs2007innovation}. 

Organizational learning is approached from two different perspectives in literature. On the one hand, learning is considered as an outcome, and on the other it is considered as a process \citep{edmondson1999psychological}. In the first perspective organizational learning is referred to be "an outcome of a process of organisations encoding interferences from history into routines that guide behaviour" \citep{levitt1988organizational}, whereas process perspective defines learning as a process of continuous trial and error \citep{argyris1978organizational}. In this thesis, learning is considered as the latter tradition of learning, which allows the growth and improved performance of individuals and organisations. \citet{edmondson1999psychological} presents and studies behaviours through which various outcomes of learning as adaptation to change, understanding or improved performance are likely to be achieved. Furthermore, \citet{edmondson1999psychological} apply the term learning behaviour to separate it from learning outcomes, and states how set of several activities form the basis of learning behaviour. 

Educational philosopher John Dewey has conceptualised learning as a process in his writings about inquiry and reflection \citep{dewey1956human}. His wok has influenced remarkably on following learning theories, such as experiential learning theory \citep{kolb1984experiential} or action approach of organisational learning \citep{schon1983reflective}. According to \citet{dewey1956human} learning is an iterative process consisting of designing, carrying out, reflecting upon and modifying actions. Dewey separates learning from humans' tendency to behave habitually or automatically. \citet{edmondson1999psychological} builds to this definition focusing on the group level of learning and defining it as an ongoing process where reflection and action occur. Integral characteristics of learning process are asking questions, seeking for feedback, performing experiments and reflecting on the results, having discussions about error and surprising or unexpected outcomes of actions. In group level, learning is enabled through testing assumptions and discussion of opinion differences transparently in order to improve team performance. \citep{edmondson1999psychological} 

Organizational learning differs from individual or team learning. Organizational learning occurs through the shared knowledge, insights and approaches of the employees of an organisation. Secondly, organisational learning is based on prior knowledge and experience, the memory of organisation, which consists of the ways of working, processes and instructions of an organisation. Even though individual and team learning are highly related to organisational learning, it is not the sum of the previously mentioned. \citep{sydanmaanlakka2007}

Various definitions of learning organisations have been presented. One of the most famous definitions is from \citet{senge1990fifth}, who describes learning organisation as follows: "Learning organisation is an organisation, where people are able to constantly develop and achieve intended results; where new ways of thinking are born and where people share goals and learn together."

Together with analysis, limitation, regeneration and technological change, learning is essential factor in improving organisational performance and strengthening competitive advantage. According to \citet{march1991exploration} each pace involves exploitation and exploration as well as adaptation. Exploitation refers to refinement and extension of competences, technologies and paradigms that already exist, whereas exploration is about experimentation with new approaches and alternatives. When results and returns of exploitation are oftentimes positive, proximate and predictable, returns of exploration are uncertain, distant and usually negative. Therefore, exploration leads to greater locus in learning and realisation of problems than exploitation, when considered the distance in time and space. \citep{march1991exploration}

Accordingly in management literature learning is considered relating and even being dependent on receiving feedback \citep{schon1983reflective}, discussion and failure \citep{sitkin1992learning} and experimenting \citep{henderson1990architectural}. As relevant information about performance is acquired through errors, discussion about them has been related with organisational effectiveness \citep{sitkin1992learning}. According to \citet{huy2003rhythm} organisations learn best through small experiments and trying out new things, and the closer and more related experimentations are to customers and customer interfaces, the more can be learned. 

Edmondson's psychological safety has roots already in early research on organisational change. \citet{schein1965personal} state that in order individuals to change and feel safe they need psychologically secure environment. However, team psychological safety should not be confused with groupthink effect that refers more to group cohesiveness, which seems to be related to decreased willingness to disagree and challenge team member's views and thus reduces interpersonal risk-taking \citep{janis1982groupthink}. Term psychological safety refers to team's confidence and shared belief and mutual trust among team members towards that speaking up in a team does not lead to embarrassment, rejection of punishment of any kind \citep{edmondson1999psychological}. 

\section{Individual learning}

Various perspectives and definitions for learning has been studied, presented, analysed and utilised in order to understand individual's process of adapting new information and skills. Experiential learning theory refers to learning as a process of knowledge-creation through experiences while experiential learning process stands as a way to describe the central process of human adaptation to the social and physical environment - a holistic adaptation process that provides bridges across life situations and underlaying the lifelong process of learning. \citep{kolb1984experiential}. Also \citet{jung1923psychological} argues how learning involves concept of human being as a whole - from feeling and thinking to perceiving and behaving.

However, everyday problem-solving and immediate reactions to situations at hand are oftentimes related to performing instead of learning. Furthermore, long-term adaptations to our previous experiences and beliefs is mainly considered as developing, not learning. Yet, when talking about development and developing in individual, team or organisational level, the question highly concerns and is related to learning. \citep{kolb1984experiential}

Experiential learning theory of \citet{kolb1984experiential} consists of four elements: experience, perception, cognition and behaviour. Immediate experience forms a basis for reflection and observation, following assimilation to a theory from which new implications for action are deducted. In order to create new experiences, these implications serve as guides. Overall, experience of an individual is a focal point of learning, and giving personal meaning to abstract concepts, which can be afterwards shared with others. Furthermore, receiving feedback is considered essential in this approach for learning, as it serves a continuous process for goal-oriented action following evaluation of that action. Feedback can thus boost effective, goal-oriented learning process. \citep{kolb1984experiential}

Continuing with the model of \citet{kolb1984experiential}, instead of conceiving learning in terms of outcomes, it should rather be conceived as a process. Ideas are not fixed and and immutable elements of thoughts, but can be formed and re-formed through experience. Furthermore, bringing the experiential learning into educational implications, all learning can be considered as relearning. Thus, all learning situations should take into account people arriving from all different experiential backgrounds to what they build their new experiences and knowledge on. This partly explains very likely resistance to new ideas, as when new information and experiences are in contradiction to old beliefs and experiences, new ideas and information is more difficult to adapt. In the education process learner's old beliefs and theories should be brought out, examined and tested, following integration of the new models and refined ideas into learner's belief systems. \citep{kolb1984experiential}

\citet{kolb1984experiential} presents Piaget's interactive process approach to learning, according to which individual learning and adaptation of new ideas occurs through integration or substitution. Integration leads to stronger part of learner's conception of the world, whereas substitution requires real questioning of previous conceptions, and thus might take longer for the learner to adopt. Learning is a mutual process between accommodation of concepts or schemas to experiences around us and assimilation of events and experiences into existing concepts and schemas. This intelligent adaptation, learning, results from the tension between accommodation and assimilation. Through this tension growth and higher-level cognitive functioning occurs. According to \citet{kolb1984experiential}, learning is a process filled with tension and conflict, and new knowledge, skills and attitudes are achieved through experiential learning, which consists of four modes and required abilities of learners: concrete experience abilities, reflective observation abilities, abstract conceptualisation abilities and active experimentation. First of all, individuals must openly involve themselves in new experiences, reflect and observe them from various perspectives, create concepts that can be integrated into more abstract theories as well as they need to be able to use these reflections and theories in active daily decision-making and problem-solving.  

Meaning of environment in learning should be emphasised. Learning concerns of transaction between an individual and the environment, learning does not happen only inside of individual's thoughts, experiences and processes but is dependent on the real world environment. Understanding of different learning styles and modes assists in supporting individuals in learning and problem-solving.

\section{Team performance and learning}

According to \citet{schein2010organizational}, concept of culture refers to and helps to explain some seemingly incomprehensible and irrational aspects of what is going on in groups, organisations and other kinds of social units, that share history. \citet{schein2010organizational} divides culture into three levels: artefacts (visible and feelable structures and processes and observed behaviour), espoused beliefs and values (ideals, goals, aspirations, ideologies and rationalisations) and basic underlying assumptions (unconscious, taking-for-granted beliefs and values). Climate of the group should not be mixed with culture of the group, it should rather be considered among artefacts. However, essential point of view \citet{schein2010organizational} provides is how culture in organisation or group level is easy to observe yet very difficult to decipher. Put in other words: researchers are able to observe and make remarks on what they see and feel, yet they are unable to reconstruct the deeper meaning of those observations to the group. Cultural analysis and understanding of dynamics of a group should begin in observing and asking members the norms, values and rules that shape practicalities of work in day-to-day level.

In order to create new value and competitive advantage in rapidly changing and uncertain organisational environments, new managerial imperative is growing focusing on teams. Thus, supporting teams in their work and understanding the aspects of team learning is required \citep{edmondson1999psychological}. Recent studies has moved the focus from individual learning to team learning. Edmondson's definition of group learning stems with definition of \citet{argote2001group}, who emphasises that knowledge is acquired, shared and combined through processes an outcomes of group interaction, focus being on processes. 

Work team refers to small group of people that exist within the context of a larger organisation, members share understanding of being member of the team and its tasks, responsibility for product or service team is working on \citep{hackman1987design}; \citep{alderfer1983intergroup} as well as its performance \citep{edmondson1999psychological}. Additionally, team members have supplementary knowledge and abilities compared to each other, and they share a goal, targets and way of working and approach \citep{edmondson1999psychological}. According to  \citet{katzenbach1993wisdom} great team performance consists of continuos work of shaping a common purpose, agreeing on performance goals, defining a common working approach, developing high level complementary skills and being transparent on the results. He emphasises that through disciplined action groups transform to teams and argues how demanding schedules, long-standing habits and unwarranted assumptions tend to threaten team efficiency and performance.
 
In previous research, structural and design-related factors have been combined to have influence on work teams's effectiveness and team performance. Well-designed tasks and goals, suitable and functional team composition, as well as physical environment and practices ensuring transparent communication and information exchange, sufficient materials, resources and motivating rewards all affect team efficiency. \citep{hackman1987design}; \citep{goodman1988groups}; \citep{campion1993relations}. Along with these factors, leader behaviour plays a major role in enhancing team effectiveness, and can be facilitated for instance through coaching and setting directions to employees (Hackman 1987) \citep{hackman1987design}. This perspective explains teams effectiveness through organisation and team structures, whereas organisational learning research puts emphasis on cognitive and interpersonal variables when explaining effectiveness in teams and individuals \citep{edmondson1999psychological}. For instance, \citet{argyris1993knowledge} has argued how individual's negative beliefs about communication and interaction may inhibit learning behaviour and lead to ineffective working in an organisation. 

In addition, in order to function team needs a clear purpose and vision what makes it a team and why it exists. Teams get energy from significant performance challenges regardless of where they are in the organisation. Set of shared, demanding performance goals usually form a team, and personal chemistry or willingness to form a team may boost that. Thus, in order to receive great results teams should focus on performance regardless of the organizational hierarchy or what team does. Thus, team performance may exceed the results of what could be achieved if employees were acting alone as individuals without the team effort. \citep{katzenbach1993wisdom}

\citet{edmondson1999psychological} have studied factors that affect and influence learning behaviour in teams by studying in which conditions and to what extent learning occurs naturally. Learning behaviour of teams refers to activities that team members carry out and through which team is able to obtain, adapt and reflect data and outcomes of actions which further shapes and improves team behaviour. Such activities consist of reflection and improvement-aiming factors such as asking for feedback, transparent information sharing, asking for help, admitting and discussing about failures and errors as well as experimenting. Through such activities teams may observe changes in environment, customer requirements and improve collective understanding. In addition, team's ability to discover and react to unexpected situations and consequences of their actions is likely to improve through learning behaviour.  Consequently, compared to low-learning teams that tend to get stuck and be unable to solve problems, teams who master in learning are greater in confronting difficult situation and improve their work. \citep{edmondson1999psychological}

The composition of the team matters. Studies have shown how team performance, especially related to innovation, is improved when team consists of individuals with various and different set of skills and characteristics\citep{buijs2007innovation}. Homogeneity in teams easily leads to groupthink, routine work and repeating traditional daily practices, while even one or two different individuals can stimulate the innovativeness of a team, and actually, the outcasts and those who stand out from the group are required in order to think outside the box, challenge the status quo and present alternative solutions and ideas that would be missing without the participations of these individuals. \citep{sternberg1997creativity}

Question of team composition comes relevant especially when forming teams for innovation. According to\citet{buijs2007innovation} innovation teams are the heart and the engine of innovation process and essential for the rest of the organisation to accept changes and innovation results. As he suggests right people in the team is the premise for innovation, all the members should be chosen carefully starting from the leader. Furthermore, the leader should be allowed to affect on the formation of the rest of the team in order to ensure positive base for teamwork and innovation process. Accordingly, team membership should be based on voluntary. \citep{buijs2007innovation}

\subsection{Psychological safety in teams}

Trust, indeed, has been widely noted in research as essential factor in organisational teams and groups \citep{golembiewski1975centrality,kramer1999trust,shalley2004leaders,edmondson1999psychological}. Trust refers to one's willingness to be vulnerable in his actions as he expects his actions will not be judged and will be favourable to one's interests \citep{robinson1997corporate}. Interpersonal trust is involved in psychological safety, yet it also includes perception of mutual respect and overall climate where team members feel free to be themselves \citep{edmondson1999psychological}. 

Team psychological safety refers to Amy Edmondson's concept of team members shared belief team being safe for interpersonal risk-taking. Together with team efficacy these have great affect on team performance and learning in an organisational team work. \citep{edmondson1999psychological} integrative perspective suggests that team performance and outcomes can be shaped through both team structure and shared beliefs, in contrast to previous studies that separate structural and interpersonal factors from each other. For instance, employee's willingness to take interpersonal risks depends highly on the experience of team safety and person's beliefs how others will respond in ideas or situations involving uncertainty. Team psychological safety refers to interpersonal trust among team but beyond that mutual respect and caring. \citep{edmondson1999psychological}

As in changing and uncertain environments the importance of teams have been recognized, pressure on managers to understand and enhance team efficiency, work and learning has increased. Fast-pace environment requires organisations to enhance the ability of teams to learn and create environment where learning can occurs safely. While uncertainty affects all working life, change is faster and job security lower, psychological safety for individuals at work can be increased through great teams and teamwork. Employees should be engouraged to ask questions, seek for help and tolerate mistakes and uncertainty. \citep{edmondson1999psychological}

Psychological safety serves as a mechanism that assists in explaining how structural and interpersonal characteristics both have effects on learning and performance in teams \citep{edmondson1999psychological}. Psychological safety can be boosted for instance through structural factors such as context support and team leader coaching affecting behavioural and performance outcomes \citep{hackman1987design};  \citep{edmondson1999psychological}. Furthermore, climate of safety and supportivenesss encourages employees to seek for feedback and ask for help in addition to admit and reflect mistakes. \citep{edmondson1999psychological}

Communication about ideas among team has been widely recognized being related to idea generation, creativity and innovation (e.g.\citep{robinson1997corporate,mumford2002social,monge1992communication,amabile1996assessing}). According to \citet{staw1989tradeoff} social influence of others plays a major role for individuals' beliefs; attitudes towards job, for instance, rise from the social labelling of work by others.  Also \citet{salancik1978social} argue the essential role opinions of others may have on individual: individual's perception of her work and organisation can be greatly influenced by opinions of others. Additionally, team member's collective view of support they get from their leader has been related to the team's creative endeavours and success in them (e.g., \citep{amabile1998kill,amabile1996assessing}). 
 
Thus, as individuals oftentimes requires support and input from several invidivuals who help to challenge ideas in constructive ways, teams are essential in generating and implementing ideas \citep{mumford2002social}. Stimulating those constructive individuals for creative actions may be valuable \citep{robinson1997corporate}. In addition, including team members in ideation assists in idea implementation and through participation new ideas are not that likely to be rejected or abandoned \citep{agrell1994team}.

Leaders have a great role enabling creative behaviour in teams and individuals, yet team members also influence essentially in others. Thus, by utilising various human resource practices leaders should create an environment where creativity is encouraged and supported. \citep{shalley2004leaders} Study of \citet{ancona1992demography} argue how changing the structure of teams is not sufficient and does not lead to improved performance, rather the leader and the team should find ways to foster positive effects of the team processes and reduce the negative ones. At team level this may mean focus on enhancing negotiation, problem-solving and conflict resolution skills while at organisational level leader should protect the team from external political pressures and reward the team from performance outcome instead of functional ones. \citep{ancona1992demography}

\chapter{Experimentation-driven developing}

\subsection{Experimentation-driven approach}

According to \citet{thomke1998modes} experimentation refers to iterative trial-and-error process in which after every trial new information of a problem is gained. When dealing with problems which outcomes are uncertain, experimentation is a fundamental for learning. Likewise, experimenting works when the most essential sources of information do not exist or are unreachable. \citep{lee2004mixed}

"Important discoveries in science (such as artificial vaccines) and technology (such as the electric lightbulb) resulted from constant trial-and-error experi- mentation through which inventors systematically built up a knowledge base (Thomke 2003). "

"Experimentation advances an engineer?s understanding of new analytical concepts, promotes new ways of think- ing, and creates new engineering knowledge (Vincente 1990). More broadly, individuals who constantly impro- vise, tinker, and experiment are able to remain adaptive in fast-paced industries where new ideas and innovations are constantly in demand (Ciborra 1996)." (lee2004mixed-artsusta)

"This avoidance can be explained by the interpersonal or social costs of failure. Specifically, failures make one?s gaps in expertise and knowledge salient to oth- ers (Lee 1997), and avoiding failure helps to maintain one?s image and professional standing among colleagues (Wolfe et al. 1986). Interpersonal costs of failure are exaggerated when people lack ?psychological safety.?" (lee2004mixed)

"For example, differences across organi- zations in psychological safety have been shown to affect the level of anxiety people feel when confronting ambiguity and uncertainty (Schein 1985). Organizational differences in psychological safety can be created by supportive structures such as information and reward systems (Edmondson and Mogelof forthcoming) and by the words and actions of high-level management;"

"Such failures can be bene- ficial because they provide the experimenter with new knowledge about the solution and thereby facilitate inno- vation and performance in the long run (Sitkin 1992)."

"even if the experiment fails, new knowl- edge is created that narrows the scope of subsequent trials." \citep{lee2004mixed}

"Each trial in experimentation generates information about a solution that the experimenter could not know in advance. Information learned in a previous trial can be used to modify subsequent experimental designs, condi- tions, or even the nature of the desired solution (Thomke et al. 1998). Tasks that are conducive to effective experi- mentation are those that allow multiple problem-solving trials and present opportunities to use knowledge gained from earlier trials to enhance learning in subsequent trials." \citep{lee2004mixed}

"Although little research has examined organizational conditions that promote experimentation, many studies identify pre- dictors of similar behaviors such as learning, creativ- ity, information seeking, and other interpersonally risky but organizationally desirable behaviors. This work has found that creativity is related to organizational cul- ture, reward systems, supervisory encouragement, trust, and resources (Amabile et al. 1996). Feedback, infor- mation, help-seeking, and issue-selling behaviors are all predicted by supportive organizational norms, leader- ship openness, and trust (Ashford and Northcraft 1992,Ashford et al. 1998, Lee 1997, Morrison 1993). Proac- tive learning behaviors are related to supportive orga- nizational contexts (access to resources, information, training, and supportive reward systems), leader coach- ing (Edmondson 2003), and routines that encourage exchange of relevant information, reduce sensitivity to feedback, decrease defensiveness, and increase trust (Argyris 1994)." (lee2004mixed)

"There is evidence that when normative values state that failures are expected and acceptable as part of learning, people are less hesitant to discuss mistakes (Edmondson 1996)"

"Rewards systems that pun- ish failures increase the costs of experimentation, and may make individuals reluctant to experiment (Thomke 2001)."

"Evaluative pressure is distinct from coaching, in which close attention or monitoring is provided to facilitate rather than evaluate performance. Indeed, monitoring in the context of supportive coaching can actually enable interpersonal risk taking (Edmondson 1999, 2002), while close and constant evaluation intended to identify and expose failures has been shown to inhibit creativity (Amabile et al. forthcoming) and make novel or unfa- miliar tasks more difficult (Zajonc 1965). " (lee2004mixed)

"Inconsistency and Experimentation. Much of the res- earch noted above has focused on how single variables? e.g., normative values, instrumental rewards, or evalu- ative pressure?independently affect innovation behav- iors. For example, Amabile et al. (1996) showed that eight organizational conditions individually predicted creative performance, and Ashford et al. (1998) exam- ined four antecedents of issue selling. These studies assume an incremental or additive model of influences on behavior. One implication of this componential per- spective is that improving any one of various organiza- tional factors should increase these behaviors.
We are interested instead in how combinations of organizational variables affect innovation behaviors. A combinational perspective assumes that the combina- tion of conditions employees face may be as influen- tial as the individual conditions themselves. The Bank of America example illustrates what can happen when normative values are changed to explicitly encourage experimentation and instrumental rewards discourage it. Inconsistency in organizational conditions may actually do more harm than good, because it creates uncertainty in which individuals do not know which factor (e.g., normative values or instrumental rewards) will shape the organization?s response to their actions." (lee2004mixed)

"In contrast, inconsistency may reduce psychological safety and thus experimentation. First, inconsistent con- ditions make the rules unpredictable and ambiguous. The uncertainty about whether one will be punished cre- ates a state of mild fear, which is antithetical to feel- ings of psychological safety. Second, facing the need to simultaneously serve contradictory aims itself may create anxiety, lowering psychological safety. Inconsis- tent messages place people in a bind (Argyris 1982) because they communicate two incompatible goals (e.g., ?experiment with new ideas, but don?t fail?). Facing this, people may experience emotions of fear or anxiety that make taking action and not taking action equally unpleasant alternatives (Argyris 1990). Third, inconsis- tency has been shown to create cognitive and emotional responses such as suspicion, mistrust, and confusion, leading to ?threat rigidity,? a tendency towards risk aversion, behavioral inhibition, suppression of activity, avoidance, lack of openness, and an inability to try novel behaviors (Masserman 1971, Staw et al. 1981)." (lee2004mixed)

"While this argument leads to the somewhat intuitive idea that consistently encouraging organizational conditions would lead to more experi- mentation behaviors than inconsistent conditions, it also suggests a less intuitive scenario. Specifically, it is possi- ble that individuals will engage in more experimentation behavior when organizational conditions consistently discourage experimentation than when some conditions encourage experimentation and others do not. In the con- sistently ?discouraging? situation, individuals are clear about the rules and constraints they encounter, and there- fore may experience more psychological safety than they would when facing the uncertainty created by inconsis- tent conditions. If so, they may experiment more.
Further, in consistently discouraging conditions, peo- ple working closely together can experience a sense of solidarity based on shared perceptions of negative work conditions (Edmondson 1999, George and Zhou 2002), whereas inconsistent conditions may lead to mistrust and suspicion that undermine psychological safety. " (lee2004mixed)

"The combinational perspective has been used to examine measures of organizational performance such as productivity, manufacturing quality, and efficiency, but it has not been applied to the study of individ- ual behaviors within the organization. Meyer et al. (1993) found that current theories of individual and group behavior in organizations, ranging from per- sonality, motivation, task design, work group design, and organizational demography, have largely adopted a componential rather than combinational approach." (lee2004mixed, t�h�n systeemiajattelupointti) 

"The componential perspective suggests that organizational conditions (such as normative values, instrumental rewards, or evaluative pressure) indepen- dently affect innovation behaviors. Thus normative val- ues that encourage experimentation will lead to higher levels of experimentation behavior, regardless of instru- mental rewards and evaluative pressure; instrumental rewards that do not punish failures will lead to higher levels of experimentation behavior, regardless of nor- mative values and evaluative pressure; and individuals under high evaluative pressure will experiment less than individuals under low evaluative pressure, regardless of normative values and instrumental rewards.
The combinational perspective suggests that the inter- action between organizational conditions is also impor- tant. This perspective suggests that when organizational conditions are consistent?for example, when norma- tive values, instrumental rewards, and evaluative pres- sure all encourage experimentation?there will be more experimentation than when organizational conditions are inconsistent?when some encourage experimentation and some discourage experimentation. This perspective also allows the possibility that experimentation will be greater when organizational conditions consistently dis- courage it than when they are inconsistent." (lee2004mixed)

"Although the literature on innovation has emphasized organizational-level structures and processes, an orga- nization?s ability to introduce a new product, develop unique processes, and leverage new technologies begins with individuals coming up with new ideas and try- ing these ideas out to assess their feasibility (Argote and Ingram 2000). Understanding conditions that enable individuals to engage in experimentation behavior is thus an important element of understanding organizational innovation (Thomke 2003)."(lee2004mixed)

\citet{lee2004mixed} argue more holistic perspective is needed, it is not sufficient to change only one organisational attribute in order to foster innovation. 
"We argued that inconsistency in organizational conditions?when some encourage experimentation but others do not? might reduce experimentation. "

"Both studies found partial support for the notion that individuals experimented more under instrumental rewards that did not penalize individuals for failures. " (lee)

"Inconsistency in organizational conditions?when some encourage and others discour- age experimentation?may undermine experimentation behavior, with one factor rendering the other ineffec- tive." (Lee)
"Yet, our results showed that these disabling effects of inconsistency only occurred for individuals under high evaluative pressure. Evalua- tive pressure might decrease psychological safety and might make individuals more vulnerable to the uncer- tainty inconsistent conditions create. Counterintuitively, both studies found that individuals under low evaluative pressure experimented more when organizational condi- tions were inconsistent than consistent. In Study 2, those with low evaluative pressure experimented more under inconsistent conditions than when both normative val- ues and instrumental rewards consistently encouraged or discouraged experimentation."

"All of these explanations have in common the obser- vation that, facing unpredictability, individuals under high evaluative pressure are more likely to become inhibited, fearful, narrowly focused, and rigid, while individuals under less evaluative pressure are more likely to become proactive, optimistic, thoughtful, and risk seeking. " (lee)
Among research it is widely recognized that in the heart of problem-solving process is continuous trial and error which are directed by some amount of insight about the possible direction of the solution \citep{barron200thinking}. 

According to \citet{thomke2001enlightened}, in the beginning every product is an idea, that was being shaped through the process of experimentation, and the ability to do experimentations is actually a measurement of company's ability to innovate. Thus, experimentation lies at the heart of company's ability to be innovative. 

Experimentation is essential in order to learn about the idea, concept and prototype and whether it actually addresses a new need or a problem or solves the one at hand. Prototyping is critical part of the process, as testing the prototype in a real environment gives instant and valuable feedback for further development. \citep{thomke2001enlightened}

Current business is remarkably dependent on services, yet innovation techniques and processes remain focused on products. Systematic learning methods are needed in order to avoid occasional successes and provide more stable base for consistency and productivity of service development. Experimentation does not only concern product development, it should and can also be applied to service design and development. \citep{thomke2003r} 
Experiments concerning service development are most useful when conducted in circumstances of real life, as the feedback is instant and customers transactions real. However, live experiments are in risk to harm customer relations, hurt the brand and may be more difficult and resource-consuming to execute and measure. \citep{thomke2003r}

Bank of America stands as an example of service prototyping by having conducted experimenting several years in order to create novel service concepts for retail banking. They have set up an experimentation laboratory in some of their banks where actual customers during normal office hours can test novel ideas. Feedback is collected and experiments measured in order to learn as much as possible for further experiments. During this process they have learned radically about the system approach and far more than only 
Bank of America has faced several \citep{thomke2003r}

"As important as the business benefits is the enormous amount of learning that the bank has gained. Carrying out experiments in a service setting poses many difficult problems. In grappling with the challenges, the bank has deepened its understanding of the unique dynamics of service innovation. It may not have solved all the prob- lems, but it has gained valuable insights into the process of service development-insights that will likely provide Bank of America with an important edge over its less ad- venturous competitors."

Conducting experiments in live settings causes distraction for both customers and employees: customers may be confused by new processes and employees find it difficult to adapt to new routines. 

"In a lab,experiments are routinely undertaken with the expectation that they'll fail but still produce value.
In the real world, there is pressure to avoid failure."

"To address this unanticipated effect of the program, senior management abandoned the traditional bonus system in the test branches in January 2001, switching all associates to fixed incentives based on team perfor- mance. Most associates welcomed the change, which am- plified their feeling of being special while also underscor-
In a lab,experiments are routinely undertaken with the expectation that they'll fail but still produce value.
In the real world, there is pressure to avoid failure.
ing top management's commitment to the experimentation process."

" But every company will have to go through its own testing period to arrive at the balance that is right for its people and that doesn't un- dermine the fidelity of its experiments. It's important to note, also, that staff turnover in the test branches dropped considerably during the program. The difficulties employees faced paled in comparison to the enthusiasm they felt about participating in the effort."

"If the capacity of the entire market was not well managed, too many experiments would have to be performed at a single branch, increasing the amount of noise sur- rounding each one. And if the team were to run out of ca- pacity, it would be forced to do tests sequentially rather than simultaneously, which would delay the process."

"Einally, there was the issue of failure. In any program of experimentation, the greatest learning comes from the most radical experiments - which also have the highest likelihood of failure. In a laboratory, radical experiments are routinely undertaken with the ex- pectation that they'll fail but still produce extraor- dinarily valuable insights, often opening up entirely new frontiers for investigation. In the real world, however, there are inevitably pressures to avoid fail- ures, particularly dramatic ones. Eirst, there's the
fear you'll alienate customers. Second, there's the fear you'll damage the bottom line, or at least shrink j'our own source of funding. Einally, there's the fear you'll alienate top managers, leading them to pull the plug
on your program."

"Any service company de- signing an experimentation program will have to care- fully weigh the risks involved against the learning that may be generated."

"experimentation in a live setting entails particular challenges. But however daunting they may initially seem, the challenges can be met, as Bank of America's effort suggests. In just a few years, the bank has achieved important benefits that are having a real impact on its business and its future."

According to \citet{thomke2003r} constant changes in practices and processes are required in order to perform experiments. Furthermore, the amount and value of learning achieved from experiment is the success measure for it: the more has been learnt and the more valuable insights, the more successful an experimentation is.  

According to \citet{thomke2001enlightened}, critical part of innovation process occurs when first prototypes are generated, as at that point they can be further tested with customers, discussed and evaluated. 

For instance, development process that lead to the innovation of a light-bulb consisted of repsated iteration of experiments, analysing the outcomes, learning from them and making changes for the next experiment. \citet{thomke2001enlightened}
In the beginning of experimenting process, possible solutions are created or selected and furthermore tested again specific requirements. Trial-and-error experiment provides new information, which is something a person conducting the experiment usually could not know beforehands, thus something referred as "error". Overall outcomes of the experiment are analysed and solutions refined accordingly. \citet{thomke1998modes} Through trial and error process various design alternatives can be tested and generated, essential being reflection after each experiment and making changes accordingly to next experimentation round. 

Even though novel experimentation techniques are useful and may reduce the cost and time used for product development, adopting experimentation techniques may require changes in organisational culture or way of doing work \citep{thomke1998modesd}. However, adapting changes may lead to increase in productivity and affect the overall competitive positioning among companies. 
 
\citet{thomke1998managing} defines experimentation as a form of problem-solving and essential part of innovation activity, and is highly related to the innovation process as a whole affecting to the cost and time of the process. Experiment highly relates trial and error learning. Experimentation process is presented as iterative four-step learning cycle consisting of designing, building, running and anaylizing the experimentation. In the first phase, experimentation is designed based on the previous experience or good guess on solution. In second phase the needed prototypes or spaces are build for experiment, following third phase where experimentation is actually conducted. Analyzing phase is essential in order to learn from the results and being able to conduct the cycle effectively again. Indeed, oftentimes experimentation is a matter of repeated trial and error. 

\citet{thomke1998managing} suggests managing different experimentation modes has affects on the productivity of the product design and development process meaning better managed time and costs. Experimentation modes are for instance computer simulating or rapid prototyping. Rapid prototyping refers to situation where developer quickly generates an inexpensive and often physical prototype that simulates the main function of the product. Fast prototypes allows testing in real environment and offers valuable feedback and reduces time and costs used for developing. 
Furthermore, \citet{thomke1998managing} defines experimentation efficiency, referring to "economic value of information learned during an experimental cycle, divided by the cost of conducting the cycle." The more inexpensive (costly) an experimentation is and the more valuable (valueless) gained information is, the higher (lower) is experimentation efficiency. 

Oftentimes experimentation is conducted by using as simple prototypes of the intended-product as possible in order to experimentation remain light and cost-efficient. As experimenting once should not be expected to lead to "right" and successful solutions, may be efficient to plan and conduct multiple experiments in order to get closer to the problem solution \citep{thomke1998modes}. 

Experimentation-driven approach for innovation differs from other methods for managing uncertain and innovation-focused projects in that it emphasises learning far more than other methods. Experimentation should not only be considered as method for learning but it's role should me further: a tool and everyday practice to guide company's strategy-making, business models and behaviour \citep{davenport2009design,mcgrath2010business}

Through experimentation new information and ideas can be generated and in even new opportunities can be found \citep{tuulenmaki2011art}. 
New possible outcomes may reveal during experimentation (McGrath 1999, samin gradusta)

According to the process of experimentation-driven innovation of \citet{tuulenmaki2011art}, there are three types of ideas. Opportunity idea, execution idea and implementation idea. 

Opportunity idea refers to an idea which is imagined to solve specific problem as well as something that brings closer to the solution. 
Zappos-esimerkki Hsieh 2010 -arstusta
Instead of writing a business plan a founder came out with experimentation idea: the easiest way possible to try out if the implementation idea is worth further development. 
Experimentation idea assists in testing the critical assumption and figuring out, whether the idea is worth taking further risks. (Sykes & Dunham 1995) 
Execution ideas are the outcomes from experimentation ideas, those ideas that have been through iterating and validating process chosen to further development and implementation. Execution ideas gather all the learnings from experiments, through which the original opportunity idea is modified in order to reach the final design plan. According to \citet{hassi2012experimentation} experimentation-driven innovation process consists of series of iteration with the three idea types, opportunity idea, experimentation idea and execution idea. 

\citet{thomke1998agile} defines a term development flexibility, which refers to 
It is an alternative for forecasting the future and works as a powerful method in risk-managing of development. As forecasting the future has become increasingly difficult, emphasis should be put on managing the risks of failed decisions. Shorter development cycles
Thus, development flexibility is critical as companies are no longer (if they ever actually were) possible to accurately forecast unknowable future. Furthermore, unknown, unstable and changing customer needs 
Through development flexibility entire product changes can be avoided, as design commitment and decisions can be made late phase in the process. Furthermore, significant reduce can be seen in time and cost used for development. 

"Development tlexiblllty can be expressed as a function of the incremental economic cost of modifying a product as a response to changes thai are external (e.g., a change Ui customer needs) or internal (e.g., discovering a better technical solu- tion) to the development process. The higher the economic cost of modifying a produa, ihe lower the development flexibility."\citep{thomke1998agile}

Other approaches
prototype-driven approach refers to a method in which customer feedback is acquired through prototyping in an early phase of the process in order to make changes affordably
Specification-driven approach, in turn, refers to a process where design is freezed after the specification is complete. Prototyping was more successful and lead to more successful products and made with fewer design resources. In other studies higher customer satisfaction and quality and company's performance have been related to more flexible development process, in which changes can be made in the very late phase of the development process. \citep{thomke1998agile}

Anticipating and exploiting early information can save a lot of resources in the development process. If problems are shown in the late-stage of the process, they can be even 100 times more costly than the ones discovered in the early stage. According to IDEO, an innovation and design-firm, using human-centered design-based approach, the key elements in the design process and prototyping is it being rough, rapid and right. The right-element reminds that even though the prototype itself is likely to be incomplete, it has to show the right specific aspects of a product. This forces developers to decide the factors that can initially be rough and those that must be right. In addition, exploiting early information serves as a good method for developers reflecting changing customer preferences. Briefly, information in the early stage of the developing process should be listened and discovered carefully, as the problems are cheaper and easier to solve. 

For enlightened experimentation \citet{thomke2001enlightened} puts emphasis on combining new and traditional technologies. 

Experimentation causes inconsistent outcomes. As organisational conditions may be inconsistent, employees are not eagerly engaging to experimentation. \citep{lee2004mixed}
"Because failures are inevitable in the experimentation process, we argue that conditions giving rise to psychological safety reduce fear of failure and promote experimentation. Based on this reason- ing, we suggest that inconsistent organizational conditions?when some support experimentation and others do not?inhibit experimentation behaviors. An exploratory study in the field, followed by a laboratory experiment, found that individu- als under high evaluative pressure were less likely to experiment when normative values and instrumental rewards were inconsistent in supporting experimentation. In contrast, individuals under low evaluative pressure responded to inconsistent conditions with increased experimentation. Our results suggest that evaluative pressure fundamentally alters an individual?s experience of and response to uncertainty and that understanding experimentation behavior requires examining effects of multiple organizational conditions in combination."

Also according to \citet{lee2004mixed} experimentation behaviour is essential for innovation. 

In the case of Bank of America failure rate was set in 30 per cent in order to show the support for essential failure

"Bank senior management voiced strong support for innovation and explicitly recognized and communicated that experimentation with new ideas necessarily pro- duced failures along the way. Indeed, a failure rate of 30percent was targeted as indicative of sufficient risk taking and novelty. Initially, however, employee compensation continued to be based on measures of routine perfor- mance (such as opening new customer accounts). The espoused goal of increasing innovation thus was incon- sistent with the reward system; individuals? compensa- tion could suffer from time spent experimenting with new ideas or from failed experiments.  "

Current research examining antecedents of behaviors such as learning, creativity, and experimentation has emphasized main effects of single organizational variables. Implicit in this approach is the idea that changing one organizational condition can lead to improvement in behaviors integral to innovation. We argue, in contrast, that changing a single organizational condition without changing others creates inconsistency that instead may inhibit innova- tion behaviors. 

However, \citet{lee2004mixed} have studied the inconsistencies that are likely to inhibit innovation behaviour. Where current research claims how affecting on one organisational condition is likely to foster behaviour essential for innovation, this approach reveals how changing only one organisational condition may lead to inconsistency between work tasks and expectations and lead to decrease in willingness to act towards innovation. Learning, creativity and experimentation are all attached to innovation. 

Combining routine work and innovative experimenting can be challenging \citep{lee2004mixed}.

\section{innovation-shitti�, siirr� pois tai mieti uudestaan mihin menis}
Why innovation and creativity should be a matter for an organisation? Studies have well established the positive relation between creativity and innovation skills of an organisation and organisational performance \citep{jung2003role,mumford2002leading}. Innovation and creativity are highly related, yet not the same thing: According to \citet{hennessey19881}, individual creativity stands for an essential building block for organisational innovation and also \citet{sethi2001cross} argue creativity being essential in new idea generation and design processes that aim for innovative solutions. Other studies also emphasise the role of creativity a first step in creating something novel, whereas innovation refers to the implementation phase of the novel ideas in individual, team or organisational level \citep{shalley2004leaders};\citep{amabile1996assessing};\citep{mumford1988creativity}. 

Innovation process itself can be approached from several angles: first of all, content of the innovation has to be clear - whether the purpose is to innovate new products, manufacturing processes, ways of organising or ways of dealing with people. Secondly, psychological process of the innovation team has to be understood, essential being shared understanding, level of comfort with ambiguity and degree of trust between team members. Thirdly, creative process of the team, meaning idea producing process, needs to be understood and efficiently facilitated. Finally, the role of leading plays a major role, and together with playful attitude innovation process is likely to succeed. \citep{buijs2007innovation}

Several factors have been recognized to affect on organisational innovation, yet many researchers have stated leadership behaviour being one of the most important. \citep{jung2003role,amabile1998kill,jung2001transformational,mumford1988creativity} \citet{jung2003role} identify four hypotheses how top managers' leadership styles may affect both directly and indirectly their companies' ability to innovate. Indirectly here stands for instance leader's possibility to empower employees and build organisational climate optimal for innovation. The study shows a positive relation between transformational leadership style and empowerment as well as innovation-supporting organisational climate. 

Social innovation refers to the generation and implementation of novel ideas concerning people in demand to organise their interpersonal or social activities and interactions in new ways in order to achieve common goals. Results and products of social innovation, like other types of innovation, are likely to vary depending on the breadth and impact of the innovation. \citep{mumford1988creativity} \citet{mumford2002social} presents four factors affecting social innovation: active exchange of ideas and information in supportive climate, tangible and low-cost ideas that can be at the fewer guessed to be beneficial, support from upper level management, and effective communication through the innovation process in order to proceed from the idea to implementation.

Innovation can be contributed by encouraging idea generation, but also creating a climate of autonomy, offering intrinsic and extrinsic rewards and engaging employees with their work \citep{amabile1996assessing}; \citep{amabile1998kill}. Furthermore, characteristics associated with innovation are integration of work units, decentralisation of control and professionalisation are likely to effect innovation in a way that through these suitable environment for innovation, dynamic idea exchange and implementation is created \citep{mumford2002social}.

Managerial practices for technological innovations have been widely studied. According to \citet{quinn1985managing} the essence lays in accepting the chaos of development. In addition, large and successful companies and their leaders listen carefully their users' needs, develop according this customer demand, define clear goals and framework for the work, encourage teams to challenge the status quo and find alternative solutions while avoiding detailed and technical or marketing plans in the beginning. Instead, they focus on early prototyping and iteration. 

According to studies creativity and innovation in an organisation requires integrated organisational approach, right climate, appropriate incentives for innovators, and a systematic way and resources to transform an idea into an innovation.  In individual level, creativity and innovation calls for various skills, such as teamwork, communication, coaching, project management, learning and learning to learn, visioning, change management and leadership, and ability to develop these skills. Oftentimes, even though the climate and practices are right for generate innovations, problems are faced when attempting to manage the change process. \citep{roffe1999innovation}

Social cohesion may inhibit innovativeness of the team and its individuals especially beyond a moderate level, while employees are more likely to settle on group think and traditional daily practices \citep{janis1982groupthink}. However, according to the study of\citet{sethi2001cross} when a team shares superordinate identity, is encouraged to take risks, lets customer's requirements be heard, and actively lets senior management monitor the project, team is more likely to present innovative ideas and perform in innovative ways. According to this study, functional diversity does not effect on innovativeness, but team's superordinate identity can be strengthened by encouraging risk-taking and weakened by social cohesion.

\citet{mumford2002social} argues in his study of Ben Franklin's social innovations, that key factor in successful social innovation lays in fast demonstrating, which he also refers as experimenting. Thus, in order to drive for social innovations, opportunism and showmanship of an individual or team may be required. \citep{mumford2002social} Furthermore, according to \citet{monge1992communication} group communication is likely to increase innovation under some circumstances,  and also \citet{katzenbach1993wisdom} argues for culture of strong team performance. However, \citet{amabile2004leader} emphasise how, ultimately, truly novel ideas raise from individuals, making them the ultimate source of any new idea or solution to a problem \citep{amabile2004leader}.

\section{Creativity, intrinsic motivation and everyday problem solving}

Creativity and innovation have gained wider acceptance as important factors creating value in organisational performance \citep{mumford2002leading}. Creativity and innovation have for instance been studied to have enhancing impact to organisations profit and growth \citep{nystrom1990organizational}. Creative thinking and actions require time, and contradictory, in fast-paced and rapidly changing world and working environment managers should allocate employees sufficient time for creative thinking and experimenting novel approaches \citep{shalley2004leaders} 

\citet{amabile1996assessing,amabile1998kill} defines creative thinking as a way how people approach problems at hand and come up with solutions. Creative thinking does not stand for intellectual capacity of an individual to create something new but rather as a combination of past experiences which creates expertise and the ability to apply creative thinking skills to these experiences and invent new solutions. 

According to \citet{sternberg1997creativity} a company can enhance its creative skills by focusing on six resources: knowledge, intellectual abilities, thinking styles, motivation, personality and environment. \citet{sternberg1997creativity} argues that too much information may hinder change and be seen as rigidity in thinking. Therefore, one should not over-weight the criticism of senior people in an organisation, and at least consider the chance for rigid thinking and intolerance for change. 
In addition, needs to be noted and understood that employees' thinking styles are shaped through what is rewarded, meaning, that if organisational environment rewards well-behaving and instruction-following thinking style and action, employees tend to implement their style to that. We are urged to adapt to organisational style and fit in, and when this is not possible, people tend to leave. \citep{sternberg1997creativity}

Divergent thinking refers to individual's ability to find multiple alternative solutions and ideas to problems at hand, and has been related to serve as a key capacity affecting creative thinking \citep{guilford1967creativity}. Accordingly \citet{mumford1988creativity} emphasis that creative people consistently and with confident tend to seek for alternative solutions, even under uncertain conditions. Even though expertise and  intelligence have been related to problem solving, series of causal analyses carried out by \citet{vincent2002divergent} revealed unique effects divergent thinking had that were not attributed to intelligence and expertise. 

\citet{shalley2004leaders}, in turn, argue that through developing extensive set of skills, employees may learn to be more comfortable and confident in thinking from different perspectives, finding various alternative solutions, trying out novel things and seizing opportunities. According to \citet{hennessey19881} individual creativity requires ability to think creatively, generate alternatives, engage in divergent thinking and tolerate or suspend judgment. Through this perspective creativity can be considered as a skill that can be learned and strengthened. Understanding of individual's creativity and ways to influence and improve it gives managers guidelines when creating an environment and leadership that support organisational innovation \citep{redmond1993putting}.

Several factors form the basis of creativity skills of an individual, such as personality, technical knowledge, expertise, motives, and the supervisor's feedback style. In group level factors form of task structure, communication styles and task autonomy, and finally in organisational level strategy, structure, culture, climate and available resources all affect how creative actions are encountered. \citep{jung2003role}

Several case studies has showed that creativity insights emerge gradually through the network and actions of an creative individual. Study of creativity is a combination of two different disciplines and research approach: sociological and historiometric lenses study the conditions in which creative actions and processes are likely to occur, whereas neurobiological approach presents neural structures and processes that are active and associated with creative outcomes. \citep{gardner1988creativity}

Generation of novel, alternative solutions requires problem-finding skills \citep{runco1988problem}, which has been indicated to be one of the best predictors of creativity in 'real world' activities, when studied 91 elementary school students \citep{runco1990evaluating}. These findings suggest leaders, in order to enhance creativity of employees, to support learning of these skills for instance by facilitating problem-construction \citep{redmond1993putting}.

\citet{kasof1997creativity} argued in his study that breadth of attention affects on creative performance of an individual: wide spread of attention is usually related to creative ability. By breadth of attention Kasof refers to "number and range of stimuli attended to at any time." Breadth of attention being narrow, individuals are able to focus on narrow range of stimuli and are better at filtering redundant stimuli from awareness. However, those individuals with wide breadth of attention tend to be more aware of irrelevant or extraneous stimuli, these individuals are strongly affected by their environment and are highly arousable.\citep{kasof1997creativity}

Studies of creative characteristics of individuals has revealed factors such as wide interest in various fields, autonomy, belief of being creative and independence in decision-making \citep{shalley2004leaders}. Intrinsic motivation is claimed to be one of the most powerful tools to creative action and non-traditional thinking \citep{amabile1996assessing,deciintrinsic,jung2001transformational}, as intrinsically motivated individuals usually prefer novel solutions, challenging status quo and trying out new ways for solving a problem at hand \citep{amabile2002creativity}. Broad interest stands for a sign of intrinsic motivation, which is also widely related to both creativity and well-being of an individual and innovation (e.g. \citep{hennessey19881}; \citep{csikszentmihalyi199916}; \citep{gardner1988creativity}). In their study  \citet{tierney1999examination}, found positive correlation between employee's level of enjoyment while working on a creative task at hand and the level of creativity.  

Without previous experience of the job routine and substance knowledge and expertise on the field creative endeavours are more rare. Even though has been argued how routine work and task familiarity is likely to lead very habitual performance \citep{ford1996theory}, knowing the status quo may provide opportunities for creative actions through reflecting and practicing skills and activities requires in the field. \citep{shalley2004leaders} Knowing the field and what has already been discovered assists in finding alternative, creative solutions \citep{andriopoulos2000enhancing}.Furthermore, creativity is not restricted to artistic occupations only; it is required in various professions in which tasks presented involve complex, ill-designed problems where novel solutions are needed and status quo challenged \citep{mumford1988creativity}. Indeed, idea implementation may require even more creativity than idea generation \citep{mumford2002leading}.

For students of creativity, there is no surprise in attaching self-efficacy to creative actions \citep{mumford1988creativity}, yet recently problem construction processes have been recognised and combined to everyday problem-solving and real-world creativity \citep{getzels1975problem}; \citep{runco1988problem}. According to study of  \citep{gardner1988creativity} correlation between creative problem solving and everyday problem solving exists: they seem to have the same roots in information processing skills. 

In their study \citet{redmond1993putting} found how leaders supporting employees problem-finding and problem construction led to more unique and novel solutions. Leaders encouraged employees to find alternative solutions, approach problems from different perspectives and overall supporting several alternative problem-solving strategies. In addition, study showed how through motivational mechanisms, such as self-set goals, involvement and commitment, problem construction may have positive influence on solution quality and originality. Thus, problem construction is likely to have its greatest impacts on performance when in the process employee is allowed to express his values, needs and interests \citep{redmond1993putting}. 

Instead of managing creativity leaders should consider new approach: managing for creativity \citep{amabile2008creativity}. According to \citet{isaksen1983toward}, in order to support employee's creativity, leaders should focus on creating and maintaining and environment of supportive empathy, respect, warmth, concreteness, genuiness, trust and flexibility. These factors have been combined to general and task-specific efficacy needs \citep{mumford1988creativity}. Furthermore, through providing enough processing time for creating novel solutions is likely to enhance creative behaviour of employees \citep{isaksen1983toward}. As creativity refers to finding novel solutions and generating understanding of problems at hand, leaders could facilitate the process of resource allocation, feedback and task management in order to support employee's creative process \citep{mumford1988creativity}. 

Leader alone is not able to boost creative solutions in employees: it is also a matter of personal characteristics, previous knowledge of the problem at hand and expertise in the field \citep{mumford1988creativity}; \citep{redmond1993putting}. Thus, in order to achieve novel solutions and fresh ideas, leaders may seek employees who have great knowledge and expertise of problem at hand or provide employees education and possibilities to develop their problem construction skills and furthermore encourage approaching problems from various perspectives.  \citep{redmond1993putting} 

Furthermore, supporting employee's feeling of self-efficacy is likely to improve creative skills of an employee \citep{redmond1993putting}, and can be done through giving positive and realistic feedback, allowing adequate resources and physical support, clarifying task assignments, providing development support for employees, and assigning employees to appropriate tasks \citep{hennessey19881}. However, often acknowledging employee's skills, potential and accomplishments is likely to push an employee to the track of creativity \citep{redmond1993putting}. 

Should also be noted that depending on the job, different level and amount of creativity is required. Certain jobs that are highly involved with novel solutions urges for creativity as major breakthrough and innovative ideas, whereas more routine and repetitive jobs such as assembly line work requires creativity in developing the job practicalities. \citep{shalley2004leaders} 

Studies show employees who consider and believe creativity as valued outcome are more willing to generate ideas, experiment, communicate openly with others about ideas and through this, overall, their behaviour will eventually lead to creative outcomes \citep{shalley2004leaders}. Accordingly, \citet{csikszentmihalyi199916} presents the belief and feeling an employee has on the capabilities, pressure, resources and sociotechnical system of work environment affects highly on the success of creativity. Furthermore, pre-set obstacle, such as deadline, assists in focusing individual's attention to an urgent problem at hand, and has been noticed to stimulate creativity \citep{andriopoulos2000enhancing}. As employee who has the feeling of autonomy performs better, setting a deadline is not likely to threaten that autonomy, whereas showing someone how to meet that deadline would do \citep{mumford2002leading}.

Creative work is resource intensive where risk is involved \citep{mumford2002leading}. It is demanding and time-consuming\citep{mumford2002leading} and requires attention over long periods of time involving high level of ambiguity and stress \citep{kasof1997creativity}.Thus, organizational environment plays a major role in employees' creative skills, and such stifling factors may be positive challenge at work, encouragement from organisational level, support from work group as well as supervisory encouragement. Furthermore, organisational impediments can lead to decreased level of creativity. \citep{amabile1998kill} Hence, leadership has a great role in ensuring that the climate and culture, structure and practises of work and work environment together with human resource practices are supportive for creative endeavours to occur \citep{shalley2004leaders,oldham1996employee,mumford2002leading}.

Problem finding and construction, making connections and evaluating ideas are important for creativity \citep{mumford2002leading,vincent2002divergent}. Thus, when improving individuals possibilities to multiple alternatives, related ideas and example solutions, they tend to make more connections leading to creative actions \citep{amabile1996assessing}. 

\section{Experimentation as a method for developing and learning}

The essence of learning from experiments is to figure out what works and what does not in an experiment or idea. Thus, experiments should be designed and planned keeping in mind how to maximise the amount of learning and valuable insights, not focus on wrong details and successful experiment. Measurement of experimentation is essential, through defining correct measures one can actually know whether the experimentation was useful and essential learned or not \citep{thomke2003r}

\citet{thomke2003r} defines seven factors that can be learnt and set as measurements for an experiment. These are fidelity, cost, iteration time, capacity, sequence, signal-to-noise ratio and type. Most can be learnt from experiment when tested under conditions that represents as closely as possible actual use of final product, process or service. This refers to fidelity of an experimentation. However, when testing in actual environment, various variables may affect on the experimentation setting, and this signal-to-noise ratio should be taken into account. Right balance between the speed of experimenting and receiving feedback in order to learn is crucial for successful experimenting, and this iteration time should be measured and estimated: time from the planning an experiment to the moment when results are available and further used. Also, cost of experiments should be analysed by estimating cost of designing, building, running and analysing experiments. Capacity concerns the realistic estimation of number of experiments possible to conduct with decent amount of fidelity in planned period of time. Experiments can be conducted in series or in parallel depending on the project at hand, and thus the sequence of experiments can be measured. Experiment type refers to the level of change, which can vary from incremental to radical. 

According to \citet{edmondson1999psychological}, learning behaviour consists of seeking feedback, sharing information, asking for help, talking about errors and experimenting. Thus, experimentation behaviour seems to relate to learning of an individual, teams and organisations and can be supported by supporting these factors. Normal business consists of repetition, risk-avoidance and focusing on business outcomes \citep{buijs2007innovation}, while innovation requires novel solutions, thinking out of the box, risk-taking, breaking the rules, challenging the status quo and questioning the future \citep{burns1961management,kanter1984change,march1991exploration}.

One definition of experimenting considers it as personal trial and error process in which employees with their full potential are involved\citep{andriopoulos2000enhancing}. Experimenting serve as a great method when testing and validating abstract concept \citep{kolb1984experiential}.

Actually, idea implementation may require even more creativity than idea generation \citep{mumford2002leading}, and according to \citet{vincent2002divergent} creative work consists creative and innovation processes. Creative processes comprises of initial idea generation, whereas innovation process goes beyond the activities underlying the implementation of those ideas \citep{vincent2002divergent}.

According to \citet{buijs2007innovation}, innovation consists of coming up with novel ideas and implementing them. Ideating begins with exploring, developing and implementing the ideas, following introducing the ideas, which have turned into products or services, into the marketplace. Innovation process is a series of stages for processing the idea, and in the end of every stage the idea is reflected and evaluated before further processing. Evaluation points stands for usable tool for measuring the quality of idea but gives also understanding of how the evaluation process is going. In addition, while evaluating, team members also need to reflect the process and the idea, through which learning occurs. Also \citep{runco1994problem} emphasises how only after evaluation of ideas implementation can be discussed and performed and several studies show the essence of evaluation \citep{mumford2002leading,vincent2002divergent}. Useful questions in evaluation process could be "What went well?", "What can be improved?" and "What has been learned?" \citep{buijs2007innovation}. 

When dealing with novel solutions and challenging status quo, we are dealing with innovations. In order for company and its employees to be innovative, they need to take risks. Yet, at the same time usual management processes avoid risk-taking and focus on managing daily routine business. As \citet{quinn1985managing} stated it in his Harvard Business Review article: we love innovation and we urge for innovation, but we can tolerate it only if it is controllable and results everything remaining the same. \citep{quinn1985managing}

\citet{andriopoulos2000enhancing} define in their study a concept of perpetual challenging - a way to enhance creativity and innovation in an organisation. According to the concept adventuring occurs when goal is idea generation and through that process individuals are encouraged to face uncertainty in order to generate novel solutions. One tool for idea generation is scenario making, which purpose is to develop possible ways to tackle situation at hand. Through scenario making employees scans what is both known and not known about current problem or situation.  Experimenting process then consists of testing different scenarios generated in ideation phase and evaluating the outcomes in order to decide and develop the scenarios further to meet the needs of clients and industry. This calls for individuals skills to tolerate risks and uncertainties, as well as skills to constructively challenge and question colleagues ideas in order to use their full potential.

According to \citet{andriopoulos2000enhancing} facing and dealing with risk serves also as positive boost to creativity, as employees' learn new skills and strengthen their capabilities constantly and adapt new knowledge to already known. This, however, requires for safe environment which \citet{andriopoulos2000enhancing} refers as safety net: environment that tolerates failure. 

According to \citet{quinn1985managing}, fast multiple-idea prototyping leads to more innovative outcomes, offers essential information about ideas or product's quality, motivates employees, and helps the company and the team to cope with anxiety and uncertainty in development. Engaging lead customers in the interactive development process instead of market research seems to elucidate more relevant information about customer's demands, required changes and entry strategies. Thus, fast prototyping serves an essential way for learning from the iterative process. Market analysis, however, remain valuable when dealing with familiar products and productions, yet with radical innovations they may easily offer misleading information. \citep{quinn1985managing}

\chapter{Factors affecting experimentation}

As in todays' business world it has been widely recognised how creativity and innovation are essential for business growth, researchers have studied factors affecting creativity and innovation in organisations. \citet{amabile1998kill} has identified three factors being important for stimulating creative behaviour in individuals and organisations: individuals' intellectual capacity (creative thinking skills), expertise based on past experience and creativity-supporting work environment. Furthermore, \citet{oldham1996employee} consider creativity skills and characteristics of and individual as important, yet they add the importance of characteristics of organisational context such as job complexity, supportive supervision or controlling supervision. 

\section{Leadership behaviour}

\section{Supportive environment for experimenting}

\subsection{Psychological safety}

\subsection{Failure and risk}

\section{Factors hindering innovation and experimentation}

Several factors may affect on the gap between idea and action in employees of an organisation. The phenomenon of threat of employees in organisations is widely studied and consensus is rising how threat effects on cognitive and behavioural flexibility and responsibility in reducing manner. \citep{argyris1982reasoning} \citep{edmondson1999psychological}

One essential factor is the beliefs, emotions and actions of an employee. An employee is likely to inhibit learning as a result of feeling the fear of being rejected, under pressure or feeling they are placing themselves at risk \citep{edmondson1999psychological}, or when facing the potential for embarrassment of threat, even though their transparency and honesty would be highly important for the behaviour of the team \citep{argyris1982reasoning}. This may occur in a situation where an employee should ask for help, yet is afraid of admitting he lacks abilities, skills or knowledge. \citep{edmondson1999psychological}. In addition, admitting mistakes, asking for help and seeking feedback are all relevant abilities in the recent organisational world, yet threatening for an individual's image of himself and his skills \citep{brown1990politeness}. 

Particularly when large new, complicated systems at hand, meaning of co-operation in production, development and communication rises exponentially. Especially in large organisations innovation can be inhibited by the errors increasing as a result of complexity of the system and inability to control, understand or make intelligent decisions. Challenging as it is for one department, faculty or company to survive on its own without communication and help of others in design, production and other business-related decisions, with management that takes the complex environment into account, the disastrous effects resulting from lack of communication can be lessen. \citep{quinn1985managing} Yet, at the same time, as a result of the difficulty of managing complex situations, innovation may denote finding the core, boiling things down and focusing on the most essential elements \citep{katz1978social}.

\citet{quinn1985managing} list several barriers to innovation, including intolerance of fanatics, short time horizons, accounting practices, excessive rationalism and bureaucracy and inappropriate incentives. \citet{hayes1982managing} supplements the list with concern of top management isolation, arguing how top management oftentimes has too little contact and understanding of the environment and conditions at factory floor or customer requirements for innovative solutions. Top managers who tend to be financially-driven and are not familiar nor have experience with current technology and its possibilities, may fear technological innovations and perceive them as too risky. Thus, more familiar traditions remain with ease. \citep{hayes1982managing} 

Furthermore, \citet{quinn1985managing} argues enthusiasm is not yet widely accepted and tolerated characteristic of employees and refers to them as entrepreneurial fanatics. Larger companies may perceive them as causing embarrassment by challenging status quo and causing troubles. 

Short time horizons require companies to stay in continuous stream of quarterly profits, oftentimes at the cost of long time benefits, that innovations demand. Especially large companies easily favour narrow-minded actions such as quick marketing fixes, cost cutting and acquisition strategies over systemic thinking and process, product or quality innovations. \citep{quinn1985managing}

Rather than managing the inevitable chaos of innovation productively, these managers soon drive out the very things that lead to innovation in order to prove their announced plans. In the name of efficiency, bureaucratic structures require many approvals and cause delays at every turn. Experiments that a small company can perform in hours may take days or weeks in large organizations. The interactive feedback that fosters innovation is lost, important time windows can be missed, and real costs and risks rise for the corporation. Inappropriate incentives. Reward and control systems in most big companies are designed to minimize surprises. Yet innovation, by definition, is full of surprises. It often disrupts well-laid plans, accepted power patterns, and entrenched organizational behavior at high costs to many. Few large companies make millionaires of those who create such disruptions, however profitable the innovations may turn out to be. When control systems neither penalize opportunities missed nor reward risks taken, the results are predictable." \citep{quinn1985managing}

Organizational structures are also likely to enhance or hinder creativity in organisational, team or individual levels \citep{shalley2004leaders}. 

\section{Factors supporting innovation and experimenting} 

According to \citet{garvin2008yours} in learning organisation employees excel at creating, acquiring and transferring knowledge. They define building blocks for learning organisation: supportive learning environment, concrete learning processes and practices and leadership behaviour that reinforces learning. Building blocks can be considered and measured as independent components yet each of them vital to the whole. In order to improve long-term learning of an organisation, strengths and weaknesses of an organisation and its unit needs to be recognised. 

\subsection{Supporting learning environment}

Experimenting requires safe and supportive environment. According to \citet{edmondson1999psychological} team psychological safety should be the first essential building block of learning behaviour in work teams. Supportive learning environment consists of four characteristics: psychological safety, appreciation of differences, openness to new ideas and time for reflection \citep{garvin2008yours}. Likewise, \citet{mumford1988creativity} emphasise the meaning of environmental variables as a means to support employee's creativity by providing resources to stimulate fresh ideas of employees. Furthermore, strong positive relations between organisational environmental variables have been found; organisational encouragement as well as support for innovation and creativity from team improve employee's creativity \citep{amabile1996assessing}

Futhermore, \citet{amabile1998kill} suggests changes in organisational environment are likely to boost intrinsic motivation of an employee leading to increased creativity skills. Role of leaders and managers is essential; being a key person in organising group work and processes a leader may encourage employees to achieve shared goals \citep{amabile1998kill}. 

A company desiring for innovation should allocate resources and define long-term goals and actions accordingly. Even though companies urge to invest most resources in current lines, sufficient resources should be allocated for long-term growth and innovation. This includes providing an environment strong enough to seize surprising opportunities and tolerate unforeseen threats in all organisational, technical and external relations levels. \citep{quinn1985managing} Level of uncertainty can be reduced through goal-setting and fast prototyping \citep{mumford2002leading}. 

\citet{mumford1988creativity} have studied the gap between an idea an action, and revealed it depending on various attributes related to individual and circumstances. As physical work environment affects on creativity, information sharing and innovation in an organisation, it should be designed to support the natural flow of traffic through the building so that informal conversations between different functional areas are enabled \citep{shalley2004leaders}. 

Learning of employees occurs when employees do not fear being rejected, ask naive questions, make mistakes or present viewpoint of minority. Psychologically safe environment enables employees comfortably express their thoughts at work. Appreciation of differences is important, as opening minds for different ideas and world views increases both energy and motivation, brings out fresh thinking. Novel approaches are relevant for learning, thus employees should be encouraged in risk-taking and exploring and testing uncertain things. Lastly, providing time for reflection is likely to foster learning in safe environment. Instead of looking and judging by numbers of hours of work or results employees should be given enough time to reflect their work. Analytic and creative thinking are prevented under stress, heavy workload and too tight schedule. Under stress ability to recognise and react to problems and learn from experiences deteriorates. In supportive learning environment time for reflection is allowed. \citep{garvin2008yours} 

Interestingly, studies show how nominal groups perform remarkably better in ideation and brainstorming processes by producing greater amount of ideas than real groups. This may be due to the learnt practices and norms of a real work group, fear of failure that prevents free idea exchange and fear of evaluation and others judgement when suggesting creative solutions. \citep{jung2001transformational}

\citet{edmondson1999psychological} defines a concept of team psychological safety, that fosters learning behaviour in work teams by reducing the risks of embarrassment or threat and increasing mutual trust between team members. Oftentimes in work teams employees are not willing to tell their ideas or errors out loud as they are afraid of being labelled as incompetent. Thus, they prefer staying silent ignoring how it may lead to negative consequences for the team performance. When team members share feeling of respect and trust of others, and stay confident on other member's not using the errors against them, they are more likely to put more weight on the benefits of telling concerns out loud. When knowing that well-intentioned interpersonal risks are not punished is a shared belief of a team, team members are more likely to take proactive actions that foster learning leading to more effective performance. 

Even though building mutual trust may not lead to mutual respect and caring among team members, it is essential for creating psychologically safe environment and through building trust a foundation for further development of team psychological safety is built. \citep{edmondson1999psychological} 

\citet{edmondson1999psychological} lists factors affecting psychological safety in teams, including context support and team leader coaching. Context support refers for instance to access to information and resources needed. Safe environment that fosters creativity also takes into account employees' perceptions of just and transparent decision-making as well as applied actions \citep{shalley2004leaders}.

Accordingly, \citet{amabile1998kill} suggested that creative thinking can be encouraged by shaping organisational culture such that employees feel encouraged to tell their ideas out loud freely and without judging, increasing idea exchange and discussion about them. In addition, studies show how creative individuals may only produce more creative outputs than less creative individuals when the context is supporting and encouraging towards creativity \citep{oldham1996employee}. 

According to \citet{mumford1988creativity} through environmental variables employee's creativity can be fostered. New solutions may be achieved through problem solving and challenging the routine ways of thinking, and environment should be designed to encourage and facilitate these skills. Environmental factors may, furthermore, affect on employee's intrinsic motivation and willingness to generate novel ideas, when social and physical environments work as a source for support and resources in idea generation and implementation. 

In order an organisation to be more innovative, according to \citet{thomke2001enlightened}, team engagement is essential, as the whole team need to understand the meaning of experimenting and developing and it should be encouraged to sharing information and ideas in as early stage of development process as possible and throughout the process. \citet{thomke2001enlightened} suggests using small teams and parallel experiments especially when the time is the most critical factor. 

\subsection{Concrete learning processes and practices}
 
According to \citet{garvin2008yours} second building block of organisational learning, consists of concrete learning processes and practices. It includes experimentation, information collection, analysis, education and training and information transfer. Organizational learning can be supported through concrete steps and activities which are tested and further developed through experimentations. Furthermore, information and intelligence about customers as well as technological trends should be collected systematically and further analysed focusing on identifying problems and solving them. Training and education of new and established employees is an essential part of practices and processes. Finally, through transparent and meaningful knowledge sharing organisational learning can be enhanced, focus being on clear, well-defined and working communication systems that employees can easily relate to. Concrete processes together with efficient knowledge sharing methods ensure essential information being available fast and efficiently for employees who to use. \citep{garvin2008yours}

Supportive leadership behaviour alone is not sufficient guarantee for organisational learning. \citet{garvin2008yours}emphasise how organisations are not monolithic and managers should sense differences in culture, department and units. In addition to cultural differences, learning requires clear and targeted processes and practices. Furthermore, learning should be considered as multidimensional, thus organisational forces should not be solely focused on a single area but to consider presented building blocks as a whole.

Also \citep{amabile2004leader} emphasise in their componential theory on creativity the support of immediate supervisors as a way to enhance employee's creativity and intrinsic motivation. Supporting actions include being a role model, defining and setting appropriate goals, showing the work group support and confidence within the organisation, showing appreciation of individuals contributions to the project, focusing on efficient and good communication, offering valuable feedback, and listening openly novel ideas. Accordingly \citep{amabile2004leader} divide required behaviours of leaders for providing support into two categories: instrumental or task-oriented and socio-emotional or relationship-oriented actions.

Organizational and team structures and hierarchies affect on innovation and experimenting. Flat organisations and small project teams are foster innovation performance in a company. Smaller team handles communication and commitment better, while as few management layers as possible decreases the jeopardy of rejection. "Since it takes a chain of yesses and only one no to kill a project, jeopardy multiplies as management layers increase."\citep{quinn1985managing}

Organizational structures can influence in many ways creativity of a team and individual. For instance, by promoting open communication, idea and ongoing information exchange with internal and external team members as well as encouraging information seeking from different perspectives and sources is likely to enhance creativity (e.g. \citep{ancona1992demography,dougherty1996sustained}). 

According to \citet{amabile2002creativity} clear time should be allocated for developing especially when the aim is to flourish idea generation, creativity, learning and experimentation of new concepts. Time pressure should be minimal in order bright ideas to glow as cognitive processing requires time of an individual and team. Yet, no sense of urgency leads employees easily to auto-pilot mode, in which routine tasks are performed without further thinking and analysing. Thus, creative time for playing with ideas, brainstorming, learning and experimenting should be allocated in an organisation in order truly new things to develop. Shared goals are once more essential in engaging team members to play with ideas and feel more motivated in developing their work. \citep{amabile2002creativity}

In addition, sufficient time and resources should be allowed for exploration\citet{amabile2008creativity,katz1985project}. 

Creativity, exchanging ideas and turning them into action requires intrinsic motivation from employees \citep{jung2001transformational}. Thus, in order to increase creativity and innovation at workplace, leaders should foster organisational culture in which individuals find their motivation in divergent thinking and trying out new ways of performing tasks \citep{amabile1998kill}. 

Leaders need to acquire resources and encourage idea generation \citep{mcgourty1996managing}, and overall create environment where idea generation is possible \citep{andrews1970social} as well as evaluate the ideas and integrate them to organisational needs \citep{mumford2002leading}. While individual characteristics affect on the creativity of an individual, creating an environment fostering creativity is likely to assist in producing novel ideas during the routine work of employees \citep{amabile1996assessing}. 

Collective organisational achievements can, in turn, be affected through affecting working environment and organisational culture and leaders influence on employee's attitudes and motivation towards work \citep{amabile1998kill}.

Prior studies show how creative efforts of employees require sufficient amount of time and energy\citep{gardner1988creativity,getzels1975problem}. Also \citet{redmond1993putting} state that leaders should allow enough time for problem solving and creative actions. 

Also \citet{shalley2004leaders} emphasises the meaning of prior knowledge and experience of an employee of area of work before demanding or anticipating creative actions from them. Naturally, job rotation and employees from different areas works as a great source for new perspectives and development, yet creativity requires sufficient level of familiarity of target area. \citep{shalley2004leaders}

\citet{shalley2004leaders} also state how appropriate level of autonomy given to employees is useful. However, too much autonomy, meaning full control over planning and conducting the work, may lead to negative consequences and contradictory goals between employee and organisation. Thus, setting appropriate goals and understandable requirements that inspire employees is essential. Furthermore, leaders need to realise whether the goals require creativity or lead to creative outcomes, and not anticipate creativity or creative outcomes and instead accept employees being less creative where it is not needed. \citep{shalley2004leaders} 

Additionally, trying out novel approaches and conducting experiments requires more energy and is overall more difficult for employees than performing and sticking to the routine tasks. As it takes more cognitive resources to generate several alternative solutions, practice divergent thinking and approach problems from different perspectives, allowing time for creative work is essential. However, engaging employees to creative activities is likely to lead better and more qualified decisions. \citep{shalley2004leaders}
Also \citet{amabile1987creativity} state how sufficient time should be allowed for creative thinking, playing with ideas and exploring multiple perspectives. \citet{katz1985project}, in turn, found in their study how uninterrupted time was considered critical for engineers working on novel technologies. 

Studies have also shown how employees working on high time pressure affects negatively on ability to engage in creative cognitive processing \citep{amabile2002creativity}.

Together with time, in order to be creative sufficient access to material resources should be allowed for employees \citep{katz1985project}. However, even though material resources are essential for creativity, studies have suggested a contradictory perspective: when employees have access to wide range of material resources, their creativity tendencies may decrease. This may happen due to the creative actions and thoughts an employee needs to perform when needing certain resources to finish his task but not having them at hand. This, in a way, stretches employees' skills to think differently and achieve goals. \citep{csikszentmihalyi199916} Thus, \citet{csikszentmihalyi199916} states how resources are likely to make employees feel too comfortable and lead to decrease in creativity. 

Giving and receiving feedback remain simultaneously a key function of leaders and one of the most challenging tasks they have. According to \citet{shalley2004leaders} giving performance feedback is essential for creativity and accordingly difficult as creativity often involves approaching problems from new approaches and trying out novel things as well as taking risks. 

Structures have their influences on creativity of employees. Relationship between formal reporting and responsibility levels, referring to bureaucracy levels are essential: highly bureaucratic organisation do not tend to encourage employees to reach for novel approaches and experiments, whereas organisation with flatter structure may enhance organisations autonomy and creativity \citep{shalley2004leaders}

Changing overall organisational climate is challenging, yet various components are reasonably manageable and should foster creativity. For instance, risk-taking, constructive feedback can be supported through role modelling of management. Employees may be likely to perceive presentations of organisations structure and hierarchy as discouraging and highlighting how employees are not allowed or encouraged to make decisions on their own leading to less enthusiasm towards trying out new ways of working and developing. In addition, heavy bureaucracy demanding lot of time and effort from employees to get novel ideas forward in the organisation is likely to destroy the enthusiasm and willingness of employees towards developing and new approaches. \citep{shalley2004leaders}

All human resource practices should be in line and systematically linked together in order to create a clear picture for employees of what is expected of them. Perceived fairness and sense of loyalty towards a company of an employee is higher when an employees understands what is expected of them, how, when and what for they are being rewarded, promoted or fired. These same attitudes are essential for fostering creativity. Committed and loyal employees are more likely to exceed what is required of them, be more motivated and committed to working towards specific goal and find novel approaches in order to succeed. \citep{shalley2004leaders}

Essentially, interaction between leaders and employees, team members and outside members, sufficient resources, employees clear expectations on their evaluation and rewards and environment which is perceived fair all have affect on employees behaviour at work. Through these variables, employees may feel they are working in supportive working environment, and foster creativity and ability find novel approaches and try out new things. \citep{shalley2004leaders}

Several types of teams function in organisations, type depending on various dimensions such as cross-functional versus single-function, time-limited versus enduring and manager-led versus self-led. These dimensions should be recognized and team learning fostered depending on the type. \citep{edmondson1999psychological}

Amabilen (1997) tekstiss\"a pdf:n sivulla 17 on hyv\"a kiteytys Amabilen osalta, mit\"a innovointi vaatii organisaatiolta. 
Expertise 
- Expert performance and its affects on implementing ideas \citep{ericsson1994expert} (tsekkaa artsu viel�)

\citet{thomke2001enlightened} suggests four steps for organizations to be more innovative. First of all, organization should allow and manage the work for the employees so that fast experimentation is possible. This usually requires challenging routine ways of working and shaping the routines, yet fast experimenting is essential in order to get rapid feedback for shaping the ideas. 

\subsection{Leadership behaviour}

\citet{quinn1985managing} emphasis top executives role over particular management. Innovation is likely to occur when top executives encourage creative and innovative endeavours and create an atmosphere and value system that supports innovation. \citet{quinn1985managing} offers an explanation why it seems easier for engineer and scientific leaders to create atmosphere supporting innovation: understanding and psychological comfort are related to familiarity, and engineers, for instance, have wider understanding and knowledge of technology, which makes newer technological innovations easier to accept and adapt. 
 
Leadership behaviour should reinforce learning. Behavior of leaders is highly related to the performance of employees \citep{kim2014blue} and organisational learning \citep{garvin2008yours}. In order to encourage employees learning, leaders should prompt dialogue and debate, ask questions and listen to employees \citep{kim2014blue,garvin2008yours}. Leaders supporting new ideas and idea exchange has been related to enhancing creativity especially among those employees who showed disposition towards creativity \citep{oldham1996employee}.  

These three building blocks overlap to some degree and reinforce one another. For instance, leadership behaviour helps in creating supportive learning environment, which supports managers and employees in creating and defining concrete learning processes and practices. Furthermore, concrete processes support leaders behaviour in a way that fosters learning and through own example cultivates that behaviour to others. \citep{garvin2008yours}

General leadership styles and practices are set for industrial management, yet at present the focus should be on leading the people, and the focus of leadership needs to change from authoritarian style to increased autonomy and trust. As \citet{mumford2002leading} state, "organisations may now need jazz group leaders rather than orchestra directors". 

Likewise, according to \citet{mumford2002leading} creative leadership highlights three key elements: encouraging employee's idea generation, creating safe environment for ideas to emerge and improving idea promotion and implementation. By idea stimulation, education of various problem solving techniques, support for novel ideas, involving employees in developing ideas and allowing them freely pursue ideas, idea generation can be enhanced by the leader. Essential elements of safe environment include diverse teams, transparent and good communication, leader acting as a role model and being in charge of conflict management. In addition to idea generation, idea structuring phase consists of creating action or project frameworks so that employee's have as much autonomy to perform the task as needed. Idea promotion, in turn, refers to leaders task to transfer ideas to broader levels of an organisation, achieve support and assist with implementation of chosen ideas. Promotional activities to upper levels of organisation serve as a major way to insure sufficient resources and support for the idea implementation. \citep{mumford2002leading} 

According to studies leaders have a strong direct impact on employee's behaviour and way of performing at workplace \citep{katz1978social,redmond1993putting}. As leaders play a major role in establishing, influencing and shaping organisational culture and climate through their communicated values and beliefs, they are able to shape the organisational culture into more innovative direction and foster creativity in an organisation \citep{jung2003role,schein2010organizational} for instance by nurturing organisational climate that supports creative efforts and learning \citep{yukl2002leadership}. Change from authority-based leadership to collaboration with employees has occurred in literature and in practice (\citep{amabile2008creativity,farson2002failuretolerantleader}). 

Leaders have a great influence on employees behaviour. \citet{avolio1988transformational} list several mechanisms through which leaders can affect employees behaviour. These include role modelling, goal definition, reward allocation, resource distribution, defining norms and values of the company, showing the way to interact as a group, condition employees' perceptions of work environment and being the lead decision maker on organisational structure and procedures. Studies also suggest leaders have a significant effect on employees' creativity \citep{hennessey19881}, and according to \citet{redmond1993putting} a leader can have an affect on employee's level of creativity through leadership behaviours such as problem construction, learning goals and feelings of self-efficacy. 

Actually, leadership behaviour is only recently recognized as essential part of enhancing creativity and innovation skills of employees \citep{mumford2002leading}. This may be due to our romantic perception of creative act, which defines creativity as an heroic act of an individual and leaders only being a hindrance to the creativity of an individual. Furthermore, conventional models of leadership are not likely to encourage employees to challenge the status quo but to achieve required goals.\citep{mumford2002leading} Current trend in research however shows leaders and their behaviour have great influence on the creativity and innovation ability of employees (eg. \citep{mumford2002leading,jung2001transformational,amabile1998kill})

Conventional leadership behaviour focuses on internal activities within the team, whereas innovative team leader needs various set of skills and approaches in order to encourage developing and growing of teams and individuals. For instance, according to \citet{barczak1989leadership} leaders of innovative teams utilise wide range of familiar and unfamiliar techniques in order to accomplish the team objectives, whereas leaders of operating teams use only a few familiar techniques. Even though in this study innovation teams were not studied, similar elements of developing by experimenting and encouragement for that may be recognised, when dealing with new tasks and developing something which result is uncertain. 

Leadership style has great impact on organisational innovation and creativity. Transformational leadership refers to leadership style and processes which emphasises longer-term and vision-based motivational processes \citep{bass1997full}. Furthermore, through offering an explanation of the importance and value of the work, leaders encourage their employees' to think beyond self-interest \citep{yukl2002leadership}. Leaders shape and define the goals and working context \citep{amabile1998kill, redmond1993putting}. Through a long-term vision (separated from short-term business outcomes, which usually focuses on quarterly profit), leader's are able to direct employee's efforts towards creativity and innovative work processes leading to likeminded outcomes\citep{amabile1996assessing}.

Leaders can affect employees' creativity and innovation skills both directly and indirectly \citep{jung2003role}. By stimulating employee's intrinsic motivation and higher level needs leaders are able to affect directly on employees' creativity \citep{tierney1999examination}, where indirect way may be through establishing a work environment where new ways of doing are encouraged and failure is not being punished \citep{amabile1996assessing}. Creating and supporting a reward-system that values creative performance, provides both intrinsic and extrinsic rewards for employee's efforts to learn new skills and to challenge status quo by experimenting new approaches, employees are constantly willing to engage in creative endeavours \citep{jung2001transformational,mumford1988creativity}.

\citep{mumford2002leading} argue organisational climate and culture being a collective social construction where the role of the leader on control and influence is remarkable. \citep{schein2010organizational} also presents a view where leaders communicated personal values and beliefs become essentially part of organisation's culture and climate. Furthermore,\citep{jung2001transformational} considers managers essential for shaping organisational culture, whether the concern is in developing, transforming or institutionalising. The way employees perceive their work environment created by their leaders, and especially the way they perceive the instrumental and socioemotional support both have influence on employees' creativity. \citep{oldham1996employee}

Also Buijs (2007) \citet{buijs2007innovation} states how leaders dealing with uncertain and new innovations should stay certain about uncertainties and provide a safe environment and encourage employees to work on current task comfortably. Thus, high level of tolerance for dealing with different states of minds and various personal feelings is required from a leader. \citep{buijs2007innovation}

In order to encourage creativity and experimenting in teams, leaders should lead by example and act as role models. Leaders should consider their own behaviour and actions in a way that stimulates employees to new and innovative, creative approaches to problems. In addition, they can even request creative and innovative solutions form the team, which may lead to better results in creativity of individuals \citep{amabile2002creativity}. \citep{mumford2002leading,amabile2008creativity,waldman1990adding}

Big ideas do not hatch overnight and creative thinking requires time. Leaders should allow team members time to think creatively, as according to studies under pressure creativity actually falls into decline. Even though individuals may feel more creative, actually they are only working more and getting things done. According to this study, employees were clearly less creative while time pressure increased. \citep{amabile2002creativity} 

Furthremore, leaders can assist their employees by recognising times with high pressure, and allowing employees to focus on certain thing at a time, leaving the expectations of creativity and new ideas into the future moment, when time pressure has decreased. On the other hand, if creativity is required under stress, leader should transparently explain the importance and reasons behind the strict schedule and required goals. Thus an employee may relate to the problem at hand and engage better at his work. Indeed, helping people to understand the importance of work is essential especially under high time pressure. \citep{amabile2002creativity} 

Failure as a part of innovation and development process begins to be generally recognised and approved. Succeeding companies even thrive for failure in order to learn fast and find the best practices and business models. Through encouraging employees in risk-taking and making mistakes, leaders are likely to boost innovation. For instance, credit company Capital One conducts continually large amount of market experiments. They now most of the tests will not pay off, yet they also know how much can be learned about customers and markets from failed tests in early phase of development. Yet leaders fail in showing their employees the support and tools for failing fast and early enough. Failing in a personal matter remains a difficult subject, as failing never feels exceptionally great, and often employees still consider failed work as failing personally. \citep{farson2002failuretolerantleader}

Failure-tolerant leaders put effort on explaining to employees how important part failure is to the development process as a whole, and how failing actually refers to a point where surprising, failed outcomes are not reflected and further analysed in order to learn. Performing accordingly, admitting own failures and not chasing anyone to blame, failure-tolerant leaders encourage failure, lower the threshold and ease the fear of failing of employees. \citep{farson2002failuretolerantleader}

Naturally management need to take seriously issues about safety and health, yet most of the failures should be seen as opportunities for growth. Furthermore, failure-tolerant leaders treat success and failure similarly, analysing and reflecting the outcomes in order to grow the intellectual capital of the team, including experience, knowledge and creativity. Other characteristics of failure-tolerant leaders are being rather collaborative than controlling, listening carefully, seeing the bigger picture, asking questions and focusing on the development and future rather than blaming on mistakes. In addition, in order to gain empathy and trust among employees, leader should admit their own mistakes, as it shows self-confidence and honesty, assisting in forming closer ties with employees. Vulnerability and transparency play a major role in trustworthy relationship between leader and employees.  \citep{farson2002failuretolerantleader}

Through the green light given and their own example leaders can change the focus from success and failure into thinking in terms of learning and experience. \citep{farson2002failuretolerantleader}

\citet{amabile2008creativity} draw a poetical picture how leader cannot manage creativity, but manages for creativity. Furthermore, they suggest that culture that fosters creativity includes leadership that enables collaboration, enhances diversity, encourages ideation, maps the stages of creativity to different needs, accepts inability and utility of failure and motivates employees with intellectual challenges. According to \citet{sosik1999leadership} leaders should concentrate on vision of work and its outcomes that is meaningful and motivational enough to inspire employees. 

In experimentation process, employees need to contribute imagination, and this may require new kind of encouragement for creativity from the leaders. Much success rises from employee's own initiatives, which results from wide amount of autonomy at work. \citep{amabile2008creativity}

A culture of creativity can be fostered in an organisation through opening the organisation to diverse perspectives and openness to various ideas. This calls for safe environment for employees to share their thinking from different fields of expertise. Furthermore, encouraging passion and knowledge of an employee is likely to result in more creative action at work. \citep{amabile2008creativity}

Empowering employees is an essential tasks of leaders, through which a work environment is created where employees desire to seek innovative approaches to perform their work tasks \citep{jung2003role}. Transformational leaders encourage employees to participate in developing by highlighting the importance of cooperation, providing the opportunity to learn from shared experience and allowing employees to perform necessary actions in order to be more effective\citep{bass1990implications}. Furthermore, autonomy and freedom to perform essential tasks has major effects on organisational creativity, as individuals are more likely to produce creative work when having the feeling of personal control over how to approach given tasks \citep{amabile1996assessing}.

Yet, in order to maintain organisational innovation and risk-taking, autonomy given to an employee can not be in contradiction with fear of failure or discouragement towards challenging status quo or trying out novel solutions \citep{yukl2002leadership}. Thus, organisational climate has to support and encourage innovation \citep{mumford1988creativity} by valuing initiative and innovative approaches that support employees in risk-taking, accepting challenging assignments and stimulate intrinsic motivation towards work \citep{jung2003role}.

\citep{jung2001transformational} has studied how leadership style affects group's creativity and performance by comparing transactional and transformational leadership styles. Transformational leader refers to a leader who encourages divergent thinking and looking at problems from unconventional perspectives, while providing and explaining clearly defined goals and facilitating the innovation process of employees \citep{bass1990implications}. Furthermore, development of clear long-term vision and practises supporting the way to achieve it is essential characteristic of transformational leaders \citep{avolio1988transformational}. The relationship between transformational leader and an employee is active and emotionally attached \citep{avolio1988transformational} and through the strong attachment resulting from tight relationship leaders can better support employees in using their personal values and self-concepts in the way that employees can pursue higher level performance and fulfil personal needs through the work. This focus of transformational leadership on value alignment is likely to lead to the root of intrinsic motivation of an employee \citep{gardner1998charismatic}, which is considered as one of the key elements in creative thinking and innovation skills of an employee (eg. \citep{jung2001transformational,amabile1998kill,deciintrinsic}).

According to \citet{bass1997full} transformational leadership consists of four unique yet interrelated behavioural components: inspirational motivation (articulating long-term vision), intellectual stimulation (promoting creativity and innovation), idealised influenced (meaning charismatic role modelling) and individualized consideration referring to coaching and mentoring leadership style. 

Transformational leaders can build environments that support creative actions (\citep{sosik1998transformational,avolio1988transformational}). According to \citet{sosik1998transformational} key characteristic of transformational leader is the intellectual stimulation, which is likely to encourage creativity and divergent thinking leading to unconventional solutions to problems at hand. 

In contrast to transformational leadership, transactional leadership refers to focus on employees ability to fulfil and achieve clearly defined goals \citep{hollander1978leadership,house1971path} and successful goal achievement is rewarded \citep{waldman1990adding}. This exchange relationship between leader and employee is based on a contract of specified goals and emphasises on the process of achievement of objectives (Avolio and Bass 1988) but does not encourage employee's to develop their creativity and innovation skills \citep{jung2001transformational}. Instead, employees are rather motivated extrinsically to perform their job under transactional leader but not expected to question and change the status quo in creative ways \citep{amabile1998kill}.  

Few studies have been made linking the transformational leadership and positive outcomes of employees' creativity in organisational level and outcomes \citep{jung2003role}, even though several studies have been made revealing the positive relation between these factors. in their study \citet{jung2003role} draw this link clearer and suggest that while leaders define the context and goals of their employes, transformational leadership can be extrapolated to an organisational level.  

According to \citet{jung2003role} transformational leadership is positively related to organisational innovation, employee's perception of empowerment and support for innovation. Furthermore, the perception of empowerment and is positively related to organisational innovation, and when perception being strong,  the relationship between transformational leadership and organisational innovation tends to be stronger. Results of the study conducted on 32 Taiwanese companies suggest that through transformational leadership by top managers organisational innovation can be affected directly or indirectly, latter referring to creating an organisational culture in innovation, discussion, novel approaches and experimenting is encouraged. \citep{jung2003role}

As undertaking novel approaches to work oftentimes involves risk-concerned decision-making, employees should be offered decent level of guidance, goals and some measure of structure \citep{jung2003role}. Leader not taking an active role in supporting and guiding the work of his employees may lead to organisational units working at cross-purpose. Thus, leadership is about maintaining a balance between empowering employees and providing guidance and structure through setting goals and agenda. However, according to \citet{mumford2002leading} leaders' planning and guidance should focus on progress, projects on general level and implementation of the results of projects instead of focusing on offering detailed guidance on piece of work. 

Study of \citet{sethi2001cross} showed how good interaction in a team and high level of commitment to the success of the team lead to more radical innovation abilities. In the study team members were highly encouraged to take risks, which lead to more motivated members in suggesting novel ideas from their perspectives. In addition, team members identified themselves strongly as part of the team, which again higher commitment level. \citep{sethi2001cross} 

However, lack of time and resources may serve as a hindrance to employee's willingness to take risks and perform experiments \citep{jung2003role}. Through leaders who allow their employees to participate in developing and ideating, reserve budget for it and set it as a part of performance standard, the hindrance for risk-taking may be lowered \citep{jung2003role}.

Under some circumstances, according to Monge et al. (1992) \citet{monge1992communication} group communication is likely to increase innovation. Thus, leaders should consider managing wide range of formal and informal meetings and facilitated discussions in order to create opportunities for ideation. Furthermore, innovation occurs over time and is a dynamic process. Leaders should be sensitive in which pace more managerial impact is needed, and in which pace of the process more freedom and autonomy should be allowed for employees. \citep{monge1992communication} 

Organizational leaders play a great role in establishing strong team performance culture. This can be achieved through addressing and demanding performance that meets the need of customers, employees and shareholders. Teams should not be fostered by the sake of the team only, rather should leaders clearly state how the team performance affects to customers and through that foster clearer performance ethics and cultures. In addition, even though people tend to have great sense of individualism, it does not have to bias the teamwork performance, as real teams find ways to support individual strengths and performance for shared goal. Furthermore, in order to team function properly and efficiently, discipline across the team and organisation is needed, focusing again on performance.  \citep{katzenbach1993wisdom}
 
Ambiguity is often perceived by individuals when lacking sufficient cues to structure a situation, and usually arises from novelty, complexity or unsolvability of situation at hand \citep{budner1962intolerance}. 
 
Leadership plays a major role in defining group goals, controlling resources and providing rewards through interactive leadership process, making leadership behaviour an essential environmental variable in stimulating creative behaviour as a means for achieving goals \citep{redmond1993putting}. \citet{katz1978social} even refer to role of the leader in a sense where leader defines by his example the reality of workplace; norms, practices and culture. According to \citet{barczak1989leadership} leader's task is also to provide clear focus for the work of employees. 

By defining organisational culture, climate and group norms leaders shape the way of working of employees. Through such role-modelling and mentoring process leaders also show employees in practise how tasks are performed. Employees, in turn, follow the example of leader in order to achieve high level of performance. \citep{redmond1993putting}
 
Role-modeling stands also as powerful tool for opening employee's eyes and attitudes to new perspectives, thinking 'out of the box' and adopting generative and exploratory thinking processes \citep{jung2003role,sternberg1997creativity}, influencing creativity of an employee \citep{shalley2004leaders}

Although different leadership styles and their effect on employee's creativity behaviour has not yet been studied widely, some studies show, how transformational leadership behaviour encourages employees look problems from different perspectives and thus widen their intellectual and creativity skills \citep{jung2001transformational,sosik1998transformational}. \citet{jung2001transformational} has studied the relation between leadership style and group creativity finding that transformational leadership is most likely to stimulate creative effort of employees. 
 
In his study, \citet{jung2001transformational} emphasises that transformational leadership skills can be practised in order to foster creativity and intellectual skills of employees and shape organisational culture. His study showed how transformational leadership; encouraging divergent thinking and solving problems at hand from unconventional perspectives, is likely to increase intrinsic motivation of employees leading to more creative problem solving and behaviour.  Through brainstorming activities that focus on non-traditional thinking and fantasising intellectual skills of employees can be enhanced \citep{sosik1998transformational}. Furthermore, \citet{jung2003role} argue that several aspects of leadership behaviour can be learned and practiced. Thus, organisations should foster and improve innovativeness by offering managers training and mentoring processes that develop transformational leadership. 
 
\citep{sosik1998transformational} furthermore suggested that anonymous ideating through nominal groups leads to better results and greater amount of ideas than brainstorming activities in real groups. When ideating in daily working groups, members may fear failing, being ashamed or measured by their performance. Overall, oftentimes it is way more difficult to take a different role and actions in group with familiar members and routines. \citep{jung2001transformational}

Innovation leaders, indeed, in managing innovation processes need to have several contradictory skills, roles and attitudes used smoothly during the day with employees; and they need to tolerate these competing and conflicting aspects within the team. Innovation team trusts their leader to be in charge and in control, yet allow them support, autonomy and enthusiasm.  At the same time, innovation leader should be few steps ahead thinking of uncertain future scenarios and support his innovation team in current step without showing doubts too strongly about team's work. \citep{buijs2007innovation} Resulting from this contradictory and challenging role of an innovation leader, according to \citet{buijs2007innovation} they should act and have characteristic of a controlled schizophrenic. 

As \citet{buijs2007innovation} argues, leaders who are to lead employees and work handling innovations need to understand the paradox and natural conflicts between routine processes (exploitation) in order to earn money in the present and the innovation processes (exploration) in order to earn money in the future. \citet{buijs2007innovation} four aspects for innovation which leader should be able to master all providing a secure environment for a team to perform in novel and creative ways. These consist of innovation process, psychological process of innovation team, creativity process, and leading and playing. 

Goals, however, should be kept broad, in order not to create undue oppositions to new ideas. Flexibility should be maintained by not defining intermediate steps in detail and by trying alternate options and routes. Identifying and solving problems at early phase fosters momentum, confidence and identity towards new approach. Furthermore, sufficient amount of information about the project and progress should be offered in order managers to follow and realise the work performed.  \citep{quinn1985managing}

Local leaders are in essential role in directing and evaluating work of employees, facilitating and allowing resources and information as well as encouraging employees to engage with the tasks and team members. \citep{amabile2004leader}

The approach of leaders oftentimes divides into task-oriented or relationship-oriented. Task-oriented leaders value performing the job, focusing on clarifying roles and responsibilities, monitoring work while managing time and resources. In turn, relationship-oriented leaders value socioemotional aspects of work through empathetic actions, showing consideration for employees, being friendly and supporting the team personally. Should be noted that in literature concerning leader behaviour, term support refers to relationship-oriented leadership behaviour, wheres in creativity literature same term refers to both task- and relationship-oriented behaviours and actions - all that are to foster creativity. In this thesis, latter and broader usage of term support is used. \citep{amabile2004leader}

According to the study of \citet{amabile2004leader}, leaders are likely to influence employees feelings, perceptions and performance as well as overall creativity do their own behaviour. By acting fairly, consulting with employees on essential decisions, offering emotional support and rewarding and recognising them for performing well leaders can enhance creativity of employees. In turn, leaders may play as hindrance for creativity by not offering support and clear task assignments, preventing autonomy of employees, treating employees unfairly and not trying to resolve important problems. 

According to componential theory of organisational creativity \citep{hennessey19881,amabile1996assessing}, employee's perception of the work environment influences individual and team creativity and emphasises the role of local leader support for creating an creativity supporting environment. 

In their study, \citet{shalley2004leaders} present how leaders should use human resource practices in order to develop work context which improves the creativity skills of employees. 

Organizational structures affect in the traditional roles of leadership as a means of direct responsibility given to employees. The trend of flatter organisations provides more autonomy to employees, whereas leaders' role transforms to more involved in external resource acquisition and managing the interfaces. \citep{shalley2004leaders}

Creative work environment is likely to be created through leaders who support and encourage employees, provide them autonomy in decision-making and everyday tasks, and communicate openly with employees \citep{oldham1996employee,tierney1999examination}. However, in addition to contextual factors and environment, studies show level of support, control and assist an employee needs depends on personal characteristics. Thus, leaders knowing and understanding their employees is essential in order to provide employees individual support needed. \citep{shalley2004leaders}

 As \citet{garvin2008yours} bring out in their article, reasonable question to ask for this fresh leadership approach is, can managers actually be excited about being a facilitator of creative process, and where to find those managers who feel engaged and aspired to that role and want to do it? \citet{lingo2010nexus} has offered one perspective to this question in her study with production of music. She claims that producer is the one bringing it all together; it is actually hard leadership exercise, where people from different fields and teams need to work together for one production, where there are no clear rules for who is controlling the output nor yardstick how good or bad the production is. Through creating a shared purpose and common goal in production team, and while still letting "other apply their distinctive expertise", a producer actually operates at the centre of the storm without being at the focus of attention as well as aims for productivity without being over controlling. According to this example, glory comes from being able to help others to find and realise their unique talents at the same time with achieving a collective goal. 
 
Team members observe and reflect other members responses and actions and attend to them, yet behaviour of the leader is often their particular concern \citep{tyler1992relational}. 

For instance, \citet{edmondson1999psychological} states team leader coaching influencing positively on team psychological safety. Psychologically safe environment includes team leaders being supportive, coaching-oriented, who doesn't response to questions or challenges in defensive manner. Employees are not likely to take interpersonal risks that might lead to learning if leader tends to act in authoritarian of punitive ways. 

\section{Attitude towards failure and risk}

\citet{garvin2008yours} divide organisational failure into three categories: unsuccessful trials, system break-downs and process deviations. In order to learn and develop, these all types of failures need to be recognized and their special characteristics analysed. For instance, unsuccessful trials may foster creative learning, but as important is to overcome failures resulting from deeply ingrained norms that inhibit experimenting. \citep{garvin2008yours}

Individuals are likely to avoid risks and uncertainty, stick to routine and prefer more certain outcomes and ways of performing \citep{bazerman2012judgment}. This does not encourage, however, creative actions that actually require several trial-and-error, iterative processes where risk is involved. Employees fearing risk-taking tend to perform the routine way instead of taking chance with new approaches. \citep{shalley2004leaders} However, in his study \citet{nystrom1990organizational} found that organisational culture reflecting challenge and risk taking lead to more innovative actions of employees and the whole organisation. 

In line with this is the psychological safe environment where new ideas and breaking with the status quo are supported \citep{edmondson1999psychological} and uncertainty is not totally avoided, preferably managed and tolerated \citep{shalley2004leaders}. In order to decrease the fear of failure, \citet{amabile2008creativity} suggest leaders should put emphasis on creating an environment where an employee feels safe to fail and speaking out loud ideas nor making mistakes does not result in punishment or humiliation. Leaders should, instead, motivate and encourage employees to ideate, break routines and learn by stating how essential experimenting, iterating and failing is for learning and developing \citep{amabile2008creativity,shalley2004leaders}. 

When individuals ask for help, admit errors or seek feedback they place themselves under risk, and perceive a threat, fear of being judged and appearing incompetent as well as fear of giving unfavourable impressions on people who have the power to give promotions, raises or who assigns projects\citep{edmondson1999psychological,brown1990politeness}. Even though knowing a team would benefit of this kind of behaviour, perceived threat appears strong. However, this is essential when initiating learning behaviour, and \citep{edmondson1999psychological}

However, innovation processes always include mistakes and failure, from which organisations, teams and leaders need to learn, preferably rather fast. Organization that learns fastest is likely to take the lead, which is the essence of innovation according to \citet{buijs2007innovation}

Feeling of threat and embarrassment is linked to reduce cognitive and behavioural flexibility and responsiveness \citep{staw1989tradeoff} and this leads easily individuals acting ways that rather inhibit than fosters learning \citep{argyris1982reasoning}. 

However, in some organisational environments people do ask help, admit errors and discuss about problems. In these environments employees seem to perceive interpersonal threat low enough to perform despite of the threat. \citet{edmondson1999psychological} has studied working environments and realised in environments employees act despite the threat, they felt safe and supported for their actions. She refers to this as psychological safety. Some studies argue how familiarity among group members is likely to encourage openness towards new information and ideas \citep{sanna1990valence}, yet this factor alone is not sufficient to explain when group members find it safe to act instead of feeling threatened \citep{edmondson1999psychological}.

In creative work risk concerns both the need to do experiments and tolerate failure \citep{andriopoulos2000enhancing,quinn1985managing}, as failing is widely considered as essential part of learning \citep{farson2002failuretolerantleader}. Thus, employees should feel being allowed to conduct experiments and despite the outcome of the experimentation \citep{jung2003role}.

Feeling of self-efficacy may affect individual's willingness to provide unique and novel ideas even when some degree of risk is involved \citep{mumford1988creativity}. Training, coaching, giving feedback and assigning tasks seem to be useful approaches for leaders, who pursue to contribute empoloyee's self-efficacy \citep{amabile1998kill}.

Fear of failure can be decreased through transformational leaders who foster the culture of intrinsic motivation and rewards from creative endeavours, idea exchange and discussion \citep{amabile1998kill}.

According to \citet{amabile1996assessing} creative solutions in an organisation can be achieved by encouraging employees to reach and experiment new perspectives and ways of performing. Essential for this is not being punished for negative outcomes. Organizational environment that allows failing is likely to assist in employees acquiring diverse perspectives and questioning the status quo and habitual way of performing. 

Uncertainty, actually, should not be considered only as a threat or inconvenience occurring in organisations, rather should appropriate level of messiness let exist, and develop opportunities where uncertainty can be exploited. Overall, uncertainty in creative processes should not be overly controlled. \citep{sternberg1997creativity}

Work needs to meet the skills and interests of employees while offering sufficient level of challenges in order to increase the motivation of employees towards work \citep{amabile1998kill}.

Furthermore, employees are more likely to contribute ideas when feeling that failing is safe. Leaders should emphasise that constant experimenting requires failing early and often and through these iterations learning is possible. Equally important is that employees should feel not being punished nor humiliated if mistakes occur or whatever ideas are spoken up. \citep{amabile2008creativity} Also \citet{de2001minority} have found a positive relation between employee's creativity and participative safety, referring to employee's perception of generating ideas without being judged. 

According to various studies, failing and negative consequences are natural part of creative, innovation and learning processes \citep{hennessey19881,shalley2004leaders,andriopoulos2000enhancing} . For instance, \citet{hennessey19881} emphasise that process is likely to have negative consequences, and in the concept of perpetual challenging of \citet{andriopoulos2000enhancing}, adventuring phase includes making mistakes. Thus, when developing novel products and processes, iteration and failure is included, employees need to feel safe to try various approaches and fail \citep{shalley2004leaders}. As discussed earlier, organisational culture has great influence on employee's perception of safe environment for failing.

Predicting the future being impossible, focus should be in managing risks involved in playing with creative ideas in both the company and individual level. As \citet{sternberg1997creativity} state, "as uncomfortable as it is, while not being able to predict and control uncertainty in creative projects, the messiness does have to let exist". \citet{kanter1983change}continues that, actually, opportunities grow from uncertainty and creative endeavours rise when struggling with uncertainty and mess, as individuals impose order where it does not exist, and thus individuals are forced to form new connections. Furthermore, allowing employees freedom to act actually arouses desire to act.

According to \citet{amabile2008creativity} essential part of creating a safe environment for creativity is managers to decrease the fear of failure. Instead, constant experimenting should be the goal of working, learning by doing and iterating until sufficiently is learnt from the process. 

Furthermore, when company grows, it usually leads to more conservative actions and increase in fear of failure. When fearing failure managers tend to deny failure and erase it from the memory instead of learning from it. \citep{amabile2008creativity} 

Also \citet{edmondson1999psychological} relates experimenting tightly to failing, and emphasises team learning, creative problem solving, reflection and overall organisational performance rising from the failure that occurs. However, willingness to interpersonal risk-taking is tightly related to how employees perceive and believe team members or leaders would react and response in uncertain actions or ideas \citet{edmondson1999psychological}. Thus, team's tolerance for imperfection and error should be increased. 

\citet{garvin2008yours} states by creating an environment that serves psychological safety for employees, organisations may capitalise on failure. Safe environment does not humiliate or punish employees for failing or coming up with novel ideas or doubts. 

However, \citet{thomke2001enlightened} does not suggest failing and making mistakes as a result of poorly planned experiments. Mistakes and failures produce most value, when the experiment is well planned and the goal or hypothesis that needs to be tested is clear. Thus, failed experiments should not be considered as failing, instead they offer valuable learning points.
Failing early and often, yet avoiding mistakes is important for experimenting. Failure can disclose important information and reveal gaps in knowledge, and is thus important as early phase of the development as possible. However, according to \citet{thomke2001enlightened}, this is not an usual way for an organisation to think about failure, thus building the capacity for rapid experimentation as well as tolerating and learning from failure is essential and often requires overcoming ingrained attitudes. Encouraging and creating a culture where failing is allowed and not being afraid of, brainstorming sessions where judgement is not allowed are important. 
