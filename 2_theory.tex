\chapter{Change and learning in an organisation}
According to \citet{edmondson1999psychological}, learning behaviour is an essential concern in the fast-paced working environment, where organisational change and complexity are increasing. Reflection and learning are critical in order to understand the circumstances of increased uncertainty, pace of change and decreased job security in future organisations. This chapter approaches change in business environment and developing in an organisation from the perspective of both organisational and individual learning.

First, change in organisational and business environment is presented, forming background for the need of new methods, such as experimenting, to deal with change. 
Then, learning in organisational level is outlined, following individual perspective for learning. This chapter forms the basis and background for the next chapter, which introduces experimenting as a tool for development and learning. 

Innovation is brought up to the discussion as it is highly related to experimentation behaviour, and even though innovation and innovation processes were not in the focus of this study, experimentation serves as a method to foster innovation. 

\section{Change in organisational and business environment}
\citet{hammer1993reengineering} have summarised aspects of change in organisational environment, beginning from the change in organisational structure; from functional departments to process teams. Work tasks change from simple and detailed tasks to multi-dimensional knowledge work while employees are becoming more autonomous instead of strict control. Furthermore, instead of educating, focus is in the learning of an employee, and evaluation of work will change from operations to outcomes. Knowledge and capability are preferred over single performance and values change to more productive behaviour than over-protective. Superiors turn from leaders of the work to coaches and hierarchical organisational structures turn lower while managers focus on leadership instead of task management. \citep{hammer1993reengineering} 

Understanding change analytically and from systems perspective in the turbulent world appears challenging. Change being hectic and fast calls for different skills and strategy than before. Only when change is understood can it be managed, and in order to survive new perspective and understanding towards change is required from an organisation. 
In the changing environment tolerance for uncertainty is needed, and while future can not be predicted, forecasting is a usable method in order to cope with the anxiousness resulting from the uncertainty. Furthermore, even more emphasis should be put on ability to learn and adapt to changes. \citep{senge1990fifth}

Current and future business environment requires continuous learning from organisations, meaning deploying the collective knowledge, skills and creative efforts of their employees \citep{dess2001changing}. Wide access to the information has put tremendous pressure on today's business and companies to increase their efficiency and effectiveness. Simultaneously, budgets are squeezed and margins of profit grow smaller. Concurrently, value of great ideas and demand for creative endeavours have rose in order to improve and develop products and processes. Employees who are able to produce those competitive ideas are precious for the current competitive business environment and organisations which strive for innovativeness. \citep{andriopoulos2000enhancing,oldham1996employee}

Furthermore, technological change, as hectic and rapid it has been, cannot be ignored. Fierce competition for market share and urge for technological innovations have increased the pace of change leading organisations in high pressure to adapt new business environment, rearrange resources, understand and meet new customer and business environment demands. \citep{andriopoulos2000enhancing}

Today, economy is driven by innovation and innovativeness, requiring new understanding and abilities to generate great ideas in order to survive in every business and business level. innovativeness calls for creativity, which again calls for new managing skills of leaders. In contrast to many leaders beliefs, creativity and creative individuals can be managed and encouraged. \citep{amabile2008creativity} 

Hand in hand with innovation comes creativity and employees' ability to be creative in their work \citep{hennessey19881,shalley2004leaders}. Organizations have realised the value of creativity across a variety of tasks, occupations and industries. Dynamic and quickly changing work environment requires new skills and approaches from managers. For instance, they need to motivate and involve employees in various ways in order to foster creativity and innovation that may lead to competitive advantage in the business field. Individual creativity forms the base for organisational creativity and innovation \citep{hennessey19881}, which has been realised to have influence on performance and survival of the company \citep{nystrom1990organizational}. However, all jobs do not require same amount of creativity, and in all jobs the weight of creativity is not as important, yet all organisations benefit from understanding where creativity is really needed and how it can be fostered and managed. \citep{shalley2004leaders} 

Although currently creativity and creative processes of an individual at work are rather well recognised and essential, even more focus should be put on organisations' ability to mobilise creative actions of employees to create novel, socially valued products or services and more efficient ways of working \citep{mumford1988creativity}. In other words, creative actions of an employee are not worthwhile to an organisation when not coordinated or harnessed to yield organisational-level outcomes \citep{jung2003role}.

Short time horizons require companies to stay in continuous stream of quarterly profits, oftentimes at the cost of long time benefits, that innovations demand. Especially large companies easily favour narrow-minded actions such as quick marketing fixes, cost cutting and acquisition strategies over systemic thinking and process, product or quality innovations. \citep{quinn1985managing}

Company's ability to learn is related to its competitive advantage, development and way to survive in hectic business environment. For instance, according to \citet{geus1997living} the only one can maintain company's competitive advantage is to make sure the company is able to learn faster than rivals. Generally organisations are considered as machines, yet recently more emphasis has been put on organisations as living organisms. When considered as machine, organisational model is mechanic and simple, which purpose is to gain profit. Whereas organisation as a living organism is a whole-systemic model, and organisations are considered as place which has deeper, permanent meaning offering people the opportunity to grow and fulfil themselves while earning money. Liable vision of the future focuses on the latter perspective of organisations, where learning and renewal form the essence of being. \citep{geus1997living}

According to \citet{edmondson1999psychological}, learning behaviour consists of seeking feedback, sharing information, asking for help, talking about errors and experimenting. Thus, experimentation behaviour seems to relate to learning of an individual, teams and organisations and can be supported by supporting these factors. Normal business consists of repetition, risk-avoidance and focusing on business outcomes \citep{buijs2007innovation}, while innovation requires novel solutions, thinking out of the box, risk-taking, breaking the rules, challenging the status quo and questioning the future \citep{burns1961management,kanter1984change,march1991exploration}.

\section{Organisational learning}
According to \citet{garvin2008yours} in learning organisation employees excel at creating, acquiring and transferring knowledge. They define building blocks for learning organisation: supportive learning environment, concrete learning processes and practices and leadership behaviour that reinforces learning. Building blocks can be considered and measured as independent components yet each of them vital to the whole. In order to improve long-term learning of an organisation, strengths and weaknesses of an organisation and its unit needs to be recognised. 

According to \citet{garvin2008yours} second building block of organisational learning, consists of concrete learning processes and practices. It includes experimentation, information collection, analysis, education and training and information transfer. Organizational learning can be supported through concrete steps and activities which are tested and further developed through experimentations. Furthermore, information and intelligence about customers as well as technological trends should be collected systematically and further analysed focusing on identifying problems and solving them. Training and education of new and established employees is an essential part of practices and processes. Finally, through transparent and meaningful knowledge sharing organisational learning can be enhanced, focus being on clear, well-defined and working communication systems that employees can easily relate to. Concrete processes together with efficient knowledge sharing methods ensure essential information being available fast and efficiently for employees who to use. \citep{garvin2008yours}

Supportive leadership behaviour alone is not sufficient guarantee for organisational learning. \citet{garvin2008yours}emphasise how organisations are not monolithic and managers should sense differences in culture, department and units. In addition to cultural differences, learning requires clear and targeted processes and practices. Furthermore, learning should be considered as multidimensional, thus organisational forces should not be solely focused on a single area but to consider presented building blocks as a whole.

Learning is essential part of developing and innovation; according to \citet{buijs2007innovation} all innovation processes are processes for organisational learning. Through the understanding of organisational learning real growth and support for learning can be offered. Learning process needs to be understood all from organisational, team and individual perspective \citep{buijs2007innovation}. In addition, \citet{senge1990fifth}, "The organizations that will truly excel in the future will be the organizations that discover how to tap people?s commitment and capacity to learn at all levels in an organization."

Organizational learning is approached from two different perspectives in literature. On the one hand, learning is considered as an outcome, and on the other it is considered as a process \citep{edmondson1999psychological}. In the first perspective organizational learning is referred to be "an outcome of a process of organisations encoding interferences from history into routines that guide behaviour" \citep{levitt1988organizational}, whereas process perspective defines learning as a process of continuous trial and error \citep{argyris1978organizational}. In this thesis, learning is considered as the latter tradition of learning, which allows the growth and improved performance of individuals and organisations. \citet{edmondson1999psychological} presents and studies behaviours through which various outcomes of learning as adaptation to change, understanding or improved performance are likely to be achieved. Furthermore, \citet{edmondson1999psychological} apply the term learning behaviour to separate it from learning outcomes, and states how set of several activities form the basis of learning behaviour. 

Educational philosopher John Dewey has conceptualised learning as a process in his writings about inquiry and reflection \citep{dewey1956human}. His wok has influenced remarkably on following learning theories, such as experiential learning theory \citep{kolb1984experiential} or action approach of organisational learning \citep{schon1983reflective}. According to \citet{dewey1956human} learning is an iterative process consisting of designing, carrying out, reflecting upon and modifying actions. Dewey separates learning from humans' tendency to behave habitually or automatically. \citet{edmondson1999psychological} builds to this definition focusing on the group level of learning and defining it as an ongoing process where reflection and action occur. Integral characteristics of learning process are asking questions, seeking for feedback, performing experiments and reflecting on the results, having discussions about error and surprising or unexpected outcomes of actions. In group level, learning is enabled through testing assumptions and discussion of opinion differences transparently in order to improve team performance. \citep{edmondson1999psychological} 

Organizational learning differs from individual or team learning. Organizational learning occurs through the shared knowledge, insights and approaches of the employees of an organisation. Secondly, organisational learning is based on prior knowledge and experience, the memory of organisation, which consists of the ways of working, processes and instructions of an organisation. Even though individual and team learning are highly related to organisational learning, it is not the sum of the previously mentioned. \citep{sydanmaanlakka2007}

Various definitions of learning organisations have been presented. One of the most famous definitions is from \citet{senge1990fifth}, who describes learning organisation as follows: "Learning organisation is an organisation, where people are able to constantly develop and achieve intended results; where new ways of thinking are born and where people share goals and learn together."

Together with analysis, limitation, regeneration and technological change, learning is essential factor in improving organisational performance and strengthening competitive advantage. According to \citet{march1991exploration} each pace involves exploitation and exploration as well as adaptation. Exploitation refers to refinement and extension of competences, technologies and paradigms that already exist, whereas exploration is about experimentation with new approaches and alternatives. When results and returns of exploitation are oftentimes positive, proximate and predictable, returns of exploration are uncertain, distant and usually negative. Therefore, exploration leads to greater locus in learning and realisation of problems than exploitation, when considered the distance in time and space. \citep{march1991exploration}

Accordingly in management literature learning is considered relating and even being dependent on receiving feedback \citep{schon1983reflective}, discussion and failure \citep{sitkin1992learning} and experimenting \citep{henderson1990architectural}. As relevant information about performance is acquired through errors, discussion about them has been related with organisational effectiveness \citep{sitkin1992learning}. According to \citet{huy2003rhythm} organisations learn best through small experiments and trying out new things, and the closer and more related experimentations are to customers and customer interfaces, the more can be learned. 

Edmondson's psychological safety has roots already in early research on organisational change. \citet{schein1965personal} state that in order individuals to change and feel safe they need psychologically secure environment. However, team psychological safety should not be confused with groupthink effect that refers more to group cohesiveness, which seems to be related to decreased willingness to disagree and challenge team member's views and thus reduces interpersonal risk-taking \citep{janis1982groupthink}. Term psychological safety refers to team's confidence and shared belief and mutual trust among team members towards that speaking up in a team does not lead to embarrassment, rejection of punishment of any kind \citep{edmondson1999psychological}. 

Learning is often related to organisation's ability to be innovative and develop; according to \citet{buijs2007innovation} all innovation processes are processes for organisational learning. Also \citep{quinn1985managing} argues how especially from the management perspective major innovations should be considered as incremental and interactive learning processes that is driven by certain goal. 

\section{Individual learning}
Various perspectives and definitions for learning has been studied, presented, analysed and utilised in order to understand individual's process of adapting new information and skills. Experiential learning theory refers to learning as a process of knowledge-creation through experiences while experiential learning process stands as a way to describe the central process of human adaptation to the social and physical environment - a holistic adaptation process that provides bridges across life situations and underlaying the lifelong process of learning. \citep{kolb1984experiential}. Also \citet{jung1923psychological} argues how learning involves concept of human being as a whole - from feeling and thinking to perceiving and behaving.

However, everyday problem-solving and immediate reactions to situations at hand are oftentimes related to performing instead of learning. Furthermore, long-term adaptations to our previous experiences and beliefs is mainly considered as developing, not learning. Yet, when talking about development and developing in individual, team or organisational level, the question highly concerns and is related to learning. \citep{kolb1984experiential}

Experiential learning theory of \citet{kolb1984experiential} consists of four elements: experience, perception, cognition and behaviour. Immediate experience forms a basis for reflection and observation, following assimilation to a theory from which new implications for action are deducted. In order to create new experiences, these implications serve as guides. Overall, experience of an individual is a focal point of learning, and giving personal meaning to abstract concepts, which can be afterwards shared with others. Furthermore, receiving feedback is considered essential in this approach for learning, as it serves a continuous process for goal-oriented action following evaluation of that action. Feedback can thus boost effective, goal-oriented learning process. \citep{kolb1984experiential}

Continuing with the model of \citet{kolb1984experiential}, instead of conceiving learning in terms of outcomes, it should rather be conceived as a process. Ideas are not fixed and and immutable elements of thoughts, but can be formed and re-formed through experience. Furthermore, bringing the experiential learning into educational implications, all learning can be considered as relearning. Thus, all learning situations should take into account people arriving from all different experiential backgrounds to what they build their new experiences and knowledge on. This partly explains very likely resistance to new ideas, as when new information and experiences are in contradiction to old beliefs and experiences, new ideas and information is more difficult to adapt. In the education process learner's old beliefs and theories should be brought out, examined and tested, following integration of the new models and refined ideas into learner's belief systems. \citep{kolb1984experiential}

\citet{kolb1984experiential} presents Piaget's interactive process approach to learning, according to which individual learning and adaptation of new ideas occurs through integration or substitution. Integration leads to stronger part of learner's conception of the world, whereas substitution requires real questioning of previous conceptions, and thus might take longer for the learner to adopt. Learning is a mutual process between accommodation of concepts or schemas to experiences around us and assimilation of events and experiences into existing concepts and schemas. This intelligent adaptation, learning, results from the tension between accommodation and assimilation. Through this tension growth and higher-level cognitive functioning occurs. According to \citet{kolb1984experiential}, learning is a process filled with tension and conflict, and new knowledge, skills and attitudes are achieved through experiential learning, which consists of four modes and required abilities of learners: concrete experience abilities, reflective observation abilities, abstract conceptualisation abilities and active experimentation. First of all, individuals must openly involve themselves in new experiences, reflect and observe them from various perspectives, create concepts that can be integrated into more abstract theories as well as they need to be able to use these reflections and theories in active daily decision-making and problem-solving.  

Meaning of environment in learning should be emphasised. Learning concerns of transaction between an individual and the environment, learning does not happen only inside of individual's thoughts, experiences and processes but is dependent on the real world environment. Understanding of different learning styles and modes assists in supporting individuals in learning and problem-solving.

Summary of organisational and individual learning

\section{Experimentation-driven developing}

In this thesis experimentation refers to a personal trial and error process in which employees utilise their full potential \citep{andriopoulos2000enhancing}. Experimenting serves as a great method when testing and validating abstract concepts \citep{kolb1984experiential} and dealing with novel products, ideas and processes, which aim to create new value. This chapter presents the experimentation-driven process for development and outlines how experimenting can be considered as a great method for learning. 

Many studies relate experimenting to innovation and creative abilities of an individual. Thus, in this chapter, innovation, creativity and intrinsic motivation are outlined. In this thesis, innovation is outlined as experimenting has been related as essential tool to foster innovation. Even though in the data and case innovation was not in focus but developing and learning through experimenting, understanding experimenting in broader perspective may assist in seeing its advantages for value-creation. 

\citet{thomke2001enlightened} suggests four steps for organizations to be more innovative. First of all, organization should allow and manage the work for the employees so that fast experimentation is possible. This usually requires challenging routine ways of working and shaping the routines, yet fast experimenting is essential in order to get rapid feedback for shaping the ideas. 

\section{Process for experimentation}

According to \citet{thomke1998modes} experimentation refers to iterative trial-and-error process in which after every trial new information of a problem is gained. When dealing with problems which outcomes are uncertain, experimentation is a fundamental for learning. Likewise, experimenting works when the most essential sources of information do not exist or are unreachable. \citep{lee2004mixed}

"Important discoveries in science (such as artificial vaccines) and technology (such as the electric lightbulb) resulted from constant trial-and-error experi- mentation through which inventors systematically built up a knowledge base (Thomke 2003). " 

"Experimentation advances an engineer?s understanding of new analytical concepts, promotes new ways of think- ing, and creates new engineering knowledge (Vincente 1990). More broadly, individuals who constantly impro- vise, tinker, and experiment are able to remain adaptive in fast-paced industries where new ideas and innovations are constantly in demand (Ciborra 1996)." (lee2004mixed-artsusta)

"This avoidance can be explained by the interpersonal or social costs of failure. Specifically, failures make one?s gaps in expertise and knowledge salient to oth- ers (Lee 1997), and avoiding failure helps to maintain one?s image and professional standing among colleagues (Wolfe et al. 1986). Interpersonal costs of failure are exaggerated when people lack ?psychological safety.?" (lee2004mixed)

"For example, differences across organi- zations in psychological safety have been shown to affect the level of anxiety people feel when confronting ambiguity and uncertainty (Schein 1985). Organizational differences in psychological safety can be created by supportive structures such as information and reward systems (Edmondson and Mogelof forthcoming) and by the words and actions of high-level management;"

"Such failures can be bene- ficial because they provide the experimenter with new knowledge about the solution and thereby facilitate inno- vation and performance in the long run (Sitkin 1992)."

"even if the experiment fails, new knowl- edge is created that narrows the scope of subsequent trials." \citep{lee2004mixed}

"Each trial in experimentation generates information about a solution that the experimenter could not know in advance. Information learned in a previous trial can be used to modify subsequent experimental designs, condi- tions, or even the nature of the desired solution (Thomke et al. 1998). Tasks that are conducive to effective experi- mentation are those that allow multiple problem-solving trials and present opportunities to use knowledge gained from earlier trials to enhance learning in subsequent trials." \citep{lee2004mixed}

"Although little research has examined organizational conditions that promote experimentation, many studies identify pre- dictors of similar behaviors such as learning, creativ- ity, information seeking, and other interpersonally risky but organizationally desirable behaviors. This work has found that creativity is related to organizational cul- ture, reward systems, supervisory encouragement, trust, and resources (Amabile et al. 1996). Feedback, infor- mation, help-seeking, and issue-selling behaviors are all predicted by supportive organizational norms, leader- ship openness, and trust (Ashford and Northcraft 1992,Ashford et al. 1998, Lee 1997, Morrison 1993). Proac- tive learning behaviors are related to supportive orga- nizational contexts (access to resources, information, training, and supportive reward systems), leader coach- ing (Edmondson 2003), and routines that encourage exchange of relevant information, reduce sensitivity to feedback, decrease defensiveness, and increase trust (Argyris 1994)." (lee2004mixed)

"There is evidence that when normative values state that failures are expected and acceptable as part of learning, people are less hesitant to discuss mistakes (Edmondson 1996)"

"Rewards systems that pun- ish failures increase the costs of experimentation, and may make individuals reluctant to experiment (Thomke 2001)."

"Evaluative pressure is distinct from coaching, in which close attention or monitoring is provided to facilitate rather than evaluate performance. Indeed, monitoring in the context of supportive coaching can actually enable interpersonal risk taking (Edmondson 1999, 2002), while close and constant evaluation intended to identify and expose failures has been shown to inhibit creativity (Amabile et al. forthcoming) and make novel or unfa- miliar tasks more difficult (Zajonc 1965). " (lee2004mixed)

"Inconsistency and Experimentation. Much of the res- earch noted above has focused on how single variables? e.g., normative values, instrumental rewards, or evalu- ative pressure?independently affect innovation behav- iors. For example, Amabile et al. (1996) showed that eight organizational conditions individually predicted creative performance, and Ashford et al. (1998) exam- ined four antecedents of issue selling. These studies assume an incremental or additive model of influences on behavior. One implication of this componential per- spective is that improving any one of various organiza- tional factors should increase these behaviors.
We are interested instead in how combinations of organizational variables affect innovation behaviors. A combinational perspective assumes that the combina- tion of conditions employees face may be as influen- tial as the individual conditions themselves. The Bank of America example illustrates what can happen when normative values are changed to explicitly encourage experimentation and instrumental rewards discourage it. Inconsistency in organizational conditions may actually do more harm than good, because it creates uncertainty in which individuals do not know which factor (e.g., normative values or instrumental rewards) will shape the organization?s response to their actions." (lee2004mixed)

"In contrast, inconsistency may reduce psychological safety and thus experimentation. First, inconsistent con- ditions make the rules unpredictable and ambiguous. The uncertainty about whether one will be punished cre- ates a state of mild fear, which is antithetical to feel- ings of psychological safety. Second, facing the need to simultaneously serve contradictory aims itself may create anxiety, lowering psychological safety. Inconsis- tent messages place people in a bind (Argyris 1982) because they communicate two incompatible goals (e.g., ?experiment with new ideas, but don?t fail?). Facing this, people may experience emotions of fear or anxiety that make taking action and not taking action equally unpleasant alternatives (Argyris 1990). Third, inconsis- tency has been shown to create cognitive and emotional responses such as suspicion, mistrust, and confusion, leading to ?threat rigidity,? a tendency towards risk aversion, behavioral inhibition, suppression of activity, avoidance, lack of openness, and an inability to try novel behaviors (Masserman 1971, Staw et al. 1981)." (lee2004mixed)

"While this argument leads to the somewhat intuitive idea that consistently encouraging organizational conditions would lead to more experi- mentation behaviors than inconsistent conditions, it also suggests a less intuitive scenario. Specifically, it is possi- ble that individuals will engage in more experimentation behavior when organizational conditions consistently discourage experimentation than when some conditions encourage experimentation and others do not. In the con- sistently ?discouraging? situation, individuals are clear about the rules and constraints they encounter, and there- fore may experience more psychological safety than they would when facing the uncertainty created by inconsis- tent conditions. If so, they may experiment more.
Further, in consistently discouraging conditions, peo- ple working closely together can experience a sense of solidarity based on shared perceptions of negative work conditions (Edmondson 1999, George and Zhou 2002), whereas inconsistent conditions may lead to mistrust and suspicion that undermine psychological safety. " (lee2004mixed)

"The combinational perspective has been used to examine measures of organizational performance such as productivity, manufacturing quality, and efficiency, but it has not been applied to the study of individ- ual behaviors within the organization. Meyer et al. (1993) found that current theories of individual and group behavior in organizations, ranging from per- sonality, motivation, task design, work group design, and organizational demography, have largely adopted a componential rather than combinational approach." (lee2004mixed, t�h�n systeemiajattelupointti) 

"The componential perspective suggests that organizational conditions (such as normative values, instrumental rewards, or evaluative pressure) indepen- dently affect innovation behaviors. Thus normative val- ues that encourage experimentation will lead to higher levels of experimentation behavior, regardless of instru- mental rewards and evaluative pressure; instrumental rewards that do not punish failures will lead to higher levels of experimentation behavior, regardless of nor- mative values and evaluative pressure; and individuals under high evaluative pressure will experiment less than individuals under low evaluative pressure, regardless of normative values and instrumental rewards.
The combinational perspective suggests that the inter- action between organizational conditions is also impor- tant. This perspective suggests that when organizational conditions are consistent?for example, when norma- tive values, instrumental rewards, and evaluative pres- sure all encourage experimentation?there will be more experimentation than when organizational conditions are inconsistent?when some encourage experimentation and some discourage experimentation. This perspective also allows the possibility that experimentation will be greater when organizational conditions consistently dis- courage it than when they are inconsistent." (lee2004mixed)

"Although the literature on innovation has emphasized organizational-level structures and processes, an orga- nization?s ability to introduce a new product, develop unique processes, and leverage new technologies begins with individuals coming up with new ideas and try- ing these ideas out to assess their feasibility (Argote and Ingram 2000). Understanding conditions that enable individuals to engage in experimentation behavior is thus an important element of understanding organizational innovation (Thomke 2003)."(lee2004mixed)

\citet{lee2004mixed} argue more holistic perspective is needed, it is not sufficient to change only one organisational attribute in order to foster innovation. 
"We argued that inconsistency in organizational conditions?when some encourage experimentation but others do not? might reduce experimentation. "

"Both studies found partial support for the notion that individuals experimented more under instrumental rewards that did not penalize individuals for failures. " (lee)

"Inconsistency in organizational conditions?when some encourage and others discour- age experimentation?may undermine experimentation behavior, with one factor rendering the other ineffec- tive." (Lee)
"Yet, our results showed that these disabling effects of inconsistency only occurred for individuals under high evaluative pressure. Evalua- tive pressure might decrease psychological safety and might make individuals more vulnerable to the uncer- tainty inconsistent conditions create. Counterintuitively, both studies found that individuals under low evaluative pressure experimented more when organizational condi- tions were inconsistent than consistent. In Study 2, those with low evaluative pressure experimented more under inconsistent conditions than when both normative val- ues and instrumental rewards consistently encouraged or discouraged experimentation."

"All of these explanations have in common the obser- vation that, facing unpredictability, individuals under high evaluative pressure are more likely to become inhibited, fearful, narrowly focused, and rigid, while individuals under less evaluative pressure are more likely to become proactive, optimistic, thoughtful, and risk seeking. " (lee)
Among research it is widely recognized that in the heart of problem-solving process is continuous trial and error which are directed by some amount of insight about the possible direction of the solution \citep{barron200thinking}. 

According to \citet{thomke2001enlightened}, in the beginning every product is an idea, that was being shaped through the process of experimentation, and the ability to do experimentations is actually a measurement of company's ability to innovate. Thus, experimentation lies at the heart of company's ability to be innovative. 

Experimentation is essential in order to learn about the idea, concept and prototype and whether it actually addresses a new need or a problem or solves the one at hand. Prototyping is critical part of the process, as testing the prototype in a real environment gives instant and valuable feedback for further development. \citep{thomke2001enlightened}

Current business is remarkably dependent on services, yet innovation techniques and processes remain focused on products. Systematic learning methods are needed in order to avoid occasional successes and provide more stable base for consistency and productivity of service development. Experimentation does not only concern product development, it should and can also be applied to service design and development. \citep{thomke2003r} 
Experiments concerning service development are most useful when conducted in circumstances of real life, as the feedback is instant and customers transactions real. However, live experiments are in risk to harm customer relations, hurt the brand and may be more difficult and resource-consuming to execute and measure. \citep{thomke2003r}

Bank of America stands as an example of service prototyping by having conducted experimenting several years in order to create novel service concepts for retail banking. They have set up an experimentation laboratory in some of their banks where actual customers during normal office hours can test novel ideas. Feedback is collected and experiments measured in order to learn as much as possible for further experiments. During this process they have learned radically about the system approach and far more than only 
Bank of America has faced several \citep{thomke2003r}

"As important as the business benefits is the enormous amount of learning that the bank has gained. Carrying out experiments in a service setting poses many difficult problems. In grappling with the challenges, the bank has deepened its understanding of the unique dynamics of service innovation. It may not have solved all the prob- lems, but it has gained valuable insights into the process of service development-insights that will likely provide Bank of America with an important edge over its less ad- venturous competitors."

Conducting experiments in live settings causes distraction for both customers and employees: customers may be confused by new processes and employees find it difficult to adapt to new routines. 

"In a lab,experiments are routinely undertaken with the expectation that they'll fail but still produce value.
In the real world, there is pressure to avoid failure."

"To address this unanticipated effect of the program, senior management abandoned the traditional bonus system in the test branches in January 2001, switching all associates to fixed incentives based on team perfor- mance. Most associates welcomed the change, which am- plified their feeling of being special while also underscor-
In a lab,experiments are routinely undertaken with the expectation that they'll fail but still produce value.
In the real world, there is pressure to avoid failure.
ing top management's commitment to the experimentation process."

" But every company will have to go through its own testing period to arrive at the balance that is right for its people and that doesn't un- dermine the fidelity of its experiments. It's important to note, also, that staff turnover in the test branches dropped considerably during the program. The difficulties employees faced paled in comparison to the enthusiasm they felt about participating in the effort."

"If the capacity of the entire market was not well managed, too many experiments would have to be performed at a single branch, increasing the amount of noise sur- rounding each one. And if the team were to run out of ca- pacity, it would be forced to do tests sequentially rather than simultaneously, which would delay the process."

"Einally, there was the issue of failure. In any program of experimentation, the greatest learning comes from the most radical experiments - which also have the highest likelihood of failure. In a laboratory, radical experiments are routinely undertaken with the ex- pectation that they'll fail but still produce extraor- dinarily valuable insights, often opening up entirely new frontiers for investigation. In the real world, however, there are inevitably pressures to avoid fail- ures, particularly dramatic ones. Eirst, there's the
fear you'll alienate customers. Second, there's the fear you'll damage the bottom line, or at least shrink j'our own source of funding. Einally, there's the fear you'll alienate top managers, leading them to pull the plug
on your program."

"Any service company de- signing an experimentation program will have to care- fully weigh the risks involved against the learning that may be generated."

"experimentation in a live setting entails particular challenges. But however daunting they may initially seem, the challenges can be met, as Bank of America's effort suggests. In just a few years, the bank has achieved important benefits that are having a real impact on its business and its future."

According to \citet{thomke2003r} constant changes in practices and processes are required in order to perform experiments. Furthermore, the amount and value of learning achieved from experiment is the success measure for it: the more has been learnt and the more valuable insights, the more successful an experimentation is.  

According to \citet{thomke2001enlightened}, critical part of innovation process occurs when first prototypes are generated, as at that point they can be further tested with customers, discussed and evaluated. 

For instance, development process that lead to the innovation of a light-bulb consisted of repsated iteration of experiments, analysing the outcomes, learning from them and making changes for the next experiment. \citet{thomke2001enlightened}
In the beginning of experimenting process, possible solutions are created or selected and furthermore tested again specific requirements. Trial-and-error experiment provides new information, which is something a person conducting the experiment usually could not know beforehands, thus something referred as "error". Overall outcomes of the experiment are analysed and solutions refined accordingly. \citet{thomke1998modes} Through trial and error process various design alternatives can be tested and generated, essential being reflection after each experiment and making changes accordingly to next experimentation round. 

Even though novel experimentation techniques are useful and may reduce the cost and time used for product development, adopting experimentation techniques may require changes in organisational culture or way of doing work \citep{thomke1998modesd}. However, adapting changes may lead to increase in productivity and affect the overall competitive positioning among companies. 
 
\citet{thomke1998managing} defines experimentation as a form of problem-solving and essential part of innovation activity, and is highly related to the innovation process as a whole affecting to the cost and time of the process. Experiment highly relates trial and error learning. Experimentation process is presented as iterative four-step learning cycle consisting of designing, building, running and anaylizing the experimentation. In the first phase, experimentation is designed based on the previous experience or good guess on solution. In second phase the needed prototypes or spaces are build for experiment, following third phase where experimentation is actually conducted. Analyzing phase is essential in order to learn from the results and being able to conduct the cycle effectively again. Indeed, oftentimes experimentation is a matter of repeated trial and error. 

\citet{thomke1998managing} suggests managing different experimentation modes has affects on the productivity of the product design and development process meaning better managed time and costs. Experimentation modes are for instance computer simulating or rapid prototyping. Rapid prototyping refers to situation where developer quickly generates an inexpensive and often physical prototype that simulates the main function of the product. Fast prototypes allows testing in real environment and offers valuable feedback and reduces time and costs used for developing. 
Furthermore, \citet{thomke1998managing} defines experimentation efficiency, referring to "economic value of information learned during an experimental cycle, divided by the cost of conducting the cycle." The more inexpensive (costly) an experimentation is and the more valuable (valueless) gained information is, the higher (lower) is experimentation efficiency. 

Oftentimes experimentation is conducted by using as simple prototypes of the intended-product as possible in order to experimentation remain light and cost-efficient. As experimenting once should not be expected to lead to "right" and successful solutions, may be efficient to plan and conduct multiple experiments in order to get closer to the problem solution \citep{thomke1998modes}. 

Experimentation-driven approach for innovation differs from other methods for managing uncertain and innovation-focused projects in that it emphasises learning far more than other methods. Experimentation should not only be considered as method for learning but it's role should me further: a tool and everyday practice to guide company's strategy-making, business models and behaviour \citep{davenport2009design,mcgrath2010business}

Through experimentation new information and ideas can be generated and in even new opportunities can be found \citep{tuulenmaki2011art}. 
New possible outcomes may reveal during experimentation (McGrath 1999, samin gradusta)

According to the process of experimentation-driven innovation of \citet{tuulenmaki2011art}, there are three types of ideas. Opportunity idea, execution idea and implementation idea. 

Opportunity idea refers to an idea which is imagined to solve specific problem as well as something that brings closer to the solution. 
Zappos-esimerkki Hsieh 2010 -arstusta
Instead of writing a business plan a founder came out with experimentation idea: the easiest way possible to try out if the implementation idea is worth further development. 
Experimentation idea assists in testing the critical assumption and figuring out, whether the idea is worth taking further risks. (Sykes and Dunham 1995) 
Execution ideas are the outcomes from experimentation ideas, those ideas that have been through iterating and validating process chosen to further development and implementation. Execution ideas gather all the learnings from experiments, through which the original opportunity idea is modified in order to reach the final design plan. According to \citet{hassi2012experimentation} experimentation-driven innovation process consists of series of iteration with the three idea types, opportunity idea, experimentation idea and execution idea. 

\citet{thomke1998agile} defines a term development flexibility, which refers to 
It is an alternative for forecasting the future and works as a powerful method in risk-managing of development. As forecasting the future has become increasingly difficult, emphasis should be put on managing the risks of failed decisions. Shorter development cycles
Thus, development flexibility is critical as companies are no longer (if they ever actually were) possible to accurately forecast unknowable future. Furthermore, unknown, unstable and changing customer needs 
Through development flexibility entire product changes can be avoided, as design commitment and decisions can be made late phase in the process. Furthermore, significant reduce can be seen in time and cost used for development. 

"Development tlexiblllty can be expressed as a function of the incremental economic cost of modifying a product as a response to changes thai are external (e.g., a change Ui customer needs) or internal (e.g., discovering a better technical solu- tion) to the development process. The higher the economic cost of modifying a produa, ihe lower the development flexibility."\citep{thomke1998agile}

Other approaches
prototype-driven approach refers to a method in which customer feedback is acquired through prototyping in an early phase of the process in order to make changes affordably
Specification-driven approach, in turn, refers to a process where design is freezed after the specification is complete. Prototyping was more successful and lead to more successful products and made with fewer design resources. In other studies higher customer satisfaction and quality and company's performance have been related to more flexible development process, in which changes can be made in the very late phase of the development process. \citep{thomke1998agile}

Anticipating and exploiting early information can save a lot of resources in the development process. If problems are shown in the late-stage of the process, they can be even 100 times more costly than the ones discovered in the early stage. According to IDEO, an innovation and design-firm, using human-centered design-based approach, the key elements in the design process and prototyping is it being rough, rapid and right. The right-element reminds that even though the prototype itself is likely to be incomplete, it has to show the right specific aspects of a product. This forces developers to decide the factors that can initially be rough and those that must be right. In addition, exploiting early information serves as a good method for developers reflecting changing customer preferences. Briefly, information in the early stage of the developing process should be listened and discovered carefully, as the problems are cheaper and easier to solve. 

For enlightened experimentation \citet{thomke2001enlightened} puts emphasis on combining new and traditional technologies. 

Experimentation causes inconsistent outcomes. As organisational conditions may be inconsistent, employees are not eagerly engaging to experimentation. \citep{lee2004mixed}
"Because failures are inevitable in the experimentation process, we argue that conditions giving rise to psychological safety reduce fear of failure and promote experimentation. Based on this reason- ing, we suggest that inconsistent organizational conditions?when some support experimentation and others do not?inhibit experimentation behaviors. An exploratory study in the field, followed by a laboratory experiment, found that individu- als under high evaluative pressure were less likely to experiment when normative values and instrumental rewards were inconsistent in supporting experimentation. In contrast, individuals under low evaluative pressure responded to inconsistent conditions with increased experimentation. Our results suggest that evaluative pressure fundamentally alters an individual?s experience of and response to uncertainty and that understanding experimentation behavior requires examining effects of multiple organizational conditions in combination."

Also according to \citet{lee2004mixed} experimentation behaviour is essential for innovation. 

In the case of Bank of America failure rate was set in 30 per cent in order to show the support for essential failure

"Bank senior management voiced strong support for innovation and explicitly recognized and communicated that experimentation with new ideas necessarily pro- duced failures along the way. Indeed, a failure rate of 30percent was targeted as indicative of sufficient risk taking and novelty. Initially, however, employee compensation continued to be based on measures of routine perfor- mance (such as opening new customer accounts). The espoused goal of increasing innovation thus was incon- sistent with the reward system; individuals? compensa- tion could suffer from time spent experimenting with new ideas or from failed experiments.  "

Current research examining antecedents of behaviors such as learning, creativity, and experimentation has emphasized main effects of single organizational variables. Implicit in this approach is the idea that changing one organizational condition can lead to improvement in behaviors integral to innovation. We argue, in contrast, that changing a single organizational condition without changing others creates inconsistency that instead may inhibit innova- tion behaviors. 

However, \citet{lee2004mixed} have studied the inconsistencies that are likely to inhibit innovation behaviour. Where current research claims how affecting on one organisational condition is likely to foster behaviour essential for innovation, this approach reveals how changing only one organisational condition may lead to inconsistency between work tasks and expectations and lead to decrease in willingness to act towards innovation. Learning, creativity and experimentation are all attached to innovation. 

Combining routine work and innovative experimenting can be challenging \citep{lee2004mixed}.

\section{innovation-shitti�, siirr� pois tai mieti uudestaan mihin menis}
Why innovation and creativity should be a matter for an organisation? Studies have well established the positive relation between creativity and innovation skills of an organisation and organisational performance \citep{jung2003role,mumford2002leading}. Innovation and creativity are highly related, yet not the same thing: According to \citet{hennessey19881}, individual creativity stands for an essential building block for organisational innovation and also \citet{sethi2001cross} argue creativity being essential in new idea generation and design processes that aim for innovative solutions. Other studies also emphasise the role of creativity a first step in creating something novel, whereas innovation refers to the implementation phase of the novel ideas in individual, team or organisational level \citep{shalley2004leaders};\citep{amabile1996assessing};\citep{mumford1988creativity}. 

Innovation process itself can be approached from several angles: first of all, content of the innovation has to be clear - whether the purpose is to innovate new products, manufacturing processes, ways of organising or ways of dealing with people. Secondly, psychological process of the innovation team has to be understood, essential being shared understanding, level of comfort with ambiguity and degree of trust between team members. Thirdly, creative process of the team, meaning idea producing process, needs to be understood and efficiently facilitated. Finally, the role of leading plays a major role, and together with playful attitude innovation process is likely to succeed. \citep{buijs2007innovation}

Several factors have been recognized to affect on organisational innovation, yet many researchers have stated leadership behaviour being one of the most important. \citep{jung2003role,amabile1998kill,jung2001transformational,mumford1988creativity} \citet{jung2003role} identify four hypotheses how top managers' leadership styles may affect both directly and indirectly their companies' ability to innovate. Indirectly here stands for instance leader's possibility to empower employees and build organisational climate optimal for innovation. The study shows a positive relation between transformational leadership style and empowerment as well as innovation-supporting organisational climate. 

Social innovation refers to the generation and implementation of novel ideas concerning people in demand to organise their interpersonal or social activities and interactions in new ways in order to achieve common goals. Results and products of social innovation, like other types of innovation, are likely to vary depending on the breadth and impact of the innovation. \citep{mumford1988creativity} \citet{mumford2002social} presents four factors affecting social innovation: active exchange of ideas and information in supportive climate, tangible and low-cost ideas that can be at the fewer guessed to be beneficial, support from upper level management, and effective communication through the innovation process in order to proceed from the idea to implementation.

Innovation can be contributed by encouraging idea generation, but also creating a climate of autonomy, offering intrinsic and extrinsic rewards and engaging employees with their work \citep{amabile1996assessing}; \citep{amabile1998kill}. Furthermore, characteristics associated with innovation are integration of work units, decentralisation of control and professionalisation are likely to effect innovation in a way that through these suitable environment for innovation, dynamic idea exchange and implementation is created \citep{mumford2002social}.

Managerial practices for technological innovations have been widely studied. According to \citet{quinn1985managing} the essence lays in accepting the chaos of development. In addition, large and successful companies and their leaders listen carefully their users' needs, develop according this customer demand, define clear goals and framework for the work, encourage teams to challenge the status quo and find alternative solutions while avoiding detailed and technical or marketing plans in the beginning. Instead, they focus on early prototyping and iteration. 

According to studies creativity and innovation in an organisation requires integrated organisational approach, right climate, appropriate incentives for innovators, and a systematic way and resources to transform an idea into an innovation.  In individual level, creativity and innovation calls for various skills, such as teamwork, communication, coaching, project management, learning and learning to learn, visioning, change management and leadership, and ability to develop these skills. Oftentimes, even though the climate and practices are right for generate innovations, problems are faced when attempting to manage the change process. \citep{roffe1999innovation}

Social cohesion may inhibit innovativeness of the team and its individuals especially beyond a moderate level, while employees are more likely to settle on group think and traditional daily practices \citep{janis1982groupthink}. However, according to the study of\citet{sethi2001cross} when a team shares superordinate identity, is encouraged to take risks, lets customer's requirements be heard, and actively lets senior management monitor the project, team is more likely to present innovative ideas and perform in innovative ways. According to this study, functional diversity does not effect on innovativeness, but team's superordinate identity can be strengthened by encouraging risk-taking and weakened by social cohesion.

\citet{mumford2002social} argues in his study of Ben Franklin's social innovations, that key factor in successful social innovation lays in fast demonstrating, which he also refers as experimenting. Thus, in order to drive for social innovations, opportunism and showmanship of an individual or team may be required. \citep{mumford2002social} Furthermore, according to \citet{monge1992communication} group communication is likely to increase innovation under some circumstances,  and also \citet{katzenbach1993wisdom} argues for culture of strong team performance. However, \citet{amabile2004leader} emphasise how, ultimately, truly novel ideas raise from individuals, making them the ultimate source of any new idea or solution to a problem \citep{amabile2004leader}.

\section{Experimentation as a method for developing and learning}

The essence of learning from experiments is to figure out what works and what does not in an experiment or idea. Thus, experiments should be designed and planned keeping in mind how to maximise the amount of learning and valuable insights, not focus on wrong details and successful experiment. Measurement of experimentation is essential, through defining correct measures one can actually know whether the experimentation was useful and essential learned or not \citep{thomke2003r}

\citet{thomke2003r} defines seven factors that can be learnt and set as measurements for an experiment. These are fidelity, cost, iteration time, capacity, sequence, signal-to-noise ratio and type. Most can be learnt from experiment when tested under conditions that represents as closely as possible actual use of final product, process or service. This refers to fidelity of an experimentation. However, when testing in actual environment, various variables may affect on the experimentation setting, and this signal-to-noise ratio should be taken into account. Right balance between the speed of experimenting and receiving feedback in order to learn is crucial for successful experimenting, and this iteration time should be measured and estimated: time from the planning an experiment to the moment when results are available and further used. Also, cost of experiments should be analysed by estimating cost of designing, building, running and analysing experiments. Capacity concerns the realistic estimation of number of experiments possible to conduct with decent amount of fidelity in planned period of time. Experiments can be conducted in series or in parallel depending on the project at hand, and thus the sequence of experiments can be measured. Experiment type refers to the level of change, which can vary from incremental to radical. 

Actually, idea implementation may require even more creativity than idea generation \citep{mumford2002leading}, and according to \citet{vincent2002divergent} creative work consists creative and innovation processes. Creative processes comprises of initial idea generation, whereas innovation process goes beyond the activities underlying the implementation of those ideas \citep{vincent2002divergent}.

According to \citet{buijs2007innovation}, innovation consists of coming up with novel ideas and implementing them. Ideating begins with exploring, developing and implementing the ideas, following introducing the ideas, which have turned into products or services, into the marketplace. Innovation process is a series of stages for processing the idea, and in the end of every stage the idea is reflected and evaluated before further processing. Evaluation points stands for usable tool for measuring the quality of idea but gives also understanding of how the evaluation process is going. In addition, while evaluating, team members also need to reflect the process and the idea, through which learning occurs. Also \citep{runco1994problem} emphasises how only after evaluation of ideas implementation can be discussed and performed and several studies show the essence of evaluation \citep{mumford2002leading,vincent2002divergent}. Useful questions in evaluation process could be "What went well?", "What can be improved?" and "What has been learned?" \citep{buijs2007innovation}. 

When dealing with novel solutions and challenging status quo, we are dealing with innovations. In order for company and its employees to be innovative, they need to take risks. Yet, at the same time usual management processes avoid risk-taking and focus on managing daily routine business. As \citet{quinn1985managing} stated it in his Harvard Business Review article: we love innovation and we urge for innovation, but we can tolerate it only if it is controllable and results everything remaining the same. \citep{quinn1985managing}

\citet{andriopoulos2000enhancing} define in their study a concept of perpetual challenging - a way to enhance creativity and innovation in an organisation. According to the concept adventuring occurs when goal is idea generation and through that process individuals are encouraged to face uncertainty in order to generate novel solutions. One tool for idea generation is scenario making, which purpose is to develop possible ways to tackle situation at hand. Through scenario making employees scans what is both known and not known about current problem or situation.  Experimenting process then consists of testing different scenarios generated in ideation phase and evaluating the outcomes in order to decide and develop the scenarios further to meet the needs of clients and industry. This calls for individuals skills to tolerate risks and uncertainties, as well as skills to constructively challenge and question colleagues ideas in order to use their full potential.

According to \citet{andriopoulos2000enhancing} facing and dealing with risk serves also as positive boost to creativity, as employees' learn new skills and strengthen their capabilities constantly and adapt new knowledge to already known. This, however, requires for safe environment which \citet{andriopoulos2000enhancing} refers as safety net: environment that tolerates failure. 

According to \citet{quinn1985managing}, fast multiple-idea prototyping leads to more innovative outcomes, offers essential information about ideas or product's quality, motivates employees, and helps the company and the team to cope with anxiety and uncertainty in development. Engaging lead customers in the interactive development process instead of market research seems to elucidate more relevant information about customer's demands, required changes and entry strategies. Thus, fast prototyping serves an essential way for learning from the iterative process. Market analysis, however, remain valuable when dealing with familiar products and productions, yet with radical innovations they may easily offer misleading information. \citep{quinn1985managing}

\section{Creativity, intrinsic motivation and everyday problem solving}
Creativity and innovation have gained wider acceptance as important factors creating value in organisational performance \citep{mumford2002leading}. Creativity and innovation have for instance been studied to have enhancing impact to organisations profit and growth \citep{nystrom1990organizational}. Creative thinking and actions require time, and contradictory, in fast-paced and rapidly changing world and working environment managers should allocate employees sufficient time for creative thinking and experimenting novel approaches \citep{shalley2004leaders} 

\citet{amabile1996assessing,amabile1998kill} defines creative thinking as a way how people approach problems at hand and come up with solutions. Creative thinking does not stand for intellectual capacity of an individual to create something new but rather as a combination of past experiences which creates expertise and the ability to apply creative thinking skills to these experiences and invent new solutions. 

According to \citet{sternberg1997creativity} a company can enhance its creative skills by focusing on six resources: knowledge, intellectual abilities, thinking styles, motivation, personality and environment. \citet{sternberg1997creativity} argues that too much information may hinder change and be seen as rigidity in thinking. Therefore, one should not over-weight the criticism of senior people in an organisation, and at least consider the chance for rigid thinking and intolerance for change. 
In addition, needs to be noted and understood that employees' thinking styles are shaped through what is rewarded, meaning, that if organisational environment rewards well-behaving and instruction-following thinking style and action, employees tend to implement their style to that. We are urged to adapt to organisational style and fit in, and when this is not possible, people tend to leave. \citep{sternberg1997creativity}

Divergent thinking refers to individual's ability to find multiple alternative solutions and ideas to problems at hand, and has been related to serve as a key capacity affecting creative thinking \citep{guilford1967creativity}. Accordingly \citet{mumford1988creativity} emphasis that creative people consistently and with confident tend to seek for alternative solutions, even under uncertain conditions. Even though expertise and  intelligence have been related to problem solving, series of causal analyses carried out by \citet{vincent2002divergent} revealed unique effects divergent thinking had that were not attributed to intelligence and expertise. 

\citet{shalley2004leaders}, in turn, argue that through developing extensive set of skills, employees may learn to be more comfortable and confident in thinking from different perspectives, finding various alternative solutions, trying out novel things and seizing opportunities. According to \citet{hennessey19881} individual creativity requires ability to think creatively, generate alternatives, engage in divergent thinking and tolerate or suspend judgment. Through this perspective creativity can be considered as a skill that can be learned and strengthened. Understanding of individual's creativity and ways to influence and improve it gives managers guidelines when creating an environment and leadership that support organisational innovation \citep{redmond1993putting}.

Several factors form the basis of creativity skills of an individual, such as personality, technical knowledge, expertise, motives, and the supervisor's feedback style. In group level factors form of task structure, communication styles and task autonomy, and finally in organisational level strategy, structure, culture, climate and available resources all affect how creative actions are encountered. \citep{jung2003role}

Several case studies has showed that creativity insights emerge gradually through the network and actions of an creative individual. Study of creativity is a combination of two different disciplines and research approach: sociological and historiometric lenses study the conditions in which creative actions and processes are likely to occur, whereas neurobiological approach presents neural structures and processes that are active and associated with creative outcomes. \citep{gardner1988creativity}

Generation of novel, alternative solutions requires problem-finding skills \citep{runco1988problem}, which has been indicated to be one of the best predictors of creativity in 'real world' activities, when studied 91 elementary school students \citep{runco1990evaluating}. These findings suggest leaders, in order to enhance creativity of employees, to support learning of these skills for instance by facilitating problem-construction \citep{redmond1993putting}.

\citet{kasof1997creativity} argued in his study that breadth of attention affects on creative performance of an individual: wide spread of attention is usually related to creative ability. By breadth of attention Kasof refers to "number and range of stimuli attended to at any time." Breadth of attention being narrow, individuals are able to focus on narrow range of stimuli and are better at filtering redundant stimuli from awareness. However, those individuals with wide breadth of attention tend to be more aware of irrelevant or extraneous stimuli, these individuals are strongly affected by their environment and are highly arousable.\citep{kasof1997creativity}

Studies of creative characteristics of individuals has revealed factors such as wide interest in various fields, autonomy, belief of being creative and independence in decision-making \citep{shalley2004leaders}. Intrinsic motivation is claimed to be one of the most powerful tools to creative action and non-traditional thinking \citep{amabile1996assessing,deciintrinsic,jung2001transformational}, as intrinsically motivated individuals usually prefer novel solutions, challenging status quo and trying out new ways for solving a problem at hand \citep{amabile2002creativity}. Broad interest stands for a sign of intrinsic motivation, which is also widely related to both creativity and well-being of an individual and innovation (e.g. \citep{hennessey19881}; \citep{csikszentmihalyi199916}; \citep{gardner1988creativity}). In their study  \citet{tierney1999examination}, found positive correlation between employee's level of enjoyment while working on a creative task at hand and the level of creativity.  

Without previous experience of the job routine and substance knowledge and expertise on the field creative endeavours are more rare. Even though has been argued how routine work and task familiarity is likely to lead very habitual performance \citep{ford1996theory}, knowing the status quo may provide opportunities for creative actions through reflecting and practicing skills and activities requires in the field. \citep{shalley2004leaders} Knowing the field and what has already been discovered assists in finding alternative, creative solutions \citep{andriopoulos2000enhancing}.Furthermore, creativity is not restricted to artistic occupations only; it is required in various professions in which tasks presented involve complex, ill-designed problems where novel solutions are needed and status quo challenged \citep{mumford1988creativity}. Indeed, idea implementation may require even more creativity than idea generation \citep{mumford2002leading}.

For students of creativity, there is no surprise in attaching self-efficacy to creative actions \citep{mumford1988creativity}, yet recently problem construction processes have been recognised and combined to everyday problem-solving and real-world creativity \citep{getzels1975problem}; \citep{runco1988problem}. According to study of  \citep{gardner1988creativity} correlation between creative problem solving and everyday problem solving exists: they seem to have the same roots in information processing skills. 

In their study \citet{redmond1993putting} found how leaders supporting employees problem-finding and problem construction led to more unique and novel solutions. Leaders encouraged employees to find alternative solutions, approach problems from different perspectives and overall supporting several alternative problem-solving strategies. In addition, study showed how through motivational mechanisms, such as self-set goals, involvement and commitment, problem construction may have positive influence on solution quality and originality. Thus, problem construction is likely to have its greatest impacts on performance when in the process employee is allowed to express his values, needs and interests \citep{redmond1993putting}. 

Instead of managing creativity leaders should consider new approach: managing for creativity \citep{amabile2008creativity}. According to \citet{isaksen1983toward}, in order to support employee's creativity, leaders should focus on creating and maintaining and environment of supportive empathy, respect, warmth, concreteness, genuiness, trust and flexibility. These factors have been combined to general and task-specific efficacy needs \citep{mumford1988creativity}. Furthermore, through providing enough processing time for creating novel solutions is likely to enhance creative behaviour of employees \citep{isaksen1983toward}. As creativity refers to finding novel solutions and generating understanding of problems at hand, leaders could facilitate the process of resource allocation, feedback and task management in order to support employee's creative process \citep{mumford1988creativity}. 

Leader alone is not able to boost creative solutions in employees: it is also a matter of personal characteristics, previous knowledge of the problem at hand and expertise in the field \citep{mumford1988creativity}; \citep{redmond1993putting}. Thus, in order to achieve novel solutions and fresh ideas, leaders may seek employees who have great knowledge and expertise of problem at hand or provide employees education and possibilities to develop their problem construction skills and furthermore encourage approaching problems from various perspectives.  \citep{redmond1993putting} 

Furthermore, supporting employee's feeling of self-efficacy is likely to improve creative skills of an employee \citep{redmond1993putting}, and can be done through giving positive and realistic feedback, allowing adequate resources and physical support, clarifying task assignments, providing development support for employees, and assigning employees to appropriate tasks \citep{hennessey19881}. However, often acknowledging employee's skills, potential and accomplishments is likely to push an employee to the track of creativity \citep{redmond1993putting}. 

Should also be noted that depending on the job, different level and amount of creativity is required. Certain jobs that are highly involved with novel solutions urges for creativity as major breakthrough and innovative ideas, whereas more routine and repetitive jobs such as assembly line work requires creativity in developing the job practicalities. \citep{shalley2004leaders} 

Studies show employees who consider and believe creativity as valued outcome are more willing to generate ideas, experiment, communicate openly with others about ideas and through this, overall, their behaviour will eventually lead to creative outcomes \citep{shalley2004leaders}. Accordingly, \citet{csikszentmihalyi199916} presents the belief and feeling an employee has on the capabilities, pressure, resources and sociotechnical system of work environment affects highly on the success of creativity. Furthermore, pre-set obstacle, such as deadline, assists in focusing individual's attention to an urgent problem at hand, and has been noticed to stimulate creativity \citep{andriopoulos2000enhancing}. As employee who has the feeling of autonomy performs better, setting a deadline is not likely to threaten that autonomy, whereas showing someone how to meet that deadline would do \citep{mumford2002leading}.

Creative work is resource intensive where risk is involved \citep{mumford2002leading}. It is demanding and time-consuming\citep{mumford2002leading} and requires attention over long periods of time involving high level of ambiguity and stress \citep{kasof1997creativity}.Thus, organizational environment plays a major role in employees' creative skills, and such stifling factors may be positive challenge at work, encouragement from organisational level, support from work group as well as supervisory encouragement. Furthermore, organisational impediments can lead to decreased level of creativity. \citep{amabile1998kill} Hence, leadership has a great role in ensuring that the climate and culture, structure and practises of work and work environment together with human resource practices are supportive for creative endeavours to occur \citep{shalley2004leaders,oldham1996employee,mumford2002leading}.

Problem finding and construction, making connections and evaluating ideas are important for creativity \citep{mumford2002leading,vincent2002divergent}. Thus, when improving individuals possibilities to multiple alternatives, related ideas and example solutions, they tend to make more connections leading to creative actions \citep{amabile1996assessing}. 
