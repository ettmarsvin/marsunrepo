\chapter{Change and learning in an organization}

Current and future business environment requires continuous innovation from organisations, meaning deploying the collective knowledge, skills and creative efforts of their employees \citep{dess2001changing}(Dess and Picken, 2000). 

"As budgets are squeezed tighter and margins
of profit grow smaller, ideas are a precious
commodity and employees who produce them
become sought-after resources. Within such a
competitive business environment
companies nowadays increasingly strive to
become innovative organisations." \citep{andriopoulos2000enhancing}

"Change. The rapid technological
advancements and the fierce competition
for market share have contributed to the
unprecedented increasing pace of change.
Therefore, organisations should be ready
to rearrange their resources to meet the
new demands." \citep{andriopoulos2000enhancing}

Wide access to the information has put tremendous pressure on today's business and companies to increase their efficiency and effectiveness. Concurrently, demand for creative endeavours has rose in order to improve and develop products and processes. (Andriopoulos and Lowe, 2000; Cummings and Oldham, 1996)

Today, economy is driven by innovation and innovativeness, requiring new understanding and abilities to generate great ideas in order to survive in every business and business level. innovativeness calls for creativity, which again calls for new managing skills of leaders. In contrast to many leaders beliefs, creativity and creative individuals can be managed and encouraged. Amabile and Khaire (2008) \citet{amabile2008creativity} 

Although currently creativity and creative processes of an individual at work are rather well recognised and essential, even more focus should be put on organisations' ability to mobilise creative actions of employees to create novel, socially valued products or services and more efficient ways of working (Mumford and Gustafson, 1988). In other words, creative actions of an employee are not worthwhile to an organisation when not coordinated or harnessed to yield organisational-level outcomes (Jung et al. 2003) 

"Increasingly, creativity has become valued across a variety of tasks, occupations, and industries. In today's fast-paced dynamic work environment, managers continue to realize that to remain competitive they need their employees to be actively involved in their work and trying to generate novel and appropriate products, processes, and approaches. Although the level of creativity required and the importance of creativity can differ depending on the tasks or job in question, most managers would agree that there is room, in almost every job, for employees to be more creative. Further, because individual creativity provides the foundation for organizational creativity and innovation (Amabile, 1988), and these have been linked to firm performance and survival (Nystrom, 1990), it is important, if not critical, that employees are creative in their work." (Shalley and Gilson 2004) 

According to Schein (2010), concept of culture refers to and helps to explain some seemingly incomprehensible and irrational aspects of what is going on in groups, organisations and other kinds of social units, that share history. Schein divides culture into three levels: artefacts (visible and feelable structures and processes and observed behaviour), espoused beliefs and values (ideals, goals, aspirations, ideologies and rationalisations) and basic underlying assumptions (unconscious, taking-for-granted beliefs and values). Climate of the group should not be mixed with culture of the group, it should rather be considered among artefacts. However, essential point of view Schein (2010) provides is how culture in organisation or group level is easy to observe yet very difficult to decipher. Put in other words: researchers are able to observe and make remarks on what they see and feel, yet they are unable to reconstruct the deeper meaning of those observations to the group. Cultural analysis and understanding of dynamics of a group should begin in observing and asking members the norms, values and rules that shape practicalities of work in day-to-day level.

\section{Learning as competitive advantage and way to survive}

Understanding change analytically in the turbulent world appears challenging. Change being hectic and fast calls for different skills and strategy than before. Only when change is understood can it be managed, and in order to survive new perspective and understanding towards change is required from an organisation. 

In the changing environment tolerance for uncertainty is needed, and while future can not be predicted, forecasting is a usable method in order to cope with the anxiousness resulting from the uncertainty. Furthermore, even more emphasis should be put on ability to learn and adapt to changes. 

"In this study, my focus on learning behavior and its accom- 
panying risk made the interpersonal context especially sa- 
lient; however, the need for learning in work teams is likely 
to become increasingly critical as organizational change and 
complexity intensify. Fast-paced work environments require 
learning behavior to make sense of what is happening as 
well as to take action. With the promise of more uncertainty, 
more change, and less job security in future organizations, 
teams are in a position to provide an important source of 
psychological safety for individuals at work. The need to ask 
questions, seek help, and tolerate mistakes in the face of 
uncertainty-while team members and other colleagues 
watch-is probably more prevalent in companies today than 
in those in which earlier team studies were conducted" (Edmondson 1999)


"Therefore, managers need to ensure that employees have enough time to be creative, which can be especially difficult in today's fast-paced, rapidly changing world." (Shalley and Gilson 2004) 

\section{Change in organisational level}
Hammer and Champy (1993) have summarised aspects of change in organisational environment, beginning from the change in organisational structure; from functional departments to process teams. Work tasks change from simple and detailed tasks to multi-dimensional knowledge work while employees are becoming more autonomous instead of strict control. Furthermore, instead of educating, focus is in the learning of an employee, and evaluation of work will change from operations to outcomes. Knowledge and capability are preferred over single performance and values change to more productive behaviour than over-protective. Superiors turn from leaders of the work to coaches and hierarchical organisational structures turn lower while managers focus on leadership instead of task management. 

According to Arie de Geus (1997) the only one can maintain company's competitive advantage is to make sure the company is able to learn faster than rivals. Generally organisations are considered as machines, yet recently more emphasis has been put on organisations as living organisms. When considered as machine, organisational model is mechanic and simple, which purpose is to gain profit. Whereas, organisation as a living organism is a whole-systemic model, and organisations are considered as place which has deeper, permanent meaning offering people the opportunity to grow and fulfil themselves while earning money.

Liable vision of the future focuses on the latter perspective of organisations, where learning and renewal form the essence of being. 

According to Huy and Minztberg (2003), organisations learn best through small experiments and trying out new things, and the closer and more related experimentations are to customers and customer interfaces, the more can be learned. 

\section{Organizational learning}

"Organizational learning is presented in the literature in two 
different ways: some discuss learning as an outcome; others 
focus on a process they define as learning. For example, 
Levitt and March (1 988: 320) conceptualized organizational 
learning as the outcome of a process of organizations "en- 
coding inferences from history into routines that guide be- 
havior"; in contrast, Argyris and Sch6n (1978) defined learn- 
ing as a process of detecting and correcting error. In this 
paper I join the latter tradition in treating learning as a pro- 
cess and attempt to articulate the behaviors through which 
such outcomes as adaptation to change, greater understand- 
ing, or improved performance in teams can be achieved. For 
clarity, I use the term "learning behavior" to avoid confusion 
with the notion of learning outcomes. " (Edmondson 1999)

"The conceptualization of learning as a process has roots in 
the work of educational philosopher John Dewey, whose 
writing on inquiry and reflection (e.g., Dewey, 1956) has had 
considerable influence on subsequent learning theories (e.g., 
Kolb, 1984; Schbn, 1983). Dewey (1956) described learning 
as an iterative process of designing, carrying out, reflecting 
upon, and modifying actions, in contrast to what he saw as 
the human tendency to rely excessively on habitual or auto- 
matic behavior. Similarly, I conceptualize learning at the 
group level of analysis as an ongoing process of reflection 
and action, characterized by asking questions, seeking feed- 
back, experimenting, reflecting on results, and discussing 
errors or unexpected outcomes of actions. For a team to dis- 
cover gaps in its plans and make changes accordingly, team 
members must test assumptions and discuss differences of 
opinion openly rather than privately or outside the group. I 
refer to this set of activities as learning behavior, as it is 
through them that learning is enacted at the group level." (Edmonsdon 1999)

"This conceptualization is consistent with a definition of group 
learning proposed recently by Argote et al. (2001) as both processes and outcomes of group interaction 
activities through which individuals acquire, share, and com- 
bine knowledge, but it focuses on the processes and leaves 
outcomes of these processes to be investigated separately. " (Edmondson 1999)

"The management literature encompasses related discussions 
of learning, for example, learning as dependent on attention 
to feedback (Schon, 1983), experimentation (Henderson and 
Clark, 1990), and discussion of failure (Sitkin, 1992)" 

"Similarly, because errors provide a source of information about performance by revealing that something did not work as planned, the ability to discuss them productively has been associated with organizational effectiveness (Michael, 1976; Sitkin, 1992; Schein, 2010) (Edmondson)

"The construct has roots in early research on organiza- 
tional change, in which Schein and Bennis (1965) discussed 
the need to create psychological safety for individuals if they 
are to feel secure and capable of changing. Team psycho- 
logical safety is not the same as group cohesiveness, as re- 
search has shown that cohesiveness can reduce willingness 
to disagree and challenge others' views, such as in the phe- 
nomenon of groupthink (Janis, 1982), implying a lack of inter- 
personal risk taking. The term is meant to suggest neither a 
careless sense of permissiveness, nor an unrelentingly posi- 
tive affect but, rather, a sense of confidence that the team 
will not embarrass, reject, or punish someone for speaking 
up. This confidence stems from mutual respect and trust 
among team members. " (Edmondson 1999)

"The importance of trust in groups and organizations has long 
been noted by researchers (e.g., Golembiewski and Mc- 
Conkie, 1975; Kramer, 1999). Trust is defined as the expec- 
tation that others' future actions will be favorable to one's 
interests, such that one is willing to be vulnerable to those 
actions (Mayer, Davis, and Schoorman, 1995; Robinson, 
1996). Team psychological safety involves but goes beyond 
interpersonal trust; it describes a team climate characterized 
by interpersonal trust and mutual respect in which people 
are comfortable being themselves. "

Learning is essential part of developing and innovation; according to \citet{buijs2007innovation} all innovation processes are processes for organisational learning.

Through the understanding of organisational learning real growth and support for learning can be offered. Learning process needs to be understood all from organisational, team and individual perspective. 

Organizational learning is approached from two different perspectives in literature. On the one hand, learning is considered as an outcome, and on the other it is considered as a process \citep{edmondson1999psychological}. In the first perspective organizational learning is referred to be an outcome of a process of organisations "encoding interferences from history into routines that guide behaviour \citep{levitt1988organizational}. Whereas process perspective define learning as a process of continuous trial and error. Further definitions of learning are presented in chapter zz. In this thesis, learning is considered as the latter tradition of learning, which allows the growth and improved performance of individuals and organisations.

Organizational learning differs from individual or team learning. Organizational learning occurs through the shared knowledge, insights and approaches of the employees of an organisation. Secondly, organisational learning is based on prior knowledge and experience, the memory of organisation, which consists of the ways of working, processes and instructions of an organisation. Even though individual and team learning are highly related to organisational learning, it is not the sum of the previously mentioned. \citep{sydanmaanlakka2007}

Various definitions of learning organisations have been presented. One of the most famous definitions is from \citet{senge1991fifth} , who describes learning organisation as follows: "Learning organisation is an organisation, where people are able to constantly develop and achieve intended results; where new ways of thinking are born and where people share goals and learn together."

"Learning, analysis, imitation, regeneration, and technological change are major
components of any effort to improve organizational performance and strengthen
competitive advantage. Each involves adaptation and a delicate trade-off between
exploration and exploitation. The present argument has been that these trade-offs are
affected by their contexts of distributed costs and benefits and ecological interaction.
The essence of exploitation is the refinement and extension of existing competences,
technologies, and paradigms. Its retums are positive, proximate, and predictable. The
essence of exploration is experimentation with new altematives. Its returns are
uncertain, distant, and often negative. Thus, the distance in time and space between
the locus of leaming and the locus for the realization of returns is generally greater in
the case of exploration than in the case of exploitation, as is the uncertainty." \citep{march1991exploration}

Organizational structures are also likely to enhance or hinder creativity in organisational, team or individual levels \citep{shalley2004leaders}. 

According to Sternberg et al. (1997) a company can enhance its creative skills by focusing on six resources: knowledge, intellectual abilities, thinking styles, motivation, personality and environment. Sternberg et al. (1997) argues that too much information of may hinder change and be seen as rigidity in thinking. Thus, one should not over-weight the criticism of senior people in an organisation, and at least consider the chance for rigid thinking and intolerance for change. 
In addition, needs to be noted and understood that employees' thinking styles are shaped through what is rewarded meaning, that if organisational environment rewards well-behaving and instruction-following thinking style and action, employees tend to implement their style to that. We are urged to adapt to organisational style and fit in, and when this is not possible, people tend to leave. 

General leadership styles and practices are set for industrial management, yet at present the focus should be on leading the people, and the focus of leadership needs to change from authoritarian style to increased autonomy and trust. As \citet{mumford2002leading} state, "organisations may now need jazz group leaders rather than orchestra directors". 

\section{Team performance and learning}

In order to create new value and competitive advantage in rapidly changing and uncertain organisational environments, new managerial imperative is growing focusing on teams. Thus, supporting teams in their work and understanding the aspects of team learning is required \citep{edmondson1999psychological}. Recent studies has moved the focus from individual learning to team learning. 

Work team refers to small group of people that exist within the context of a larger organisation, members share understanding of being member of the team and its tasks, responsibility for product or service team is working on \citep{hackman1987design}; \citep{alderfer1983intergroup} as well as its performance \citep{edmondson1999psychological}. Additionally, team members have supplementary knowledge and abilities compared to each other, and they share a goal, targets and way of working and approach \citep{edmondson1999psychological}. According to  \citet{katzenbach1993wisdom} great team performance consists of continuos work of shaping a common purpose, agreeing on performance goals, defining a common working approach, developing high level complementary skills and being transparent on the results. He emphasises that through disciplined action groups transform to teams and argues how demanding schedules, long-standing habits and unwarranted assumptions tend to threaten team efficiency and performance.
 
In previous research, structural and design-related factors have been combined to have influence on work teams's effectiveness and team performance. Well-designed tasks and goals, suitable and functional team composition, as well as physical environment and practices ensuring transparent communication and information exchange, sufficient materials, resources and motivating rewards all affect team efficiency. \citep{hackman1987design}; \citep{goodman1988groups}; \citep{campion1993relations}. Along with these factors, leader behaviour plays a major role in enhancing team effectiveness, and can be facilitated for instance through coaching and setting directions to employees (Hackman 1987) \citep{hackman1987design}. This perspective explains teams effectiveness through organisation and team structures, whereas organisational learning research puts emphasis on cognitive and interpersonal variables when explaining effectiveness in teams and individuals \citep{edmondson1999psychological}. For instance, \citet{argyris1993knowledge} has argued how individual's negative beliefs about communication and interaction may inhibit learning behaviour and lead to ineffective working in an organisation. 

In addition, in order to function team needs a clear purpose and vision what makes it a team and why it exists. Teams get energy from significant performance challenges regardless of where they are in the organisation. Set of shared, demanding performance goals usually form a team, and personal chemistry or willingness to form a team may boost that. Thus, in order to receive great results teams should focus on performance regardless of the organizational hierarchy or what team does. Thus, team performance may exceed the results of what could be achieved if employees were acting alone as individuals without the team effort. \citep{katzenbach1993wisdom}

\citet{edmondson1999psychological} have studied factors that affect and influence learning behaviour in teams by studying in which conditions and to what extent learning occurs naturally. Learning behaviour of teams refers to activities that team members carry out and through which team is able to obtain, adapt and reflect data and outcomes of actions which further shapes and improves team behaviour. Such activities consist of reflection and improvement-aiming factors such as asking for feedback, transparent information sharing, asking for help, admitting and discussing about failures and errors as well as experimenting. Through such activities teams may observe changes in environment, customer requirements and improve collective understanding. In addition, team's ability to discover and react to unexpected situations and consequences of their actions is likely to improve through learning behaviour.  Consequently, compared to low-learning teams that tend to get stuck and be unable to solve problems, teams who master in learning are greater in confronting difficult situation and improve their work. \citep{edmondson1999psychological}

The composition of the team matters. Studies have shown how team performance, especially related to innovation, is improved when team consists of individuals with various and different set of skills and characteristics\citep{buijs2007innovation}. Homogeneity in teams easily leads to groupthink, routine work and repeating traditional daily practices, while even one or two different individuals can stimulate the innovativeness of a team, and actually, the outcasts and those who stand out from the group are required in order to think outside the box, challenge the status quo and show alternative solutions and ideas that would be missing without the participations of these individuals. \citep{sternberg1997creativity}

Question of team composition comes relevant especially when forming teams for innovation. According to\citet{buijs2007innovation} innovation teams are the heart and the engine of innovation process and essential for the rest of the organisation to accept changes and innovation results. As he suggests right people in the team is the premise for innovation, all the members should be chosen carefully starting from the leader. Furthermore, the leader should be allowed to affect on the formation of the rest of the team in order to ensure positive base for teamwork and innovation process. Accordingly, team membership should be based on voluntary. \citep{buijs2007innovation}

\subsection{Psychological safety in teams}

As in changing and uncertain environments the importance of teams have been recognized, pressure on managers to understand and enhance team efficiency, work and learning has increased. Fast-pace environment requires organisations to enhance the ability of teams to learn and create environment where learning can occurs safely. While uncertainty affects all working life, change is faster and job security lower, psychological safety for individuals at work can be increased through great teams and teamwork. Employees should be engouraged to ask questions, seek for help and tolerate mistakes and uncertainty. \citep{edmondson1999psychological}

Team psychological safety refers to Amy Edmondson's concept of team members shared belief team being safe for interpersonal risk-taking. Together with team efficacy these have great affect on team performance and learning in an organisational team work. \citep{edmondson1999psychological} integrative perspective suggests that team performance and outcomes can be shaped through both team structure and shared beliefs, in contrast to previous studies that separate structural and interpersonal factors from each other. For instance, employee's willingness to take interpersonal risks depends highly on the experience of team safety and person's beliefs how others will respond in ideas or situations involving uncertainty. Team psychological safety refers to interpersonal trust among team but beyond that mutual respect and caring. \citep{edmondson1999psychological}

Psychological safety serves as a mechanism that assists in explaining how structural and interpersonal characteristics both have effects on learning and performance in teams \citep{edmondson1999psychological}. Psychological safety can be boosted for instance through structural factors such as context support and team leader coaching affecting behavioural and performance outcomes \citep{hackman1987design};  \citep{edmondson1999psychological}. Furthermore, climate of safety and supportivenesss encourages employees to seek for feedback and ask for help in addition to admit and reflect mistakes. \citep{edmondson1999psychological}

Communication about ideas among team has been widely recognized being related to idea generation, creativity and innovation (e.g.\citep{robinson1997coprorate};\citep{mumford2002social};\citep{monge1992communication}\citep{amabile1996assessing}). Organizational structures can influence in many ways creativity of a team and individual. For instance, by promoting open communication, idea and ongoing information exchange with internal and external team members as well as encouraging information seeking from different perspectives and sources is likely to enhance creativity (e.g. \citep{ancona1992demogpraphy}\citep{dougherty1996sustained}). Accorgind to \citep{staw1989tradeoff} social influence of others plays a major role for individuals' beliefs; attitudes towards job, for instance, rise from the social labelling of work by others.  Also \citep{salancik1978social} argue the essential role opinions of others may have on individual: individual's perception of her work and organisation can be greatly influenced by opinions of others. Additionally, team member's collective view of support they get from their leader has been related to the team's creative endeavours and success in them (e.g., \citep{amabile1998kill} \citep{amabile1996assessing}). 
 
Thus, as individuals oftentimes requires support and input from several invidivuals who help to challenge ideas in constructive ways, teams are essential in generating and implementing ideas \citep{mumford2002social}. Stimulating those constructive individuals for creative actions may be valuable \citep{robinson1997corporate}. In addition, including team members in ideation assists in idea implementation and through participation new ideas are not that likely to be rejected or abandoned \citep{agrell1994team}.

Leaders have a great role enabling creative behaviour in teams and individuals, yet team members also influence essentially in others. Thus, by utilising various human resource practices leaders should create an environment where creativity is encouraged and supported. \citep{shalley2004leaders} Study of \citep{ancona1992demography} argue how changing the structure of teams is not sufficient and does not lead to improved performance, rather the leader and the team should find ways to foster positive effects of the team processes and reduce the negative ones. At team level this may mean focus on enhancing negotiation, problem-solving and conflict resolution skills while at organisational level leader should protect the team from external political pressures and reward the team from performance outcome instead of functional ones. \citep{ancona1992demography}

\section{Individual learning}

Various perspectives and definitions for learning has been studied, presented, analysed and utilised in order to understand individual's process of adapting new information and skills. Experiential learning theory refers to learning as a process of knowledge-creation through experiences while experiential learning process stands as a way to describe the central process of human adaptation to the social and physical environment - a holistic adaptation process that provides bridges across life situations and underlaying the lifelong process of learning. \citep{kolb1984experiential}. Also\citet{jung1923psychological} argues how learning involves concept of human being as a whole - from feeling and thinking to perceiving and behaving.

However, everyday problem-solving and immediate reactions to situations at hand are oftentimes related to performing instead of learning. Furthermore, long-term adaptations to our previous experiences and beliefs is mainly considered as developing, not learning. Yet, when talking about development and developing in individual, team or organisational level, the question highly concerns and is related to learning. \citep{kolb1984experiential}

Experiential learning theory of \citet{kolb1984experiential} consists of four elements: experience, perception, cognition and behaviour. Immediate experience forms a basis for reflection and observation, following assimilation to a theory from which new implications for action are deducted. In order to create new experiences, these implications serve as guides. Overall, experience of an individual is a focal point of learning, and giving personal meaning to abstract concepts, which can be afterwards shared with others. Furthermore, receiving feedback is considered essential in this approach for learning, as it serves a continuous process for goal-oriented action following evaluation of that action. Feedback can thus boost effective, goal-oriented learning process. \citep{kolb1984experiential}

Continuing with the model of \citet{kolb1984experiential}, instead of conceiving learning in terms of outcomes, it should rather be conceived as a process. Ideas are not fixed and and immutable elements of thoughts, but can be formed and re-formed through experience. Furthermore, bringing the experiential learning into educational implications, all learning can be considered as relearning. Thus, all learning situations should take into account people arriving from all different experiential backgrounds to what they build their new experiences and knowledge on. This partly explains very likely resistance to new ideas, as when new information and experiences are in contradiction to old beliefs and experiences, new ideas and information is more difficult to adapt. In the education process learner's old beliefs and theories should be brought out, examined and tested, following integration of the new models and refined ideas into learner's belief systems. \citep{kolb1984experiential} (GAP BETWEEN IDEA AND EXPE?)

\citet{kolb1984experiential} presents Piaget's interactive process approach to learning, according to which individual learning and adaptation of new ideas occurs through integration or substitution. Integration leads to stronger part of learner's conception of the world, whereas substitution requires real questioning of previous conceptions, and thus might take longer for the learner to adopt. Learning is a mutual process between accommodation of concepts or schemas to experiences around us and assimilation of events and experiences into existing concepts and schemas. This intelligent adaptation, learning, results from the tension between accommodation and assimilation. Through this tension growth and higher-level cognitive functioning occurs. According to \citet{kolb1984experiential}, learning is a process filled with tension and conflict, and new knowledge, skills and attitudes are achieved through experiential learning, which consists of four modes and required abilities of learners: concrete experience abilities, reflective observation abilities, abstract conceptualisation abilities and active experimentation. First of all, individuals must openly involve themselves in new experiences, reflect and observe them from various perspectives, create concepts that can be integrated into more abstract theories as well as they need to be able to use these reflections and theories in active daily decision-making and problem-solving. 

Understanding of different learning styles and modes assists in supporting individuals in learning and problem-solving. 

Meaning of environment in learning should be emphasised. Learning concerns of transaction between an individual and the environment, learning does not happen only inside of individual's thoughts, experiences and processes but is dependent on the real world environment. 

\chapter{Innovation to the rescue}
Why innovation and creativity should be a matter for an organisation? Studies have well established the positive relation between creativity and innovation skills of an organisation and organisational performance \citep{jung2003role,mumford2002leading}

Social innovation refers to the generation and implementation of novel ideas concerning people in demand to organise their interpersonal or social activities and interactions in new ways in order to achieve common goals. Results and products of social innovation, like other types of innovation, are likely to vary depending on the breadth and impact of the innovation. \citep{mumford1988creativity} \citet{mumford2002social} presents four factors affecting social innovation: active exchange of ideas and information in supportive climate, tangible and low-cost ideas that can be at the fewer guessed to be beneficial, support from upper level management, and effective communication through the innovation process in order to proceed from the idea to implementation.

Innovation can be contributed by encouraging idea generation, but also creating a climate of autonomy, offering intrinsic and extrinsic rewards and engaging employees with their work \citep{amabile1996assessing}; \citep{amabile1998kill}. Furthermore, characteristics associated with innovation are integration of work units, decentralisation of control and professionalisation are likely to effect innovation in a way that through these suitable environment for innovation, dynamic idea exchange and implementation is created \citep{mumford2002social}.

Managerial practices for technological innovations have been widely studied. According to \citet{quinn1985managing} the essence lays in accepting the chaos of development. In addition, large and successful companies and their leaders listen carefully their users' needs, develop according this customer demand, define clear goals and framework for the work, encourage teams to challenge the status quo and find alternative solutions while avoiding detailed and technical or marketing plans in the beginning. Instead, they focus on early prototyping and iteration. 

According to studies creativity and innovation in an organisation requires integrated organisational approach, right climate, appropriate incentives for innovators, and a systematic way and resources to transform an idea into an innovation.  In individual level, creativity and innovation calls for various skills, such as teamwork, communication, coaching, project management, learning and learning to learn, visioning, change management and leadership, and ability to develop these skills. Oftentimes, even though the climate and practices are right for generate innovations, problems are faced when attempting to manage the change process. \citep{roffe1999innovation}

Innovation and creativity are highly related, yet not the same thing: According to \citet{hennessey19881}, individual creativity stands for an essential building block for organisational innovation and also \citet{sethi2001cross} argue creativity being essential in new idea generation and design processes that aim for innovative solutions. Other studies also emphasise the role of creativity a first step in creating something novel, whereas innovation refers to the implementation phase of the novel ideas in individual, team or organisational level \citep{shalley2004leaders};\citep{amabile1996assessing};\citep{mumford1988creativity}. 

Social cohesion may inhibit innovativeness of the team and its individuals especially beyond a moderate level, while employees are more likely to settle on group think and traditional daily practices. However, according to the study of\citet{sethi2001cross} when a team shares superordinate identity, is encouraged to take risks, lets customer's requirements be heard, and actively lets senior management monitor the project, team is more likely to present innovative ideas and perform in innovative ways. According to this study, functional diversity does not effect on innovativeness, but team's superordinate identity can be strengthened by encouraging risk-taking and weakened by social cohesion.

\citet{mumford2002social} argues in his study of Ben Franklin's social innovations, that key factor in successful social innovation lays in fast demonstrating, which he also refers as experimenting. Thus, in order to drive for social innovations, opportunism and showmanship of an individual or team may be required. \citep{mumford2002social} Furthermore, according to \citet{monge1992communication} group communication is likely to increase innovation under some circumstances,  and also \citet{katzenbach1993wisdom} argues for culture of strong team performance. However, \citet{amabile2004leader} emphasise how ,ultimately, truly novel ideas raise from individuals, making them the ultimate source of any new idea or solution to a problem \citep{amabile2004leader}.

\section{Creativity, intrinsic motivation and everyday problem solving}

Creativity and innovation have gained wider acceptance as important factors creating value in organisational performance \citep{mumford2002leading}. Creativity and innovation have for instance been studied to have enhancing impact to organisations profit and growth \citep{nystrom1990organizational}. 

Intrinsic motivation has been attached to creative and innovation performance of an individual as intrinsically motivated individuals usually prefer novel solutions, challenging status quo and trying out new ways for solving a problem at hand.\citep{amabile2002creativity} Intrinsic motivation of an individual is one of the most powerful tools to creative action and non-traditional thinking \citep{amabile1996assessing,deciintrinsic,jung2001transformational}.

\citet{amabile1996assessing,amabile1998kill} defines creative thinking as a way how people approach problems at hand and come up with solutions. Creative thinking does not stand for intellectual capacity of an individual to create something new but rather as a combination of past experiences which creates expertise and the ability to apply creative thinking skills to these experiences and invent new solutions. 

Divergent thinking refers to individual's ability to find multiple alternative solutions and ideas to problems at hand, and has been related to serve as a key capacity affecting creative thinking \citep{guilford1967creativity}. Accordingly \citet{mumford1988creativity} emphasis that creative people consistently and with confident tend to seek for alternative solutions, even under uncertain conditions. Even though expertise and  intelligence have been related to problem solving, series of causal analyses carried out by \citet{vincent2002divergent} revealed unique effects divergent thinking had that were not attributed to intelligence and expertise. 

\citet{shalley2004leaders}, in turn, argue that through developing extensive set of skills, employees may learn to be more comfortable and confident in thinking from different perspectives, finding various alternative solutions, trying out novel things and seizing opportunities. According to \citet{hennessey19881} individual creativity requires ability to think creatively, generate alternatives, engage in divergent thinking and tolerate or suspend judgment. Through this perspective creativity can be considered as a skill that can be learned and strengthened. Understanding of individual's creativity and ways to influence and improve it gives managers guidelines when creating an environment and leadership that support organisational innovation \citep{redmond1993putting}.

Several case studies has showed that creativity insights emerge gradually through the network and actions of an creative individual. Study of creativity is a combination of two different disciplines and research approach: sociological and historiometric lenses study the conditions in which creative actions and processes are likely to occur, whereas neurobiological approach presents neural structures and processes that are active and associated with creative outcomes. \citep{gardner1988creativity}

Generation of novel, alternative solutions requires problem-finding skills \citep{runco1988problem}, which has been indicated to be one of the best predictors of creativity in 'real world' activities, when studied 91 elementary school students \citep{runco1990evaluating}. These findings suggest leaders, in order to enhance creativity of employees, to support learning of these skills for instance by facilitating problem-construction \citep{redmond1993putting}.

\citet{kasof1997creativity} argued in his study that breadth of attention affects on creative performance of an individual: wide spread of attention is usually related to creative ability. By breadth of attention Kasof refers to "number and range of stimuli attended to at any time." Breadth of attention being narrow, individuals are able to focus on narrow range of stimuli and are better at filtering redundant stimuli from awareness. However, those individuals with wide breadth of attention tend to be more aware of irrelevant or extraneous stimuli, these individuals are strongly affected by their environment and are highly arousable.\citep{kasof1997creativity}

Studies of creative characteristics of individuals has revealed factors such as wide interest in various fields, autonomy, belief of being creative and independence in decision-making \citep{shalley2004leaders}. Indeed, broad interest stands for a sign of intrinsic motivation, which is also widely related to both creativity and well-being of an individual and innovation (e.g. \citep{hennessey19881}; \citep{csikszentmihalyi199916}; \citep{gardner1988creativity}). Indeed, intrinsic motivation is claimed to be one of the most powerful tools to creative action and non-traditional thinking \citep{amabile1996assessing,deciintrinsic,jung2001transformational}. In their study  \citet{tierney1999examination}, found positive correlation between employee's level of enjoyment while working on a creative task at hand and the level of creativity.  

For students of creativity, there is no surprise in attaching self-efficacy to creative actions \citep{mumford1988creativity}, yet recently problem construction processes have been recognised and combined to everyday problem-solving and real-world creativity \citep{getzels1975problem}; \citep{runco1988problem}. According to study of  \citep{gardner1988creativity} correlation between creative problem solving and everyday problem solving exists: they seem to have the same roots in information processing skills. 

Without previous experience of the job routine and substance knowledge and expertise on the field creative endeavours are more rare. Even though has been argued how routine work and task familiarity is likely to lead very habitual performance \citep{ford1996theory}, knowing the status quo may provide opportunities for creative actions through reflecting and practicing skills and activities requires in the field. \citep{shalley2004leaders} Knowing the field and what has already been discovered assists in finding alternative, creative solutions \citep{andriopoulos2000enhancing}.Furthermore, creativity is not restricted to artistic occupations only; it is required in various professions in which tasks presented involve complex, ill-designed problems where novel solutions are needed and status quo challenged \citep{mumford1988creativity}. Indeed, idea implementation may require even more creativity than idea generation \citep{mumford2002leading}.

In their study \citet{redmond1993putting} found how leaders supporting employees problem-finding and problem construction led to more unique and novel solutions. Leaders encouraged employees to find alternative solutions, approach problems from different perspectives and overall supporting several alternative problem-solving strategies. In addition, study showed how through motivational mechanisms, such as self-set goals, involvement and commitment, problem construction may have positive influence on solution quality and originality. Thus, problem construction is likely to have its greatest impacts on performance when in the process employee is allowed to express his values, needs and interests \citep{redmond1993putting}. 

Instead of managing creativity leaders should consider new approach: managing for creativity \citep{amabile2008creativity}. According to \citet{isaksen1983toward}, in order to support employee's creativity, leaders should focus on creating and maintaining and environment of supportive empathy, respect, warmth, concreteness, genuiness, trust and flexibility. These factors have been combined to general and task-specific efficacy needs \citep{mumford1988creativity}. Furthermore, through providing enough processing time for creating novel solutions is likely to enhance creative behaviour of employees \citep{isaksen1983toward}. As creativity refers to finding novel solutions and generating understanding of problems at hand, leaders could facilitate the process of resource allocation, feedback and task management in order to support employee's creative process \citep{mumford1988creativity}. 

Leader alone is not able to boost creative solutions in employees: it is also a matter of personal characteristics, previous knowledge of the problem at hand and expertise in the field \citep{mumford1988creativity}; \citep{redmond1993putting}. Thus, in order to achieve novel solutions and fresh ideas, leaders may seek employees who have great knowledge and expertise of problem at hand or provide employees education and possibilities to develop their problem construction skills and furthermore encourage approaching problems from various perspectives.  \citep{redmond1993putting} 

Furthermore, supporting employee's feeling of self-efficacy is likely to improve creative skills of an employee \citep{redmond1993putting}, and can be done through giving positive and realistic feedback, allowing adequate resources and physical support, clarifying task assignments, providing development support for employees, and assigning employees to appropriate tasks \citep{hennessey19881}. However, often acknowledging employee's skills, potential and accomplishments is likely to push an employee to the track of creativity \citep{redmond1993putting}. 

Should also be noted that depending on the job, different level of creativity may be required. Certain jobs that are highly involved with novel solutions urges for creativity as major breakthrough and innovative ideas, whereas more routine and repetitive jobs such as assembly line work requires creativity in developing the job practicalities. \citep{shalley2004leaders} 

Employees who consider and believe creativity as valued outcome are more willing to generate ideas, experiment, communicate openly with others about ideas and through this, overall, their behaviour will eventually lead to creative outcomes. \citep{shalley2004leaders} Accordingly, \citet{csikszentmihalyi199916} presents the belief and feeling an employee has on the capabilities, pressure, resources and sociotechnical system of work environment affects highly on the success of creativity. Furthermore, pre-set obstacle, such as deadline, assists in focusing individual's attention to an urgent problem at hand, and has been noticed to stimulate creativity \citep{andriopoulos2000enhancing}. As employee who has the feeling of autonomy performs better, setting a deadline is not likely to threaten that autonomy, whereas showing someone how to meet that deadline would do \citep{mumford2002leading}.

Creative work is resource intensive where risk is involved \citep{mumford2002leading}. It is demanding and time-consuming\citep{mumford2002leading} and requires attention over long periods of time involving high level of ambiguity and stress \citep{kasof1997creativity}.

Thus, organizational environment plays a major role in employees' creative skills, and such stifling factors may be positive challenge at work, encouragement from organisational level, support from work group as well as supervisory encouragement. Furthermore, organisational impediments can lead to decreased level of creativity. \citep{amabile1998kill} Hence, leadership has a great role in ensuring that the climate and culture, structure and practises of work and work environment together with human resource practices are supportive for creative endeavours to occur \citep{shalley2004leaders}; \citep{oldham1996employee}; \citep{mumford2002leading}. 

\chapter{Experimentation as a method for innovation and learning}

One definition of experimenting considers it as personal trial and error process in which employees with their full potential are involved\citep{andriopoulos2000enhancing}. Experimenting serve as a great method when testing and validating abstract concept \citep{kolb1984experiential}.

Normal business consists of repetition, risk-avoidance and focusing on business outcomes \citep{buijs2007innovation}, while innovation requires novel solutions, thinking out of the box, risk-taking, breaking the rules, challenging the status quo and questioning the future \citep{burns1961management,kanter1984change,march1991exploration}

According to \citet{edmondson1999psychological}, learning behaviour consists of seeking feedback, sharing information, asking for help, talking about errors and experimenting. Thus, experimentation behaviour seems to relate to learning of an individual, teams and organisations and can be supported by supporting these factors. 

Actually, idea implementation may require even more creativity than idea generation \citep{mumford2002leading}, and according to \citet{vincent2002divergent} creative work consists creative and innovation processes. Creative processes comprises of initial idea generation, whereas innovation process goes beyond the activities underlying the implementation of those ideas.

According to \citet{buijs2007innovation}, innovation consists of coming up with novel ideas and implementing them. Ideating begins with exploring, developing and implementing the ideas, following introducing the ideas, which have turned into products or services, into the marketplace. Innovation process is a series of stages for processing the idea, and in the end of every stage the idea is reflected and evaluated before further processing. Evaluation points stands for usable tool for measuring the quality of idea but gives also understanding of how the evaluation process is going. In addition, while evaluating, team members also need to reflect the process and the idea, through which learning occurs. Also \citep{runco1994problem} emphasises how only after evaluation of ideas implementation can be discussed and performed and several studies show the essence of evaluation \citep{mumford2002leading,vincent2002divergent}. Useful questions in evaluation process could be "What went well?", "What can be improved?" and "What has been learned?". 

Learning is essential part of developing and innovation; according to \citet{buijs2007innovation} all innovation processes are processes for organisational learning. Also \citep{quinn1985managing} argues how especially from the management perspective major innovations should be considered as incremental and interactive learning processes that is driven by certain goal. 

Innovation process itself can be approached from several angles: first of all, content of the innovation has to be clear - whether the purpose is to innovate new products, manufacturing processes, ways of organising or ways of dealing with people. Secondly, psychological process of the innovation team has to be understood, essential being shared understanding, level of comfort with ambiguity and degree of trust between team members. Thirdly, creative process of the team, meaning idea producing process, needs to be understood and efficiently facilitated. Finally, the role of leading plays a major role, and together with playful attitude innovation process is likely to succeed. \citep{buijs2007innovation}

Several factors have been recognized to affect on organisational innovation, yet many researchers have stated leadership behaviour being one of the most important. \citep{jung2003role,amabile1998kill,jung2001transformational,mumford1988creativity} \citet{jung2003role} identify four hypotheses how top managers' leadership styles may affect both directly and indirectly their companies' ability to innovate. Indirectly here stands for instance leader's possibility to empower employees and build organisational climate optimal for innovation. The study shows a positive relation between transformational leadership style and empowerment as well as innovation-supporting organisational climate. 

As in todays' business world it has been widely recognised how creativity and innovation are essential for business growth, researchers have studied factors affecting creativity and innovation in organisations. \citet{amabile1998kill} has identified three factors being important for stimulating creative behaviour in individuals and organisations: individuals' intellectual capacity (creative thinking skills), expertise based on past experience and creativity-supporting work environment. Furthermore, \citet{oldham1996employee} consider creativity skills and characteristics of and individual as important, yet they add the importance of characteristics of organisational context such as job complexity, supportive supervision or controlling supervision. 

Problem finding and construction, making connections and evaluating ideas are important for creativity \citep{mumford2002leading,vincent2002divergent}. Thus, when improving individuals possibilities to multiple alternatives, related ideas and example solutions, they tend to make more connections leading to creative actions \citep{amabile1996assessing}. 

Several factors form the basis of creativity skills of an individual, such as personality, technical knowledge, expertise, motives, and the supervisor's feedback style. In group level factors form of task structure, communication styles and task autonomy, and finally in organisational level strategy, structure, culture, climate and available resources all affect how creative actions are encountered. \citep{jung2003role}

When dealing with novel solutions and challenging status quo, we are dealing with innovations. In order for company and its employees to be innovative, they need to take risks. Yet, at the same time usual management processes avoid risk-taking and focus on managing daily routine business. As \citet{quinn1985managing} stated it in his Harvard Business Review article: we love innovation and we urge for innovation, but we can tolerate it only if it is controllable and results everything remaining the same. \citep{quinn1985managing}

\citet{andriopoulos2000enhancing} define in their study a concept of perpetual challenging - a way to enhance creativity and innovation in an organisation. According to the concept adventuring occurs when goal is idea generation and through that process individuals are encouraged to face uncertainty in order to generate novel solutions. One tool for idea generation is scenario making, which purpose is to develop possible ways to tackle situation at hand. Through scenario making employees scans what is both known and not known about current problem or situation.  Experimenting process then consists of testing different scenarios generated in ideation phase and evaluating the outcomes in order to decide and develop the scenarios further to meet the needs of clients and industry. This calls for individuals skills to tolerate risks and uncertainties, as well as skills to constructively challenge and question colleagues ideas in order to use their full potential.

According to \citet{andriopoulos2000enhancing} facing and dealing with risk serves also as positive boost to creativity, as employees' learn new skills and strengthen their capabilities constantly and adapt new knowledge to already known. This, however, requires for safe environment which \citet{andriopoulos2000enhancing} refers as safety net: environment that tolerates failure. 
 
\chapter{Factors hindering innovation and experimentation}

Several factors may affect on the gap between idea and action in employees of an organisation. The phenomenon of threat of employees in organisations is widely studied and consensus is rising how threat effects on cognitive and behavioural flexibility and responsibility in reducing manner. \citep{argyris1982reasoning} \citep{edmondson1999psychological}

One essential factor is the beliefs, emotions and actions of an employee. An employee is likely to inhibit learning as a result of feeling the fear of being rejected, under pressure or feeling they are placing themselves at risk \citep{edmondson1999psychological}, or when facing the potential for embarrassment of threat, even though their transparency and honesty would be highly important for the behaviour of the team \citep{argyris1982reasoning}. This may occur in a situation where an employee should ask for help, yet is afraid of admitting he lacks abilities, skills or knowledge. \citep{edmondson1999psychological}. In addition, admitting mistakes, asking for help and seeking feedback are all relevant abilities in the recent organisational world, yet threatening for an individual's image of himself and his skills \citep{brown1990politeness}. 

Particularly when large new, complicated systems at hand, meaning of co-operation in production, development and communication rises exponentially. Especially in large organisations innovation can be inhibited by the errors increasing as a result of complexity of the system and inability to control, understand or make intelligent decisions. Challenging as it is for one department, faculty or company to survive on its own without communication and help of others in design, production and other business-related decisions, with management that takes the complex environment into account, the disastrous effects resulting from lack of communication can be lessen. \citep{quinn1985managing} Yet, at the same time, as a result of the difficulty of managing complex situations, innovation may denote finding the core, boiling things down and focusing on the most essential elements \citep{katz1978social}.

\citet{quinn1985managing} list several barriers to innovation, including intolerance of fanatics, short time horizons, accounting practices, excessive rationalism and bureaucracy and inappropriate incentives. \citet{hayes1982managing} supplements the list with concern of top management isolation, arguing how top management oftentimes has too little contact and understanding of the environment and conditions at factory floor or customer requirements for innovative solutions. Top managers who tend to be financially-driven and are not familiar nor have experience with current technology and its possibilities, may fear technological innovations and perceive them as too risky. Thus, more familiar traditions remain with ease. \citep{hayes1982managing} 

Furthermore, \citet{quinn1985managing} argues enthusiasm is not yet widely accepted and tolerated characteristic of employees and refers to them as entrepreneurial fanatics. Larger companies may perceive them as causing embarrassment by challenging status quo and causing troubles. 

Short time horizons require companies to stay in continuous stream of quarterly profits, oftentimes at the cost of long time benefits, that innovations demand. Especially large companies easily favour narrow-minded actions such as quick marketing fixes, cost cutting and acquisition strategies over systemic thinking and process, product or quality innovations. Accounting practices

Rather than managing the inevitable chaos of innovation productively, these managers soon drive out the very things that lead to innovation in order to prove their announced plans.Excessive bureaucracy. In the name of efficiency, bureaucratic structures require many approvals and cause delays at every turn. Experiments that a small company can perform in hours may take days or weeks in large organizations. The interactive feedback that fosters innovation is lost, important time windows can be missed, and real costs and risks rise for the corporation. Inappropriate incentives. Reward and control systems in most big companies are designed to minimize surprises. Yet innovation, by definition, is full of surprises. It often disrupts well-laid plans, accepted power patterns, and entrenched organizational behavior at high costs to many. Few large companies make millionaires of those who create such disruptions, however profitable the innovations may turn out to be. When control systems neither penalize opportunities missed nor reward risks taken, the results are predictable." \citep{quinn1985managing}

\chapter{Factors supporting innovation and experimenting} 

According to \citet{garvin2008yours} in learning organisation employees excel at creating, acquiring and transferring knowledge. They define building blocks for learning organisation: supportive learning environment, concrete learning processes and practices and leadership behaviour that reinforces learning. Building blocks can be considered and measured as independent components yet each of them vital to the whole. In order to improve long-term learning of an organisation, strengths and weaknesses of an organisation and its unit needs to be recognised. 

Experimenting requires safe and supportive environment. According to \citet{garvin2008yours} supportive learning environment consists of four characteristics: psychological safety, appreciation of differences, openness to new ideas and time for reflection. 

Likewise, \citet{mumford1988creativity} emphasise the meaning of environmental variables as a means to support employee's creativity by providing resources to stimulate fresh ideas of employees. Furthermore, strong positive relations between organisational environmental variables have been found; organisational encouragement as well as support for innovation and creativity from team improve employee's creativity \citep{amabile1996assessing}

Futhermore, \citet{amabile1998kill} suggests changes in organisational environment are likely to boost intrinsic motivation of an employee leading to increased creativity skills. Role of leaders and managers is essential; being a key person in organising group work and processes a leader may encourage employees to achieve shared goals \citep{amabile1998kill}. 

Also \citet{quinn1985managing} emphasis top executives role over particular management. Innovation is likely to occur when top executives encourage creative and innovative endeavours and create an atmosphere and value system that supports innovation. \citet{quinn1985managing} offers an explanation why it seems easier for engineer and scientific leaders to create atmosphere supporting innovation: understanding and psychological comfort are related to familiarity, and engineers, for instance, have wider understanding and knowledge of technology, which makes newer technological innovations easier to accept and adapt. 

Organizational and team structures and hierarchies affect on innovation and experimenting. Flat organisations and small project teams are foster innovation performance in a company. Smaller team handles communication and commitment better, while as few management layers as possible decreases the jeopardy of rejection. "Since it takes a chain of yesses and only one no to kill a project, jeopardy multiplies as management layers increase."\citep{quinn1985managing}

According to \citet{quinn1985managing}, fast multiple-idea prototyping leads to more innovative outcomes, offers essential information about ideas or product's quality, motivates employees, and helps the company and the team to cope with anxiety and uncertainty in development. Engaging lead customers in the interactive development process instead of market research seems to elucidate more relevant information about customer's demands, required changes and entry strategies. Thus, fast prototyping serves an essential way for learning from the iterative process. 
Market analysis, however, remain valuable when dealing with familiar products and productions, yet with radical innovations they may easily offer misleading information. 

A company desiring for innovation should allocate resources and define long-term goals and actions accordingly. Even though companies urge to invest most resources in current lines, sufficient resources should be allocated for long-term growth and innovation. This includes providing an environment strong enough to seize surprising opportunities and tolerate unforeseen threats in all organisational, technical and external relations levels. \citep{quinn1985managing} Level of uncertainty can be reduced through goal-setting and fast prototyping \citep{mumford2002leading}. 

\citet{mumford1988creativity} have studied the gap between an idea an action, and revealed it depending on various attributes related to individual and circumstances. As physical work environment affects on creativity, information sharing and innovation in an organisation, it should be designed to support the natural flow of traffic through the building so that informal conversations between different functional areas are enabled \citep{shalley2004leaders}. 

Learning of employees occurs when employees do not fear being rejected, ask naive questions, make mistakes or present viewpoint of minority. Psychologically safe environment enables employees comfortably to express their thoughts at work. Appreciation of differences is important, as opening minds for different ideas and world views increases both energy and motivation, brings out fresh thinking. Novel approaches are relevant for learning, thus employees should be encouraged in risk-taking and exploring and testing uncertain things. Lastly, through providing time for reflection learning in safe environment occurs. Instead of looking and judging by numbers of hours of work or results employees should be given enough time to reflect their work. Analytic and creative thinking are prevented under stress, heavy workload and too tight schedule. Under stress ability to recognise and react to problems and learn from experiences deteriorates. In supportive learning environment time for reflection is allowed. \citep{garvin2008yours} 

According to \citet{garvin2008yours} second building block of organisational learning, consists of concrete learning processes and practices. It includes experimentation, information collection, analysis, education and training and information transfer. Organizational learning can be supported through concrete steps and activities which are tested and further developed through experimentations. Furthermore, information and intelligence about customers as well as technological trends should be collected systematically and further analysed focusing on identifying problems and solving them. Training and education of new and established employees is an essential part of practices and processes. Finally, through transparent and meaningful knowledge sharing organisational learning can be enhanced, focus being on clear, well-defined and working communication systems that employees can easily relate to. Concrete processes together with efficient knowledge sharing methods ensure essential information being available fast and efficiently for employees who to use. \citep{garvin2008yours}

Creating and defining concrete learning processes and practices.
 
Thirdly, leadership behaviour should reinforce learning. Behavior of leaders is highly related to the performance of employees \citep{kim2014blue} and organisational learning \citep{garvin2008yours}. In order to encourage employees learning, leaders should prompt dialogue and debate, ask questions and listen to employees \citep{kim2014blue,garvin2008yours}. Leaders supporting new ideas and idea exchange has been related to enhancing creativity especially among those employees who showed disposition towards creativity \citep{oldham1996employee}.  

These three building blocks overlap to some degree and reinforce one another. For instance, leadership behaviour helps in creating supportive learning environment, which supports managers and employees in creating and defining concrete learning processes and practices. Furthermore, concrete processes support leaders behaviour in a way that fosters learning and through own example cultivates that behaviour to others. \citep{garvin2008yours}

Supportive leadership behaviour alone is not sufficient guarantee for organisational learning. \citet{garvin2008yours}emphasise how organisations are not monolithic and managers should sense differences in culture, department and units. In addition to cultural differences, learning requires clear and targeted processes and practices. Furthermore, learning should be considered as multidimensional, thus organisational forces should not be solely focused on a single area but to consider presented building blocks as a whole.

Likewise, according to \citet{mumford2002leading} creative leadership highlights three key elements: encouraging employee's idea generation, creating safe environment for ideas to emerge and improving idea promotion and implementation. By idea stimulation, education of various problem solving techniques, support for novel ideas, involving employees in developing ideas and allowing them freely pursue ideas, idea generation can be enhanced by the leader. Essential elements of safe environment include diverse teams, transparent and good communication, leader acting as a role model and being in charge of conflict management. In addition to idea generation, idea structuring phase consists of creating action or project frameworks so that employee's have as much autonomy to perform the task as needed. Idea promotion, in turn, refers to leaders task to transfer ideas to broader levels of an organisation, achieve support and assist with implementation of chosen ideas. Promotional activities to upper levels of organisation serve as a major way to insure sufficient resources and support for the idea implementation. \citep{mumford2002leading} 


Also \citep{amabile2004leader} emphasise in their componential theory on creativity the support of immediate supervisors as a way to enhance employee's creativity and intrinsic motivation. Supporting actions include being a role model, defining and setting appropriate goals, showing the work group support and confidence within the organisation, showing appreciation of individuals contributions to the project, focusing on efficient and good communication, offering valuable feedback, and listening openly novel ideas. Accordingly \citep{amabile2004leader} divide required behaviours of leaders for providing support into two categories: instrumental or task-oriented and socio-emotional or relationship-oriented actions.

Expertise 
- Expert performance and its affects on implementing ideas \citep{ericsson1994expert} (artsu l�ytyy)

\section{Psychological safety}

Interestingly, studies show how nominal groups perform remarkably better in ideation and brainstorming processes by producing greater amount of ideas than real groups. This may be due to the learnt practices and norms of a real work group, fear of failure that prevents free idea exchange and fear of evaluation and others judgement when suggesting creative solutions. \citep{jung2001transformational}

\citet{edmondson1999psychological} defines a concept of team psychological safety, that fosters learning behaviour in work teams by reducing the risks of embarrassment or threat and increasing mutual trust between team members. Oftentimes in work teams employees are not willing to tell their ideas or errors out loud as they are afraid of being labelled as incompetent. Thus, they prefer staying silent ignoring how it may lead to negative consequences for the team performance. When team members share feeling of respect and trust of others, and stay confident on other member's not using the errors against them, they are more likely to put more weight on the benefits of telling concerns out loud. When knowing that well-intentioned interpersonal risks are not punished is a shared belief of a team, team members are more likely to take proactive actions that foster learning leading to more effective performance. 

Even though building mutual trust may not lead to mutual respect and caring among team members, it is essential for creating psychologically safe environment and through building trust a foundation for further development of team psychological safety is built. \citep{edmondson1999psychological} 

\citet{edmondson1999psychological} lists factors affecting psychological safety in teams, including context support and team leader coaching. Context support refers for instance to access to information and resources needed. 

According to \citet{edmondson1999psychological} team psychological safety should be the first essential building block of learning behaviour in work teams. 

Accordingly, \citet{amabile1998kill} suggested that creative thinking can be encouraged by shaping organisational culture such that employees feel encouraged to tell their ideas out loud freely and without judging, increasing idea exchange and discussion about them. In addition, studies show how creative individuals may only produce more creative outputs than less creative individuals when the context is supporting and encouraging towards creativity \citep{oldham1996employee}.

\citep{mumford2002leading} argue organisational climate and culture being a collective social construction where the role of the leader on control and influence is remarkable. \citep{schein2010organizational} also presents a view where leaders communicated personal values and beliefs become essentially part of organisation's culture and climate. Furthermore,\citep{jung2001transformational} considers managers essential for shaping organisational culture, whether the concern is in developing, transforming or institutionalising. The way employees perceive their work environment created by their leaders, and especially the way they perceive the instrumental and socioemotional support both have influence on employees' creativity. \citep{oldham1996employee}

According to \citet{mumford1988creativity} through environmental variables employee's creativity can be fostered. New solutions may be achieved through problem solving and challenging the routine ways of thinking, and environment should be designed to encourage and facilitate these skills. Environmental factors may, furthermore, affect on employee's intrinsic motivation and willingness to generate novel ideas, when social and physical environments work as a source for support and resources in idea generation and implementation. 

\section{Leadership behaviour}

According to studies leaders have a strong direct impact on employee's behaviour and way of performing at workplace \citep{katz1978social,redmond1993putting}. As leaders play a major role in establishing, influencing and shaping organisational culture and climate through their communicated values and beliefs, they are able to shape the organisational culture into more innovative direction and foster creativity in an organisation \citep{jung2003role,schein2010organizational} for instance by nurturing organisational climate that supports creative efforts and learning \citep{yukl2002leadership}. Change from authority-based leadership to collaboration with employees has occurred in literature and in practice (\citep{amabile2008creativity,farson2002failuretolerantleader}). 

Leaders have a great influence on employees behaviour. \citet{avolio1988transformational} list several mechanisms through which leaders can affect employees behaviour. These include role modelling, goal definition, reward allocation, resource distribution, defining norms and values of the company, showing the way to interact as a group, condition employees' perceptions of work environment and being the lead decision maker on organisational structure and procedures. Studies also suggest leaders have a significant effect on employees' creativity \citep{hennessey19881}, and according to \citet{redmond1993putting} a leader can have an affect on employee's level of creativity through leadership behaviours such as problem construction, learning goals and feelings of self-efficacy. 

Actually, leadership behaviour is only recently recognized as essential part of enhancing creativity and innovation skills of employees \citep{mumford2002leading}. This may be due to our romantic perception of creative act, which defines creativity as an heroic act of an individual and leaders only being a hindrance to the creativity of an individual. Furthermore, conventional models of leadership are not likely to encourage employees to challenge the status quo but to achieve required goals.\citep{mumford2002leading} Current trend in research however shows leaders and their behaviour have great influence on the creativity and innovation ability of employees (eg. \citep{mumford2002leading,jung2001transformational,amabile1998kill})

Conventional leadership behaviour focuses on internal activities within the team, whereas innovative team leader needs various set of skills and approaches in order to encourage developing and growing of teams and individuals. For instance, according to \citet{barczak1989leadership} leaders of innovative teams utilise wide range of familiar and unfamiliar techniques in order to accomplish the team objectives, whereas leaders of operating teams use only a few familiar techniques. Even though in this study innovation teams were not studied, similar elements of developing by experimenting and encouragement for that may be recognised, when dealing with new tasks and developing something which result is uncertain. 

Leadership style has great impact on organisational innovation and creativity. Transformational leadership refers to leadership style and processes which emphasises longer-term and vision-based motivational processes \citep{bass1997full}. Furthermore, through offering an explanation of the importance and value of the work, leaders encourage their employees' to think beyond self-interest \citep{yukl2002leadership}. Leaders shape and define the goals and working context \citep{amabile1998kill, redmond1993putting}. Through a long-term vision (separated from short-term business outcomes, which usually focuses on quarterly profit), leader's are able to direct employee's efforts towards creativity and innovative work processes leading to likeminded outcomes\citep{amabile1996assessing}.

Leaders can affect employees' creativity and innovation skills both directly and indirectly \citep{jung2003role}. By stimulating employee's intrinsic motivation and higher level needs leaders are able to affect directly on employees' creativity \citep{tierney1999examination}, where indirect way may be through establishing a work environment where new ways of doing are encouraged and failure is not being punished \citep{amabile1996assessing}. Creating and supporting a reward-system that values creative performance, provides both intrinsic and extrinsic rewards for employee's efforts to learn new skills and to challenge status quo by experimenting new approaches, employees are constantly willing to engage in creative endeavours \citep{jung2001transformational,mumford1988creativity}.

Also Buijs (2007) \citet{buijs2007innovation}states how leaders dealing with uncertain and new innovations should stay certain about uncertainties and provide a safe environment and encourage employees to work on current task comfortably. Thus, high level of tolerance for dealing with different states of minds and various personal feelings is required from a leader. \citep{buijs2007innovation}

In order to encourage creativity and experimenting in teams, leaders should lead by example and act as role models. Leaders should consider their own behaviour and actions in a way that stimulates employees to new and innovative, creative approaches to problems. In addition, they can even request creative and innovative solutions form the team, which may lead to better results in creativity of individuals \citep{amabile2002creativity}. \citep{mumford2002leading,amabile2008creativity,waldman1990adding}

Big ideas do not hatch overnight and creative thinking requires time. Leaders should allow team members time to think creatively, as according to studies under pressure creativity actually falls into decline. Even though individuals may feel more creative, actually they are only working more and getting things done. According to this study, employees were clearly less creative while time pressure increased. \citep{amabile2002creativity} 

Furthremore, leaders can assist their employees by recognising times with high pressure, and allowing employees to focus on certain thing at a time, leaving the expectations of creativity and new ideas into the future moment, when time pressure has decreased. On the other hand, if creativity is required under stress, leader should transparently explain the importance and reasons behind the strict schedule and required goals. Thus an employee may relate to the problem at hand and engage better at his work. Indeed, helping people to understand the importance of work is essential especially under high time pressure. \citep{amabile2002creativity} 

Failure as a part of innovation and development process begins to be generally recognised and approved. Succeeding companies even thrive for failure in order to learn fast and find the best practices and business models. Through encouraging employees in risk-taking and making mistakes, leaders are likely to boost innovation. For instance, credit company Capital One conducts continually large amount of market experiments. They now most of the tests will not pay off, yet they also know how much can be learned about customers and markets from failed tests in early phase of development. Yet leaders fail in showing their employees the support and tools for failing fast and early enough. Failing in a personal matter remains a difficult subject, as failing never feels exceptionally great, and often employees still consider failed work as failing personally. \citep{farson2002failuretolerantleader}

Failure-tolerant leaders put effort on explaining to employees how important part failure is to the development process as a whole, and how failing actually refers to a point where surprising, failed outcomes are not reflected and further analysed in order to learn. Performing accordingly, admitting own failures and not chasing anyone to blame, failure-tolerant leaders encourage failure, lower the threshold and ease the fear of failing of employees. \citep{farson2002failuretolerantleader}

Naturally management need to take seriously issues about safety and health, yet most of the failures should be seen as opportunities for growth. Furthermore, failure-tolerant leaders treat success and failure similarly, analysing and reflecting the outcomes in order to grow the intellectual capital of the team, including experience, knowledge and creativity. Other characteristics of failure-tolerant leaders are being rather collaborative than controlling, listening carefully, seeing the bigger picture, asking questions and focusing on the development and future rather than blaming on mistakes. In addition, in order to gain empathy and trust among employees, leader should admit their own mistakes, as it shows self-confidence and honesty, assisting in forming closer ties with employees. Vulnerability and transparency play a major role in trustworthy relationship between leader and employees.  \citep{farson2002failuretolerantleader}

Through the green light given and their own example leaders can change the focus from success and failure into thinking in terms of learning and experience. \citep{farson2002failuretolerantleader}

\citet{amabile2008creativity} draw a poetical picture how leader cannot manage creativity, but manages for creativity. Furthermore, they suggest that culture that fosters creativity includes leadership that enables collaboration, enhances diversity, encourages ideation, maps the stages of creativity to different needs, accepts inability and utility of failure and motivates employees with intellectual challenges. According to \citet{sosik1999leadership} leaders should concentrate on vision of work and its outcomes that is meaningful and motivational enough to inspire employees. 

In experimentation process, employees need to contribute imagination, and this may require new kind of encouragement for creativity from the leaders. Much success rises from employee's own initiatives, which results from wide amount of autonomy at work. \citep{amabile2008creativity}

A culture of creativity can be fostered in an organisation through opening the organisation to diverse perspectives and openness to various ideas. This calls for safe environment for employees to share their thinking from different fields of expertise. Furthermore, encouraging passion and knowledge of an employee is likely to result in more creative action at work. \citep{amabile2008creativity}

Empowering employees is an essential tasks of leaders, through which a work environment is created where employees desire to seek innovative approaches to perform their work tasks \citep{jung2003role}. Transformational leaders encourage employees to participate in developing by highlighting the importance of cooperation, providing the opportunity to learn from shared experience and allowing employees to perform necessary actions in order to be more effective\citep{bass1990implications}. Furthermore, autonomy and freedom to perform essential tasks has major effects on organisational creativity, as individuals are more likely to produce creative work when having the feeling of personal control over how to approach given tasks \citep{amabile1996assessing}.

Yet, in order to maintain organisational innovation and risk-taking, autonomy given to an employee can not be in contradiction with fear of failure or discouragement towards challenging status quo or trying out novel solutions \citep{yukl2002leadership}. Thus, organisational climate has to support and encourage innovation \citep{mumford1988creativity} by valuing initiative and innovative approaches that support employees in risk-taking, accepting challenging assignments and stimulate intrinsic motivation towards work \citep{jung2003role}.

According to \citet{bass1997full} transformational leadership consists of four unique yet interrelated behavioural components: inspirational motivation (articulating long-term vision), intellectual stimulation (promoting creativity and innovation), idealised influenced (meaning charismatic role modelling) and individualized consideration referring to coaching and mentoring leadership style. 

Transformational leaders can build environments that support creative actions (\citep{sosik1998transformational,avolio1988transformational}). According to \citet{sosik1998transformational} key characteristic of transformational leader is the intellectual stimulation, which is likely to encourage creativity and divergent thinking leading to unconventional solutions to problems at hand. 

\citep{jung2001transformational} has studied how leadership style affects group's creativity and performance by comparing transactional and transformational leadership styles. Transformational leader refers to a leader who encourages divergent thinking and looking at problems from unconventional perspectives, while providing and explaining clearly defined goals and facilitating the innovation process of employees \citep{bass1990implications}. Furthermore, development of clear long-term vision and practises supporting the way to achieve it is essential characteristic of transformational leaders \citep{avolio1988transformational}. The relationship between transformational leader and an employee is active and emotionally attached \citep{avolio1988transformational} and through the strong attachment resulting from tight relationship leaders can better support employees in using their personal values and self-concepts in the way that employees can pursue higher level performance and fulfil personal needs through the work. This focus of transformational leadership on value alignment is likely to lead to the root of intrinsic motivation of an employee \citep{gardner1998charismatic}, which is considered as one of the key elements in creative thinking and innovation skills of an employee (eg. \citep{jung2001transformational,amabile1998kill,deciintrinsic}).

In contrast to transformational leadership, transactional leadership refers to focus on employees ability to fulfil and achieve clearly defined goals \citep{hollander1978leadership,house1971path} and successful goal achievement is rewarded \citep{waldman1990adding}. This exchange relationship between leader and employee is based on a contract of specified goals and emphasises on the process of achievement of objectives (Avolio and Bass 1988) but does not encourage employee's to develop their creativity and innovation skills \citep{jung2001transformational}. Instead, employees are rather motivated extrinsically to perform their job under transactional leader but not expected to question and change the status quo in creative ways \citep{amabile1998kill}.  

Few studies have been made linking the transformational leadership and positive outcomes of employees' creativity in organisational level and outcomes \citep{jung2003role}, even though several studies have been made revealing the positive relation between these factors. in their study \citet{jung2003role} draw this link clearer and suggest that while leaders define the context and goals of their employes, transformational leadership can be extrapolated to an organisational level.  

According to \citet{jung2003role} transformational leadership is positively related to organisational innovation, employee's perception of empowerment and support for innovation. Furthermore, the perception of empowerment and is positively related to organisational innovation, and when perception being strong,  the relationship between transformational leadership and organisational innovation tends to be stronger. Results of the study conducted on 32 Taiwanese companies suggest that through transformational leadership by top managers organisational innovation can be affected directly or indirectly, latter referring to creating an organisational culture in innovation, discussion, novel approaches and experimenting is encouraged. \citep{jung2003role}

As undertaking novel approaches to work oftentimes involves risk-concerned decision-making, employees should be offered decent level of guidance, goals and some measure of structure \citep{jung2003role}. Leader not taking an active role in supporting and guiding the work of his employees may lead to organisational units working at cross-purpose. Thus, leadership is about maintaining a balance between empowering employees and providing guidance and structure through setting goals and agenda. However, according to \citet{mumford2002leading} leaders' planning and guidance should focus on progress, projects on general level and implementation of the results of projects instead of focusing on offering detailed guidance on piece of work. 

Employees should feel being allowed to conduct experiments \citep{jung2003role}.

Study of \citet{sethi2001cross} showed how good interaction in a team and high level of commitment to the success of the team lead to more radical innovation abilities. In the study team members were highly encouraged to take risks, which lead to more motivated members in suggesting novel ideas from their perspectives. In addition, team members identified themselves strongly as part of the team, which again higher commitment level. \citep{sethi2001cross} 

However, lack of time and resources may serve as a hindrance to employee's willingness to take risks and perform experiments \citep{jung2003role}. Through leaders who allow their employees to participate in developing and ideating, reserve budget for it and set it as a part of performance standard, the hindrance for risk-taking may be lowered \citep{jung2003role}.

Under some circumstances, according to Monge et al. (1992) \citet{monge1992communication} group communication is likely to increase innovation. Thus, leaders should consider managing wide range of formal and informal meetings and facilitated discussions in order to create opportunities for ideation. Furthermore, innovation occurs over time and is a dynamic process. Leaders should be sensitive in which pace more managerial impact is needed, and in which pace of the process more freedom and autonomy should be allowed for employees. \citep{monge1992communication} 

Organizational leaders play a great role in establishing strong team performance culture. This can be achieved through addressing and demanding performance that meets the need of customers, employees and shareholders. Teams should not be fostered by the sake of the team only, rather should leaders clearly state how the team performance affects to customers and through that foster clearer performance ethics and cultures. In addition, even though people tend to have great sense of individualism, it does not have to bias the teamwork performance, as real teams find ways to support individual strengths and performance for shared goal. Furthermore, in order to team function properly and efficiently, discipline across the team and organisation is needed, focusing again on performance.  \citep{katzenbach1993wisdom}
 
Ambiguity is often perceived by individuals when lacking sufficient cues to structure a situation, and usually arises from novelty, complexity or unsolvability of situation at hand \citep{budner1962intolerance}. 
 
Leadership plays a major role in defining group goals, controlling resources and providing rewards through interactive leadership process, making leadership behaviour an essential environmental variable in stimulating creative behaviour as a means for achieving goals \citep{redmond1993putting}. \citet{katz1978social} even refer to role of the leader in a sense where leader defines by his example the reality of workplace; norms, practices and culture. According to \citet{barczak1989leadership} leader's task is also to provide clear focus for the work of employees. 

By defining organisational culture, climate and group norms leaders shape the way of working of employees. Through such role-modelling and mentoring process leaders also show employees in practise how tasks are performed. Employees, in turn, follow the example of leader in order to achieve high level of performance. \citep{redmong1993putting}
 
Role-modeling stands also as powerful tool for opening employee's eyes and attitudes to new perspectives, thinking 'out of the box' and adopting generative and exploratory thinking processes \citep{jung2003role,sternberg1997creativity}, influencing creativity of an employee \citep{shalley2004leaders}

Although different leadership styles and their effect on employee's creativity behaviour has not yet been studied widely, some studies show, how transformational leadership behaviour encourages employees look problems from different perspectives and thus widen their intellectual and creativity skills \citep{jung2001transformational,sosik1998transformational}. \citet{jung2001transformational} has studied the relation between leadership style and group creativity finding that transformational leadership is most likely to stimulate creative effort of employees. 
 
In his study, \citet{jung2001transformational} emphasises that transformational leadership skills can be practised in order to foster creativity and intellectual skills of employees and shape organisational culture. His study showed how transformational leadership; encouraging divergent thinking and solving problems at hand from unconventional perspectives, is likely to increase intrinsic motivation of employees leading to more creative problem solving and behaviour.  Through brainstorming activities that focus on non-traditional thinking and fantasising intellectual skills of employees can be enhanced \citep{sosik1998transformational}. Furthermore, \citet{jung2003role} argue that several aspects of leadership behaviour can be learned and practiced. Thus, organisations should foster and improve innovativeness by offering managers training and mentoring processes that develop transformational leadership. 
 
\citep{sosik1998transformational} furthermore suggested that anonymous ideating through nominal groups leads to better results and greater amount of ideas than brainstorming activities in real groups. When ideating in daily working groups, members may fear failing, being ashamed or measured by their performance. Overall, oftentimes it is way more difficult to take a different role and actions in group with familiar members and routines. \citep{jung2001transformational}

Innovation leaders, indeed, in managing innovation processes need to have several contradictory skills, roles and attitudes used smoothly during the day with employees; and they need to tolerate these competing and conflicting aspects within the team. Innovation team trusts their leader to be in charge and in control, yet allow them support, autonomy and enthusiasm.  At the same time, innovation leader should be few steps ahead thinking of uncertain future scenarios and support his innovation team in current step without showing doubts too strongly about team's work. \citep{buijs2007innovation} Resulting from this contradictory and challenging role of an innovation leader, according to \citet{buijs2007innovation} they should act and have characteristic of a controlled schizophrenic. 

As \citet{buijs2007innovation} argues, leaders who are to lead employees and work handling innovations need to understand the paradox and natural conflicts between routine processes (exploitation) in order to earn money in the present and the innovation processes (exploration) in order to earn money in the future. \citet{bujis2007innovation} four aspects for innovation which leader should be able to master all providing a secure environment for a team to perform in novel and creative ways. These consist of innovation process, psychological process of innovation team, creativity process, and leading and playing. 

Goals, however, should be kept broad, in order not to create undue oppositions to new ideas. Flexibility should be maintained by not defining intermediate steps in detail and by trying alternate options and routes. Identifying and solving problems at early phase fosters momentum, confidence and identity towards new approach. Furthermore, sufficient amount of information about the project and progress should be offered in order managers to follow and realise the work performed.  \citep{quinn1985managing}

Local leaders are in essential role in directing and evaluating work of employees, facilitating and allowing resources and information as well as encouraging employees to engage with the tasks and team members. \citep{amabile2004leader}

The approach of leaders oftentimes divides into task-oriented or relationship-oriented. Task-oriented leaders value performing the job, focusing on clarifying roles and responsibilities, monitoring work while managing time and resources. In turn, relationship-oriented leaders value socioemotional aspects of work through empathetic actions, showing consideration for employees, being friendly and supporting the team personally. Should be noted that in literature concerning leader behaviour, term support refers to relationship-oriented leadership behaviour, wheres in creativity literature same term refers to both task- and relationship-oriented behaviours and actions - all that are to foster creativity. In this thesis, latter and broader usage of term support is used. \citep{amabile2004leader}

According to the study of \citet{amabile2004leader}, leaders are likely to influence employees feelings, perceptions and performance as well as overall creativity do their own behaviour. By acting fairly, consulting with employees on essential decisions, offering emotional support and rewarding and recognising them for performing well leaders can enhance creativity of employees. In turn, leaders may play as hindrance for creativity by not offering support and clear task assignments, preventing autonomy of employees, treating employees unfairly and not trying to resolve important problems. 

According to componential theory of organisational creativity \citep{hennessey19881,amabile1996assessing}, employee's perception of the work environment influences individual and team creativity and emphasises the role of local leader support for creating an creativity supporting environment. 

In their study, \citet{shalley2004leaders} present how leaders should use human resource practices in order to develop work context which improves the creativity skills of employees. 

Organizational structures affect in the traditional roles of leadership as a means of direct responsibility given to employees. The trend of flatter organisations provides more autonomy to employees, whereas leaders' role transforms to more involved in external resource acquisition and managing the interfaces. \citep{shalley2004leaders}

Creative work environment is likely to be created through leaders who support and encourage employees, provide them autonomy in decision-making and everyday tasks, and communicate openly with employees \citep{oldham1996employee,tierney1999examination}. However, in addition to contextual factors and environment, studies show level of support, control and assist an employee needs depends on personal characteristics. Thus, leaders knowing and understanding their employees is essential in order to provide employees individual support needed. \citep{shaley2004}

 As \citet{garvin2008yours} bring out in their article, reasonable question to ask for this fresh leadership approach is, can managers actually be excited about being a facilitator of creative process, and where to find those managers who feel engaged and aspired to that role and want to do it? \citet{lingo2010nexus} has offered one perspective to this question in her study with production of music. She claims that producer is the one bringing it all together; it is actually hard leadership exercise, where people from different fields and teams need to work together for one production, where there are no clear rules for who is controlling the output nor yardstick how good or bad the production is. Through creating a shared purpose and common goal in production team, and while still letting "other apply their distinctive expertise", a producer actually operates at the centre of the storm without being at the focus of attention as well as aims for productivity without being over controlling. According to this example, glory comes from being able to help others to find and realise their unique talents at the same time with achieving a collective goal. 
 
Team members observe and reflect other members responses and actions and attend to them, yet behaviour of the leader is often their particular concern \citep{tyler1992relational}. 

For instance, \citet{edmondson1999psychological} states team leader coaching influencing positively on team psychological safety. Psychologically safe environment includes team leaders being supportive, coaching-oriented, who doesn't response to questions or challenges in defensive manner. Employees are not likely to take interpersonal risks that might lead to learning if leader tends to act in authoritarian of punitive ways. 

Several types of teams function in organisations, type depending on various dimensions such as cross-functional versus single-function, time-limited versus enduring and manager-led versus self-led. These dimensions should be recognized and team learning fostered depending on the type. \citep{edmondson1999psychological}

\section{Practices and structures for developing and experimenting}

According to \citet{amabile2002creativity} clear time should be allocated for developing especially when the aim is to flourish idea generation, creativity, learning and experimentation of new concepts. Time pressure should be minimal in order bright ideas to glow as cognitive processing requires time of an individual and team. Yet, no sense of urgency leads employees easily to auto-pilot mode, in which routine tasks are performed without further thinking and analysing. Thus, creative time for playing with ideas, brainstorming, learning and experimenting should be allocated in an organisation in order truly new things to develop. Shared goals are once more essential in engaging team members to play with ideas and feel more motivated in developing their work. \citep{amabile2002creativity}

In addition, sufficient time and resources should be allowed for exploration\citet{amabile2008creativity,katz1985project}. 

Creativity, exchanging ideas and turning them into action requires intrinsic motivation from employees \citep{jung2001transformational}. Thus, in order to increase creativity and innovation at workplace, leaders should foster organisational culture in which individuals find their motivation in divergent thinking and trying out new ways of performing tasks \citep{amabile1998kill}. 

Leaders need to acquire resources and encourage idea generation \citep{mcgourty1996managing}, and overall create environment where idea generation is possible \citep{andrews1970social} as well as evaluate the ideas and integrate them to organisational needs \citep{mumford2002leading}. While individual characteristics affect on the creativity of an individual, creating an environment fostering creativity is likely to assist in producing novel ideas during the routine work of employees \citep{amabile1996assessing}. 

Collective organisational achievements can, in turn, be affected through affecting working environment and organisational culture and leaders influence on employee's attitudes and motivation towards work \citep{amabile1998kill}.

Prior studies show how creative efforts of employees require sufficient amount of time and energy\citep{gardner1988creativity,getzels1975problem}. Also \citet{redmond1993putting} state that leaders should allow enough time for problem solving and creative actions. 

Also \citet{shalley2004leaders} emphasises the meaning of prior knowledge and experience of an employee of area of work before demanding or anticipating creative actions from them. Naturally, job rotation and employees from different areas works as a great source for new perspectives and development, yet creativity requires sufficient level of familiarity of target area. \citep{shalley2004leaders}

\citet{shalley2004leaders} also state how appropriate level of autonomy given to employees is useful. However, too much autonomy, meaning full control over planning and conducting the work, may lead to negative consequences and contradictory goals between employee and organisation. Thus, setting appropriate goals and understandable requirements that inspire employees is essential. Furthermore, leaders need to realise whether the goals require creativity or lead to creative outcomes, and not anticipate creativity or creative outcomes and instead accept employees being less creative where it is not needed. \citep{shalley2004leaders} 

Additionally, trying out novel approaches and conducting experiments requires more energy and is overall more difficult for employees than performing and sticking to the routine tasks. As it takes more cognitive resources to generate several alternative solutions, practice divergent thinking and approach problems from different perspectives, allowing time for creative work is essential. However, engaging employees to creative activities is likely to lead better and more qualified decisions. \citep{shalley2004leaders}
Also \citet{amabile1987creativity} state how sufficient time should be allowed for creative thinking, playing with ideas and exploring multiple perspectives. \citet{katz1985project}, in turn, found in their study how uninterrupted time was considered critical for engineers working on novel technologies. 

Studies have also shown how employees working on high time pressure affects negatively on ability to engage in creative cognitive processing \{amabile2002creativity}.

Together with time, in order to be creative sufficient access to material resources should be allowed for employees \citep{katz1985project}. However, even though material resources are essential for creativity, studies have suggested a contradictory perspective: when employees have access to wide range of material resources, their creativity tendencies may decrease. This may happen due to the creative actions and thoughts an employee needs to perform when needing certain resources to finish his task but not having them at hand. This, in a way, stretches employees' skills to think differently and achieve goals. \citep{csikszentmihalyi199916} Thus, \citet{csikszentmihalyi199916} states how resources are likely to make employees feel too comfortable and lead to decrease in creativity. 

Giving and receiving feedback remain simultaneously a key function of leaders and one of the most challenging tasks they have. According to \citet{shalley2004leaders} giving performance feedback is essential for creativity and accordingly difficult as creativity often involves approaching problems from new approaches and trying out novel things as well as taking risks. 

Structures have their influences on creativity of employees. Relationship between formal reporting and responsibility levels, referring to bureaucracy levels are essential: highly bureaucratic organisation do not tend to encourage employees to reach for novel approaches and experiments, whereas organisation with flatter structure may enhance organisations autonomy and creativity \citep{shalley2004leaders}

Safe environment that fosters creativity also takes into account employees' perceptions of just and transparent decision-making as well as applied actions \citep{shalley2004leaders}.

Changing organisational climate is challenging, yet various components are reasonably manageable. 
"While overall climate is often regarded as a hard thing to change, there are several components of climate that are reasonably manageable and should have an effect on creativity. For example, fostering a climate where risk taking and constructive task conflict are encouraged can be role modeled and actively encouraged and supported by management. Likewise, a review of a division's or organization's management hierarchy and reporting structure may highlight that employees are not encouraged to make decisions on their own and thus may be less likely to try new ways of doing their work. Finally, if the bureaucracy associated with changing anything is such that it takes a great deal of time and effort to get new ideas considered, employees also may be less likely to try new approaches to work." (Shalley and Garvin 2004) 

"The human resource practices used to select, train, appraise, and reward employees all need to be systematically linked together so employees know what is expected of them and when and how. This also ties back to the importance of procedural justice in that if employees understand how, when, and for what they will be rewarded, promoted, or even fired, then they should have a stronger sense of fairness and subsequent organizational commitment, loyalty, and increased levels of organizational citizenship behavior. In addition, it is specifically these types of attitudes that need to be fostered for creativity to occur. For instance, employees who are not loyal or committed to their organizations will not be willing to give more than is required by their job and therefore will be more likely to stick to the tried and true ways of performing their tasks rather than searching for alternative solutions." (Shalley and Garvin 2004) 

"The practical implications of our review for the day-to-day management of creative people should be highlighted. First, across the empirical studies reviewed, one common theme is that individuals need to feel they are working in a supportive work context. This applies to how leaders interact with employees, how coworkers, team members, and even others outside of work interact with employees, whether sufficient resources are available, how employees expect to be evaluated and rewarded, and whether the climate is perceived to be supportive (e.g., a perceived fair environment). Thus, managers should attempt to increase the supportiveness of the work context." (Shalley and Garvin 2004) 

Amabilen (1997) tekstiss\"a pdf:n sivulla 17 on hyv\"a kiteytys Amabilen osalta, mit\"a innovointi vaatii organisaatiolta. 

\section{Attitude towards failure and risk}

In order to decrease the fear of failure, Amabile and Khaire (2008) \citet{amabile2008creativity} suggest leaders should put emphasis on creating an environment where an employee feels it is safe to fail and speaking out loud ideas nor making mistakes does not result in punishment or humiliation. Leaders should, instead, motivate and encourage employees to ideate and learn by stating how essential experimenting, iterating and failing is for learning and developing. 

"those in a 
position to initiate learning behavior may believe they are 
placing themselves at risk; for example, by admitting an er- 
ror or asking for help, an individual may appear incompetent 
and thus suffer a blow to his or her image. In addition, such 
individuals may incur more tangible costs if their actions cre- 
ate unfavorable impressions on people who influence deci- 
sions about promotions, raises, or project assignments" (Edmondson 1999)

"Asking for help, admitting errors, and 
seeking feedback exemplify the kinds of behaviors that pose 
a threat to face (Brown, 1990) even when 
doing so would provide benefits for the team or organization (edmondson 1999)"

"there is no innovation process without failures and mistakes. Organizations need to learn from them as quickly as possible. If the organization out-learns its competitors, they then take the lead. And taking the lead is what innovation is all about!" (Buijs 2007)

"The phenomenon of threat 
rigidity has been explored at multiple levels of analysis, 
showing that threat has the effect of reducing cognitive and 
behavioral flexibility and responsiveness, despite the implicit 
need for these to address the source of threat (Staw, Sand- 
elands, and Dutton, 1981)."

"In sum, people tend to act in 
ways that inhibit learning when they face the potential for 
threat or embarrassment (Argyris, 1982)." 

"Nonetheless, in some environments, people perceive the 
career and interpersonal threat as sufficiently low that they 
do ask for help, admit errors, and discuss problems. Some 
insight into this may be found in research showing that fa- 
miliarity among group members can reduce the tendency to 
conform and suppress unusual information (Sanna and Shot- 
land, 1990); however, this does not directly address the 
question of when group members will be comfortable with 
interpersonally threatening actions." (Edmondson 1999)

Failing is widely considered as essential part of learning \citep{farson2002failuretolerantleader}.

Andriopoulos and Lowe (2000) :
"The risk attached to creative work implies both a need to experiment and a need to tolerate failure Andriopoulos and Lowe, 2000 and Quinn, 1989."(T\"a\"a andriopoulos l�ytyy PDF:n\"a, k\"ay l\"api, on varmaan hyv\"a\"a setti\"a!) 

Suoraa lainausta Mumford et alilta (2002) 
"Thus, creative work is contextualized with the success of creative ventures depending on an awareness of the capabilities of, and pressures on, extant socio-technical systems. In fact, it is this contextualization of creative activities that accounts for such well known phenomena as simultaneous invention and the tendency for innovation to occur in spurts within a given industry Csikszentmihalyi, 1999"

Feeling of self-efficacy may affect individual's willingness to provide unique and novel ideas even when some degree of risk is involved \citep{mumford1988creativity}. Training, coaching, giving feedback and assigning tasks seem to be useful approaches for leaders, who pursue to contribute empoloyee's self-efficacy \citep{amabile1998kill}.

Fear of failure can be decreased through transformational leaders who foster the culture of intrinsic motivation and rewards from creative endeavours, idea exchange and discussion \citep{amabile1998kill}.

According to \citet{amabile1996assessing} creative solutions in an organisation can be achieved by encouraging employees to reach and experiment new perspectives and ways of performing. Essential for this is not being punished for negative outcomes. Organizational environment that allows failing is likely to assist in employees acquiring diverse perspectives and questioning the status quo and habitual way of performing. 

"In addition, established theory recognises the need for developing opportunities where employees can exploit uncertainty. Sternberg et al. (1997) argue that managers should let messiness exist. In other words, they suggest that uncertainty associated with creative projects must not be controlled in order to establish some order."
"The importance of
challenging work has also been emphasised
by Amabile (1997), who states that matching
creative employees to assignments, based on
their skills and interests, enhances their
motivation toward work." 

"Motivate people to contribute ideas by 
making it safe to fail. Stress that the goal is to 
experiment constantly, fail early and often?
and learn as much as possible in the process. 
Convince people that they won?t be punished 
or humiliated if they speak up or make 
mistakes." (Amabile and Khaire 2008)

"The comparison of perpetual
challenging with established corporate
creativity theory reveals that the emergent
category of adventuring comes closer to the
five stages of Amabile's (1988) componential
framework of creativity. Specifically, the
properties of adventuring, such as
introspecting, scenario making and
experimenting can be associated with
Amabile's preparation, response generation
and validation stages of the creative process.
Moreover, her conceptual model emphasises
the fact that the process can have a negative
consequence, which can also be found also in
the emergent category of ``adventuring''
under the title ``mistake making'' within this
grounded theory study." \citep{andriopoulos2000enhancing}


"to develop new and useful products or processes, individuals have to be willing to try and to possibly fail. For many, this is not an easy thing to do and can, in part, depend on the individual's predisposition toward risk as well as the organizations culture, which will be discussed later in this article. " (Shalley and Gilson 2004) 

"Research has indicated that people tend to avoid risk and prefer more certain outcomes (Bazerman, 1994). However, because creativity does not just happen but rather evolves through a trial-and-error process that involves risk taking, failure will often occur along with success. If employees are risk averse, it is much easier for them to continue performing in more routine ways rather than take a chance with a new, and potentially better, approach. Therefore, a key in the motivation of employees toward creativity is to ensure that they feel encouraged to take risks and break out of routine, safe ways of doing things." (Shalley and Gilson 2004) 

"If leaders value and want employees to be creative, a critical contextual factor they need to attend to is fostering an environment where risk taking is encouraged and uncertainty is not avoided. This has been referred to as providing a culture where employees feel psychologically safe such that blame or punishment will not be assigned for new ideas or breaking with the status quo (e.g Edmondson, 1999). In support of these arguments, Nystrom (1990) found that organizational divisions were more innovative when their cultures reflected challenge and risk taking, " (Shalley and Gilson 2004) 

"participative safety, being able to give input without being judged or ridiculed, has been positively linked to creativity (De Dreu and West, 2001)"

Predicting the future being impossible, focus should be in managing risks involved in playing with creative ideas in both the company and individual level. As \citet{sternberg1997creativity} state, "as uncomfortable as it is, while not being able to predict and control uncertainty in creative projects, the messiness does have to let exist". \citet{kanter1983change}continues that, actually, opportunities grow from uncertainty and creative endeavours rise when struggling with uncertainty and mess, as individuals impose order where it does not exist, and thus individuals are forced to form new connections. Furthermore, allowing employees freedom to act actually arouses desire to act.

According to \citet{amabile2008creativity} essential part of creating a safe environment for creativity is managers to decrease the fear of failure. Instead, constant experimenting should be the goal of working, learning by doing and iterating until sufficiently is learnt from the process. 

Furthermore, when company grows, it usually leads to more conservative actions and increase in fear of failure. When fearing failure managers tend to deny failure and erase it from the memory instead of learning from it. \citep{amabile2008creativity} 

According to Edmonsdon  (1999) "any business that experiments vigorously experience failure?which, when it happens, should be mined to improve creative problem solving, team learning, and organizational performance" 

"the premise that learning behavior in so- 
cial settings is risky but can be mitigated by a team's toler- 
ance of imperfection and error. This appeared to be a 
tolerance (or lack of tolerance) that was understood by all 
team members--
The implication of this result is that people's beliefs about how oth- 
ers will respond if they engage in behavior for which the out- 
come is uncertain affects their willingness to take 
interpersonal risks.. " (Edmondson 1999)

By creating an environment that serves psychological safety for employees, organisations may capitalise on failure. In a safe environment employees are convinced they are not humiliated or punished when failing or saying out loud their ideas, concerns or raise discussion. Furthermore, failure can be divided in three separate categories: unsuccessful trials, system break-downs and process deviations, which all need to be recognized and analysed and dealt with in order learning to happen. Especially unsuccessful trials are fruitful and essential for creative learning, yet overcoming "deep ingrained norms that stigmatise failure and thereby inhibit experimenting" is needed. (Garvin et al. 2008)
