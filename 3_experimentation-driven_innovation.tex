\chapter{Experimentation-driven innovation}

Failed experiments should not be considered as failing, instead they offer valuable learning points. 

According to Thomke (2003), in the beginning every product is an idea, that was being shaped through the process of experimentation, and the ability to do experimentations is actually a measurement of company?s ability to innovate.

Experimentation is essential in order to learn about the idea, concept and prototype and whether it actually addresses a new need or a problem or solves the one at hand. Prototyping is critical part of the process, as testing the prototype in a real environment gives instant and valuable feedback for further development. 

Thomke (2003) suggests four steps for organizations to be more innovative. First of all, organization should allow and manage the work for the employees so that fast experimentation is possible. This usually requires challenging routine ways of working and shaping the routines, yet fast experimenting is essential in order to get rapid feedback for shaping the ideas. In addition, team engagement is essential, as the whole team need to understand the meaning of experimenting and developing and it should be encouraged to sharing information and ideas in as early stage of development process as possible and throughout the process. Thomke suggests using small teams and parallel experiments especially when the time is the most critical factor. 

Secondly, failing early and often, yet avoiding mistakes is important for experimenting. Failure can disclose important information and reveal gaps in knowledge, and is thus important as early phase of the development as possible. However, according to Thomke (2003), this is not an usual way for an organisation to think about failure, thus building the capacity for rapid experimentation as well as tolerating and learning from failure is essential and often requires overcoming ingrained attitudes. Encouraging and creating a culture where failing is allowed and not being afraid of, brainstorming sessions where judgement is not allowed are important for 

However, Thomke (2003) does not suggest failing and making mistakes as a result of poorly planned experiments. Mistakes and failures produce most value, when the experiment is well planned and the goal or hypothesis that needs to be tested is clear.

Thirdly, anticipating and exploiting early information can save a lot of resources in the development process. If problems are shown in the late-stage of the process, they can be even 100 times more costly than the ones discovered in the early stage. According to IDEO, an innovation and design-firm, using human-centered design-based approach, the key elements in the design process and prototyping is it being rough, rapid and right. The right-element reminds that even though the prototype itself is likely to be incomplete, it has to show the right specific aspects of a product. This forces developers to decide the factors that can initially be rough and those that must be right. In addition, exploiting early information serves as a good method for developers reflecting changing customer preferences. Briefly, information in the early stage of the developing process should be listened and discovered carefully, as the problems are cheaper and easier to solve. 

Lastly, for enlightened experimentation Thomke (2003) puts emphasis on combining new and traditional technologies. For company


\section{Approach of Mind}
Dream and think big, but act small. 

%In the Mind approach, there are three types of ideas: (Voiko t�t� kirjottaa kun ei taida olla viel� ihan "tieteellist�"

Experimentation approach suggests only after one has tried out an idea and performed an experiment, has he enough information and experience to continue to the execution of the idea.  
