\chapter{Experimentation-driven innovation}

poisj��nytt� tavaraa, ehk� k�yttist� my�hemmin

In order teams to function under ambiguous circumstances, they need to feel safe to ask questions, seek help and tolerate mistakes. Team members, with the help and support of managers and leaders can assist each other in providing required safety. \citep{edmondson1999psychological}

Work needs to meet the skills and interests of employees while offering sufficient level of challenges in order to increase the motivation of employees towards work \citep{amabile1998kill}.

Question of team composition comes relevant especially when forming teams for innovation. According to\citet{buijs2007innovation} innovation teams are the heart and the engine of innovation process and essential for the rest of the organisation to accept changes and innovation results. As he suggests right people in the team is the premise for innovation, all the members should be chosen carefully starting from the leader. Furthermore, the leader should be allowed to affect on the formation of the rest of the team in order to ensure positive base for teamwork and innovation process. Accordingly, team membership should be based on voluntary. \citep{buijs2007innovation}


According to \citet{schein2010organizational}, concept of culture refers to and helps to explain some seemingly incomprehensible and irrational aspects of what is going on in groups, organisations and other kinds of social units, that share history. \citet{schein2010organizational} divides culture into three levels: artefacts (visible and feelable structures and processes and observed behaviour), espoused beliefs and values (ideals, goals, aspirations, ideologies and rationalisations) and basic underlying assumptions (unconscious, taking-for-granted beliefs and values). Climate of the group should not be mixed with culture of the group, it should rather be considered among artefacts. However, essential point of view \citet{schein2010organizational} provides is how culture in organisation or group level is easy to observe yet very difficult to decipher. Put in other words: researchers are able to observe and make remarks on what they see and feel, yet they are unable to reconstruct the deeper meaning of those observations to the group. Cultural analysis and understanding of dynamics of a group should begin in observing and asking members the norms, values and rules that shape practicalities of work in day-to-day level.

The approach of leaders oftentimes divides into task-oriented or relationship-oriented. Task-oriented leaders value performing the job, focusing on clarifying roles and responsibilities, monitoring work while managing time and resources. In turn, relationship-oriented leaders value socioemotional aspects of work through empathetic actions, showing consideration for employees, being friendly and supporting the team personally. Should be noted that in literature concerning leader behaviour, term support refers to relationship-oriented leadership behaviour, wheres in creativity literature same term refers to both task- and relationship-oriented behaviours and actions - all that are to foster creativity. In this thesis, latter and broader usage of term support is used. \citep{amabile2004leader}

Rather than managing the inevitable chaos of innovation productively, these managers soon drive out the very things that lead to innovation in order to prove their announced plans. In the name of efficiency, bureaucratic structures require many approvals and cause delays at every turn. Experiments that a small company can perform in hours may take days or weeks in large organizations. The interactive feedback that fosters innovation is lost, important time windows can be missed, and real costs and risks rise for the corporation. Inappropriate incentives. Reward and control systems in most big companies are designed to minimize surprises. Yet innovation, by definition, is full of surprises. It often disrupts well-laid plans, accepted power patterns, and entrenched organizational behavior at high costs to many. Few large companies make millionaires of those who create such disruptions, however profitable the innovations may turn out to be. When control systems neither penalize opportunities missed nor reward risks taken, the results are predictable." \citep{quinn1985managing}

 
Particularly when large new, complicated systems at hand, meaning of co-operation in production, development and communication rises exponentially. Especially in large organisations innovation can be inhibited by the errors increasing as a result of complexity of the system and inability to control, understand or make intelligent decisions. Challenging as it is for one department, faculty or company to survive on its own without communication and help of others in design, production and other business-related decisions, with management that takes the complex environment into account, the disastrous effects resulting from lack of communication can be lessen. \citep{quinn1985managing} Yet, at the same time, as a result of the difficulty of managing complex situations, innovation may denote finding the core, boiling things down and focusing on the most essential elements \citep{katz1978social}.

\citet{quinn1985managing} list several barriers to innovation, including intolerance of fanatics, short time horizons, accounting practices, excessive rationalism and bureaucracy and inappropriate incentives. \citet{hayes1982managing} supplements the list with concern of top management isolation, arguing how top management oftentimes has too little contact and understanding of the environment and conditions at factory floor or customer requirements for innovative solutions. Top managers who tend to be financially-driven and are not familiar nor have experience with current technology and its possibilities, may fear technological innovations and perceive them as too risky. Thus, more familiar traditions remain with ease. \citep{hayes1982managing} 

Amabilen (1997) tekstiss pdf:n sivulla 17 on hyva kiteytys Amabilen osalta, mita innovointi vaatii organisaatiolta. 
Expertise 
- Expert performance and its affects on implementing ideas \citep{ericsson1994expert} (tsekkaa artsu viel�)


Several types of teams function in organisations, type depending on various dimensions such as cross-functional versus single-function, time-limited versus enduring and manager-led versus self-led. These dimensions should be recognized and team learning fostered depending on the type. \citep{edmondson1999psychological}

Thus, leaders should allow team members time to think creatively \citep{amabile2002creativity}.

In previous research, structural and design-related factors have been combined to have influence on work teams's effectiveness and team performance. Well-designed tasks and goals, suitable and functional team composition, as well as physical environment and practices ensuring transparent communication and information exchange, sufficient materials, resources and motivating rewards all affect team efficiency. \citep{hackman1987design,goodman1988groups,campion1993relations}. 

 \citet{quinn1985managing} offers an explanation why it seems easier for engineer and scientific leaders to create atmosphere supporting innovation: understanding and psychological comfort are related to familiarity, and engineers, for instance, have wider understanding and knowledge of technology, which makes newer technological innovations easier to accept and adapt.