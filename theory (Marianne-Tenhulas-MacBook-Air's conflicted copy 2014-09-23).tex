\documentclass[11pt, oneside]{article}   	% use "amsart" instead of "article" for AMSLaTeX format
\usepackage{geometry}                		% See geometry.pdf to learn the layout options. There are lots.
\geometry{letterpaper}                   		% ... or a4paper or a5paper or ... 
%\geometry{landscape}                		% Activate for for rotated page geometry
\usepackage[parfill]{parskip}    		% Activate to begin paragraphs with an empty line rather than an indent
\usepackage{graphicx}				% Use pdf, png, jpg, or eps§ with pdflatex; use eps in DVI mode
								% TeX will automatically convert eps --> pdf in pdflatex		
\usepackage{amssymb}
\setcounter{secnumdepth}{5}

\begin{document}


\section{Systems thinking and learning}

\subsection{fifht-discipline -background, why is this important?}
\subsection{Learning as competitive advantage and way to survive}

Understanding change analytically in the turbulent world appears challenging. Change being hectic and fast calls for different skills and strategy than before. Only when change is understood can it be managed, and in order to survive new perspective and understanding towards change is required from an organisation. (Syd�nmaanlakka) 

In the changing environment tolerance for uncertainty is needed, and while future can not be predicted, forecasting is a usable method in order to cope with the anxiousness resulting from the uncertainty. Furthermore, even more emphasis should be put on ability to learn and adapt to changes. 

\subsection{Change in organisational level}
Hammer and Champy (1993) have summarised aspects of change in organisational environment, beginning from the change in organisational structure; from functional departments to process teams. Work tasks change from simple and detailed tasks to multi-dimensional knowledge work while employees are becoming more autonomous instead of strict control. Furthermore, instead of educating, focus is in the learning of an employee, and evaluation of work will change from operations to outcomes. Knowledge and capability are preferred over single performance and values change to more productive behaviour than over-protective. Superiors turn from leaders of the work to coaches and hierarchical organisational structures turn lower while managers focus on leadership instead of task management. 

According to Arie de Geus (1997) the only one can maintain company's competitive advantage is to make sure the company is able to learn faster than rivals. Generally organisations are considered as machines, yet recently more emphasis has been put on organisations as living organisms. When considered as machine, organisational model is mechanic and simple, which purpose is to gain profit. Whereas, organisation as a living organism is a whole-systemic model, and organisations are considered as place which has deeper, permanent meaning offering people the opportunity to grow and fulfil themselves while earning money.

Liable vision of the future focuses on the latter perspective of organisations, where learning and renewal form the essence of being. 

According to Huy and Minztberg (2003), organisations learn best through small experiments and trying out new things, and the closer and more related experimentations are to customers and customer interfaces, the more can be learned. 

\section{Organizational learning}

From the data rose various factors affecting experimentation behaviour, and the literature review revealed how similar factors are related to learning in an organisation and organisational learning. 

Through the understanding of organisational learning real growth and support for learning can be offered. Learning process needs to be understood all from organisational, team and individual perspective. 


\subsection{What is learning?}
Generally, learning process consists of experience, evaluation, understanding and applying new knowledge into use. Syd�nmaalakka (2007) claims clarifying and understanding phases of learning process assists in supporting learning and organisational recurrence. 

Learning is changing, growing, evolving and developing. Generally time for reflection is needed in learning process, when new information or experience is reflected and compared to prior knowledge and experiences and attached to previous frameworks, and when considered useful, new experiences and knowledge is combined to previous knowledge providing new way for performing in the future. (Syd�nmaanlakka, 2007)

Various definitions of learning have been suggested. Syd�nmaanlakka (2007) defines learning as a process, where an individual acquires new information, abilities, attitudes, experiences and contacts, which result changes in his actions. According to this definition, learning is a process which is affected by individual's cognitive, emotional and physical skills. Information acquirement always concerns of individual's interpretation, thus learning is not mechanical action but making meanings through interpretations. Furthermore, learning is not only acquiring new information; the process involves abilities, attitudes, emotions and values. In addition, all experiences are essential in learning, and the more experiences individual has, the easier it is for him to learn from the new experiences. In order to truly understand, several skills even require prior experience from the field. Furthermore, contacts and networks serve as significant factors in learning process. True learning consists of adapting new information and processes resulting in change in action, behaviour or thinking. 

In addition, motivation is essential part of learning. 

Experimenting can be seen as one way of gaining experiences, thus improving learning. 

According to experiential learning theory, learning is "the process whereby knowledge is created through the transformation of experiences." (Kolb)

\subsection{Levels/stages of learning}

Learning occurs in several levels. In the bottom level is the belief of knowing something. Already a step further is the comprehension of not knowing something, right before the level of knowing, which consists of acquiring new information, following the step of understanding. Understanding consists of actual comprehension of phenomena at hand, and this level also concerns the attitudes and emotions of an individual. More time is required for actually understanding something than only acquiring new information. Level of applying proves how recently learnt things is successfully applied in practise. The level of applying requires management of previous levels - knowing and understanding. According to Syd�nmaanlakka (2007) true and deep learning comprises all these levels and enables developing and renewal of already known things and approaches. 

Learning types
Briefly, learning styles of individuals, teams or organisations can be divided into four categories. 
Reactive learning
Proactive learning
Action learning
Questioning learning


Learning process 
Another definition of learning process is presented by Kolb (1984). This general model for learning begins in experiences of an individual and the ability and motivation to learn from them. Motivation stands for key to learning, following time for reflecting and processing new perspectives to previous knowledge and experiences. After reflection the knowledge is processed and understood, following the final stage of applying knowledge in practice. Learning can best be supported when all these phases are taken into account. 

Factors enhancing learning process consists of motivation, joy of success, experimenting and documenting. 

Based on the learning process of Kolb, several models have been developed to describe different learning styles. One is Honey and Mumford's model, where individuals are divided into four categories based on their learning style: activist, reflector, theorist, and pragmatist.

Critique on these models sitaatit cassidy2004learning

\subsection{Organizational capabilities for learning}

Organizational learning differs from individual or team learning. Organizational learning occurs through the shared knowledge, insights and approaches of the employees of an organisation. Secondly, organisational learning is based on prior knowledge and experience, the memory of organisation, which consists of the ways of working, processes and instructions of an organisation. Even though individual and team learning are highly related to organisational learning, it is not the sum of the previously mentioned. 

Various definitions of learning organisations have been presented. One of the most famous definitions is from Peter Senge (1991), who describes learning organisation as follows: "Learning organisation is an organisation, where people are able to constantly develop and achieve intended results; where new ways of thinking are born and where people share goals and learn together."

Pedler et al Define 
\subsection{Team learning or other team-related stuff}
Definition of team consists of a small group of people who have supplementary knowledge and abilities compared to each other, and who share a goal, targets and way of working and approach. Team also shares a feeling of being responsible for its performance. 

Team opportunities and full advantage of team effort can be threatened by demanding schedules, long-standing habits and unwarranted assumptions. In addition, in order to function, team needs a clear purpose and vision what makes it a team and why it exists. Teams get energy from significant performance challenges regardless of where they are in the organisation. Set of shared, demanding performance goals usually form a team, and personal chemistry or willingness to form a team may boost that, yet teamwork is essential for an actual team. Indeed, in order to receive great results teams should focus on performance regardless of the organizational hierarchy or what team does. Thus, team performance may exceed the results of what could be achieved if employees were acting alone as individuals without the team effort.  (Katzenbach)

Leaders play a great role in establishing strong team performance culture. This can be achieved through addressing and demanding performance that meets the need of the customers, employees and shareholders. Teams should not be fostered by the sake of the team only, rather should leaders clearly state how the team performance affects to customers and through that clearer performance ethics and cultures are fostered. In addition, even though people tend to have great sense of individualism, it does not have to bias the teamwork performance. (Katzenbach)



\subsection{Individual learning}

Learning stands for essential part of experimentation-driven innovation. 

Experiential learning process stands as a way describe the central process of human adaptation to the social and physical environment. (Kolb) Learning involves total concept of human being - feeling, thinking, perceiving and behaving. (Jung, 1923) Learning should be conceived as a holistic adaptation process providing bridges across life situations and underlaying the lifelong process of learning. (Kolb)

However, learning may not be the first thought in mind when considering an immediate reaction to the situation or problem at hand, rather is it considered as performance. Furthermore, long-term adaptations to our previous experiences and beliefs is mainly considered as developing, not learning. Thus, when talking about developing an individual or an organisation, the question highly concerns and is related to learning. (Kolb)

According to Kolb (..) experience plays a central role in learning, and learning is a process consisting of experience, perception, cognition and behaviour. Immediate experience forms a basis for reflection and observation, following assimilation to a theory from which new implications for action are deducted. In order to create new experiences, these implications serve as guides. 

This Lewinian approach emphasises here and now -concrete experiments in testing and validating abstract concepts. Thus, experience of an individual is a focal point of learning, and giving personal meaning to abstract concepts, which can be afterwards shared with others. Furthermore, receiving feedback is considered essential in this approach for learning, as it serves a continuous process for goal-oriented action following evaluation of that action. Feedback can thus boost effective, goal-oriented learning process. (Kolb)

Kolb presents Dewey's model of Learning, which adds to Lewinian model how through learning the impulses, feelings, and desires of concrete experience into higher-order purposeful action is transformed. Both Lewin and Dewey consider learning as a dialectic process integrating experiences and concepts, observations and actions. 

Piaget has formed in 1970 model of learning and cognitive development, which presents similar aspects to Lewin, Dewey and Kolb. Piaget also stated how learning is an interactive process between individual and environment. Learning is a mutual process between accommodation of concepts or schemas to experiences around us and assimilation of events and experiences into existing concepts and schemas. This intelligent adaptation, learning, results from the tension between accommodation and assimilation. Through this tension growth and higher-level cognitive functioning occurs. 

Kolb summarises above models. First of all, learning should not be conceived in terms of outcomes, but it is best conceived as a process. Ideas are not fixed and and immutable elements of thoughts, but can be formed and re-formed through experience,  

Bringing the experiential learning into education implications, all learning can be considered as relearning. Thus, all learning situations should take into account people arriving from all different experiential backgrounds to what they build their new experiences and knowledge on. This partly explains very likely resistance to new ideas, as when new information and experiences are in contradiction to old beliefs and experiences, new ideas and information is more difficult to adapt. In the education process learner's old beliefs and theories should be brought out, examined and tested, following integration of the new models and refined ideas into learner's belief systems. (Kolb)

According to Piaget individual learning and adaptation of new ideas occurs through integration or substitution, integration leading to stronger part of learner's conception of the world, whereas substitution requires real questioning of previous conceptions, and thus might take longer for the learner to adopt. 

According to Kolb, learning is a process filled with tension and conflict, and new knowledge, skills and attitudes are achieved through experiential learning, which consists of four modes and required abilities of learners: concrete experience abilities, reflective observation abilities, abstract conceptualisation abilities and active experimentation. First of all, individuals must openly involve themselves in new experiences, reflect and observe them from various perspectives, create concepts that can be integrated into more abstract theories as well as they need to be able to use these reflections and theories in active daily decision-making and problem-solving. Quite a task and ability list for individuals, and this is to show how learning requires various abilities from different perspectives and to choose which learning perspective to use in a situation at hand. Furthermore, Kolb divides the learning into two dimensions: "first of all representing the concrete experiences at one end and abstract conceptualisation at the other. " The other dimension includes active experimentation at one end and reflective observation at other. Thus, the level of learning depends on the way in which conflicts ... Kesken, kolb page 15. One mode dominating the conflict resolving process leads to learning dominated by that mode. However, the aim is to create a synthesis between the different modes in order to boost the process of learning and personal development. 

Also Syd�nmaanlakka emphasises different learning styles and modes, which should be understood and used variously in learning and problem-solving situations. 

In addition the meaning of environment in learning should be emphasised. Learning concerns of transaction between an individual and the environment, learning does not happen only inside of individual's thoughts, experiences and processes but is dependent on the real world environment. 

\subsection{Learning and feedback}
\subsection{Learning and change}

Learning is a process that creates knowledge through transaction of subjective and objective knowledge of an individual. (Kolb)

\subsection{Building blocks: How can organisational learning be supported?}

According to Garvin et al. (2008) in learning organisation employees excel at creating, acquiring and transferring knowledge. They define building blocks for learning organisation: supportive learning environment, concrete learning processes and practices and leadership behaviour that reinforces learning. Building blocks can be considered and measured as independent components yet each of them vital to the whole. In order to improve long-term learning of an organisation, strengths and weaknesses of an organisation and its unit needs to be recognised. 

In order experimenting to happen in an organisation safe and supportive environment has to be created. According to Garvin et al. (2008) supportive learning environment consists of four characteristics: psychological safety, appreciation of differences, openness to new ideas and time for reflection. 

Psychological safety means that learning of employees occurs when employees do not fear being rejected, ask naive questions, make mistakes or present viewpoint of minority. Psychologically safe environment enables employees comfortably to express their thoughts at work. Appreciation of differences is important as opening minds for different ideas and world views increases both energy and motivation, brings out fresh thinking. Learning occurs when employees become aware of opposing ideas in a safe environment, and additionally openness to new ideas is required. Novel approaches are relevant for learning, thus employees should be encouraged in risk-taking and exploring and testing uncertain things. Lastly, through providing time for reflection learning in safe environment occurs. Instead of looking and judging by numbers of hours of work or results employees should be given enough time to reflect their work. Analytic and creative thinking will not occur under stress, heavy workload and too tight schedule. Under stress ability to recognise and react to problems and learn from experiences deteriorates. In supportive learning environment time for reflection is allowed. 

Similar characteristics were found in data and are further presented in chapter xx. 

According to Garvin et al. (2008) second building block of organisational learning, consists of concrete learning processes and practices. It includes experimentation, information collection, analysis, education and training and information transfer. Organizational learning can be supported trough concrete steps and activities which are tested and further developed through experimentations. Furthermore, information and intelligence about customers as well as technological trends should be collected systematically and further analysed focusing on identifying problems and solving them. Training and education of new and established employees is an essential part of practices and processes. Finally, through transparent and meaningful knowledge sharing organisational learning can be enhanced, focus being on clear, well-defined and working communication systems that employees can easily relate and feel useful. Concrete processes together with efficient knowledge sharing methods ensures that essential information is available quickly and efficiently for employees who need it. 

 Thus, emphasis should be put on creating and defining concrete learning processes and practices.
 
Thirdly, leadership behaviour should reinforce learning. Behavior of leaders is highly related to the performance of employees (Kim and Mauborgne 2014) and organisational learning (Garvin et al. 2008). In order to encourage employees to learn, leaders should prompt dialogue and debate, ask questions and listen to employees. Example of the leaders was also recocgized from the data as an important factor affecting experimentation behaviour, and Garvin et al. also emphasises how through own example leaders can encourage employees to offer new ideas and options. 

These three building blocks overlap to some degree and reinforce one another. For instance, leadership behaviour helps in creating supportive learning environment and this supports managers and employees in creating and defining concrete learning processes and practices. Furthermore, concrete processes support leaders behaviour in a way that fosters learning and through own example cultivates that behaviour to others. 

Supportive leadership behaviour alone is not sufficient for guarantee organisational learning. Garvin et al. (2008) emphasise how organisations are not monolithic and managers should be sensitive to differences in culture, department and units. In addition to cultural differences, learning requires clear and targeted processes and practices. Furthermore, learning should be considered as multidimensional, thus organisational forces should not be solely focused on a single area but to consider presented building blocks as a whole. 




 Tolerance for ambiguity 
 
 Ambiguity is often perceived by individuals when lacking sufficient cues to structure a situation, and usually arises from novelty, complexity or unsolvability of situation at hand (Budner 1962). 


\section{Experimentation-driven innovation}
Failed experiments should not be considered as failing, instead they offer valuable learning points. 

According to Thomke (2003), in the beginning every product is an idea, that was being shaped through the process of experimentation, and the ability to do experimentations is actually a measurement of company?s ability to innovate.

Experimentation is essential in order to learn about the idea, concept and prototype and whether it actually addresses a new need or a problem or solves the one at hand. Prototyping is critical part of the process, as testing the prototype in a real environment gives instant and valuable feedback for further development. 

Thomke (2003) suggests four steps for organizations to be more innovative. First of all, organization should allow and manage the work for the employees so that fast experimentation is possible. This usually requires challenging routine ways of working and shaping the routines, yet fast experimenting is essential in order to get rapid feedback for shaping the ideas. In addition, team engagement is essential, as the whole team need to understand the meaning of experimenting and developing and it should be encouraged to sharing information and ideas in as early stage of development process as possible and throughout the process. Thomke suggests using small teams and parallel experiments especially when the time is the most critical factor. 

Secondly, failing early and often, yet avoiding mistakes is important for experimenting. Failure can disclose important information and reveal gaps in knowledge, and is thus important as early phase of the development as possible. However, according to Thomke (2003), this is not an usual way for an organisation to think about failure, thus building the capacity for rapid experimentation as well as tolerating and learning from failure is essential and often requires overcoming ingrained attitudes. Encouraging and creating a culture where failing is allowed and not being afraid of, brainstorming sessions where judgement is not allowed are important for 

However, Thomke (2003) does not suggest failing and making mistakes as a result of poorly planned experiments. Mistakes and failures produce most value, when the experiment is well planned and the goal or hypothesis that needs to be tested is clear.

Thirdly, anticipating and exploiting early information can save a lot of resources in the development process. If problems are shown in the late-stage of the process, they can be even 100 times more costly than the ones discovered in the early stage. According to IDEO, an innovation and design-firm, using human-centered design-based approach, the key elements in the design process and prototyping is it being rough, rapid and right. The right-element reminds that even though the prototype itself is likely to be incomplete, it has to show the right specific aspects of a product. This forces developers to decide the factors that can initially be rough and those that must be right. In addition, exploiting early information serves as a good method for developers reflecting changing customer preferences. Briefly, information in the early stage of the developing process should be listened and discovered carefully, as the problems are cheaper and easier to solve. 

Lastly, for enlightened experimentation Thomke (2003) puts emphasis on combining new and traditional technologies. For company

\subsection{Experimenting and organizational learning }


Discussioniin

Relation between personal characteristics such as age, study background or education and experimentation behaviour was not studied in this thesis, yet some correlations were recognised, such as employees who had more working experience were more confident in performing experiments and they did not consider failing as a personal failure, but a reflective opportunity for learning. 

	
\end{document}  