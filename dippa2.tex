% General information
% ==================================================================
% ------------------------------------------------------------------
% Document class for the thesis is report
% ------------------------------------------------------------------
% You can change this but do so at your own risk - it may break other things.
% Note that the option pdftext is used for pdflatex; there is no
% pdflatex option.
% ------------------------------------------------------------------
\documentclass[12pt,a4paper,oneside,pdftex]{report}
%\documentclass[12pt,a4paper,oneside,pdftex]{article}
% The input files (tex files) are encoded with the latin-1 encoding
% (ISO-8859-1 works). Change the latin1-option if you use UTF8
% (at some point LaTeX did not work with UTF8, but I'm not sure
% what the current situation is)
\usepackage[latin1]{inputenc}
%\DeclareUnicodeCharacter{2265}{\ensuremath{\ge}}
% OT1 font encoding seems to work better than T1. Check the rendered
% PDF file to see if the fonts are encoded properly as vectors (instead
% of rendered bitmaps). You can do this by zooming very close to any letter
% - if the letter is shown pixelated, you should change this setting
% (try commenting out the entire line, for example).

\usepackage[OT1]{fontenc}
% The babel package provides hyphenating instructions for LaTeX. Give
% the languages you wish to use in your thesis as options to the babel
% package (as shown below). You can remove any language you are not
% going to use.
% Examples of valid language codes: english (or USenglish), british,
% finnish, swedish; and so on.
\usepackage[finnish,swedish,english]{babel}
\usepackage{csquotes}
% Font selection
% ------------------------------------------------------------------
% you can try out these font variants by uncommenting one of the following lines
% Changing the font causes the layouts to shift a bit; you many
% need to manually adjust some layouts. 
% ------------------------------------------------------------------
%\usepackage{palatino}
%\usepackage{tgpagella}

% Changing citation style to biblatex
% ------------------------------------
% Options: natbib allows to use \citep & \citet. 
% isbn removes isbn from ref. list, firstinits truncates first and middle names, 
% maxbibname lists all authors in ref. list, , hyperref: citation marks are hyperlinks
\usepackage[natbib=true, style=authoryear, isbn=false, doi=false, firstinits=true, 
maxbibnames=99, maxcitenames=2, hyperref=true, urldate=long, dateabbrev=true, 
dashed=false, uniquename=init, backend=bibtex]{biblatex}

%\citet{jon90}	    		-->    	Jones et al. (1990)
%\citet[chap. 2]{jon90}	    	-->    	Jones et al. (1990, chap. 2)
%\citep{jon90}	    		-->    	(Jones et al., 1990)
%\citep[chap. 2]{jon90}	    	-->    	(Jones et al., 1990, chap. 2)
%\citep[see][]{jon90}	    	-->    	(see Jones et al., 1990)
%\citep[see][chap. 2]{jon90}	-->    	(see Jones et al., 1990, chap. 2)
%\citet*{jon90}	    		-->    	Jones, Baker, and Williams (1990)
%\citep*{jon90}	    		-->    	(Jones, Baker, and Williams, 1990)


%, bibencoding=utf8
\bibliography{references.bib}% ONLY selects .bib file; syntax for version <= 1.1b
%\addbibresource{references.bib}% Syntax for version >= 1.2

% Customized bibliography list:
% ---------------------------
% Order names: "Last, F. M."
\DeclareNameAlias{sortname}{last-first}
% title of these formats is printed as normal text
\DeclareFieldFormat[article, inbook, thesis, incollection, inproceedings, techreport, misc]{title}{#1}

% Change "visited on" to "accessed on" on urls
% \DefineBibliographyStrings{english}{urlseen = {Accessed:}} 

% Customizing tables:
% -------------------
% Longtable provides a tabular environment that can span multiple pages. This is used in the example acronyms file.
\usepackage{longtable}

% For tables: allow left-aligned/centered, multirow and to use \newline or \\ for manual page breaks
\usepackage{array}
\newcolumntype{L}[1]{>{\raggedright\let\newline\\\arraybackslash\hspace{0pt}}m{#1}}
\newcolumntype{C}[1]{>{\centering\let\newline\\\arraybackslash\hspace{0pt}}m{#1}}
% to allow multirow columns in tables
\usepackage{multirow}

% customizing tables
\usepackage{booktabs}
% more space between rows (or 1.3)
\renewcommand{\arraystretch}{1.2} 

% Other:
% -----------------------------------------------------------

% The eurosym package provides a euro symbol. Use with \euro{}
\usepackage{eurosym}

% Loading calcualtion packege
\usepackage{calc}

% Verbatim provides a standard teletype environment that renderes
% the text exactly as written in the tex file. Useful for code
% snippets (although you can also use the listings package to get
% automatic code formatting).
\usepackage{verbatim}

% Subfigure package allows you to use subfigures (i.e. many subfigures within one figure environment). 
% These can have different labels and they are numbered automatically. Check the package documentation.
\usepackage{subfigure}

% Defining a red-color macro for highlighing drafts
\usepackage{color}
\newcommand{\red}[1]{\textcolor{red}{#1}}

% Adding appendix package:
% ------------------------
% appendices environment gives you, is that once the environment ends, you can carry on with sections or chapters as before ? numbering isn't affected by the intervening appendixes. 
\usepackage[titletoc]{appendix}

% code for list of appendices
\newcommand\listappendixname{List of Appendices}
\newcommand\appcaption[1]{%
  \addcontentsline{app}{section}{#1}}
\makeatletter
\newcommand\listofappendices{%
  \section*{\listappendixname}\@starttoc{app}}
\makeatother

% The titlesec package can be used to alter the look of the titles
% of sections, chapters, and so on. This example uses the ``medium''
% package option which sets the titles to a medium size, making them
% a bit smaller than what is the default. You can fine-tune the
% title fonts and sizes by using the package options. See the package
% documentation.
\usepackage[medium]{titlesec}

% Customization chapter headers:
% -----------------------------
% no chapter name in the beginning
\titleformat{\chapter}[hang]{\bfseries\huge}{\thechapter}{2pc}{}{}
% Spacing before and after headings
\titlespacing*{\chapter}{0pt}{12pt}{18pt}{}

% The aalto-thesis package provides typesetting instructions for the
% standard master's thesis parts (abstracts, front page, and so on)
% Load this package second-to-last, just before the hyperref package.
% Options that you can use:
%   mydraft - renders the thesis in draft mode.
%             Do not use for the final version.
%   doublenumbering - [optional] number the first pages of the thesis
%                     with roman numerals (i, ii, iii, ...); and start
%                     arabic numbering (1, 2, 3, ...) only on the
%                     first page of the first chapter
%   twoinstructors  - changes the title of instructors to plural form
%   twosupervisors  - changes the title of supervisors to plural form
%\usepackage[mydraft,twosupervisors]{aalto-thesis}
%\DeclareUnicodeCharacter{00A0}{~}
\usepackage[doublenumbering]{aalto-thesis}
%\usepackage{aalto-thesis}

% Hyperref
% ------------------------------------------------------------------
% Hyperref creates links from URLs, for references, and creates a TOC in the PDF file.
% This package must be the last one you include, because it has compatibility issues 
% with many other packages and it fixes those issues when it is loaded.
\RequirePackage[pdftex]{hyperref}
% Setup hyperref so that links are clickable but do not look different
\hypersetup{colorlinks=false,raiselinks=false,breaklinks=true}
\hypersetup{pdfborder={0 0 0}}
\hypersetup{bookmarksnumbered=true}
% The following line suggests the PDF reader that it should show the
% first level of bookmarks opened in the hierarchical bookmark view.
\hypersetup{bookmarksopen=true,bookmarksopenlevel=1}
% Hyperref can also set up the PDF metadata fields. These are
% set a bit later on, after the thesis setup.

% Add hyperref working if citing only the year (for ISO standards)
\DeclareCiteCommand{\citeyear}
    {}
    {\bibhyperref{\printdate}}
    {, }
    {}
    
% Thesis setup
% ==================================================================
% Change these to fit your own thesis.
% \COMMAND always refers to the English version;
% \FCOMMAND refers to the Finnish version
% ------------------------------------------------------------------
% If you do not find the command for a text that is shown in the cover page or
% in the abstract texts, check the aalto-thesis.sty file and locate the text
% from there.
% All the texts are configured in language-specific blocks (lots of commands
% that look like this: \renewcommand{\ATCITY}{Espoo}.
% You can just fix the texts there. Just remember to check all the language
% variants you use (they are all there in the same place).
% ------------------------------------------------------------------
\newcommand{\TITLE}{Experimentation-driven development and learning in organisations}
\newcommand{\FTITLE}{Kokeilemalla kehitt�minen ja oppiminen organisaatioissa}
\newcommand{\SUBTITLE}{}
\newcommand{\FSUBTITLE}{}
\newcommand{\DATE}{January 7, 2015}
\newcommand{\FDATE}{1. tammikuuta 2015}

% Supervisors and instructors
% ------------------------------------------------------------------
% If you have two supervisors, write both names here, separate them with a
% double-backslash (see below for an example)
% Also remember to add the package option ``twoinstructors'' to the aalto-thesis package 
% so that the titles are in plural.
\newcommand{\SUPERVISOR}{Prof. Marko Nieminen}
\newcommand{\FSUPERVISOR}{Prof. Marko Nieminen}

% If you have two instructors, separate them with \\ to create linefeeds
\newcommand{\INSTRUCTOR}{Satu Rekonen, Lic.Sc (Tech.) and M.Sc.(Econ.)}
\newcommand{\FINSTRUCTOR}{Lic.Sc (Tech.) and M.Sc (Econ.) Satu Rekonen}
 
% Other stuff
% ------------------------------------------------------------------
\newcommand{\PROFESSORSHIP}{Usability and user interfaces}
\newcommand{\FPROFESSORSHIP}{K�ytett�vyys ja k�ytt�liittym�t}
% Professorship code is the same in all languages
\newcommand{\PROFCODE}{T-121}
\newcommand{\KEYWORDS}{Organisational development, learning, creativity, experimentation-driven development, experience of experimenting}
\newcommand{\FKEYWORDS}{Organisaatiokehitys, oppiminen, luovuus, kokeilemalla kehitt�minen, kokeilemalla kehitt�misen kokemus}
\newcommand{\LANGUAGE}{English}
\newcommand{\FLANGUAGE}{englanti}

% Author is the same for all languages
\newcommand{\AUTHOR}{Marianne Tenhula}

% Currently the English versions are used for the PDF file metadata
% Set the PDF title
\hypersetup{pdftitle={\TITLE\ \SUBTITLE}}
% Set the PDF author
\hypersetup{pdfauthor={\AUTHOR}}
% Set the PDF keywords
\hypersetup{pdfkeywords={\KEYWORDS}}
% Set the PDF subject
\hypersetup{pdfsubject={Master's Thesis}}

% Layout settings
% ------------------------------------------------------------------

% When you write in English, you should use the standard LaTeX
% paragraph formatting: paragraphs are indented, and there is no
% space between paragraphs.

% Use this to control how much space there is between each line of text.
% 1 is normal (no extra space), 1.3 is about one-half more space, and
% 1.6 is about double line spacing.
% \linespread{1} % This is the default
% \linespread{1.3}

% Extra hyphenation settings
% ------------------------------------------------------------------
% You can list here all the files that are not hyphenated correctly.
% You can provide many \hyphenation commands and/or separate each word
% with a space inside a single command. Put hyphens in the places where
% a word can be hyphenated.
% Note that (by default) LaTeX will not hyphenate words that already
% have a hyphen in them (for example, if you write ``structure-modification
% operation'', the word structure-modification will never be hyphenated).
% You need a special package to hyphenate those words.
\hyphenation{re-searcher stud-ies con-tex-tual}

% Table of contents levels
%\setcounter{tocdepth}{2}

% The preamble ends here, and the document begins.
% Place all formatting commands and such before this line.
% ------------------------------------------------------------------
\begin{document}
% This command adds a PDF bookmark to the cover page. You may leave
% it out if you don't like it...
%\pdfbookmark[0]{Cover page}{bookmark.0.cover}
% This command is defined in aalto-thesis.sty. It controls the page
% numbering based on whether the doublenumbering option is specified
\startcoverpage

% Cover page
% ------------------------------------------------------------------
% Options: finnish, english, and swedish
% These control in which language the cover-page information is shown
\coverpage{english}

% Abstracts
% ------------------------------------------------------------------
% Include an abstract in the language that the thesis is written in and in Finnish

% Abstract in English
% ------------------------------------------------------------------
\thesisabstract{english}{
Current unpredictable, complex and uncertain business environments require ability from both organisation and its employees to adapt to changes in creative ways that support learning. Together with creative performance, organisations that are able to learn faster than rivals and are thus better at adapting to changes in business environments are claimed to gain better competitive advantage. Therefore need for non-predictive approaches for developing that support learning and growth in organisational and individual level exists.

This thesis studied what kinds of factors affect on experimenting behaviour of an employee and how experimentation-driven development can be supported in organisation. In addition, experimentation-driven development as a tool for learning was studied. 

Case study method was used in the study where client organisation was instructed to apply experimentation-driven approach during a six-week experimentation challenge aiming for employees to create novel ideas to develop their work and rapidly experiment those ideas. To study the factors affecting experimentation behaviour, an interpretive approach together with thematic analysis was used. The data consisted of 14 semi-structured interviews. 

Analysis of the data resulted in two classes: factors having affects on experimentation behaviour of an employee and how experimenting affects an employee. First class consists of five categories including leadership, team, individual and structural perspectives and the gap between an idea and experimentation. Second class consists of two categories: emotional perspective of experimenting and learning. 

Experimentation behaviour is likely to be supported by assuring safe environment for experimenting, supportive leadership behaviour, allocating resources for experiments and carefully designing experiments.

This thesis was done as a part of the two-year MINDexpe research project, undertaken by the MIND research group of Aalto University and funded by Tekes. MIND studies how through experimentation strategic innovations can be created. 
}

% Abstract in Finnish
% ------------------------------------------------------------------
\thesisabstract{finnish}{
Nykyiset ennalta arvaamattomat, monimutkaiset ja ep�varmat ymp�rist�t vaativat sek� organisaatiolta ett� ty�ntekij�ilt� kyky� mukautua muutoksiin luovilla tavoilla, jotka edist�v�t oppimista. Luovan ongelmanratkaisun lis�ksi organisaatiot, jotka oppivat kilpailijoitaan nopeammin ovat parempia mukautumaan yritysel�m�n muutoksiin ja saavuttavan parempaa kilpailuetua. N�in on syntynyt tarve organisaation kehitt�misen l�hestymistavoille, jotka eiv�t pyri ennustamaan tulevaa, vaan tukevat organisaation sek� yksil�n oppimista ja kasvua.

T�ss� diplomity�ss� tutkittiin mink�laiset tekij�t vaikuttavat ty�ntekij�n kokeilevaan k�ytt�ytymiseen ja miten kokeilemalla kehitt�mist� voidaan tukea organisaatiossa. Lis�ksi kokeilemalla kehitt�mist� oppimisen v�lineen� tutkittiin. 

Tutkimuksessa k�ytettiin case-tutkimusta, jossa asiakasorganisaatio tutustutettiin kokeilemalla kehitt�miseen kuuden viikon kokeilukilpailun kautta. Kilpailun tarkoitus oli luoda uusia ideoita ty�n parantamiseksi ja kokeilla niit� nopeasti. Tulkitsevaa tutkimusta yhdess� temaattisen analyysin kanssa k�ytettiin tutkimaan tekij�it�, jotka vaikuttavat kokeilemalla kehitt�miseen. Empiriinen materiaali koostui 14 puolistrukturoidusta haastattelusta. 

Empiirisen datan analyysi johti kahteen luokkaan: tekij�t, jotka vaikuttavat ty�ntekij�n kokeilevaan k�ytt�ytymiseen ja kuinka kokeileminen vaikuttaa ty�ntekij��n. Ensimm�inen luokka koostu viidest� kategoriasta, johtajuuden, tiimin, yksil�n ja rakenteiden n�k�kulmasta. Toinen luokka sis�lt�� kaksi kategoria: kokeilemisen tunnekokemuksen sek� oppimisen. 

Kokeilevaa k�ytt�ytymist� voidaan tukea varmistamalla turvallinen ymp�rist� kokeilemiselle, kokeilemista tukevalla johtamisella, varaamalla riitt�v�sti resursseja kokeilulle sek� huolellisella kokeilujen suunnittelulla. 

T�m� diplomity� tehdiin osana Aalto-yliopiston MIND-tutkimusryhm�n kaksivuotista MINDexpe-tutkimushanketta, jonka rahoittaja oli Tekes. MIND tutkii miten kokeilemalla voidaan synnytt�� strategisia innovaatioita. 
}



% ------------------------------------------------------------------
% Select the language you use in your acknowledgements
\selectlanguage{english}

% Uncomment this line if you wish acknoledgements to appear in the
% table of contents
%\addcontentsline{toc}{chapter}{Acknowledgements}
\chapter*{Acknowledgements}

Acknowledgements here! (acknowledgements.tex) 

\vskip 10mm
\noindent Espoo, <Insert date here> %\DATE
\vskip 20mm
\noindent\AUTHOR


% Acronyms
% ------------------------------------------------------------------
% Use \cleardoublepage so that IF two-sided printing is used
% (which is not often for masters theses), then the pages will still
% start correctly on the right-hand side.
\cleardoublepage

\input{0_acronyms.tex}

% Table of contents
% ------------------------------------------------------------------
\cleardoublepage
% This command adds a PDF bookmark that links to the contents.
% You can use \addcontentsline{} as well, but that also adds contents
% entry to the table of contents, which is kind of redundant.
% The text ``Contents'' is shown in the PDF bookmark.
%\pdfbookmark[0]{Contents}{bookmark.0.contents}
\tableofcontents

% The following label is used for counting the prelude pages
\label{pages-prelude}
\cleardoublepage

%%%%%%%%%%%%%%%%% The main content starts here %%%%%%%%%%%%%%%%%%%%%
% ------------------------------------------------------------------
% This command is defined in aalto-thesis.sty. It controls the page
% numbering based on whether the doublenumbering option is specified
\startfirstchapter

% Add headings to pages (the chapter title is shown)
% Page number is top right, and it is possible to control the rest of the header. No chapter title on header.
\pagestyle{myheadings}

\chapter{Introduction}
\citet{hammer1993reengineering} have summarised aspects of change in organisational environment, beginning from the change in organisational structure; from functional departments to process teams. Work tasks change from simple and detailed tasks to multi-dimensional knowledge work while employees are becoming more autonomous instead of strict control. Furthermore, instead of educating, focus shifts to learning of an employee, and evaluation of work will change from operations to outcomes. Knowledge and capability are preferred over single performance and values change to more productive behaviour than over-protective. Superiors turn from leaders of the work to coaches and hierarchical organisational structures turn lower while managers focus on leadership instead of task management. \citep{hammer1993reengineering} 

Fierce competition for market share and urge for technological innovations have increased the pace of change leading organisations in high pressure to adapt new business environment, rearrange resources, understand and meet new customer and business environment demands. \citep{andriopoulos2000enhancing} Wide access to the information has put tremendous pressure on today's business and companies to increase their efficiency and effectiveness and to develop novel products and processes. Simultaneously, budgets are squeezed and margins of profit grow smaller. \citep{andriopoulos2000enhancing,oldham1996employee} Short time horizons require companies to stay in continuous stream of quarterly profits, oftentimes at the cost of long time benefits. Especially large companies easily favour narrow-minded actions such as quick marketing fixes, cost cutting and acquisition strategies over systemic thinking and process, product or quality innovations. \citep{quinn1985managing}

However, current economy is driven by innovation and innovativeness, requiring new understanding and abilities to generate great ideas \citep{amabile2008creativity} as well as new way of leadership \citep{shalley2004leaders}. Conventional business consists of repetition, risk-avoidance and focusing on business outcomes \citep{buijs2007innovation}, while innovation requires novel solutions, thinking out of the box, risk-taking, breaking the rules, challenging the status quo and questioning the future \citep{burns1961management,kanter1984change,march1991exploration}. 

Understanding change analytically and from systems perspective in the turbulent world appears challenging, with the need of different skills and strategies than before. However, adapting to change and tolerating uncertainty are keys to successful organisation. \citep{senge1990fifth} According to \citet{edmondson1999psychological}, reflection and learning are critical in order to understand the circumstances of increased uncertainty and complexity, pace of change and decreased job security in future organisations. According to \citet{geus1997living} to maintain company's competitive advantage company needs to to learn faster than rivals. Current and future business environment requires continuous learning from organisations, meaning deploying the collective knowledge, skills and creative efforts of their employees \citep{dess2001changing}. 

In addition to company's ability to learn, organisations have begun to value great ideas and demand creative endeavours of employees. Employees who are able to produce those competitive ideas are precious for the current business environment and organisations which strive for innovativeness. \citep{andriopoulos2000enhancing,oldham1996employee} \citet{shalley2004leaders} argues creative employees create competitive advantage in the business field, and various studies recognise creativity influencing on performance and survival of the company across variety of tasks, occupations and industries\citep{hennessey19881,shalley2004leaders,amabile2008creativity}.

Furthermore, studies have well established the positive relation between creativity and innovation skills of an organisation and organisational performance \citep{jung2003role,mumford2002leading}. According to \citet{hennessey19881}, individual creativity stands for an essential building block for organisational innovation \citep{hennessey19881} and is essential in new idea generation and design processes that aim for innovative solutions\citep{sethi2001cross}. The significance of creativity lays in its first step in creating something novel, whereas innovation refers to the implementation phase of the novel ideas in individual, team or organisational level \citep{shalley2004leaders,amabile1996assessing,mumford1988creativity}. 

However, not all jobs require same amount of creativity, yet all organisations benefit from understanding where creativity is required and how it can be fostered and managed \citep{shalley2004leaders}. Likewise, creative actions of an employee are not worthwhile for an organisation when not coordinated or harnessed to yield organisational-level outcomes \citep{jung2003role}. Thus, the future focus should be in organisations' ability to mobilise creative actions of employees to create novel, socially valued products or services and more efficient ways of working \citep{mumford1988creativity}. 

\section{Scope of the study}
Unpredictable, complex an uncertain environments require ability from both organisation and its employees to learn faster than rivals and adapt to changes in creative ways that foster innovation. Need for non-predictive approaches that support learning and growth in organisational and individual level occurs. In this thesis, experimentation-driven approach for development is presented as a method for learning and building competitive advantage in an organisation. Furthermore, factors affecting experimentation behaviour are examined and framework for organisational support for experimenting is created. 

Theoretical part of the thesis forms a synthesis of organisational and individual learning and presents experimenting as a method for developing and learning. Furthermore, thesis presents factors that affect experimentation behaviour in an organisation, forming a picture of optimal environment for experimentation-driven development. When talking about new-value creation, innovation, creativity comes to the topic constantly. Thus, in this study, perspectives of creativity are also presented together with arguments of innovation. 

The thesis was written as a part of a two-year research project called MindExpe studying experimentation-driven innovation at MIND research group, 
Aalto University. MIND operates under the Business, Innovation and Technology (BIT) research centre, which is part of Department of Industrial Engineering 
in Aalto University School of Science. MIND research group is based on Aalto Design Factory. Tekes-funded MINDexpe project studies innovation and development 
in established organisations through experimentation-driven approach. 

The research team instructed a client organisation on using experimentation-driven approach by organising an experimentation challenge where the units of the 
client organisation were tasked to create, develop and report new ideas to develop their work during a six-week time period. Instructions were given before the challenge; during the challenge further instructions were not provided. 

\section{Research objectives}
In MINDexpe client organisations are tasked to use the experimentation-driven approach instead of more traditional planning-based approaches to development. 
This thesis aims to discover how various organisational conditions may affect the experimentation behaviour. The larger aim of the MINDexpe project is to 
widen the understanding of experimentation-driven innovation itself.

The motivation for this study was to reveal factors affecting experimentation behaviour in organisations. In addition the aim was to study
the experimentation-driven approach as a method for learning in organisations and identify how this approach could be supported in an organisation culture. More specifically, research questions that this thesis aims to answer are as follow. 

\begin{enumerate}
  \item How can experimenting support organisational learning?
  \item What kinds of factors affect on experimenting behaviour of an employee? 
  \item How can experimenting behaviour be encouraged in organisations?
\end{enumerate}

The first research question is mainly answered through theoretical research and complemented with empirical findings. Second question is answered through theoretical research and empirical findings. As experimentation-driven development has not been widely studied, important findings from interviews on experimenting in an organisation are gained. To support these empirical findings, literature on creativity and innovation are researched in order to form understanding of the third research question.  

\section{Motivation for the study}
The key motivation for Mind research group is to study how and why some business ideas or businesses work better than others, how new-value can be created and strategic innovations emerged. Mind approaches these broad questions through three agendas. First of all, in order extraordinary innovations to emerge, great ideas are needed. Thus, in the interest of Mind is to find methods and tools for improving the quality of ideas. 

Leader of the research group Anssi Tuulenm�ki states how new value cannot be planned, it needs to be developed through experimenting. Thus, second agenda of Mind is to study experimentation-driven development and its impacts on organisational and individual level. Experimenting is mainly described and used as 
a tool for developing, creating something new. This thesis focuses on this second agenda of Mind group, and deepens the understanding of how experimenting can be used as tool for learning and creates a synthesis on organisational conditions in which experimenting is likely to happen. 

Third agenda of Mind relates to organisational structures and networks, aiming to understand the essence in structures and utilise that to create the most simple organisational structures that support business. 

\section{Methodology}
In this thesis research design is based on case study method. Client organisation Service Foundation for People with Intellectual Disabilities (KVPS) were interested in applying novel approach towards developing in their organisation, and through participating in the MINDexpe research project KVPS experienced experimentation-driven approach in action while MINDexpe benefitted from the real life research context. Case study refers to a method dealing with contemporary phenomenon in real life context and is ideal in studies where boundaries between the phenomenon and its context cannot be clearly delimited. \citep{yin2014case} When studying factors affecting experimentation behaviour in organisations, real life context is significant, yet boundaries of the work and the context of developing remain complex and unclear. 

Researchers of the Mind group organised together with the management of KVPS a six-week experimentation challenge to employees of KVPS in order to generate novel ideas to improve work and test them in action. Researchers gave very brief introduction poster on experimentation and instructions for the challenge, further instructions during the challenge were not provided. 

In the analysis process thematic analysis was used \citep{braun2006using} as a method for analysing the data. The data consists of 14 semi-structured interviews of client organisations members from five different units. The analysis of the data demonstrates various requirements for supporting experimentation behaviour in developing in an organisation. Furthermore, the data revealed that experimentation behaviour has various affects on an individual's performance and the way an employee experiences his work.

Research methodology and study surrounding are introduced further in \ref{reserachdesign}. 

\section{Structure of the thesis}
First chapter briefly introduces background, research objectives and motives for the thesis as well as methodology used. 

This thesis consists of theoretical and empirical part. Chapters 2, 3 and 4 form the theoretical basis for the thesis, and chapter 5 and 6 present the empirical part of the thesis. In current and future organisations in order to create competitive advantage, focus will be on organisations who learn faster than rivals. Additionally, creative ideas of employees has been related to improve competitive advantage for companies. Thus, in the second chapter learning as a tool for an organisation to be better than rivals is presented, together with introduction to creative aspects. 

In the chapter 2, current change in business environment is described in order to form understanding of the need for novel approaches towards development and new-value creation. Through innovations new business and competitive advantage is created, and thus aspects for innovation are outlined. Furthermore, organisational learning and supporting conditions for learning are described. Behind every successful innovation, product or service stands an individual employee or a team with a great idea, so individual and team perspectives on learning are described. Furthermore, in order great ideas to emerge, organisational conditions need to support individual learning, creativity and innovation processes. These aspects are presented in the end of the first chapter. 

Chapter 3 presents experimentation-driven approach for development and learning. Understanding of experimenting and experimentation process is formed, which is the focus at Mind research group. Furthermore, this chapter provides insights on occasions when experimentation-driven approach should be adapted as a way of developing and creating something new. Experimentation-driven developing works best when uncertainty is high and under development is a process with many unfamiliar factors. Experimenting stands as a method for learn on the way of the development process and through iterative experiments and reflections better products, services or ways of working are formed. 

Chapter 4 outlines factors affecting experimentation behaviour in organisations based on literature on innovation, creativity, and organisational organisational management and behaviour. It provides understanding how through organisational conditions creative actions of employees, willingness to conduct experiments and courage to say out ideas can be fostered. 

Chapter 5 presents the research design, including surroundings, case company description and methodology used in the study. It clarifies the experimentation 
challenge organised for the case company, explains data gathering methods and sheds light on the analysis process. 

After this, in chapter 6, the results of the data are presented. Two main concepts were recognised from the data: factors affecting experimentation behaviour and effects 
experimenting has on individual. 

Chapter 7 consists of the discussion, where implications of the results are analysed. Furthermore, both theoretical and managerial
implications of the study are being evaluated as well as suggestions for future research. Lastly, reliability of the thesis is analysed. 

The final chapter of the thesis consists of the brief conclusion drawn from both the theoretical framework, empirical results and discussion. 


\chapter{Change and learning in an organization}

Current and future business environment requires continuous innovation from organisations, meaning deploying the collective knowledge, skills and creative efforts of their employees \citep{dess2001changing}. Wide access to the information has put tremendous pressure on today's business and companies to increase their efficiency and effectiveness. Simultaneously, budgets are squeezed and margins of profit grow smaller. Concurrently, value of great ideas and demand for creative endeavours have rose in order to improve and develop products and processes. Employees who are able to produce those competitive ideas are precious for the current competitive business environment and organisations which strive for innovativeness. \citep{andriopoulos2000enhancing,oldham1996employee}

Furthermore, technological change, as hectic as rapid it has been, cannot be ignored. Fierce competition for market share and urge for technological innovations have increased the pace of change leading organisations in high pressure to adapt new business environment, rearrange resources, understand and meet new customer and business environment demands. \citep{andriopoulos2000enhancing}

Today, economy is driven by innovation and innovativeness, requiring new understanding and abilities to generate great ideas in order to survive in every business and business level. innovativeness calls for creativity, which again calls for new managing skills of leaders. In contrast to many leaders beliefs, creativity and creative individuals can be managed and encouraged. \citep{amabile2008creativity} 

Hand in hand with innovation comes creativity and employees' ability to be creative in their work \citep{hennessey19881,shalley2004leaders}. Organizations have realised the value of creativity across a variety of tasks, occupations and industries. Dynamic and quickly changing work environment requires new skills and approaches from managers. For instance, they need to motivate and involve employees in various ways in order to foster creativity and innovation that may lead to competitive advantage in the business field. Individual creativity forms the base for organisational creativity and innovation \citep{hennessey19881}, which has been realised to have influence on performance and survival of the company \citep{nystrom1990organizational}. However, all jobs do not require same amount of creativity, and in all jobs the weight of creativity is not as important, yet all organisations benefit from understanding where creativity is really needed and how it can be fostered and managed. \citep{shalley2004leaders} 

Although currently creativity and creative processes of an individual at work are rather well recognised and essential, even more focus should be put on organisations' ability to mobilise creative actions of employees to create novel, socially valued products or services and more efficient ways of working \citep{mumford1988creativity}. In other words, creative actions of an employee are not worthwhile to an organisation when not coordinated or harnessed to yield organisational-level outcomes \citep{jung2003role}.

\citet{hammer1993reengineering} have summarised aspects of change in organisational environment, beginning from the change in organisational structure; from functional departments to process teams. Work tasks change from simple and detailed tasks to multi-dimensional knowledge work while employees are becoming more autonomous instead of strict control. Furthermore, instead of educating, focus is in the learning of an employee, and evaluation of work will change from operations to outcomes. Knowledge and capability are preferred over single performance and values change to more productive behaviour than over-protective. Superiors turn from leaders of the work to coaches and hierarchical organisational structures turn lower while managers focus on leadership instead of task management. \citep{hammer1993reengineering} 

\section{Learning as competitive advantage and way to change and survive}

According to \citet{geus1997living} the only one can maintain company's competitive advantage is to make sure the company is able to learn faster than rivals. Generally organisations are considered as machines, yet recently more emphasis has been put on organisations as living organisms. When considered as machine, organisational model is mechanic and simple, which purpose is to gain profit. Whereas, organisation as a living organism is a whole-systemic model, and organisations are considered as place which has deeper, permanent meaning offering people the opportunity to grow and fulfil themselves while earning money. Liable vision of the future focuses on the latter perspective of organisations, where learning and renewal form the essence of being.  \citep{geus1997living}

Learning is essential part of developing and innovation; according to \citet{buijs2007innovation} all innovation processes are processes for organisational learning. Also \citep{quinn1985managing} argues how especially from the management perspective major innovations should be considered as incremental and interactive learning processes that is driven by certain goal. 

Understanding change analytically and from systems perspective in the turbulent world appears challenging. Change being hectic and fast calls for different skills and strategy than before. Only when change is understood can it be managed, and in order to survive new perspective and understanding towards change is required from an organisation. 
In the changing environment tolerance for uncertainty is needed, and while future can not be predicted, forecasting is a usable method in order to cope with the anxiousness resulting from the uncertainty. Furthermore, even more emphasis should be put on ability to learn and adapt to changes. \citep{senge1990fifth}

According to \citet{edmondson1999psychological}, learning behaviour is an essential concern in this fast-paced working environment, where organisational change and complexity are increasing. In order to understand what is going on and what actions to take, reflection and learning are critical. Defining concepts of today's business and working environment are increased uncertainty and change as well as decreased job security in future organisations. Thus, in order teams to function under ambiguous circumstances, they need to feel safe and are actually in a position to provide required safety to each others, with the help and support of managers and leaders. Today's environment require that team members do not fear asking questions, seek help and tolerate mistakes. \citep{edmondson1999psychological}

\section{Organizational learning}

Learning is essential part of developing and innovation; according to \citet{buijs2007innovation} all innovation processes are processes for organisational learning. Through the understanding of organisational learning real growth and support for learning can be offered. Learning process needs to be understood all from organisational, team and individual perspective \citep{buijs2007innovation}. 

Organizational learning is approached from two different perspectives in literature. On the one hand, learning is considered as an outcome, and on the other it is considered as a process \citep{edmondson1999psychological}. In the first perspective organizational learning is referred to be "an outcome of a process of organisations encoding interferences from history into routines that guide behaviour" \citep{levitt1988organizational}, whereas process perspective defines learning as a process of continuous trial and error \citep{argyris1978organizational}. In this thesis, learning is considered as the latter tradition of learning, which allows the growth and improved performance of individuals and organisations. \citet{edmondson1999psychological} presents and studies behaviours through which various outcomes of learning as adaptation to change, understanding or improved performance are likely to be achieved. Furthermore, \citet{edmondson1999psychological} apply the term learning behaviour to separate it from learning outcomes, and states how set of several activities form the basis of learning behaviour. 

Educational philosopher John Dewey has conceptualised learning as a process in his writings about inquiry and reflection \citep{dewey1956human}. His wok has influenced remarkably on following learning theories, such as experiential learning theory \citep{kolb1984experiential} or action approach of organisational learning \citep{schon1983reflective}. According to \citet{dewey1956human} learning is an iterative process consisting of designing, carrying out, reflecting upon and modifying actions. Dewey separates learning from humans' tendency to behave habitually or automatically. \citet{edmondson1999psychological} builds to this definition focusing on the group level of learning and defining it as an ongoing process where reflection and action occur. Integral characteristics of learning process are asking questions, seeking for feedback, performing experiments and reflecting on the results, having discussions about error and surprising or unexpected outcomes of actions. In group level, learning is enabled through testing assumptions and discussion of opinion differences transparently in order to improve team performance. \citep{edmondson1999psychological} 

Organizational learning differs from individual or team learning. Organizational learning occurs through the shared knowledge, insights and approaches of the employees of an organisation. Secondly, organisational learning is based on prior knowledge and experience, the memory of organisation, which consists of the ways of working, processes and instructions of an organisation. Even though individual and team learning are highly related to organisational learning, it is not the sum of the previously mentioned. \citep{sydanmaanlakka2007}

Various definitions of learning organisations have been presented. One of the most famous definitions is from \citet{senge1990fifth}, who describes learning organisation as follows: "Learning organisation is an organisation, where people are able to constantly develop and achieve intended results; where new ways of thinking are born and where people share goals and learn together."

Together with analysis, limitation, regeneration and technological change, learning is essential factor in improving organisational performance and strengthening competitive advantage. According to \citet{march1991exploration} each pace involves exploitation and exploration as well as adaptation. Exploitation refers to refinement and extension of competences, technologies and paradigms that already exist, whereas exploration is about experimentation with new approaches and alternatives. When results and returns of exploitation are oftentimes positive, proximate and predictable, returns of exploration are uncertain, distant and usually negative. Therefore, exploration leads to greater locus in learning and realisation of problems than exploitation, when considered the distance in time and space. \citep{march1991exploration}

Accordingly in management literature learning is considered relating and even being dependent on receiving feedback \citep{schon1983reflective}, discussion and failure \citep{sitkin1992learning} and experimenting \citep{henderson1990architectural}. As relevant information about performance is acquired through errors, discussion about them has been related with organisational effectiveness \citep{sitkin1992learning}. According to \citet{huy2003rhythm} organisations learn best through small experiments and trying out new things, and the closer and more related experimentations are to customers and customer interfaces, the more can be learned. 

Edmondson's psychological safety has roots already in early research on organisational change. \citet{schein1965personal} state that in order individuals to change and feel safe they need psychologically secure environment. However, team psychological safety should not be confused with groupthink effect that refers more to group cohesiveness, which seems to be related to decreased willingness to disagree and challenge team member's views and thus reduces interpersonal risk-taking \citep{janis1982groupthink}. Term psychological safety refers to team's confidence and shared belief and mutual trust among team members towards that speaking up in a team does not lead to embarrassment, rejection of punishment of any kind \citep{edmondson1999psychological}. 

\section{Team performance and learning}

According to \citet{schein2010organizational}, concept of culture refers to and helps to explain some seemingly incomprehensible and irrational aspects of what is going on in groups, organisations and other kinds of social units, that share history. \citet{schein2010organizational} divides culture into three levels: artefacts (visible and feelable structures and processes and observed behaviour), espoused beliefs and values (ideals, goals, aspirations, ideologies and rationalisations) and basic underlying assumptions (unconscious, taking-for-granted beliefs and values). Climate of the group should not be mixed with culture of the group, it should rather be considered among artefacts. However, essential point of view \citet{schein2010organizational} provides is how culture in organisation or group level is easy to observe yet very difficult to decipher. Put in other words: researchers are able to observe and make remarks on what they see and feel, yet they are unable to reconstruct the deeper meaning of those observations to the group. Cultural analysis and understanding of dynamics of a group should begin in observing and asking members the norms, values and rules that shape practicalities of work in day-to-day level.

In order to create new value and competitive advantage in rapidly changing and uncertain organisational environments, new managerial imperative is growing focusing on teams. Thus, supporting teams in their work and understanding the aspects of team learning is required \citep{edmondson1999psychological}. Recent studies has moved the focus from individual learning to team learning. Edmondson's definition of group learning stems with definition of \citet{argote2001group}, who emphasises that knowledge is acquired, shared and combined through processes an outcomes of group interaction, focus being on processes. 

Work team refers to small group of people that exist within the context of a larger organisation, members share understanding of being member of the team and its tasks, responsibility for product or service team is working on \citep{hackman1987design}; \citep{alderfer1983intergroup} as well as its performance \citep{edmondson1999psychological}. Additionally, team members have supplementary knowledge and abilities compared to each other, and they share a goal, targets and way of working and approach \citep{edmondson1999psychological}. According to  \citet{katzenbach1993wisdom} great team performance consists of continuos work of shaping a common purpose, agreeing on performance goals, defining a common working approach, developing high level complementary skills and being transparent on the results. He emphasises that through disciplined action groups transform to teams and argues how demanding schedules, long-standing habits and unwarranted assumptions tend to threaten team efficiency and performance.
 
In previous research, structural and design-related factors have been combined to have influence on work teams's effectiveness and team performance. Well-designed tasks and goals, suitable and functional team composition, as well as physical environment and practices ensuring transparent communication and information exchange, sufficient materials, resources and motivating rewards all affect team efficiency. \citep{hackman1987design}; \citep{goodman1988groups}; \citep{campion1993relations}. Along with these factors, leader behaviour plays a major role in enhancing team effectiveness, and can be facilitated for instance through coaching and setting directions to employees (Hackman 1987) \citep{hackman1987design}. This perspective explains teams effectiveness through organisation and team structures, whereas organisational learning research puts emphasis on cognitive and interpersonal variables when explaining effectiveness in teams and individuals \citep{edmondson1999psychological}. For instance, \citet{argyris1993knowledge} has argued how individual's negative beliefs about communication and interaction may inhibit learning behaviour and lead to ineffective working in an organisation. 

In addition, in order to function team needs a clear purpose and vision what makes it a team and why it exists. Teams get energy from significant performance challenges regardless of where they are in the organisation. Set of shared, demanding performance goals usually form a team, and personal chemistry or willingness to form a team may boost that. Thus, in order to receive great results teams should focus on performance regardless of the organizational hierarchy or what team does. Thus, team performance may exceed the results of what could be achieved if employees were acting alone as individuals without the team effort. \citep{katzenbach1993wisdom}

\citet{edmondson1999psychological} have studied factors that affect and influence learning behaviour in teams by studying in which conditions and to what extent learning occurs naturally. Learning behaviour of teams refers to activities that team members carry out and through which team is able to obtain, adapt and reflect data and outcomes of actions which further shapes and improves team behaviour. Such activities consist of reflection and improvement-aiming factors such as asking for feedback, transparent information sharing, asking for help, admitting and discussing about failures and errors as well as experimenting. Through such activities teams may observe changes in environment, customer requirements and improve collective understanding. In addition, team's ability to discover and react to unexpected situations and consequences of their actions is likely to improve through learning behaviour.  Consequently, compared to low-learning teams that tend to get stuck and be unable to solve problems, teams who master in learning are greater in confronting difficult situation and improve their work. \citep{edmondson1999psychological}

The composition of the team matters. Studies have shown how team performance, especially related to innovation, is improved when team consists of individuals with various and different set of skills and characteristics\citep{buijs2007innovation}. Homogeneity in teams easily leads to groupthink, routine work and repeating traditional daily practices, while even one or two different individuals can stimulate the innovativeness of a team, and actually, the outcasts and those who stand out from the group are required in order to think outside the box, challenge the status quo and present alternative solutions and ideas that would be missing without the participations of these individuals. \citep{sternberg1997creativity}

Question of team composition comes relevant especially when forming teams for innovation. According to\citet{buijs2007innovation} innovation teams are the heart and the engine of innovation process and essential for the rest of the organisation to accept changes and innovation results. As he suggests right people in the team is the premise for innovation, all the members should be chosen carefully starting from the leader. Furthermore, the leader should be allowed to affect on the formation of the rest of the team in order to ensure positive base for teamwork and innovation process. Accordingly, team membership should be based on voluntary. \citep{buijs2007innovation}

\subsection{Psychological safety in teams}

Trust, indeed, has been widely noted in research as essential factor in organisational teams and groups \citep{golembiewski1975centrality,kramer1999trust,shalley2004leaders,edmondson1999psychological}. Trust refers to one's willingness to be vulnerable in his actions as he expects his actions will not be judged and will be favourable to one's interests \citep{robinson1997corporate}. Interpersonal trust is involved in psychological safety, yet it also includes perception of mutual respect and overall climate where team members feel free to be themselves \citep{edmondson1999psychological}. 

Team psychological safety refers to Amy Edmondson's concept of team members shared belief team being safe for interpersonal risk-taking. Together with team efficacy these have great affect on team performance and learning in an organisational team work. \citep{edmondson1999psychological} integrative perspective suggests that team performance and outcomes can be shaped through both team structure and shared beliefs, in contrast to previous studies that separate structural and interpersonal factors from each other. For instance, employee's willingness to take interpersonal risks depends highly on the experience of team safety and person's beliefs how others will respond in ideas or situations involving uncertainty. Team psychological safety refers to interpersonal trust among team but beyond that mutual respect and caring. \citep{edmondson1999psychological}

As in changing and uncertain environments the importance of teams have been recognized, pressure on managers to understand and enhance team efficiency, work and learning has increased. Fast-pace environment requires organisations to enhance the ability of teams to learn and create environment where learning can occurs safely. While uncertainty affects all working life, change is faster and job security lower, psychological safety for individuals at work can be increased through great teams and teamwork. Employees should be engouraged to ask questions, seek for help and tolerate mistakes and uncertainty. \citep{edmondson1999psychological}

Psychological safety serves as a mechanism that assists in explaining how structural and interpersonal characteristics both have effects on learning and performance in teams \citep{edmondson1999psychological}. Psychological safety can be boosted for instance through structural factors such as context support and team leader coaching affecting behavioural and performance outcomes \citep{hackman1987design};  \citep{edmondson1999psychological}. Furthermore, climate of safety and supportivenesss encourages employees to seek for feedback and ask for help in addition to admit and reflect mistakes. \citep{edmondson1999psychological}

Communication about ideas among team has been widely recognized being related to idea generation, creativity and innovation (e.g.\citep{robinson1997corporate,mumford2002social,monge1992communication,amabile1996assessing}). According to \citet{staw1989tradeoff} social influence of others plays a major role for individuals' beliefs; attitudes towards job, for instance, rise from the social labelling of work by others.  Also \citet{salancik1978social} argue the essential role opinions of others may have on individual: individual's perception of her work and organisation can be greatly influenced by opinions of others. Additionally, team member's collective view of support they get from their leader has been related to the team's creative endeavours and success in them (e.g., \citep{amabile1998kill,amabile1996assessing}). 
 
Thus, as individuals oftentimes requires support and input from several invidivuals who help to challenge ideas in constructive ways, teams are essential in generating and implementing ideas \citep{mumford2002social}. Stimulating those constructive individuals for creative actions may be valuable \citep{robinson1997corporate}. In addition, including team members in ideation assists in idea implementation and through participation new ideas are not that likely to be rejected or abandoned \citep{agrell1994team}.

Leaders have a great role enabling creative behaviour in teams and individuals, yet team members also influence essentially in others. Thus, by utilising various human resource practices leaders should create an environment where creativity is encouraged and supported. \citep{shalley2004leaders} Study of \citet{ancona1992demography} argue how changing the structure of teams is not sufficient and does not lead to improved performance, rather the leader and the team should find ways to foster positive effects of the team processes and reduce the negative ones. At team level this may mean focus on enhancing negotiation, problem-solving and conflict resolution skills while at organisational level leader should protect the team from external political pressures and reward the team from performance outcome instead of functional ones. \citep{ancona1992demography}

\section{Individual learning}

Various perspectives and definitions for learning has been studied, presented, analysed and utilised in order to understand individual's process of adapting new information and skills. Experiential learning theory refers to learning as a process of knowledge-creation through experiences while experiential learning process stands as a way to describe the central process of human adaptation to the social and physical environment - a holistic adaptation process that provides bridges across life situations and underlaying the lifelong process of learning. \citep{kolb1984experiential}. Also \citet{jung1923psychological} argues how learning involves concept of human being as a whole - from feeling and thinking to perceiving and behaving.

However, everyday problem-solving and immediate reactions to situations at hand are oftentimes related to performing instead of learning. Furthermore, long-term adaptations to our previous experiences and beliefs is mainly considered as developing, not learning. Yet, when talking about development and developing in individual, team or organisational level, the question highly concerns and is related to learning. \citep{kolb1984experiential}

Experiential learning theory of \citet{kolb1984experiential} consists of four elements: experience, perception, cognition and behaviour. Immediate experience forms a basis for reflection and observation, following assimilation to a theory from which new implications for action are deducted. In order to create new experiences, these implications serve as guides. Overall, experience of an individual is a focal point of learning, and giving personal meaning to abstract concepts, which can be afterwards shared with others. Furthermore, receiving feedback is considered essential in this approach for learning, as it serves a continuous process for goal-oriented action following evaluation of that action. Feedback can thus boost effective, goal-oriented learning process. \citep{kolb1984experiential}

Continuing with the model of \citet{kolb1984experiential}, instead of conceiving learning in terms of outcomes, it should rather be conceived as a process. Ideas are not fixed and and immutable elements of thoughts, but can be formed and re-formed through experience. Furthermore, bringing the experiential learning into educational implications, all learning can be considered as relearning. Thus, all learning situations should take into account people arriving from all different experiential backgrounds to what they build their new experiences and knowledge on. This partly explains very likely resistance to new ideas, as when new information and experiences are in contradiction to old beliefs and experiences, new ideas and information is more difficult to adapt. In the education process learner's old beliefs and theories should be brought out, examined and tested, following integration of the new models and refined ideas into learner's belief systems. \citep{kolb1984experiential}

\citet{kolb1984experiential} presents Piaget's interactive process approach to learning, according to which individual learning and adaptation of new ideas occurs through integration or substitution. Integration leads to stronger part of learner's conception of the world, whereas substitution requires real questioning of previous conceptions, and thus might take longer for the learner to adopt. Learning is a mutual process between accommodation of concepts or schemas to experiences around us and assimilation of events and experiences into existing concepts and schemas. This intelligent adaptation, learning, results from the tension between accommodation and assimilation. Through this tension growth and higher-level cognitive functioning occurs. According to \citet{kolb1984experiential}, learning is a process filled with tension and conflict, and new knowledge, skills and attitudes are achieved through experiential learning, which consists of four modes and required abilities of learners: concrete experience abilities, reflective observation abilities, abstract conceptualisation abilities and active experimentation. First of all, individuals must openly involve themselves in new experiences, reflect and observe them from various perspectives, create concepts that can be integrated into more abstract theories as well as they need to be able to use these reflections and theories in active daily decision-making and problem-solving.  

Meaning of environment in learning should be emphasised. Learning concerns of transaction between an individual and the environment, learning does not happen only inside of individual's thoughts, experiences and processes but is dependent on the real world environment. Understanding of different learning styles and modes assists in supporting individuals in learning and problem-solving.

\chapter{Innovation}

Why innovation and creativity should be a matter for an organisation? Studies have well established the positive relation between creativity and innovation skills of an organisation and organisational performance \citep{jung2003role,mumford2002leading}. Innovation and creativity are highly related, yet not the same thing: According to \citet{hennessey19881}, individual creativity stands for an essential building block for organisational innovation and also \citet{sethi2001cross} argue creativity being essential in new idea generation and design processes that aim for innovative solutions. Other studies also emphasise the role of creativity a first step in creating something novel, whereas innovation refers to the implementation phase of the novel ideas in individual, team or organisational level \citep{shalley2004leaders};\citep{amabile1996assessing};\citep{mumford1988creativity}. 

Innovation process itself can be approached from several angles: first of all, content of the innovation has to be clear - whether the purpose is to innovate new products, manufacturing processes, ways of organising or ways of dealing with people. Secondly, psychological process of the innovation team has to be understood, essential being shared understanding, level of comfort with ambiguity and degree of trust between team members. Thirdly, creative process of the team, meaning idea producing process, needs to be understood and efficiently facilitated. Finally, the role of leading plays a major role, and together with playful attitude innovation process is likely to succeed. \citep{buijs2007innovation}

Several factors have been recognized to affect on organisational innovation, yet many researchers have stated leadership behaviour being one of the most important. \citep{jung2003role,amabile1998kill,jung2001transformational,mumford1988creativity} \citet{jung2003role} identify four hypotheses how top managers' leadership styles may affect both directly and indirectly their companies' ability to innovate. Indirectly here stands for instance leader's possibility to empower employees and build organisational climate optimal for innovation. The study shows a positive relation between transformational leadership style and empowerment as well as innovation-supporting organisational climate. 

Social innovation refers to the generation and implementation of novel ideas concerning people in demand to organise their interpersonal or social activities and interactions in new ways in order to achieve common goals. Results and products of social innovation, like other types of innovation, are likely to vary depending on the breadth and impact of the innovation. \citep{mumford1988creativity} \citet{mumford2002social} presents four factors affecting social innovation: active exchange of ideas and information in supportive climate, tangible and low-cost ideas that can be at the fewer guessed to be beneficial, support from upper level management, and effective communication through the innovation process in order to proceed from the idea to implementation.

Innovation can be contributed by encouraging idea generation, but also creating a climate of autonomy, offering intrinsic and extrinsic rewards and engaging employees with their work \citep{amabile1996assessing}; \citep{amabile1998kill}. Furthermore, characteristics associated with innovation are integration of work units, decentralisation of control and professionalisation are likely to effect innovation in a way that through these suitable environment for innovation, dynamic idea exchange and implementation is created \citep{mumford2002social}.

Managerial practices for technological innovations have been widely studied. According to \citet{quinn1985managing} the essence lays in accepting the chaos of development. In addition, large and successful companies and their leaders listen carefully their users' needs, develop according this customer demand, define clear goals and framework for the work, encourage teams to challenge the status quo and find alternative solutions while avoiding detailed and technical or marketing plans in the beginning. Instead, they focus on early prototyping and iteration. 

According to studies creativity and innovation in an organisation requires integrated organisational approach, right climate, appropriate incentives for innovators, and a systematic way and resources to transform an idea into an innovation.  In individual level, creativity and innovation calls for various skills, such as teamwork, communication, coaching, project management, learning and learning to learn, visioning, change management and leadership, and ability to develop these skills. Oftentimes, even though the climate and practices are right for generate innovations, problems are faced when attempting to manage the change process. \citep{roffe1999innovation}

Social cohesion may inhibit innovativeness of the team and its individuals especially beyond a moderate level, while employees are more likely to settle on group think and traditional daily practices \citep{janis1982groupthink}. However, according to the study of\citet{sethi2001cross} when a team shares superordinate identity, is encouraged to take risks, lets customer's requirements be heard, and actively lets senior management monitor the project, team is more likely to present innovative ideas and perform in innovative ways. According to this study, functional diversity does not effect on innovativeness, but team's superordinate identity can be strengthened by encouraging risk-taking and weakened by social cohesion.

\citet{mumford2002social} argues in his study of Ben Franklin's social innovations, that key factor in successful social innovation lays in fast demonstrating, which he also refers as experimenting. Thus, in order to drive for social innovations, opportunism and showmanship of an individual or team may be required. \citep{mumford2002social} Furthermore, according to \citet{monge1992communication} group communication is likely to increase innovation under some circumstances,  and also \citet{katzenbach1993wisdom} argues for culture of strong team performance. However, \citet{amabile2004leader} emphasise how, ultimately, truly novel ideas raise from individuals, making them the ultimate source of any new idea or solution to a problem \citep{amabile2004leader}.

\section{Creativity, intrinsic motivation and everyday problem solving}

Creativity and innovation have gained wider acceptance as important factors creating value in organisational performance \citep{mumford2002leading}. Creativity and innovation have for instance been studied to have enhancing impact to organisations profit and growth \citep{nystrom1990organizational}. Creative thinking and actions require time, and contradictory, in fast-paced and rapidly changing world and working environment managers should allocate employees sufficient time for creative thinking and experimenting novel approaches \citep{shalley2004leaders} 

\citet{amabile1996assessing,amabile1998kill} defines creative thinking as a way how people approach problems at hand and come up with solutions. Creative thinking does not stand for intellectual capacity of an individual to create something new but rather as a combination of past experiences which creates expertise and the ability to apply creative thinking skills to these experiences and invent new solutions. 

According to \citet{sternberg1997creativity} a company can enhance its creative skills by focusing on six resources: knowledge, intellectual abilities, thinking styles, motivation, personality and environment. \citet{sternberg1997creativity} argues that too much information may hinder change and be seen as rigidity in thinking. Therefore, one should not over-weight the criticism of senior people in an organisation, and at least consider the chance for rigid thinking and intolerance for change. 
In addition, needs to be noted and understood that employees' thinking styles are shaped through what is rewarded, meaning, that if organisational environment rewards well-behaving and instruction-following thinking style and action, employees tend to implement their style to that. We are urged to adapt to organisational style and fit in, and when this is not possible, people tend to leave. \citep{sternberg1997creativity}

Divergent thinking refers to individual's ability to find multiple alternative solutions and ideas to problems at hand, and has been related to serve as a key capacity affecting creative thinking \citep{guilford1967creativity}. Accordingly \citet{mumford1988creativity} emphasis that creative people consistently and with confident tend to seek for alternative solutions, even under uncertain conditions. Even though expertise and  intelligence have been related to problem solving, series of causal analyses carried out by \citet{vincent2002divergent} revealed unique effects divergent thinking had that were not attributed to intelligence and expertise. 

\citet{shalley2004leaders}, in turn, argue that through developing extensive set of skills, employees may learn to be more comfortable and confident in thinking from different perspectives, finding various alternative solutions, trying out novel things and seizing opportunities. According to \citet{hennessey19881} individual creativity requires ability to think creatively, generate alternatives, engage in divergent thinking and tolerate or suspend judgment. Through this perspective creativity can be considered as a skill that can be learned and strengthened. Understanding of individual's creativity and ways to influence and improve it gives managers guidelines when creating an environment and leadership that support organisational innovation \citep{redmond1993putting}.

Several factors form the basis of creativity skills of an individual, such as personality, technical knowledge, expertise, motives, and the supervisor's feedback style. In group level factors form of task structure, communication styles and task autonomy, and finally in organisational level strategy, structure, culture, climate and available resources all affect how creative actions are encountered. \citep{jung2003role}

Several case studies has showed that creativity insights emerge gradually through the network and actions of an creative individual. Study of creativity is a combination of two different disciplines and research approach: sociological and historiometric lenses study the conditions in which creative actions and processes are likely to occur, whereas neurobiological approach presents neural structures and processes that are active and associated with creative outcomes. \citep{gardner1988creativity}

Generation of novel, alternative solutions requires problem-finding skills \citep{runco1988problem}, which has been indicated to be one of the best predictors of creativity in 'real world' activities, when studied 91 elementary school students \citep{runco1990evaluating}. These findings suggest leaders, in order to enhance creativity of employees, to support learning of these skills for instance by facilitating problem-construction \citep{redmond1993putting}.

\citet{kasof1997creativity} argued in his study that breadth of attention affects on creative performance of an individual: wide spread of attention is usually related to creative ability. By breadth of attention Kasof refers to "number and range of stimuli attended to at any time." Breadth of attention being narrow, individuals are able to focus on narrow range of stimuli and are better at filtering redundant stimuli from awareness. However, those individuals with wide breadth of attention tend to be more aware of irrelevant or extraneous stimuli, these individuals are strongly affected by their environment and are highly arousable.\citep{kasof1997creativity}

Studies of creative characteristics of individuals has revealed factors such as wide interest in various fields, autonomy, belief of being creative and independence in decision-making \citep{shalley2004leaders}. Intrinsic motivation is claimed to be one of the most powerful tools to creative action and non-traditional thinking \citep{amabile1996assessing,deciintrinsic,jung2001transformational}, as intrinsically motivated individuals usually prefer novel solutions, challenging status quo and trying out new ways for solving a problem at hand \citep{amabile2002creativity}. Broad interest stands for a sign of intrinsic motivation, which is also widely related to both creativity and well-being of an individual and innovation (e.g. \citep{hennessey19881}; \citep{csikszentmihalyi199916}; \citep{gardner1988creativity}). In their study  \citet{tierney1999examination}, found positive correlation between employee's level of enjoyment while working on a creative task at hand and the level of creativity.  

Without previous experience of the job routine and substance knowledge and expertise on the field creative endeavours are more rare. Even though has been argued how routine work and task familiarity is likely to lead very habitual performance \citep{ford1996theory}, knowing the status quo may provide opportunities for creative actions through reflecting and practicing skills and activities requires in the field. \citep{shalley2004leaders} Knowing the field and what has already been discovered assists in finding alternative, creative solutions \citep{andriopoulos2000enhancing}.Furthermore, creativity is not restricted to artistic occupations only; it is required in various professions in which tasks presented involve complex, ill-designed problems where novel solutions are needed and status quo challenged \citep{mumford1988creativity}. Indeed, idea implementation may require even more creativity than idea generation \citep{mumford2002leading}.

For students of creativity, there is no surprise in attaching self-efficacy to creative actions \citep{mumford1988creativity}, yet recently problem construction processes have been recognised and combined to everyday problem-solving and real-world creativity \citep{getzels1975problem}; \citep{runco1988problem}. According to study of  \citep{gardner1988creativity} correlation between creative problem solving and everyday problem solving exists: they seem to have the same roots in information processing skills. 

In their study \citet{redmond1993putting} found how leaders supporting employees problem-finding and problem construction led to more unique and novel solutions. Leaders encouraged employees to find alternative solutions, approach problems from different perspectives and overall supporting several alternative problem-solving strategies. In addition, study showed how through motivational mechanisms, such as self-set goals, involvement and commitment, problem construction may have positive influence on solution quality and originality. Thus, problem construction is likely to have its greatest impacts on performance when in the process employee is allowed to express his values, needs and interests \citep{redmond1993putting}. 

Instead of managing creativity leaders should consider new approach: managing for creativity \citep{amabile2008creativity}. According to \citet{isaksen1983toward}, in order to support employee's creativity, leaders should focus on creating and maintaining and environment of supportive empathy, respect, warmth, concreteness, genuiness, trust and flexibility. These factors have been combined to general and task-specific efficacy needs \citep{mumford1988creativity}. Furthermore, through providing enough processing time for creating novel solutions is likely to enhance creative behaviour of employees \citep{isaksen1983toward}. As creativity refers to finding novel solutions and generating understanding of problems at hand, leaders could facilitate the process of resource allocation, feedback and task management in order to support employee's creative process \citep{mumford1988creativity}. 

Leader alone is not able to boost creative solutions in employees: it is also a matter of personal characteristics, previous knowledge of the problem at hand and expertise in the field \citep{mumford1988creativity}; \citep{redmond1993putting}. Thus, in order to achieve novel solutions and fresh ideas, leaders may seek employees who have great knowledge and expertise of problem at hand or provide employees education and possibilities to develop their problem construction skills and furthermore encourage approaching problems from various perspectives.  \citep{redmond1993putting} 

Furthermore, supporting employee's feeling of self-efficacy is likely to improve creative skills of an employee \citep{redmond1993putting}, and can be done through giving positive and realistic feedback, allowing adequate resources and physical support, clarifying task assignments, providing development support for employees, and assigning employees to appropriate tasks \citep{hennessey19881}. However, often acknowledging employee's skills, potential and accomplishments is likely to push an employee to the track of creativity \citep{redmond1993putting}. 

Should also be noted that depending on the job, different level and amount of creativity is required. Certain jobs that are highly involved with novel solutions urges for creativity as major breakthrough and innovative ideas, whereas more routine and repetitive jobs such as assembly line work requires creativity in developing the job practicalities. \citep{shalley2004leaders} 

Studies show employees who consider and believe creativity as valued outcome are more willing to generate ideas, experiment, communicate openly with others about ideas and through this, overall, their behaviour will eventually lead to creative outcomes \citep{shalley2004leaders}. Accordingly, \citet{csikszentmihalyi199916} presents the belief and feeling an employee has on the capabilities, pressure, resources and sociotechnical system of work environment affects highly on the success of creativity. Furthermore, pre-set obstacle, such as deadline, assists in focusing individual's attention to an urgent problem at hand, and has been noticed to stimulate creativity \citep{andriopoulos2000enhancing}. As employee who has the feeling of autonomy performs better, setting a deadline is not likely to threaten that autonomy, whereas showing someone how to meet that deadline would do \citep{mumford2002leading}.

Creative work is resource intensive where risk is involved \citep{mumford2002leading}. It is demanding and time-consuming\citep{mumford2002leading} and requires attention over long periods of time involving high level of ambiguity and stress \citep{kasof1997creativity}.Thus, organizational environment plays a major role in employees' creative skills, and such stifling factors may be positive challenge at work, encouragement from organisational level, support from work group as well as supervisory encouragement. Furthermore, organisational impediments can lead to decreased level of creativity. \citep{amabile1998kill} Hence, leadership has a great role in ensuring that the climate and culture, structure and practises of work and work environment together with human resource practices are supportive for creative endeavours to occur \citep{shalley2004leaders,oldham1996employee,mumford2002leading}.

Problem finding and construction, making connections and evaluating ideas are important for creativity \citep{mumford2002leading,vincent2002divergent}. Thus, when improving individuals possibilities to multiple alternatives, related ideas and example solutions, they tend to make more connections leading to creative actions \citep{amabile1996assessing}. 

\section{Experimentation as a method for innovation and learning}

According to \citet{edmondson1999psychological}, learning behaviour consists of seeking feedback, sharing information, asking for help, talking about errors and experimenting. Thus, experimentation behaviour seems to relate to learning of an individual, teams and organisations and can be supported by supporting these factors. Normal business consists of repetition, risk-avoidance and focusing on business outcomes \citep{buijs2007innovation}, while innovation requires novel solutions, thinking out of the box, risk-taking, breaking the rules, challenging the status quo and questioning the future \citep{burns1961management,kanter1984change,march1991exploration}.

One definition of experimenting considers it as personal trial and error process in which employees with their full potential are involved\citep{andriopoulos2000enhancing}. Experimenting serve as a great method when testing and validating abstract concept \citep{kolb1984experiential}.

Actually, idea implementation may require even more creativity than idea generation \citep{mumford2002leading}, and according to \citet{vincent2002divergent} creative work consists creative and innovation processes. Creative processes comprises of initial idea generation, whereas innovation process goes beyond the activities underlying the implementation of those ideas \citep{vincent2002divergent}.

According to \citet{buijs2007innovation}, innovation consists of coming up with novel ideas and implementing them. Ideating begins with exploring, developing and implementing the ideas, following introducing the ideas, which have turned into products or services, into the marketplace. Innovation process is a series of stages for processing the idea, and in the end of every stage the idea is reflected and evaluated before further processing. Evaluation points stands for usable tool for measuring the quality of idea but gives also understanding of how the evaluation process is going. In addition, while evaluating, team members also need to reflect the process and the idea, through which learning occurs. Also \citep{runco1994problem} emphasises how only after evaluation of ideas implementation can be discussed and performed and several studies show the essence of evaluation \citep{mumford2002leading,vincent2002divergent}. Useful questions in evaluation process could be "What went well?", "What can be improved?" and "What has been learned?" \citep{buijs2007innovation}. 

When dealing with novel solutions and challenging status quo, we are dealing with innovations. In order for company and its employees to be innovative, they need to take risks. Yet, at the same time usual management processes avoid risk-taking and focus on managing daily routine business. As \citet{quinn1985managing} stated it in his Harvard Business Review article: we love innovation and we urge for innovation, but we can tolerate it only if it is controllable and results everything remaining the same. \citep{quinn1985managing}

\citet{andriopoulos2000enhancing} define in their study a concept of perpetual challenging - a way to enhance creativity and innovation in an organisation. According to the concept adventuring occurs when goal is idea generation and through that process individuals are encouraged to face uncertainty in order to generate novel solutions. One tool for idea generation is scenario making, which purpose is to develop possible ways to tackle situation at hand. Through scenario making employees scans what is both known and not known about current problem or situation.  Experimenting process then consists of testing different scenarios generated in ideation phase and evaluating the outcomes in order to decide and develop the scenarios further to meet the needs of clients and industry. This calls for individuals skills to tolerate risks and uncertainties, as well as skills to constructively challenge and question colleagues ideas in order to use their full potential.

According to \citet{andriopoulos2000enhancing} facing and dealing with risk serves also as positive boost to creativity, as employees' learn new skills and strengthen their capabilities constantly and adapt new knowledge to already known. This, however, requires for safe environment which \citet{andriopoulos2000enhancing} refers as safety net: environment that tolerates failure. 

According to \citet{quinn1985managing}, fast multiple-idea prototyping leads to more innovative outcomes, offers essential information about ideas or product's quality, motivates employees, and helps the company and the team to cope with anxiety and uncertainty in development. Engaging lead customers in the interactive development process instead of market research seems to elucidate more relevant information about customer's demands, required changes and entry strategies. Thus, fast prototyping serves an essential way for learning from the iterative process. Market analysis, however, remain valuable when dealing with familiar products and productions, yet with radical innovations they may easily offer misleading information. \citep{quinn1985managing}

\chapter{Factors affecting innovation and experimentation}

As in todays' business world it has been widely recognised how creativity and innovation are essential for business growth, researchers have studied factors affecting creativity and innovation in organisations. \citet{amabile1998kill} has identified three factors being important for stimulating creative behaviour in individuals and organisations: individuals' intellectual capacity (creative thinking skills), expertise based on past experience and creativity-supporting work environment. Furthermore, \citet{oldham1996employee} consider creativity skills and characteristics of and individual as important, yet they add the importance of characteristics of organisational context such as job complexity, supportive supervision or controlling supervision. 

\section{Factors hindering innovation and experimentation}

Several factors may affect on the gap between idea and action in employees of an organisation. The phenomenon of threat of employees in organisations is widely studied and consensus is rising how threat effects on cognitive and behavioural flexibility and responsibility in reducing manner. \citep{argyris1982reasoning} \citep{edmondson1999psychological}

One essential factor is the beliefs, emotions and actions of an employee. An employee is likely to inhibit learning as a result of feeling the fear of being rejected, under pressure or feeling they are placing themselves at risk \citep{edmondson1999psychological}, or when facing the potential for embarrassment of threat, even though their transparency and honesty would be highly important for the behaviour of the team \citep{argyris1982reasoning}. This may occur in a situation where an employee should ask for help, yet is afraid of admitting he lacks abilities, skills or knowledge. \citep{edmondson1999psychological}. In addition, admitting mistakes, asking for help and seeking feedback are all relevant abilities in the recent organisational world, yet threatening for an individual's image of himself and his skills \citep{brown1990politeness}. 

Particularly when large new, complicated systems at hand, meaning of co-operation in production, development and communication rises exponentially. Especially in large organisations innovation can be inhibited by the errors increasing as a result of complexity of the system and inability to control, understand or make intelligent decisions. Challenging as it is for one department, faculty or company to survive on its own without communication and help of others in design, production and other business-related decisions, with management that takes the complex environment into account, the disastrous effects resulting from lack of communication can be lessen. \citep{quinn1985managing} Yet, at the same time, as a result of the difficulty of managing complex situations, innovation may denote finding the core, boiling things down and focusing on the most essential elements \citep{katz1978social}.

\citet{quinn1985managing} list several barriers to innovation, including intolerance of fanatics, short time horizons, accounting practices, excessive rationalism and bureaucracy and inappropriate incentives. \citet{hayes1982managing} supplements the list with concern of top management isolation, arguing how top management oftentimes has too little contact and understanding of the environment and conditions at factory floor or customer requirements for innovative solutions. Top managers who tend to be financially-driven and are not familiar nor have experience with current technology and its possibilities, may fear technological innovations and perceive them as too risky. Thus, more familiar traditions remain with ease. \citep{hayes1982managing} 

Furthermore, \citet{quinn1985managing} argues enthusiasm is not yet widely accepted and tolerated characteristic of employees and refers to them as entrepreneurial fanatics. Larger companies may perceive them as causing embarrassment by challenging status quo and causing troubles. 

Short time horizons require companies to stay in continuous stream of quarterly profits, oftentimes at the cost of long time benefits, that innovations demand. Especially large companies easily favour narrow-minded actions such as quick marketing fixes, cost cutting and acquisition strategies over systemic thinking and process, product or quality innovations. \citep{quinn1985managing}

Rather than managing the inevitable chaos of innovation productively, these managers soon drive out the very things that lead to innovation in order to prove their announced plans. In the name of efficiency, bureaucratic structures require many approvals and cause delays at every turn. Experiments that a small company can perform in hours may take days or weeks in large organizations. The interactive feedback that fosters innovation is lost, important time windows can be missed, and real costs and risks rise for the corporation. Inappropriate incentives. Reward and control systems in most big companies are designed to minimize surprises. Yet innovation, by definition, is full of surprises. It often disrupts well-laid plans, accepted power patterns, and entrenched organizational behavior at high costs to many. Few large companies make millionaires of those who create such disruptions, however profitable the innovations may turn out to be. When control systems neither penalize opportunities missed nor reward risks taken, the results are predictable." \citep{quinn1985managing}

Organizational structures are also likely to enhance or hinder creativity in organisational, team or individual levels \citep{shalley2004leaders}. 

\section{Factors supporting innovation and experimenting} 

According to \citet{garvin2008yours} in learning organisation employees excel at creating, acquiring and transferring knowledge. They define building blocks for learning organisation: supportive learning environment, concrete learning processes and practices and leadership behaviour that reinforces learning. Building blocks can be considered and measured as independent components yet each of them vital to the whole. In order to improve long-term learning of an organisation, strengths and weaknesses of an organisation and its unit needs to be recognised. 

\subsection{Supporting learning environment}

Experimenting requires safe and supportive environment. According to \citet{edmondson1999psychological} team psychological safety should be the first essential building block of learning behaviour in work teams. Supportive learning environment consists of four characteristics: psychological safety, appreciation of differences, openness to new ideas and time for reflection \citep{garvin2008yours}. Likewise, \citet{mumford1988creativity} emphasise the meaning of environmental variables as a means to support employee's creativity by providing resources to stimulate fresh ideas of employees. Furthermore, strong positive relations between organisational environmental variables have been found; organisational encouragement as well as support for innovation and creativity from team improve employee's creativity \citep{amabile1996assessing}

Futhermore, \citet{amabile1998kill} suggests changes in organisational environment are likely to boost intrinsic motivation of an employee leading to increased creativity skills. Role of leaders and managers is essential; being a key person in organising group work and processes a leader may encourage employees to achieve shared goals \citep{amabile1998kill}. 

A company desiring for innovation should allocate resources and define long-term goals and actions accordingly. Even though companies urge to invest most resources in current lines, sufficient resources should be allocated for long-term growth and innovation. This includes providing an environment strong enough to seize surprising opportunities and tolerate unforeseen threats in all organisational, technical and external relations levels. \citep{quinn1985managing} Level of uncertainty can be reduced through goal-setting and fast prototyping \citep{mumford2002leading}. 

\citet{mumford1988creativity} have studied the gap between an idea an action, and revealed it depending on various attributes related to individual and circumstances. As physical work environment affects on creativity, information sharing and innovation in an organisation, it should be designed to support the natural flow of traffic through the building so that informal conversations between different functional areas are enabled \citep{shalley2004leaders}. 

Learning of employees occurs when employees do not fear being rejected, ask naive questions, make mistakes or present viewpoint of minority. Psychologically safe environment enables employees comfortably express their thoughts at work. Appreciation of differences is important, as opening minds for different ideas and world views increases both energy and motivation, brings out fresh thinking. Novel approaches are relevant for learning, thus employees should be encouraged in risk-taking and exploring and testing uncertain things. Lastly, providing time for reflection is likely to foster learning in safe environment. Instead of looking and judging by numbers of hours of work or results employees should be given enough time to reflect their work. Analytic and creative thinking are prevented under stress, heavy workload and too tight schedule. Under stress ability to recognise and react to problems and learn from experiences deteriorates. In supportive learning environment time for reflection is allowed. \citep{garvin2008yours} 

Interestingly, studies show how nominal groups perform remarkably better in ideation and brainstorming processes by producing greater amount of ideas than real groups. This may be due to the learnt practices and norms of a real work group, fear of failure that prevents free idea exchange and fear of evaluation and others judgement when suggesting creative solutions. \citep{jung2001transformational}

\citet{edmondson1999psychological} defines a concept of team psychological safety, that fosters learning behaviour in work teams by reducing the risks of embarrassment or threat and increasing mutual trust between team members. Oftentimes in work teams employees are not willing to tell their ideas or errors out loud as they are afraid of being labelled as incompetent. Thus, they prefer staying silent ignoring how it may lead to negative consequences for the team performance. When team members share feeling of respect and trust of others, and stay confident on other member's not using the errors against them, they are more likely to put more weight on the benefits of telling concerns out loud. When knowing that well-intentioned interpersonal risks are not punished is a shared belief of a team, team members are more likely to take proactive actions that foster learning leading to more effective performance. 

Even though building mutual trust may not lead to mutual respect and caring among team members, it is essential for creating psychologically safe environment and through building trust a foundation for further development of team psychological safety is built. \citep{edmondson1999psychological} 

\citet{edmondson1999psychological} lists factors affecting psychological safety in teams, including context support and team leader coaching. Context support refers for instance to access to information and resources needed. Safe environment that fosters creativity also takes into account employees' perceptions of just and transparent decision-making as well as applied actions \citep{shalley2004leaders}.

Accordingly, \citet{amabile1998kill} suggested that creative thinking can be encouraged by shaping organisational culture such that employees feel encouraged to tell their ideas out loud freely and without judging, increasing idea exchange and discussion about them. In addition, studies show how creative individuals may only produce more creative outputs than less creative individuals when the context is supporting and encouraging towards creativity \citep{oldham1996employee}. 

According to \citet{mumford1988creativity} through environmental variables employee's creativity can be fostered. New solutions may be achieved through problem solving and challenging the routine ways of thinking, and environment should be designed to encourage and facilitate these skills. Environmental factors may, furthermore, affect on employee's intrinsic motivation and willingness to generate novel ideas, when social and physical environments work as a source for support and resources in idea generation and implementation. 

\subsection{Concrete learning processes and practices}
 
According to \citet{garvin2008yours} second building block of organisational learning, consists of concrete learning processes and practices. It includes experimentation, information collection, analysis, education and training and information transfer. Organizational learning can be supported through concrete steps and activities which are tested and further developed through experimentations. Furthermore, information and intelligence about customers as well as technological trends should be collected systematically and further analysed focusing on identifying problems and solving them. Training and education of new and established employees is an essential part of practices and processes. Finally, through transparent and meaningful knowledge sharing organisational learning can be enhanced, focus being on clear, well-defined and working communication systems that employees can easily relate to. Concrete processes together with efficient knowledge sharing methods ensure essential information being available fast and efficiently for employees who to use. \citep{garvin2008yours}

Supportive leadership behaviour alone is not sufficient guarantee for organisational learning. \citet{garvin2008yours}emphasise how organisations are not monolithic and managers should sense differences in culture, department and units. In addition to cultural differences, learning requires clear and targeted processes and practices. Furthermore, learning should be considered as multidimensional, thus organisational forces should not be solely focused on a single area but to consider presented building blocks as a whole.

Also \citep{amabile2004leader} emphasise in their componential theory on creativity the support of immediate supervisors as a way to enhance employee's creativity and intrinsic motivation. Supporting actions include being a role model, defining and setting appropriate goals, showing the work group support and confidence within the organisation, showing appreciation of individuals contributions to the project, focusing on efficient and good communication, offering valuable feedback, and listening openly novel ideas. Accordingly \citep{amabile2004leader} divide required behaviours of leaders for providing support into two categories: instrumental or task-oriented and socio-emotional or relationship-oriented actions.

Organizational and team structures and hierarchies affect on innovation and experimenting. Flat organisations and small project teams are foster innovation performance in a company. Smaller team handles communication and commitment better, while as few management layers as possible decreases the jeopardy of rejection. "Since it takes a chain of yesses and only one no to kill a project, jeopardy multiplies as management layers increase."\citep{quinn1985managing}

Organizational structures can influence in many ways creativity of a team and individual. For instance, by promoting open communication, idea and ongoing information exchange with internal and external team members as well as encouraging information seeking from different perspectives and sources is likely to enhance creativity (e.g. \citep{ancona1992demography,dougherty1996sustained}). 

According to \citet{amabile2002creativity} clear time should be allocated for developing especially when the aim is to flourish idea generation, creativity, learning and experimentation of new concepts. Time pressure should be minimal in order bright ideas to glow as cognitive processing requires time of an individual and team. Yet, no sense of urgency leads employees easily to auto-pilot mode, in which routine tasks are performed without further thinking and analysing. Thus, creative time for playing with ideas, brainstorming, learning and experimenting should be allocated in an organisation in order truly new things to develop. Shared goals are once more essential in engaging team members to play with ideas and feel more motivated in developing their work. \citep{amabile2002creativity}

In addition, sufficient time and resources should be allowed for exploration\citet{amabile2008creativity,katz1985project}. 

Creativity, exchanging ideas and turning them into action requires intrinsic motivation from employees \citep{jung2001transformational}. Thus, in order to increase creativity and innovation at workplace, leaders should foster organisational culture in which individuals find their motivation in divergent thinking and trying out new ways of performing tasks \citep{amabile1998kill}. 

Leaders need to acquire resources and encourage idea generation \citep{mcgourty1996managing}, and overall create environment where idea generation is possible \citep{andrews1970social} as well as evaluate the ideas and integrate them to organisational needs \citep{mumford2002leading}. While individual characteristics affect on the creativity of an individual, creating an environment fostering creativity is likely to assist in producing novel ideas during the routine work of employees \citep{amabile1996assessing}. 

Collective organisational achievements can, in turn, be affected through affecting working environment and organisational culture and leaders influence on employee's attitudes and motivation towards work \citep{amabile1998kill}.

Prior studies show how creative efforts of employees require sufficient amount of time and energy\citep{gardner1988creativity,getzels1975problem}. Also \citet{redmond1993putting} state that leaders should allow enough time for problem solving and creative actions. 

Also \citet{shalley2004leaders} emphasises the meaning of prior knowledge and experience of an employee of area of work before demanding or anticipating creative actions from them. Naturally, job rotation and employees from different areas works as a great source for new perspectives and development, yet creativity requires sufficient level of familiarity of target area. \citep{shalley2004leaders}

\citet{shalley2004leaders} also state how appropriate level of autonomy given to employees is useful. However, too much autonomy, meaning full control over planning and conducting the work, may lead to negative consequences and contradictory goals between employee and organisation. Thus, setting appropriate goals and understandable requirements that inspire employees is essential. Furthermore, leaders need to realise whether the goals require creativity or lead to creative outcomes, and not anticipate creativity or creative outcomes and instead accept employees being less creative where it is not needed. \citep{shalley2004leaders} 

Additionally, trying out novel approaches and conducting experiments requires more energy and is overall more difficult for employees than performing and sticking to the routine tasks. As it takes more cognitive resources to generate several alternative solutions, practice divergent thinking and approach problems from different perspectives, allowing time for creative work is essential. However, engaging employees to creative activities is likely to lead better and more qualified decisions. \citep{shalley2004leaders}
Also \citet{amabile1987creativity} state how sufficient time should be allowed for creative thinking, playing with ideas and exploring multiple perspectives. \citet{katz1985project}, in turn, found in their study how uninterrupted time was considered critical for engineers working on novel technologies. 

Studies have also shown how employees working on high time pressure affects negatively on ability to engage in creative cognitive processing \citep{amabile2002creativity}.

Together with time, in order to be creative sufficient access to material resources should be allowed for employees \citep{katz1985project}. However, even though material resources are essential for creativity, studies have suggested a contradictory perspective: when employees have access to wide range of material resources, their creativity tendencies may decrease. This may happen due to the creative actions and thoughts an employee needs to perform when needing certain resources to finish his task but not having them at hand. This, in a way, stretches employees' skills to think differently and achieve goals. \citep{csikszentmihalyi199916} Thus, \citet{csikszentmihalyi199916} states how resources are likely to make employees feel too comfortable and lead to decrease in creativity. 

Giving and receiving feedback remain simultaneously a key function of leaders and one of the most challenging tasks they have. According to \citet{shalley2004leaders} giving performance feedback is essential for creativity and accordingly difficult as creativity often involves approaching problems from new approaches and trying out novel things as well as taking risks. 

Structures have their influences on creativity of employees. Relationship between formal reporting and responsibility levels, referring to bureaucracy levels are essential: highly bureaucratic organisation do not tend to encourage employees to reach for novel approaches and experiments, whereas organisation with flatter structure may enhance organisations autonomy and creativity \citep{shalley2004leaders}

Changing overall organisational climate is challenging, yet various components are reasonably manageable and should foster creativity. For instance, risk-taking, constructive feedback can be supported through role modelling of management. Employees may be likely to perceive presentations of organisations structure and hierarchy as discouraging and highlighting how employees are not allowed or encouraged to make decisions on their own leading to less enthusiasm towards trying out new ways of working and developing. In addition, heavy bureaucracy demanding lot of time and effort from employees to get novel ideas forward in the organisation is likely to destroy the enthusiasm and willingness of employees towards developing and new approaches. \citep{shalley2004leaders}

All human resource practices should be in line and systematically linked together in order to create a clear picture for employees of what is expected of them. Perceived fairness and sense of loyalty towards a company of an employee is higher when an employees understands what is expected of them, how, when and what for they are being rewarded, promoted or fired. These same attitudes are essential for fostering creativity. Committed and loyal employees are more likely to exceed what is required of them, be more motivated and committed to working towards specific goal and find novel approaches in order to succeed. \citep{shalley2004leaders}

Essentially, interaction between leaders and employees, team members and outside members, sufficient resources, employees clear expectations on their evaluation and rewards and environment which is perceived fair all have affect on employees behaviour at work. Through these variables, employees may feel they are working in supportive working environment, and foster creativity and ability find novel approaches and try out new things. \citep{shalley2004leaders}

Several types of teams function in organisations, type depending on various dimensions such as cross-functional versus single-function, time-limited versus enduring and manager-led versus self-led. These dimensions should be recognized and team learning fostered depending on the type. \citep{edmondson1999psychological}

Amabilen (1997) tekstiss\"a pdf:n sivulla 17 on hyv\"a kiteytys Amabilen osalta, mit\"a innovointi vaatii organisaatiolta. 
Expertise 
- Expert performance and its affects on implementing ideas \citep{ericsson1994expert} (tsekkaa artsu viel�)

\subsection{Leadership behaviour}

\citet{quinn1985managing} emphasis top executives role over particular management. Innovation is likely to occur when top executives encourage creative and innovative endeavours and create an atmosphere and value system that supports innovation. \citet{quinn1985managing} offers an explanation why it seems easier for engineer and scientific leaders to create atmosphere supporting innovation: understanding and psychological comfort are related to familiarity, and engineers, for instance, have wider understanding and knowledge of technology, which makes newer technological innovations easier to accept and adapt. 
 
Leadership behaviour should reinforce learning. Behavior of leaders is highly related to the performance of employees \citep{kim2014blue} and organisational learning \citep{garvin2008yours}. In order to encourage employees learning, leaders should prompt dialogue and debate, ask questions and listen to employees \citep{kim2014blue,garvin2008yours}. Leaders supporting new ideas and idea exchange has been related to enhancing creativity especially among those employees who showed disposition towards creativity \citep{oldham1996employee}.  

These three building blocks overlap to some degree and reinforce one another. For instance, leadership behaviour helps in creating supportive learning environment, which supports managers and employees in creating and defining concrete learning processes and practices. Furthermore, concrete processes support leaders behaviour in a way that fosters learning and through own example cultivates that behaviour to others. \citep{garvin2008yours}

General leadership styles and practices are set for industrial management, yet at present the focus should be on leading the people, and the focus of leadership needs to change from authoritarian style to increased autonomy and trust. As \citet{mumford2002leading} state, "organisations may now need jazz group leaders rather than orchestra directors". 

Likewise, according to \citet{mumford2002leading} creative leadership highlights three key elements: encouraging employee's idea generation, creating safe environment for ideas to emerge and improving idea promotion and implementation. By idea stimulation, education of various problem solving techniques, support for novel ideas, involving employees in developing ideas and allowing them freely pursue ideas, idea generation can be enhanced by the leader. Essential elements of safe environment include diverse teams, transparent and good communication, leader acting as a role model and being in charge of conflict management. In addition to idea generation, idea structuring phase consists of creating action or project frameworks so that employee's have as much autonomy to perform the task as needed. Idea promotion, in turn, refers to leaders task to transfer ideas to broader levels of an organisation, achieve support and assist with implementation of chosen ideas. Promotional activities to upper levels of organisation serve as a major way to insure sufficient resources and support for the idea implementation. \citep{mumford2002leading} 

According to studies leaders have a strong direct impact on employee's behaviour and way of performing at workplace \citep{katz1978social,redmond1993putting}. As leaders play a major role in establishing, influencing and shaping organisational culture and climate through their communicated values and beliefs, they are able to shape the organisational culture into more innovative direction and foster creativity in an organisation \citep{jung2003role,schein2010organizational} for instance by nurturing organisational climate that supports creative efforts and learning \citep{yukl2002leadership}. Change from authority-based leadership to collaboration with employees has occurred in literature and in practice (\citep{amabile2008creativity,farson2002failuretolerantleader}). 

Leaders have a great influence on employees behaviour. \citet{avolio1988transformational} list several mechanisms through which leaders can affect employees behaviour. These include role modelling, goal definition, reward allocation, resource distribution, defining norms and values of the company, showing the way to interact as a group, condition employees' perceptions of work environment and being the lead decision maker on organisational structure and procedures. Studies also suggest leaders have a significant effect on employees' creativity \citep{hennessey19881}, and according to \citet{redmond1993putting} a leader can have an affect on employee's level of creativity through leadership behaviours such as problem construction, learning goals and feelings of self-efficacy. 

Actually, leadership behaviour is only recently recognized as essential part of enhancing creativity and innovation skills of employees \citep{mumford2002leading}. This may be due to our romantic perception of creative act, which defines creativity as an heroic act of an individual and leaders only being a hindrance to the creativity of an individual. Furthermore, conventional models of leadership are not likely to encourage employees to challenge the status quo but to achieve required goals.\citep{mumford2002leading} Current trend in research however shows leaders and their behaviour have great influence on the creativity and innovation ability of employees (eg. \citep{mumford2002leading,jung2001transformational,amabile1998kill})

Conventional leadership behaviour focuses on internal activities within the team, whereas innovative team leader needs various set of skills and approaches in order to encourage developing and growing of teams and individuals. For instance, according to \citet{barczak1989leadership} leaders of innovative teams utilise wide range of familiar and unfamiliar techniques in order to accomplish the team objectives, whereas leaders of operating teams use only a few familiar techniques. Even though in this study innovation teams were not studied, similar elements of developing by experimenting and encouragement for that may be recognised, when dealing with new tasks and developing something which result is uncertain. 

Leadership style has great impact on organisational innovation and creativity. Transformational leadership refers to leadership style and processes which emphasises longer-term and vision-based motivational processes \citep{bass1997full}. Furthermore, through offering an explanation of the importance and value of the work, leaders encourage their employees' to think beyond self-interest \citep{yukl2002leadership}. Leaders shape and define the goals and working context \citep{amabile1998kill, redmond1993putting}. Through a long-term vision (separated from short-term business outcomes, which usually focuses on quarterly profit), leader's are able to direct employee's efforts towards creativity and innovative work processes leading to likeminded outcomes\citep{amabile1996assessing}.

Leaders can affect employees' creativity and innovation skills both directly and indirectly \citep{jung2003role}. By stimulating employee's intrinsic motivation and higher level needs leaders are able to affect directly on employees' creativity \citep{tierney1999examination}, where indirect way may be through establishing a work environment where new ways of doing are encouraged and failure is not being punished \citep{amabile1996assessing}. Creating and supporting a reward-system that values creative performance, provides both intrinsic and extrinsic rewards for employee's efforts to learn new skills and to challenge status quo by experimenting new approaches, employees are constantly willing to engage in creative endeavours \citep{jung2001transformational,mumford1988creativity}.

\citep{mumford2002leading} argue organisational climate and culture being a collective social construction where the role of the leader on control and influence is remarkable. \citep{schein2010organizational} also presents a view where leaders communicated personal values and beliefs become essentially part of organisation's culture and climate. Furthermore,\citep{jung2001transformational} considers managers essential for shaping organisational culture, whether the concern is in developing, transforming or institutionalising. The way employees perceive their work environment created by their leaders, and especially the way they perceive the instrumental and socioemotional support both have influence on employees' creativity. \citep{oldham1996employee}

Also Buijs (2007) \citet{buijs2007innovation} states how leaders dealing with uncertain and new innovations should stay certain about uncertainties and provide a safe environment and encourage employees to work on current task comfortably. Thus, high level of tolerance for dealing with different states of minds and various personal feelings is required from a leader. \citep{buijs2007innovation}

In order to encourage creativity and experimenting in teams, leaders should lead by example and act as role models. Leaders should consider their own behaviour and actions in a way that stimulates employees to new and innovative, creative approaches to problems. In addition, they can even request creative and innovative solutions form the team, which may lead to better results in creativity of individuals \citep{amabile2002creativity}. \citep{mumford2002leading,amabile2008creativity,waldman1990adding}

Big ideas do not hatch overnight and creative thinking requires time. Leaders should allow team members time to think creatively, as according to studies under pressure creativity actually falls into decline. Even though individuals may feel more creative, actually they are only working more and getting things done. According to this study, employees were clearly less creative while time pressure increased. \citep{amabile2002creativity} 

Furthremore, leaders can assist their employees by recognising times with high pressure, and allowing employees to focus on certain thing at a time, leaving the expectations of creativity and new ideas into the future moment, when time pressure has decreased. On the other hand, if creativity is required under stress, leader should transparently explain the importance and reasons behind the strict schedule and required goals. Thus an employee may relate to the problem at hand and engage better at his work. Indeed, helping people to understand the importance of work is essential especially under high time pressure. \citep{amabile2002creativity} 

Failure as a part of innovation and development process begins to be generally recognised and approved. Succeeding companies even thrive for failure in order to learn fast and find the best practices and business models. Through encouraging employees in risk-taking and making mistakes, leaders are likely to boost innovation. For instance, credit company Capital One conducts continually large amount of market experiments. They now most of the tests will not pay off, yet they also know how much can be learned about customers and markets from failed tests in early phase of development. Yet leaders fail in showing their employees the support and tools for failing fast and early enough. Failing in a personal matter remains a difficult subject, as failing never feels exceptionally great, and often employees still consider failed work as failing personally. \citep{farson2002failuretolerantleader}

Failure-tolerant leaders put effort on explaining to employees how important part failure is to the development process as a whole, and how failing actually refers to a point where surprising, failed outcomes are not reflected and further analysed in order to learn. Performing accordingly, admitting own failures and not chasing anyone to blame, failure-tolerant leaders encourage failure, lower the threshold and ease the fear of failing of employees. \citep{farson2002failuretolerantleader}

Naturally management need to take seriously issues about safety and health, yet most of the failures should be seen as opportunities for growth. Furthermore, failure-tolerant leaders treat success and failure similarly, analysing and reflecting the outcomes in order to grow the intellectual capital of the team, including experience, knowledge and creativity. Other characteristics of failure-tolerant leaders are being rather collaborative than controlling, listening carefully, seeing the bigger picture, asking questions and focusing on the development and future rather than blaming on mistakes. In addition, in order to gain empathy and trust among employees, leader should admit their own mistakes, as it shows self-confidence and honesty, assisting in forming closer ties with employees. Vulnerability and transparency play a major role in trustworthy relationship between leader and employees.  \citep{farson2002failuretolerantleader}

Through the green light given and their own example leaders can change the focus from success and failure into thinking in terms of learning and experience. \citep{farson2002failuretolerantleader}

\citet{amabile2008creativity} draw a poetical picture how leader cannot manage creativity, but manages for creativity. Furthermore, they suggest that culture that fosters creativity includes leadership that enables collaboration, enhances diversity, encourages ideation, maps the stages of creativity to different needs, accepts inability and utility of failure and motivates employees with intellectual challenges. According to \citet{sosik1999leadership} leaders should concentrate on vision of work and its outcomes that is meaningful and motivational enough to inspire employees. 

In experimentation process, employees need to contribute imagination, and this may require new kind of encouragement for creativity from the leaders. Much success rises from employee's own initiatives, which results from wide amount of autonomy at work. \citep{amabile2008creativity}

A culture of creativity can be fostered in an organisation through opening the organisation to diverse perspectives and openness to various ideas. This calls for safe environment for employees to share their thinking from different fields of expertise. Furthermore, encouraging passion and knowledge of an employee is likely to result in more creative action at work. \citep{amabile2008creativity}

Empowering employees is an essential tasks of leaders, through which a work environment is created where employees desire to seek innovative approaches to perform their work tasks \citep{jung2003role}. Transformational leaders encourage employees to participate in developing by highlighting the importance of cooperation, providing the opportunity to learn from shared experience and allowing employees to perform necessary actions in order to be more effective\citep{bass1990implications}. Furthermore, autonomy and freedom to perform essential tasks has major effects on organisational creativity, as individuals are more likely to produce creative work when having the feeling of personal control over how to approach given tasks \citep{amabile1996assessing}.

Yet, in order to maintain organisational innovation and risk-taking, autonomy given to an employee can not be in contradiction with fear of failure or discouragement towards challenging status quo or trying out novel solutions \citep{yukl2002leadership}. Thus, organisational climate has to support and encourage innovation \citep{mumford1988creativity} by valuing initiative and innovative approaches that support employees in risk-taking, accepting challenging assignments and stimulate intrinsic motivation towards work \citep{jung2003role}.

\citep{jung2001transformational} has studied how leadership style affects group's creativity and performance by comparing transactional and transformational leadership styles. Transformational leader refers to a leader who encourages divergent thinking and looking at problems from unconventional perspectives, while providing and explaining clearly defined goals and facilitating the innovation process of employees \citep{bass1990implications}. Furthermore, development of clear long-term vision and practises supporting the way to achieve it is essential characteristic of transformational leaders \citep{avolio1988transformational}. The relationship between transformational leader and an employee is active and emotionally attached \citep{avolio1988transformational} and through the strong attachment resulting from tight relationship leaders can better support employees in using their personal values and self-concepts in the way that employees can pursue higher level performance and fulfil personal needs through the work. This focus of transformational leadership on value alignment is likely to lead to the root of intrinsic motivation of an employee \citep{gardner1998charismatic}, which is considered as one of the key elements in creative thinking and innovation skills of an employee (eg. \citep{jung2001transformational,amabile1998kill,deciintrinsic}).

According to \citet{bass1997full} transformational leadership consists of four unique yet interrelated behavioural components: inspirational motivation (articulating long-term vision), intellectual stimulation (promoting creativity and innovation), idealised influenced (meaning charismatic role modelling) and individualized consideration referring to coaching and mentoring leadership style. 

Transformational leaders can build environments that support creative actions (\citep{sosik1998transformational,avolio1988transformational}). According to \citet{sosik1998transformational} key characteristic of transformational leader is the intellectual stimulation, which is likely to encourage creativity and divergent thinking leading to unconventional solutions to problems at hand. 

In contrast to transformational leadership, transactional leadership refers to focus on employees ability to fulfil and achieve clearly defined goals \citep{hollander1978leadership,house1971path} and successful goal achievement is rewarded \citep{waldman1990adding}. This exchange relationship between leader and employee is based on a contract of specified goals and emphasises on the process of achievement of objectives (Avolio and Bass 1988) but does not encourage employee's to develop their creativity and innovation skills \citep{jung2001transformational}. Instead, employees are rather motivated extrinsically to perform their job under transactional leader but not expected to question and change the status quo in creative ways \citep{amabile1998kill}.  

Few studies have been made linking the transformational leadership and positive outcomes of employees' creativity in organisational level and outcomes \citep{jung2003role}, even though several studies have been made revealing the positive relation between these factors. in their study \citet{jung2003role} draw this link clearer and suggest that while leaders define the context and goals of their employes, transformational leadership can be extrapolated to an organisational level.  

According to \citet{jung2003role} transformational leadership is positively related to organisational innovation, employee's perception of empowerment and support for innovation. Furthermore, the perception of empowerment and is positively related to organisational innovation, and when perception being strong,  the relationship between transformational leadership and organisational innovation tends to be stronger. Results of the study conducted on 32 Taiwanese companies suggest that through transformational leadership by top managers organisational innovation can be affected directly or indirectly, latter referring to creating an organisational culture in innovation, discussion, novel approaches and experimenting is encouraged. \citep{jung2003role}

As undertaking novel approaches to work oftentimes involves risk-concerned decision-making, employees should be offered decent level of guidance, goals and some measure of structure \citep{jung2003role}. Leader not taking an active role in supporting and guiding the work of his employees may lead to organisational units working at cross-purpose. Thus, leadership is about maintaining a balance between empowering employees and providing guidance and structure through setting goals and agenda. However, according to \citet{mumford2002leading} leaders' planning and guidance should focus on progress, projects on general level and implementation of the results of projects instead of focusing on offering detailed guidance on piece of work. 

Study of \citet{sethi2001cross} showed how good interaction in a team and high level of commitment to the success of the team lead to more radical innovation abilities. In the study team members were highly encouraged to take risks, which lead to more motivated members in suggesting novel ideas from their perspectives. In addition, team members identified themselves strongly as part of the team, which again higher commitment level. \citep{sethi2001cross} 

However, lack of time and resources may serve as a hindrance to employee's willingness to take risks and perform experiments \citep{jung2003role}. Through leaders who allow their employees to participate in developing and ideating, reserve budget for it and set it as a part of performance standard, the hindrance for risk-taking may be lowered \citep{jung2003role}.

Under some circumstances, according to Monge et al. (1992) \citet{monge1992communication} group communication is likely to increase innovation. Thus, leaders should consider managing wide range of formal and informal meetings and facilitated discussions in order to create opportunities for ideation. Furthermore, innovation occurs over time and is a dynamic process. Leaders should be sensitive in which pace more managerial impact is needed, and in which pace of the process more freedom and autonomy should be allowed for employees. \citep{monge1992communication} 

Organizational leaders play a great role in establishing strong team performance culture. This can be achieved through addressing and demanding performance that meets the need of customers, employees and shareholders. Teams should not be fostered by the sake of the team only, rather should leaders clearly state how the team performance affects to customers and through that foster clearer performance ethics and cultures. In addition, even though people tend to have great sense of individualism, it does not have to bias the teamwork performance, as real teams find ways to support individual strengths and performance for shared goal. Furthermore, in order to team function properly and efficiently, discipline across the team and organisation is needed, focusing again on performance.  \citep{katzenbach1993wisdom}
 
Ambiguity is often perceived by individuals when lacking sufficient cues to structure a situation, and usually arises from novelty, complexity or unsolvability of situation at hand \citep{budner1962intolerance}. 
 
Leadership plays a major role in defining group goals, controlling resources and providing rewards through interactive leadership process, making leadership behaviour an essential environmental variable in stimulating creative behaviour as a means for achieving goals \citep{redmond1993putting}. \citet{katz1978social} even refer to role of the leader in a sense where leader defines by his example the reality of workplace; norms, practices and culture. According to \citet{barczak1989leadership} leader's task is also to provide clear focus for the work of employees. 

By defining organisational culture, climate and group norms leaders shape the way of working of employees. Through such role-modelling and mentoring process leaders also show employees in practise how tasks are performed. Employees, in turn, follow the example of leader in order to achieve high level of performance. \citep{redmond1993putting}
 
Role-modeling stands also as powerful tool for opening employee's eyes and attitudes to new perspectives, thinking 'out of the box' and adopting generative and exploratory thinking processes \citep{jung2003role,sternberg1997creativity}, influencing creativity of an employee \citep{shalley2004leaders}

Although different leadership styles and their effect on employee's creativity behaviour has not yet been studied widely, some studies show, how transformational leadership behaviour encourages employees look problems from different perspectives and thus widen their intellectual and creativity skills \citep{jung2001transformational,sosik1998transformational}. \citet{jung2001transformational} has studied the relation between leadership style and group creativity finding that transformational leadership is most likely to stimulate creative effort of employees. 
 
In his study, \citet{jung2001transformational} emphasises that transformational leadership skills can be practised in order to foster creativity and intellectual skills of employees and shape organisational culture. His study showed how transformational leadership; encouraging divergent thinking and solving problems at hand from unconventional perspectives, is likely to increase intrinsic motivation of employees leading to more creative problem solving and behaviour.  Through brainstorming activities that focus on non-traditional thinking and fantasising intellectual skills of employees can be enhanced \citep{sosik1998transformational}. Furthermore, \citet{jung2003role} argue that several aspects of leadership behaviour can be learned and practiced. Thus, organisations should foster and improve innovativeness by offering managers training and mentoring processes that develop transformational leadership. 
 
\citep{sosik1998transformational} furthermore suggested that anonymous ideating through nominal groups leads to better results and greater amount of ideas than brainstorming activities in real groups. When ideating in daily working groups, members may fear failing, being ashamed or measured by their performance. Overall, oftentimes it is way more difficult to take a different role and actions in group with familiar members and routines. \citep{jung2001transformational}

Innovation leaders, indeed, in managing innovation processes need to have several contradictory skills, roles and attitudes used smoothly during the day with employees; and they need to tolerate these competing and conflicting aspects within the team. Innovation team trusts their leader to be in charge and in control, yet allow them support, autonomy and enthusiasm.  At the same time, innovation leader should be few steps ahead thinking of uncertain future scenarios and support his innovation team in current step without showing doubts too strongly about team's work. \citep{buijs2007innovation} Resulting from this contradictory and challenging role of an innovation leader, according to \citet{buijs2007innovation} they should act and have characteristic of a controlled schizophrenic. 

As \citet{buijs2007innovation} argues, leaders who are to lead employees and work handling innovations need to understand the paradox and natural conflicts between routine processes (exploitation) in order to earn money in the present and the innovation processes (exploration) in order to earn money in the future. \citet{buijs2007innovation} four aspects for innovation which leader should be able to master all providing a secure environment for a team to perform in novel and creative ways. These consist of innovation process, psychological process of innovation team, creativity process, and leading and playing. 

Goals, however, should be kept broad, in order not to create undue oppositions to new ideas. Flexibility should be maintained by not defining intermediate steps in detail and by trying alternate options and routes. Identifying and solving problems at early phase fosters momentum, confidence and identity towards new approach. Furthermore, sufficient amount of information about the project and progress should be offered in order managers to follow and realise the work performed.  \citep{quinn1985managing}

Local leaders are in essential role in directing and evaluating work of employees, facilitating and allowing resources and information as well as encouraging employees to engage with the tasks and team members. \citep{amabile2004leader}

The approach of leaders oftentimes divides into task-oriented or relationship-oriented. Task-oriented leaders value performing the job, focusing on clarifying roles and responsibilities, monitoring work while managing time and resources. In turn, relationship-oriented leaders value socioemotional aspects of work through empathetic actions, showing consideration for employees, being friendly and supporting the team personally. Should be noted that in literature concerning leader behaviour, term support refers to relationship-oriented leadership behaviour, wheres in creativity literature same term refers to both task- and relationship-oriented behaviours and actions - all that are to foster creativity. In this thesis, latter and broader usage of term support is used. \citep{amabile2004leader}

According to the study of \citet{amabile2004leader}, leaders are likely to influence employees feelings, perceptions and performance as well as overall creativity do their own behaviour. By acting fairly, consulting with employees on essential decisions, offering emotional support and rewarding and recognising them for performing well leaders can enhance creativity of employees. In turn, leaders may play as hindrance for creativity by not offering support and clear task assignments, preventing autonomy of employees, treating employees unfairly and not trying to resolve important problems. 

According to componential theory of organisational creativity \citep{hennessey19881,amabile1996assessing}, employee's perception of the work environment influences individual and team creativity and emphasises the role of local leader support for creating an creativity supporting environment. 

In their study, \citet{shalley2004leaders} present how leaders should use human resource practices in order to develop work context which improves the creativity skills of employees. 

Organizational structures affect in the traditional roles of leadership as a means of direct responsibility given to employees. The trend of flatter organisations provides more autonomy to employees, whereas leaders' role transforms to more involved in external resource acquisition and managing the interfaces. \citep{shalley2004leaders}

Creative work environment is likely to be created through leaders who support and encourage employees, provide them autonomy in decision-making and everyday tasks, and communicate openly with employees \citep{oldham1996employee,tierney1999examination}. However, in addition to contextual factors and environment, studies show level of support, control and assist an employee needs depends on personal characteristics. Thus, leaders knowing and understanding their employees is essential in order to provide employees individual support needed. \citep{shalley2004leaders}

 As \citet{garvin2008yours} bring out in their article, reasonable question to ask for this fresh leadership approach is, can managers actually be excited about being a facilitator of creative process, and where to find those managers who feel engaged and aspired to that role and want to do it? \citet{lingo2010nexus} has offered one perspective to this question in her study with production of music. She claims that producer is the one bringing it all together; it is actually hard leadership exercise, where people from different fields and teams need to work together for one production, where there are no clear rules for who is controlling the output nor yardstick how good or bad the production is. Through creating a shared purpose and common goal in production team, and while still letting "other apply their distinctive expertise", a producer actually operates at the centre of the storm without being at the focus of attention as well as aims for productivity without being over controlling. According to this example, glory comes from being able to help others to find and realise their unique talents at the same time with achieving a collective goal. 
 
Team members observe and reflect other members responses and actions and attend to them, yet behaviour of the leader is often their particular concern \citep{tyler1992relational}. 

For instance, \citet{edmondson1999psychological} states team leader coaching influencing positively on team psychological safety. Psychologically safe environment includes team leaders being supportive, coaching-oriented, who doesn't response to questions or challenges in defensive manner. Employees are not likely to take interpersonal risks that might lead to learning if leader tends to act in authoritarian of punitive ways. 

\section{Attitude towards failure and risk}

\citet{garvin2008yours} divide organisational failure into three categories: unsuccessful trials, system break-downs and process deviations. In order to learn and develop, these all types of failures need to be recognized and their special characteristics analysed. For instance, unsuccessful trials may foster creative learning, but as important is to overcome failures resulting from deeply ingrained norms that inhibit experimenting. \citep{garvin2008yours}

Individuals are likely to avoid risks and uncertainty, stick to routine and prefer more certain outcomes and ways of performing \citep{bazerman2012judgment}. This does not encourage, however, creative actions that actually require several trial-and-error, iterative processes where risk is involved. Employees fearing risk-taking tend to perform the routine way instead of taking chance with new approaches. \citep{shalley2004leaders} However, in his study \citet{nystrom1990organizational} found that organisational culture reflecting challenge and risk taking lead to more innovative actions of employees and the whole organisation. 

In line with this is the psychological safe environment where new ideas and breaking with the status quo are supported \citep{edmondson1999psychological} and uncertainty is not totally avoided, preferably managed and tolerated \citep{shalley2004leaders}. In order to decrease the fear of failure, \citet{amabile2008creativity} suggest leaders should put emphasis on creating an environment where an employee feels safe to fail and speaking out loud ideas nor making mistakes does not result in punishment or humiliation. Leaders should, instead, motivate and encourage employees to ideate, break routines and learn by stating how essential experimenting, iterating and failing is for learning and developing \citep{amabile2008creativity,shalley2004leaders}. 

When individuals ask for help, admit errors or seek feedback they place themselves under risk, and perceive a threat, fear of being judged and appearing incompetent as well as fear of giving unfavourable impressions on people who have the power to give promotions, raises or who assigns projects\citep{edmondson1999psychological,brown1990politeness}. Even though knowing a team would benefit of this kind of behaviour, perceived threat appears strong. However, this is essential when initiating learning behaviour, and \citep{edmondson1999psychological}

However, innovation processes always include mistakes and failure, from which organisations, teams and leaders need to learn, preferably rather fast. Organization that learns fastest is likely to take the lead, which is the essence of innovation according to \citet{buijs2007innovation}

Feeling of threat and embarrassment is linked to reduce cognitive and behavioural flexibility and responsiveness \citep{staw1989tradeoff} and this leads easily individuals acting ways that rather inhibit than fosters learning \citep{argyris1982reasoning}. 

However, in some organisational environments people do ask help, admit errors and discuss about problems. In these environments employees seem to perceive interpersonal threat low enough to perform despite of the threat. \citet{edmondson1999psychological} has studied working environments and realised in environments employees act despite the threat, they felt safe and supported for their actions. She refers to this as psychological safety. Some studies argue how familiarity among group members is likely to encourage openness towards new information and ideas \citep{sanna1990valence}, yet this factor alone is not sufficient to explain when group members find it safe to act instead of feeling threatened \citep{edmondson1999psychological}.

In creative work risk concerns both the need to do experiments and tolerate failure \citep{andriopoulos2000enhancing,quinn1985managing}, as failing is widely considered as essential part of learning \citep{farson2002failuretolerantleader}. Thus, employees should feel being allowed to conduct experiments and despite the outcome of the experimentation \citep{jung2003role}.

Feeling of self-efficacy may affect individual's willingness to provide unique and novel ideas even when some degree of risk is involved \citep{mumford1988creativity}. Training, coaching, giving feedback and assigning tasks seem to be useful approaches for leaders, who pursue to contribute empoloyee's self-efficacy \citep{amabile1998kill}.

Fear of failure can be decreased through transformational leaders who foster the culture of intrinsic motivation and rewards from creative endeavours, idea exchange and discussion \citep{amabile1998kill}.

According to \citet{amabile1996assessing} creative solutions in an organisation can be achieved by encouraging employees to reach and experiment new perspectives and ways of performing. Essential for this is not being punished for negative outcomes. Organizational environment that allows failing is likely to assist in employees acquiring diverse perspectives and questioning the status quo and habitual way of performing. 

Uncertainty, actually, should not be considered only as a threat or inconvenience occurring in organisations, rather should appropriate level of messiness let exist, and develop opportunities where uncertainty can be exploited. Overall, uncertainty in creative processes should not be overly controlled. \citep{sternberg1997creativity}

Work needs to meet the skills and interests of employees while offering sufficient level of challenges in order to increase the motivation of employees towards work \citep{amabile1998kill}.

Furthermore, employees are more likely to contribute ideas when feeling that failing is safe. Leaders should emphasise that constant experimenting requires failing early and often and through these iterations learning is possible. Equally important is that employees should feel not being punished nor humiliated if mistakes occur or whatever ideas are spoken up. \citep{amabile2008creativity} Also \citet{de2001minority} have found a positive relation between employee's creativity and participative safety, referring to employee's perception of generating ideas without being judged. 

According to various studies, failing and negative consequences are natural part of creative, innovation and learning processes \citep{hennessey19881,shalley2004leaders,andriopoulos2000enhancing} . For instance, \citet{hennessey19881} emphasise that process is likely to have negative consequences, and in the concept of perpetual challenging of \citet{andriopoulos2000enhancing}, adventuring phase includes making mistakes. Thus, when developing novel products and processes, iteration and failure is included, employees need to feel safe to try various approaches and fail \citep{shalley2004leaders}. As discussed earlier, organisational culture has great influence on employee's perception of safe environment for failing.

Predicting the future being impossible, focus should be in managing risks involved in playing with creative ideas in both the company and individual level. As \citet{sternberg1997creativity} state, "as uncomfortable as it is, while not being able to predict and control uncertainty in creative projects, the messiness does have to let exist". \citet{kanter1983change}continues that, actually, opportunities grow from uncertainty and creative endeavours rise when struggling with uncertainty and mess, as individuals impose order where it does not exist, and thus individuals are forced to form new connections. Furthermore, allowing employees freedom to act actually arouses desire to act.

According to \citet{amabile2008creativity} essential part of creating a safe environment for creativity is managers to decrease the fear of failure. Instead, constant experimenting should be the goal of working, learning by doing and iterating until sufficiently is learnt from the process. 

Furthermore, when company grows, it usually leads to more conservative actions and increase in fear of failure. When fearing failure managers tend to deny failure and erase it from the memory instead of learning from it. \citep{amabile2008creativity} 

Also \citet{edmondson1999psychological} relates experimenting tightly to failing, and emphasises team learning, creative problem solving, reflection and overall organisational performance rising from the failure that occurs. However, willingness to interpersonal risk-taking is tightly related to how employees perceive and believe team members or leaders would react and response in uncertain actions or ideas \citet{edmondson1999psychological}. Thus, team's tolerance for imperfection and error should be increased. 

\citet{garvin2008yours} states by creating an environment that serves psychological safety for employees, organisations may capitalise on failure. Safe environment does not humiliate or punish employees for failing or coming up with novel ideas or doubts. 


\chapter{Experimentation-driven innovation}

Failed experiments should not be considered as failing, instead they offer valuable learning points. 

According to Thomke (2003), in the beginning every product is an idea, that was being shaped through the process of experimentation, and the ability to do experimentations is actually a measurement of company?s ability to innovate.

Experimentation is essential in order to learn about the idea, concept and prototype and whether it actually addresses a new need or a problem or solves the one at hand. Prototyping is critical part of the process, as testing the prototype in a real environment gives instant and valuable feedback for further development. 

Thomke (2003) suggests four steps for organizations to be more innovative. First of all, organization should allow and manage the work for the employees so that fast experimentation is possible. This usually requires challenging routine ways of working and shaping the routines, yet fast experimenting is essential in order to get rapid feedback for shaping the ideas. In addition, team engagement is essential, as the whole team need to understand the meaning of experimenting and developing and it should be encouraged to sharing information and ideas in as early stage of development process as possible and throughout the process. Thomke suggests using small teams and parallel experiments especially when the time is the most critical factor. 

Secondly, failing early and often, yet avoiding mistakes is important for experimenting. Failure can disclose important information and reveal gaps in knowledge, and is thus important as early phase of the development as possible. However, according to Thomke (2003), this is not an usual way for an organisation to think about failure, thus building the capacity for rapid experimentation as well as tolerating and learning from failure is essential and often requires overcoming ingrained attitudes. Encouraging and creating a culture where failing is allowed and not being afraid of, brainstorming sessions where judgement is not allowed are important for 

However, Thomke (2003) does not suggest failing and making mistakes as a result of poorly planned experiments. Mistakes and failures produce most value, when the experiment is well planned and the goal or hypothesis that needs to be tested is clear.

Thirdly, anticipating and exploiting early information can save a lot of resources in the development process. If problems are shown in the late-stage of the process, they can be even 100 times more costly than the ones discovered in the early stage. According to IDEO, an innovation and design-firm, using human-centered design-based approach, the key elements in the design process and prototyping is it being rough, rapid and right. The right-element reminds that even though the prototype itself is likely to be incomplete, it has to show the right specific aspects of a product. This forces developers to decide the factors that can initially be rough and those that must be right. In addition, exploiting early information serves as a good method for developers reflecting changing customer preferences. Briefly, information in the early stage of the developing process should be listened and discovered carefully, as the problems are cheaper and easier to solve. 

Lastly, for enlightened experimentation Thomke (2003) puts emphasis on combining new and traditional technologies. For company


\section{Approach of Mind}
Dream and think big, but act small. 

%In the Mind approach, there are three types of ideas: (Voiko t�t� kirjottaa kun ei taida olla viel� ihan "tieteellist�"

Experimentation approach suggests only after one has tried out an idea and performed an experiment, has he enough information and experience to continue to the execution of the idea.  


\chapter{Research surroundings and methodology} \label{researchdesign}
\section{Company description}
The case company in this study is a client organisation of the MindExpe project; Service Foundation for People with an Intellectual Disability (KVPS). The Service Foundation for People with an Intellectual Disability was founded by Inclusion Finland KVTL which is a non-governmental organisation aiming to promote equal opportunities in society for people with intellectual disabilities and their families. The aim of the foundation is to promote a good life for people with intellectual disabilities and their families by lobbying decision-makers and legislators. They co-operate in advocacy work with NGO?s and other parties involved in the field. \citep{kvps.fi}
 
KVPS promotes a person-centred approach to the lives of people with intellectual disabilities, promotes their full citizenship rights and carry out development projects and organise various kinds of trainings. In addition they offer wide variety of respite care services to cater for the different needs and situations of families and people with special support needs and acquire apartments for young people and adults with intellectual disabilities who wish to live on their own. \citep{kvps.fi}
 
KVPS Tukena Ltd (later will be referred as Tukena), as part of KVPS, focuses on providing diverse, person-centred support services in partnership with local authorities and other providers. KVPS Tukena provides different solutions and housing services for young people and adults with intellectual disabilities who wish to live on their own, one of them being group housing. 10 out of 14 interviewees in this study were employees in several Tukena housing units in Finland. Rest of the interviewees worked in the operations development unit in various projects. \citep{tukena.fi}

\section{Experimentation challenge description}
In order to study experimentation behaviour in an organisation, an experimentation challenge was designed and organised. 
The challenge was organised separately for two client organisations of MindExpe project: The K-Retailer's Association and 
Service Foundation for People with an Intellectual Disability (In Finnish, Kehitysvammaisten Palvelus\"a\"ati�, KVPS). The data 
analysed in this thesis is gathered from different service units of KVPS and KVPS Tukena Ltd, which is a part of KVPS, 
focusing on providing support services for people with an intellectual disability. 

The kick-off for the experimentation challenge for the management level was held in April 2013. As the approach for 
developing through experimenting is not yet widely studied nor recognised way of working in the client organisation, 
during the launching two researchers of Mind told briefly through examples about the approach. Furthermore, practicalities 
and frames for the competition were presented. The managers of the units were thus given the responsibility to bring the 
information of experimentation challenge to their units. Mind team offered posters where steps for ideating and 
experimenting were presented.

Time for experimenting was from 24th of April until 11th of June. However, as it took few days for the leaders to inform their employees about the challenge, actual time for experiments was six weeks. During the challenge participants were asked to 
perform quick and easy experiments, reflect the learnings of them and report the experimentations through either an online or paper formula. In the formula employees were asked to describe the idea, experiment, how they conducted it, how it went and what they learnt; what was successful and what left something to improve. 

Each unit participating experimentation challenge was responsible for its own activity. After the kick-off for experimentation 
challenge project leader called once to immediate superior of some units in order to gain knowledge how the team is 
contributing to the challenge and whether experimentations are conducted or not. However, no additional support or advising 
was given to units, and teams were self-driven in their activity. 

During an experimentation challenge, participants, employees of the company units, were asked to ideate ways to improve the work life, from the perspective of an employee and especially from the perspective of a customer. In addition to only ideate, they were encouraged to plan as small and easy way to test the idea as possible, in order to perform it during the time frame. Intentionally, Mind team did not restrict the style, theme or ways of experimenting. This let participants participate in a way feasible for them, their unit and working pace.

Experiments were then reported to the jury, which consisted of members from both the development and management team of KVPS and Mind researchers. Best experiments and best reflections (experiments that helped the team to reflect and learn more about the idea, whether or not the experiment itself was successful) were rewarded in the closing session of the experimentation challenge as well as the unit that performed most experiments. In the evaluation process, jury focused on how well the goal of an experiment was recognised and kept in mind, how useful the experiment was (for instance for work efficiency or customer satisfaction), and what was learnt from the experiment. 

During the experimentation challenge KVPS Tukena reported 33 experiments and 20 were reported by the foundation, so altogether 53 experiments were reported through an online form or traditional paper form. Experiments themselves were not further analysed in this study, as in the focus and interest of this study is the experience of an employee of the experimentation process. 

Experimentation challenge was essential part of empirical study, which overall took place during the year 2013. The experimentation challenge was organised during the spring and summer, following the closing session with rewards and interviews of 14 employees during the autumn 2013. Detailed dates of the challenge are described in table \ref{tbl:schedule}

\begin{table}[htcb]
\begin{center}
\caption{Schedule of experimentation challenge}
\begin{tabular}{ | p{2.3cm} | p{4.5cm} | }
\hline
	23rd of April 2013 & The experimentation challenge was launched to the whole KVPS and Tukena Group  \\  \hline
	24rd April to 11th June 2013 & Experimentation challenge   \\ \hline
	20th of September 2013 & Closing session of the challenge and rewarding winners \\ \hline
\end{tabular}
\label{tbl:schedule}
\end{center}
\end{table}

In order to better understand the practicalities and structure of the challenge, experimentation challenge was first pivoted with two units of KVPS and two stores of K-Retailer's Association, before the actual challenge for the whole organisation was launched. However, the data gathered for this study does not consist of the interviews made from the pilot challenge, yet they gave the direction and frames for the actual experimentation challenge and assisted in framing the structure for the interviews.


\section{Research methods}
Considering the complex and uncertain characteristics of organisational and human behaviour under study, qualitative research was used. According to \citet {morgan1980case} qualitative research should be seen as an approach rather than a set of techniques, and especially when exploring social phenomenon, as in this study, qualitative research serves a relevant approach.

To enhance the credibility of this research and address the concerns associated with interpretive research, systematic and transparent description of the research methodology, data gathering and analysis process is offered. In addition, quotations from the data are presented widely in the results and in thematic and interpretive analysis process two researchers and supervisor of this thesis have been involved. 

\subsection{Case study}
The most recognised case study research is based on the work of \citet{yin1989case} and \citet{eisenhardt1989building}. According to their perspective, case study refers to a method dealing with contemporary phenomenon in real life context. Case study method is ideal in studies where boundaries between the phenomenon and its context cannot be clearly delimited. \citep{yin2014case} This is rather accurate when considering organisational change and behaviour, learning and applying novel approaches for developing. 

In case studies research question and boundaries of the study are expected and allowed to change during the study. Thus, in the beginning of the study too strict premises might create biases and limit the results. \citep{eisenhardt1989building}
According to \citet{eisenhardt1989building} no theory is necessarily needed in the beginning of the study, yet some basic theoretical assumptions are required to use as a guidance in the empirical world. 

When in the field of qualitative research, case study method can be used both in theory building and theory testing. Furthermore, it can also serve as a method for interpretive research design, which allows the constructs of interest emerging from the data and not to be defined and known in advance. In interpretive research social reality is seen as embedded within their social settings, as well as it is impossible to abstract it from them. Researchers then focus on interpreting the reality using sense-making process in comparison to hypothesis testing process. \citep{bhattacherjee2012social} 

While case studies are allowed to bring forth theories from empirical data, the empirical findings need to emerge through theory. Theory and findings are mutually dependent, the empirical data affects theories and theories need the verification from empirical findings. Thus, oftentimes case studies are conducted iteratively, the whole process including retesting and redefining of theories. \citep{dubois2004research}

\citet{yin2014case} concludes ideal usage for case study: when aiming to answer 'how' or 'why' -questions, when participants cannot be manipulated and when real life context is significant for the study as well as when the case and the context boundaries remain complex or unclear. In these occasions, case study is an appropriate approach. These aspects resonate well with this thesis, as the focus is on answering "How experimentation-driven developing can be fostered in organisations?". Furthermore, participants were only given an introduction to experimentation-driven development, and their behaviour during the challenge depended solely on themselves and the work team, and researchers could not manipulate participants. When studying factors affecting experimentation behaviour in organisations, real life context is significant, yet boundaries of the work and the context of developing remain complex and unclear. \citep{yin2014case}

In this thesis, initial research questions were formed based on previous empirical and theoretical findings of experimentation-driven approach on organisational innovation. Urge to study how experimentation-driven process could be fostered in organisations served as an inspiration for the case study setting and literature review. Deeper study on literature review was conducted after gathering empirical data and rising relevant themes to study further: learning and creativity in organisations, experimentation-driven approach as a tool for learning and factors affecting experimentation behaviour. 

\section{Data gathering}
\textit{"I want to understand the world from your point of view. I want to know what you know in the way you know it. I want to understand the meaning of your experience, to walk in your shoes, to feel things as you feel them, to explain things as you explain them. Will you become my teacher and help me understand?"} \citep{spradley1979ethnographic}
\newline

As quoted above, interviews are an essential method for gathering information of interviewees thoughts and experiences, feelings and knowledge, ideas and preferences. Open-ended and semi-structured interviews leave space for all of the above mentioned, leading to highly qualitative data. \citep{monroe2001evaluation} In order to form understanding of employees perspective and experience of experimenting, semi-structured interviews were conducted. The structure of the interviews can be found in appendix \ref{haastisrunko}. 

The empirical data comprises of 14 semi-structured interviews. Interviewees were employees from KVPS Tukena housing service units and KVPS foundation. The author carried out all the interviews face-to-face with the interviewees. Interviews lasted from half an hour to an hour. The interviews were recorded and transcribed. All interviews were held in the interviewee's mother tongue, Finnish, therefore all the quotes presented in the thesis have been translated into English. Interviews concentrated on finding advantages and challenges concerning experimentation behaviour and experimentation-driven development. The interviewees' roles and work experience are summarised in table \ref{tbl:interviewees}. 

\begin{table}[htcb]
\begin{center}
\caption{Work experience and work description of interviewees}
\begin{tabular}{ | p{2.3cm} | p{2.5cm} | p{2.5cm} | p{3.7cm} | }
\hline
	\textbf{Interviewee} & \textbf{Years in current unit} & \textbf{Years of work experience} & \textbf{Work description}   \\ \hline
	1 & over 10  & over 10 &   \\  \cline{1-3}
	2 & 1,5  & over 10 & Development of    \\ \cline{1-3}
	3 & under 1 & over 10 &  operations   \\ \cline{1-3}
	4 & over 2 & over 20 &   \\ \hline
	5 & over 1 & over 10 &    \\ \cline{1-3}
	6 & under 1 & under 1 &    \\ \cline{1-3}
	7 & under 1 & over 10 &    \\ \cline{1-3}
	8 & over  2 & over 15 & Daily routines in   \\ \cline{1-3}
	9 & over 1 & over 1 &  housing service unit   \\ \cline{1-3}
	10 & over 1 & over 25 &    \\ \cline{1-3}
	11 & round 2 & over 10 &    \\ \cline{1-3}
	12 & round 2 & over 5 &    \\ \cline{1-3}
	13 & over 1 & over 10 &    \\ \cline{1-3}
	14 & over 1 & over 10 &    \\ \hline
\end{tabular}
\label{tbl:interviewees}
\end{center}
\end{table}

As this thesis is written as a part of a MINDexpe project, and the data collected will be used as a part of other MIND researchers doctoral studies, collecting data from the perspective of factors affecting experimenting was not the only topic of concern. Thus, all of the data in interviews were not straightly relevant to interest of this thesis. This, in turn, was a fruitful position to conduct a case study with thematic analysis, as too strict hypothesis were not formed in the beginning of the study and more space were left to essential themes to rise from the interviews. 

Interviews were carried out after the experimentation challenge. Interviewees were chosen from the units of KVPS and Tukena Group so that both active and less active units were heard. Interviews focused on identifying factors affecting experimentation behaviour of an individual in organisational context. The structure of the interview can be found in the appendix \ref{haastisrunko}. Even though factors affecting experimentation behaviour were mainly in focus, during the analysis process another theme was recognised from the data: effects experimenting has on an individual. 

If the interviewee had not taken part in the experimentation challenge the interview focused on finding whether the routine work of an interviewee consisted of characteristics of experimentation behaviour, for instance ideating. Discussions with immediate superiors of the interviewees as well as interview notes served as a tool for gaining an overall understanding of the routine work and attitude towards experimentation behaviour, but were not included in the research data. 

\subsection{Analysis process}
Thematic analysis was used to analyse the data collected. The focus was on recognising factors affecting experimentation behaviour of an individual. Analysis process followed the idea of \citet{braun2006using} step-by-step process, which is used to identify, analyse and report patterns within data without being tied to any pre-existing theoretical framework. Whereas closely related method grounded theory focuses on theory building about the social phenomenon being studied, thematic analysis can be used more flexibly without detailed knowledge of theoretical framework. \citep{bhattacherjee2012social} In this thesis, thematic analysis assisted in recognising and identifying factors affecting experimentation behaviour in organisations. In this section analysis process is presented. Along the analysis method, author wrote research diary, which collected the phases of empirical analysis process.
\newline
\newline
1. Transcribing of interviews and overview of interview notes \newline
The author transcribed four of the interviews; the transcribing of the rest was outsourced. In addition, interview notes were read in order to form understanding of highlights and to get into mindset of categorisation.
\newline
\newline
2. Coding of the interviews\newline
After all interviews were transcribed, they were read through in order to create a preliminary understanding of the data collected.  
To further analyse the material, transcriptions were coded into initial themes arising from the interviews. In this phase, large amount of themes were identified. 
\newline
\newline
3. Initial categorisation \newline
Together with two researchers initial categorising under themes that seemed most appropriate was made. Even though the main focus of the study was to find factors affecting experimentation behaviour, in the beginning of the coding various interesting aspects of experience of experimenting rose from the data. Thus, another perspective, how experimenting affects an individual was refined. The initial categorisation can be seen in the table \ref{tbl:initialcategories}.
\begin{table}[htcb]
\caption{Initial categorisation}
\begin{tabular}{| p{4cm} | p{6.5cm} |}
  \hline
    & Structures and practices \\ 
    Factors affecting & Business field \\  
   experimentation& Job description \\ 
   & Process \\  
   & Idea \\   
   & Influence of an immediate superior  \\  
   & Understanding of experimentation  \\ 
   & Climate  \\  
   & Individual characteristics \\ 
   & Licence to conduct experiments \\  
   & Co-creation\\  
   & Expertise  \\  
   & Knowing of the customer  \\  \hline
    & Individual \\ 
    Experience   & Idea \\ 
    of experimentation & Process \\ 
    & Team \\ 
    & Making abstract ideas concrete\\ 
\hline
\end{tabular}
\label{tbl:initialcategories}
\end{table}

4. Theme creation and category refinement \newline
As can be seen in table \ref{tbl:initialcategories}, various categories were identified in the initial categorisation. However, the categories were rough and needed specification and further analysis. The analysis process continued with combining and examining the categories, leading to the second categorisation, which is described in table \ref{tbl:secondcategories}. More appropriate themes and refinement of categories was done by bringing together themes that were closely connected and eliminate those without many quotations. This phase was done with assist of the supervisor categories in order to enhance the credibility of the study. 

In this phase, context dependent organisational and business variables (business field) were considered as one category, yet in order to keep the focus, this was further left out of the review and analysis. 

\begin{table}[htcb]
\caption{Second categorisation}
\begin{tabular}{| p{4cm} | p{6.5cm} |}
  \hline
    & Structures and practices \\ 
    Factors affecting & Business field \\  
   experimentation& Idea \\ 
   & Leadership behaviour  \\   
   & Climate  \\  
   & Individual know-how \\ 
   & Individual characteristics \\ \hline 
    & Emotional level \\ 
    Experience & Perspective towards work \\ 
    of experimentation & Personal development \\ 
\hline
\end{tabular}
\label{tbl:secondcategories}
\end{table}

5. Refinement of final themes and categories \newline
Final step of the analysis process resulted in describing the classes and categories in detail. The refinement focused on describing categories and subcategories better to response to research objectives, as many of them were described in rather abstract level. Results are presented in chapter \ref{results}. The analysis process resulted in two classes, which were divided into categories presented in table \ref{tbl:classes}: 
\begin{table}[htcb]
\caption{Final classes and categories}
\begin{tabular}{| p{6cm} | p{7cm} |}
  \hline
   & Role of the immediate superior \\ 
   Class 1: & Role of the team \\ 
  Factors affecting experimentation & Structures and practices of developing \\ 
   & Characteristics and know-how of an employee\\ \
   & The gap between an idea and experiment \\ 
    \hline
    Class 2:  & Emotional experience and engagement \\ 
     How experimenting affects an individual & Learning\\
  \hline
\end{tabular}
\label{tbl:classes}
\end{table}

\chapter{Results and analysis}
Two classes were identified in the analysis process: factors that have an effect on experimentation and effects that experimentation has on individual. This chapter introduces the results of the study. The results are described in two categories which are further divided into several subcategories. 

\section{Factors affecting experimentation behaviour}
A suitable context for experimenting was defined by the interviewees through interviews, and several factors were identified that affect in a way or another on experimentation behaviour of an employee in an organisation. In the analysis process, five different categories were formed of factors affecting experimentation. Table x summarises those categories and subcategories. 

The experimentation process consists of ideating, planning an experiment and conducting the experiment as well as reflecting and learning from it in order to start the iterative experimentation process. Factors affecting these phases were identified from the data. 

\subsection{Role of the immediate superior}
Different kind of leadership behaviour that affects experimenting was recognised from the data. This category consists of actions an immediate superior can perform in order to encourage or discourage experimentation. Three main themes were recognised from the data, which are presented as subcategories Leading by example, Supporting ideation and experimentation and Giving license to do experiments.

\subsubsection{Leading by example}

According to the study attitude and actions of an immediate superior towards developing are important factors for the organisational unit as a whole. Interviewees claimed that experiments rarely happen if the immediate superior is not involved in the experimentation process and his attitude towards new ideas and developing is passive or negative. 

As interviewee 5 noted, especially when the work environment is passive towards developing and experiments, immediate superiors should act as role models and by own example create a trustworthy environment where employees can ideate and conduct experiments without fearing failure. Especially in a situation where an employee lacks support from colleagues for his idea and leading to disappointment, immediate superior can lead by example and join in the experiment in order it to occur and encourage the whole team to conduct experiments.
   
\begin{quote}
``Often they get shut down, new ideas, which is very sad, then I do not feel like even trying anymore, and in this point the leader is required. That he joins and says that now we will try this.. then it will succeed, but if it is only among colleagues, they [experiments] usually do not happen.'' [Interviewee 5]
\end{quote}
\begin{quote}
 ``Our immediate superior is such a lovely person, real idea bank herself! -- Luckily she is very development-oriented.. I mean it is nice that she does not stick to routines either.'' [Interviewee 11]
\end{quote}
Altogether, experiments are more likely to happen in workplace when immediate superior leads by example in ideating and conducting experiments himself, as can be seen in the comment above from interviewee 11.

\subsubsection{Supporting ideation and experimentation}

According to the study immediate superior?s support and encouragement towards experimenting was experienced highly important in order experiments and ideation to occur among employees. Interviewees reported how the support from the immediate superior gives freedom to try out new practices, be creative and ideate together. This can be recognised from the comment of interviewee 14. Support from the immediate superior also encourages an employee to test ones limits, utilize one?s working experience and abilities. 
\begin{quote}
 ``Both the immediate superior and the nurse in charge supported right away when they knew I am good at handwork, so they told me to use it as much as possible and experiment with customers.. and they told everyone the same.''[Interviewee 14]
\end{quote}
The immediate superior of an employee acts also as a bridge between the employees and upper level management bringing ideas from the organisation unit to upper levels. Few interviewees reported their immediate superior being extremely supporting and fighting for employees? ideas. According to interviewees, these superiors received support from upper level management as well. 
\begin{quote}
 ``And if we talk about even bigger experiments, so that we have to ask from upper level management, she [immediate superior] usually conveys our ideas further. So we get quite well support from there as well and it is only rarely when some idea is being shut down right away.'' [Interviewee 10]
\end{quote}
In most occasions where interviewees experienced support and encouragement towards experimentation, an immediate superior was himself very keen to developing, trying out new practices and experimentation-driven approach in work.   
\begin{quote}
``It [feedback and appreciation from the upper level management] makes the experiment more viable, appropriate and bold.'' [Interviewee 3]
\end{quote}
According to the study some immediate superiors show appreciation and support by noticing and rewarding conducted experimentations or successful ideas. This gives employees feeling that the experimentation and their work is meaningful and is thus likely to encourage experimenting behaviour. Interviewee 3 states above, how the experimentation becomes bigger through appreciation and support. 

\subsubsection{Giving licenses to do experiments}
However, immediate superior can not only encourage and support his employees in ideation and experimenting with words, he also has to allocate and allow resources for experimentation to happen. Thus, immediate superior is responsible for creating both environment and tools where experimenting is possible and resources are allocated for it. Many similar comments like interviewee 14 states below were recognised from the data. 
\begin{quote}
``Experimenting is possible exactly because the management is positive towards things like that. It has direct influence.. and that I can buy equipment I need and I get a possibility to organize new kind of activities and they don?t resist it..'' [Interviewee 14]
\end{quote}
Important part of allowing experiments and giving license to conduct them is allowing them to fail. If failing when trying something new is considered punishable or the goals are exaggerated, it is likely to discourage employees to conduct experiments. In contrary, one interviewee described how his immediate superior gave license to do one experiment and promised to uphold an employee if some negative feedback or results occur from it. 

Furthermore, immediate superiors may even demand developing and doing experiments by explicitly requesting creative and innovative solutions, as can be seen in the comment of interviewee 1. 
\begin{quote}
``She [immediate superior] will give space to that [experimenting] and actually even requires that we start experimenting and ideating. So, that is the idea of all these projects, to be able to create new ways of doing things.'' [Interviewee 1]
\end{quote}
However, in some units there was a clear contradiction between the request of experiments and the resources allocated for ideation and experimentation. Again, in few units immediate superiors had taken this into account and provided time and resources for experimentation behaviour. 

Furthermore, as interviewee 8 noted, an essential part of experimenting is it being voluntary.
\begin{quote}
``In a certain way I think that workplace should encourage [to do experiments], but it cannot be forced..''[Interviewee 8]
\end{quote}
Freedom to ideate and participate experimenting depending on own motivation and interests as well as freedom to not do so was experienced important among interviewees. According to the study there is a significant difference whether the leader or a team encourages an employee to do experiments or if he is forced to do so. When feeling free to try out new things, ideate and develop himself and his work without asking permission constantly from different parts, an employee is more likely to perform experiments and develop his work.

\subsection{Role of the team}
Several aspects on how different characteristics of the team affects on experimenting behaviour were identified in the study. Subcategories Democracy and low hierarchy, Supportive climate and team practices towards ideating and experimenting, Attitude towards failure as well as Engagement of the team are described below.

\subsubsection{Democracy and low hierarchy}
Few interviewees reported their organisational unit having a low hierarchy making it possible to perform spontaneous experiments and tasks without asking permission and opinion from many parts. As interviewee 13 summarises, this was seen to lower the threshold and encourage experimenting.
\begin{quote}
``The low hierarchy kind of.. when you are in a big institution you always have to sort out if you can get the car of the institution and many other things, so here we don?t have those kinds of things.''. [Interviewee 13]
\end{quote}
However, most of the interviewees described the democratic view being strong in addition to low hierarchy. Overly democratic environment and decision-making in a team can either courage employees to participate in ideating and in experimenting or make the environment too passive for actually performing experimentations, when one always has to have the majority on his side in order to try out new things. Interviewee 6 emphasises this in the quotation below. 
\begin{quote}
``There is a lot this kind of where you have to take the whole work group into account, big workgroup, as team work of course takes its own time so that everyone will then be, involved in developing.'' [Interviewee 6]
\end{quote}
\begin{quote}
``So there are once in a while divergent opinions. But then we discuss, and we decide together what will be done. And everyone has to kind of work like has been agreed. So if the majority [of employees] says we will do like this, then we will do like that.''[Interviewee 9]
\end{quote}
Furthermore, when ideas are turned into experiments and practice, interviewees reported team size being a factor affecting on how efficiently or well experimentation is executed. When a size of the team is large, meaning over five employees, it is more challenging to get all employees involved and hear everyone?s opinion. In turn, if there are no people to reflect one?s ideas with, ideas may remain in one?s head and never come alive. A compromise would be needed in between these aspects. 

In addition, ways of sharing information and ideas among team is essential for experimentation. These aspects are presented more deeply in the next subcategory Supportive climate and practices. 


\subsubsection{Supportive climate and team practices}
Climate among the team seems to affect a lot on how easily ideas are said out loud in a team and experimentations performed. Factors such as open and creative atmosphere and support and positive feedback from colleagues have clearly a positive boost towards experimenting in an organisation.

A following pattern was identified from the data: Employees tend to ask opinion and permission from immediate superiors while also having a need for support from the team. Emerging ideas are preferably discussed with closest and most trustworthy colleagues in order to receive support, feedback, deeper understanding and reflection for employee?s idea. Only after receiving other opinion and support are they brought to a team meeting under discussion. If the climate does not support ideating nor is safe for throwing ideas, employees are very unlikely to tell their ideas to others. 

However, interviewees, reported conversations and ideating sessions together with the whole team being important as they may encourage others to ideate and tell their suggestions out loud. In addition, one is likely to achieve better results with a team than only ideating alone. One person reported realizing it being important to say ideas out loud despite feeling insecure as it may lead to surprising outcomes and encourage more silent colleagues to participate in ideating. 

In addition, interviewees described ideas starting to grow the more they are thrown into discussion, and the heterogeneity of the team being mostly inspirational and beneficial in ideation phase. Interviewee 6 summarises the power of ideating in teams. 
\begin{quote}
``But then I realize that the more I say my ideas out loud, others also get excited and ideas keep coming. So it is worth speaking, even though sometimes one might think that I cannot be always talking, so it is good to keep on talking and others will follow..'' [Interviewee 6]
\end{quote}
\begin{quote}
``We also have few who are not that active in throwing ideas or performing and they like the routines, I guess the workplace could somehow support them in developing..'' [Interviewee 11]
\end{quote}
Furthermore, as interviewee 11 states above, team and climate should encourage and support employees to participate in ideating and experimenting in order an employee to overcome oneself and gently push towards new ways of doing things. 

\subsubsection{Attitude towards failing}
While discussion and giving and receiving feedback are essential in ideating phase, license to fail plays a major role when an actual experimentation is performed. Team being judgemental towards ideas and experiments can prevent actual learning and reflection of experimentations.  

Thus, one part of the supportive climate towards ideating and experimenting is the attitude towards the results of experiments. Seems that if the climate in the team allows failing and does not take it too seriously, ideation and experimentation occur more often than if a team is afraid of failing. Quote from interviewee 1 describes attitude in their unit towards failing, several similar comments about trying again together and allowing failure were recognised from the data. 
\begin{quote}
``Well no one will get punished or be thrown tomatoes at [if an experimentation fails].. I think we go through the idea and experiment, and try another way.. We do not have here that kind of attitude that we would not be allowed to fail.'' [Interviewee 1]
\end{quote}

Most interviewees described the team being very supportive and attitude towards failure positive and constructive. They rather see failure as an opportunity or a learning point. Some interviewees described going through a failed idea or experiment among team in order to find an alternative way to test the idea. This attitude seemed to help the team in experimentation behaviour. 
\begin{quote}
``I think the team reacts very well to it [failing], kind of laughing and saying that these things happen and are part of this work.''[Interviewee 6]
\end{quote}
As interviewee 6 described above, humor was also described as an important way to cope with failures and to support team members in their work. 

\subsubsection{Team engagement}
While most of the interviewees reported the workplace environment being very democratic and discursive, they described a high possibility that experimentations are not likely to happen if everyone in the team is not involved and engaged to turning idea into an experiment. Comment from interviewee 9 describes this further.
\begin{quote}
 ``I think that the workplace has a major role [in experiments to happen], and especially that everyone are engaged. Thus we get things going and forward..'' [Interviewee 9]
\end{quote}
Interviewees reported feeling frustrated when realizing how all colleagues who are involved in the experiment are not engaging to it, thus preventing an experiment to happen and get relevant feedback from it. One way to motivate and engage the team was found from the data and is presented in the subcategory Characteristics of the idea and experiment under category Gap between idea and experiment. Furthermore, the team is more likely to engage to an experiment when the purpose of the experiment is clear and shared goal exists.  

In addition, as interviewee 14 mentioned below, in order to foster the engagement of employees and the team, a team could encourage new employees to utilize their own strengths and ideate courageously. Few interviewees described their colleagues being highly supportive towards one?s special abilities and skills, and encouraging everyone to use them freely at work. This was experienced as improving the level of engagement towards developing and ideating.
\begin{quote}
``Every employee is allowed to use those resources and creativity that one has in the job..'' [Interviewee 14]
\end{quote}
Shared and understandable goal for an experiment forms a strong basis for idea turning into an experiment. According to the interviewees employees and the whole team is more likely to engage to the experiment if they understand in a deeper level the reason and goal for experiment. Thus, as following quote from interviewee 11 suggests, employees are more easily involved if the goal for the experiment can be clearly justified. 
\begin{quote}
``There we had a clear goal, that somehow we just have to make it work. Then we just processed it and nothing special, it was that kind [of an experiment] with a clear goal, yep.'' [Interviewee 11]
\end{quote}

\subsection{Structures and practices of developing}
According to the study, various organisational structures and daily practices can support or prevent experimentation and ideation. First of all, the meaning of Resources allocated for ideation and development is presented. Secondly, seems that most of the units interviewed lack of systematic way for Collecting ideas, resulting to inefficient way of developing. In addition, in order the change actually happen Implementing new ways of working has to be considered. 

\subsubsection{Resources allocated for ideation and development}
Only little or no solid structure for ideation and developing was found in organisational units throughout the analyzing process. Interviewees described ideas usually emerging when a problem is encountered and an alternative way of performing is needed. In some occasions, ideas emerge accidentally  in conversations with colleagues, team meetings and rarely in meetings where developing is a specific agenda. Rather than developing purposefully interviewees described their daily work as practice-driven. 
\begin{quote}
``It can be that some person suddenly brings a good idea or then we have a specific team meeting, where the idea is to develop something. Or then some new idea might come up in some bigger meeting by accident..''[Interviewee 4]
\end{quote}
No time allocated for ideation and development of ideas leads easily to a situation where routines are repeated. In every interview time came to prominence as a lacking factor preventing ideation and experimentation from happening. Interviewees described the usual way for telling ideas and planning experiments being weekly or monthly team meetings with colleagues. As interviewee 1 states, however, these meeting usually did not include specific time for ideating and developing, more did they concentrate on routine issues. Most of the interviewees wished to have more time for ideation, developing and implementing experimentation-driven development to daily routine. Only one interviewee described having enough time for ideating. 
\begin{quote}
``Well there is not that much time to ideate during them [weekly meetings].'' [Interviewee 1]
\end{quote}
At present, most interviewees described how developing is not seen as a routine part of work but as an additional part that needs time allocated for it. Few interviewees, like interviewee 2 above, experienced experimenting challenge as a refreshing way to remind the workplace of challenging conventions and the daily routine and even thought about it becoming an annual tradition. The interviewees described that reflection is more likely to happen in between different projects and in project-type work, and when the emphasis of the work is not project-like, developing is more likely to be put aside. 
\begin{quote}
``Once a year could be kind of more intensive period or so, maybe it would maintain that no one would be too routinized. And especially in these projects that last long, so long that they are actually no longer projects, it could be quite good..'' [Interviewee 2]
\end{quote}
One interviewee described peer resources being used only little in order to exchange ideas and best practices. He suggested more meetings and ideation sessions with peer colleagues throughout Finland or even abroad in order to exchange opinions, gain perspective and find fresh, new and valuable practices to daily routine work. 
\begin{quote}
``Well, I guess the workload of some people is already so huge, causing also that people start doing things in the same way, continuing the routine..'' [Interviewee 5]
\end{quote}
Quote below from interviewee 5 clarifies how heavy workload can lead to repeating routine way of working. Sick leaves, heavy workload and hectic pace of work all affect on motivation and possibilities for developing and ideating. Furthermore, resources such as money were mentioned during the interviews as a lacking factor preventing experimentation. 

In turn, interviewees who described successful experiments said an essential factor for the success was that all the resources such as people, equipment and time were at the right place at the same time and there where no hindrances preventing experiment. Thus seems that an aspiration for using resources efficiently is essential. 

\subsubsection{Collecting of ideas}
In addition to rather usual communication problems in an organisation, the interviewed units lacked a working system for collecting ideas or feedback from experiments. The need for collecting ideas systematically was however recognised, as interviewee 5 emphasises.. Interviewees report daily work and notions in a system called DomaCare, and all employees are responsible for reading both those notes as well as ones from team meetings. 
\begin{quote}
``So of course during this one year we have had all kinds of good and bad ideas, and they are not documented.. So.. It could have been a good idea to document them..'' [Interviewee 5]
\end{quote}
However, DomaCare is not a place for new ideas, and people not reading what has been written in DomaCare remain a problem. Information is exchanged when work shift ends and other begins. Interviewees described through conversation essential information is likely to come up, thus telling about ideas and experiments should be obvious. For instance a situation where something has been already experimented before, but no reporting was made; without conversation the same experiment may be performed again, resulting to waisting time and other resources.In turn, interviewee 4 describes how successful experimentations are shared with a team. 
\begin{quote}
``But yep, if someone has experimented some good thing in his own project, it will be informed to others as well. Like hey this is what we have and this is worth experimenting.We have regular team meetings where information is shared widely, so we now what others are doing.''[Interviewee 4]
\end{quote}
The interviewees regarding reporting of experiments brought up somewhat contradictory points of views. Some interviewees claimed positive experiments being more under discussion and reporting, whereas others emphasized how failed ones raise more conversation and opinions among colleagues. However, clear and systematic practice for this was not recognised, and seems the most popular mode to share knowledge remains face-to-face conversations.

\subsubsection{Implementing new ways of working}
After experimentation being successfully executed, interviewees described the major difficulty lying behind the implementation process. Interviewees reported insightful experiments and solutions that would be important to implement in the daily routine. However, seems that structures and practices easily prevent implementing new ways of performing. For instance, new practice taking more resources in the implementation phase yet being more efficient and helpful in the long run, is more likely to be turned down and workplace sticking to old routines. 

Interviewee 5 describes one way how interviewees have tried to ease the difficulty of implementing new way of working; Deciding person or people who are in charge of the change to happen in the beginning. 
\begin{quote}
``And then, if it requires actions and processing we agree on who will start to do it. So that it will not remain only in speech, as happens so often.'' [Interviewee 5]
\end{quote}
Same phenomena and strong synergy to the difficulty of the implementation of a new routine is described in the subcategory Static Friction, which can be found under a category Gap between an idea and experiment. 

\subsection{Characteristics and know-how of an employee}
In the analysis process occurred that individual characteristics and knowledge of an employee could assist experimentation-driven approach in development as well as prevent experimentation from happening. In this category these factors are presented in three subcategories. Substance know-how explains the extent to which prior working experience can both encourage experimenting and in turn lead to repeating routines and resisting change. Individual characteristics consist of the factors how tolerance of uncertainty, employee?s self-criticism and confidence affect experimenting. Attitude towards development of an employee also has a major impact on experimenting behaviour, whether an employee considers developing part of the work or not and whether he is open for new ideas and breaking routines. 

\subsubsection{Substance know-how}
According to the study prior work experience affects on the threshold for experimenting especially when experiments concern customers. Through substance knowledge an employee can gain wide understanding of the field, customer and ways of working, which can assist experimenting by adding the courage of the employee to perform an experiment. Knowing the customer, stakeholder and field of work are likely to add self-esteem and employees self-image as workers. 

Those interviewees who were recently graduated and had little previous working experience described it taking time to get to know the routines and the organisational culture as well as customers before actually being ready to suggest anything new or perform experiments. They felt easily insecure and described it important listening to more experienced colleagues and asking their opinion about new ideas before experimenting. When asked what is needed from an employee to begin with performing experiments the interviewees reported experience and knowing what to do being essential. Interviewee 6 describes the meaning of work experience to the ability to be creative and ideate. 
\begin{quote}
``In the beginning it of course took some time for me to learn the basics of the work, so maybe my own creativity and ability to ideate now grows with the experience of this work..'' [Interviewee 6]
\end{quote}
Furthermore, interviewees who had several years working experience, like interviewee 14 below, considered as a positive factor that they were able to combine previous experience to present work environment and customers. Working with same customer segment in different units gives perspective of what kind of ideas can be easily experimented and what may need more effort and resources. 
\begin{quote}
``I used my previous experience as an instructor.. It was useful for this experiment..'' [Interviewee 14]
\end{quote}
In addition, working many years with same customers leads to a high mutual trust between an employee and a customer. This again eases suggesting new ideas and performing experiments with customers. 
\begin{quote}
 ``I think there is the gained trust that has grown during this working journey. So that.. I think there is no resident [customer] who could not join doing these [experiments].''[Interviewee 7]
\end{quote}
In turn, many years of working experience from the same field and similar customer segments can also lead to repeating familiar routines and resisting change. Few interviewees described being essential that there are also newly graduated people or trainees with no prior experience in order to more clearly perceive unnecessary routines and bring new and fresh ideas into workplace. 

\subsubsection{Tolerance for uncertainty, self-criticism and confidence}
According to the study, the way an employee tolerates uncertainty may have high impact on the experimentation behaviour. When asking how interviewees feel experimentation-driven approach affecting on their work few of them described the difficulty lying in the feeling of uncertainty and incompleteness. Where some experienced uncertainty and incompleteness as threats and anxious factors in work, others emphasized those being factors that make working interesting. Interviewee 4 emphasises the ability to tolerate uncertainty and own failures.
\begin{quote}
``On the other hand it [uncertainty] is richness in work, so that one will not get too routinized. But of course one has to tolerate the uncertainty and has to tolerate your own failures, that ?ok, this time I chose wrong?..''[Interviewee 4]
\end{quote}
Furthermore, high level of self-criticism may prevent an employee from conducting experiments. Few interviewees described how they usually like to spend a lot of time in planning and refining a new idea before trying it in action. However, through the deadline and the pressure of experimentation challenge they were able to lower the level of self-criticism and try something incomplete. Interviewees learnt surprising facts already from the small experiment, and were overall satisfied experimenting even though they were not totally satisfied with the idea or experiment. As interviewee 4 states, a trifle of pressure can boost experimenting.
\begin{quote}
``I would have probably thought about this idea for ages and be like this is not good enough yet and it is not perfect''. [Interviewee 4]
\end{quote}
Self-criticism is also likely to prevent an employee saying ideas out loud. An employee may feel insecure and that his idea is actually poor and not worth sharing. An employee may even feel scared of team member shooting down his idea. As mentioned in the category Role of the team, a team can support its members to lower the level of self-criticism and encourage in ideating and experimenting. 

In turn, some interviewees described throwing also wild ideas among a team, as they may lead to something good and encourage others in ideating as well, as interviewee 6 points out. Their level of self-criticism was considerably lower and confidence higher than the ones who were not that enthusiastic in sharing ideas out loud. According to interviewee 11, personality affects on employee's opinion on failure. 
\begin{quote}
 ``I sometimes say out loud stupid ideas as well.. It?s that sometimes stupid ideas can lead to anything.'' [Interviewee 6]
\end{quote}
\begin{quote}
``I do think it depends on a personality. One has to be ready for failing and not fear it.'' [Interviewee 11]
\end{quote}
Even though saying ideas out loud requires courage and confidence, it can be learnt by experience. According to the study the employees who had more previous work experience were also the ones who were more confident in telling ideas out loud and conducting experiments without fearing failure.

\subsubsection{Attitude and motivation towards developing}
According to the interviews attitude towards developing affects on experimentation behaviour. If an employee considers developing and learning new things important and part of the work, he is more likely to reframe failures as opportunities, be resilient in developing and find alternative ways of doing things that needs change. 

If a team is not supporting employee?s idea and experimenting, an employee has to have motivation to try again and find another way. For instance, if the team is likely to be negative towards new ideas, interviewees described telling their ideas first to a trusted colleague. After the support from a close colleague an employee feels he has enough confidence to tell the idea to the whole team. Interviewee 9 describes this phenomena in the quote below. 
\begin{quote}
``I guess I find the certain people to whom I? [tell ideas out loud]. Then I dare to tell to others and as I have an urge to develop and learn, so through that I try..'' [Interviewee 9]
\end{quote}
Thus, the meaning of an employee?s motivation and attitude towards developing and learning is remarkable. In one experiment an employee ideated and prepared a prototype during his leisure time and during the process learnt new skills he had not known before. His motivation was so high it took him less than two weeks from the idea to actual prototype and first tests with customers. In turn, if an employee is not excited about learning new things, enjoys routines and prefers little change, he most likely will not be the first one to ideate or conduct experiments. As interviewee 14 states, developing requires action from an individual. 
\begin{quote}
 ``It [developing] requires activity from oneself and that an employee is willing to act and knows what he wants.'' [Interviewee 14]
\end{quote}
Furhtermore, in order ideation and experimentation to begin an employee has to be motivated and have the resilient attitude towards failing and learning from it. 

\subsection{The gap between an idea and experiment}
According to the study an idea that is said out loud in an organisation is not always easily developed into an experiment or a new routine; there seems to be a gap between an idea and experiment.  In this category factors related to idea and people involved that are critical for experimentation to happen are presented.

Interviewees reported several factors that need to be taken into account when moving from an idea to experiment, and those factors are here divided into three subcategories. Characteristics of an idea and experiment focus on how the size, riskiness and relevance of the idea and experiment can prevent or support an experiment to happen. In addition, a phenomenon called Static friction was recognised in the study, meaning that even though employees are excited about ideating and experimenting in the beginning, for some reason experiments still do not take place. Stakeholder distance and customer involvement consists of the importance of stakeholder and customer opinion on the experiment as well as mutual trust between different parties that experiment concerns. 

\subsubsection{Characteristics of an idea and experiment}
Characteristics of an idea seem to have an impact on whether or not it is experimented. For instance, the simpler and more concrete the idea is, the more likely it is to be experimented. Experimentation seems to help in making abstract ideas into concrete things and reflect the problem more clearly. However, even though the idea gains positive feedback among workplace, it still might not be experimented. The more resources, planning and opinions from different parts are needed in experiment and the more complex it feels among participants, the more likely it is to remain in ideation phase and not evolved into an experiment.  Interviewee 1 describes below why an idea actually turned into an actual experiment.
\begin{quote}
``It was as concrete thing as possible, that did not take too much.. or more 
negotiation with different parts..'' [Interviewee 1]
\end{quote}
The risk level of an idea affects on the bridge between throwing ideas and actually experimenting. When talking about performed experimentations interviewees, like interviewee 10 below, described the first ideas behind them being easy and simple, and especially possible to experiment with a low risk of anything bad to happen to customers or people involved in the experiment. In addition, interviewees reported that suitable experiments take into account the characteristics and possible limitations of the team or experiment. 
\begin{quote}
``But of course one has to think through that the experiment benefits everyone and it will not cause any harm to anyone.''[Interviewee 10]
\end{quote}
In addition, relevance and importance of an idea and the problem it attempts to solve are essential for the gap between an idea and experiment. Be the problem widely recognised among workplace, an attempt to experiment something new is rather likely to get support and engagement from colleagues and stakeholders. Likewise, according to the study if employees do not consider an idea important, and are not motivated to perform it, the idea will presumably be shut down by colleagues rather than considered from different perspectives or experimented nevertheless. 

\subsubsection{Static friction}
Static friction in organisational environment came to prominence in the study.  Static friction here means a workplace, despite of eagerness towards new ideas and ideating, sticking to the routines and not being able to act in a different way and experimenting new ways of working. Employees are likely to get excited of ideas and even ideate eagerly together as a team, yet when it comes to implementation and actually performing experiments or do something differently, employees are no longer willing to take responsibility or be that excited about the idea. Interviewee 8 describes the phenomenon.
\begin{quote}
``I think we are always so excited about everything, but when we start going through details and who will actually take charge of this and who will be involved, then I think we are no longer that excited..'' [Interviewee 8]
\end{quote}
In turn, as interviewee 3 put it, few interviewees described a situation where little or no static friction is recognised. Common factor to these descriptions is that when coming up with an idea, employees begin right away going through different possibilities and actions on how to perform an experiment or implement the idea and actually proceed with them. Yet in the study descriptions about static friction were in majority. 
\begin{quote}
``I start quickly ideating where I can contact next [in order to make perform the experiment or gain a certain goal]..'' [Interviewee 3]
\end{quote}
\begin{quote}
``One really learns from that [experimenting and developing], definitely yes. One only needs to begin and get involved, which is usually the hard part..'' [Interviewee 5]
\end{quote}
Even though interviewees emphasized learning and experimenting new ideas being essential for work, lot of resistance and inactivity occurred when actual experimenting was supposed to happen. Interviewees described even being surprised how after all the enthusiasm towards experimenting, no one was willing to take the lead and the experiment was never performed. Interviewees supposed and admitted that at times it feels too exhausting and difficult to break routines and it is easier to continue performing tasks as is used to. 


\subsubsection{Stakeholder distance and customer involvement}
According to the study the relevance and closeness of the idea to the customer enhances the engagement for experiment of employees. According to interviewee 7, the need for a change rising from a customer, improves the likelihood of the idea taken seriously and experimented. The same applies when a clear need to try something new is present. This is usually faced as a problem in present way of working, yet it can also be an attempt to improve the quality of customer?s life or the atmosphere at the workplace.  
\begin{quote}
 ``Especially when the idea rises from customer himself, we take every idea into account and consider everything that concerns a customer.'' [Interviewee 7]
\end{quote}
In this specific working field experimentation that concerns customers needs permission usually from both customer and relatives. According to the study this is occasionally a challenging network to deal with, as the requirements and wishes from customers and stakeholders may be highly contradictory. Yet, worthy ideas also rise from relatives as well as relevant information of what has already been experimented with the customers and what was learnt from that. Interviewee 10 describes this phenomenon. 
\begin{quote}
``So relatives play a very central role in our customers? lives, and almost everything is still discussed and checked with them and ask for support from them, like can we do this. And some very good ideas may also rise from them. Or then it can be like ?Oh no, this has been experimented for 15 years now, and it doesn?t work, so you should not start doing this.?.. So we have to remember that there is the network outside this workplace that is usually also involved in these experiments.'' [Interviewee 10]
\end{quote}
Mutual trust between people involved in experiments is needed. Interviewees described mutual trust being a relevant part of experimenting, and experienced the trust especially among customers and stakeholders as highly important factor in order experiments to happen. 

\section{The effects of experimenting on individual}
Even though the interview focused on the factors affecting experimentation in an organisation, the study revealed that experimenting also has an impact on employees. 

Experimentation has an effect on employee on different levels. First of all, wide variety of emotions, such as excitement, fear of failure, disappointment and uncertainty is involved and rises up at different phases of the experimentation process. Secondly, experimenting helps an employee to learn and reflect on one?s work as well as to gain process know-how of experimenting. It seems that through experimenting an employee is likely to encounter surprising outcomes that would not have been realized and learnt through planning.  

\subsection{Emotional experience and engagement}
The study revealed that during the experimentation process an individual experiences wide range of emotions. Those factors are presented here in three subcategories, which are Positive emotions: Happiness, excitement, inspiration, boost to self-esteem, Negative emotions: Frustration, disappointment, fear of failure and fatigue and Engagement and motivation towards work. These subcategories are described and explained, in which part of the experimentation process they are faced. 

\subsubsection{Positive emotions: happiness, excitement, inspiration, boost to self-esteem}

Experimentation usually begins with ideation, and in this phase interviewees described feeling creative, happy and excited. Ideating feels inspiring when colleagues support and join the ideating and plan together how the idea could be experimented. Interviewees described being excited and happy especially when the idea was their own or they were highly involved in ideating and planning the experiment.  Furthermore, ideating as a group also felt more empowering than ideating alone and not getting support for one?s ideas. 

As Interviewee 4 states, positive results of experimenting, meaning that new way of doing things works better than the previous way, is likely to raise positive emotions. Interviewees described that getting good feedback from the experiments and ideas as well as getting support from both customers and colleagues raise positive emotions and give boost to self-confidence. Furthermore, a successful experimentation encourages experimentation behaviour and ideating and was described to nourish one?s creativity, as interviewee 6 emphasises.  
\begin{quote}
``If something works better than before [experimenting], you get a good feeling out of that.'' [Interviewee 4]
\end{quote}
\begin{quote}
``Experimenting nourishes creativity and ability to throw oneself to something new.'' [Interviewee 6]
\end{quote}
Among some interviewees the uncertain outcomes of experiments can also be seen as exciting and refreshing possibilities. As by experimenting new ways of performing work tasks are tested, experimenting brings exciting new aspects and challenges to routine work. Good ones tend to spread and may lead to something new and energetic and are likely to bring energy and stimulation to an employee. Interviewee 10 describes how experimentation brings stimulation to work. 
\begin{quote}
``So the experimentation kind of spreads. And I do also like routines and stuff: that things go in a certain way. But, it does bring stimulation to work, that we try out new routines.'' [Interviewee 10]
\end{quote}

\subsubsection{Negative emotions: frustration, disappointment, fear of failure and fatigue}
Among positive emotions, interviewees described encountering various uncomfortable and complicated feelings throughout the experimentation process. 

In cases where an idea of an employee gets only little or no support from the team or a manager, feelings of frustration and disappointment may occur. This phenomenon is likely to discourage employees to say their ideas out loud in the future and thus makes the level of employee?s self-criticism higher. In the quote below of interviewee 14, is described the feeling of disappointment when encountering resistance from colleagues or team.
\begin{quote}
``When you are totally excited about something [idea].. for sure there comes the disappointment like what is it now, why this idea cannot go through, what is it that is so difficult in this.''[Interviewee 14]
\end{quote}
While the outcome of experimentation is usually difficult to forecast and cannot be planned in beforehand, this has an influence on emotions of an employee. Tight schedule of experimenting and little planning combined to uncertain outcomes can raise anxious emotions. Saying out loud one?s ideas might feel scary as the employees feel insecure about their idea. This forces them to encounter an uncomfortable fear of failure of their idea being shot down. 

In cases where the experimentation in where all the people involved are highly excited about, does not reach the goals set for the experiment, it is likely to cause frustration, disappointment and sadness among employees. In most cases, interviewees felt failing personally when they felt that experimentation failed. Interviewees usually described experimentation as failed if it did not reach the goals set for the experiment. 

In situations where there is no specific closure for the experiment or for some reason the experimentation is not finished, interviewees described feeling disappointed. This rather typical need for getting things done is described in the quote of interviewee 5.
 \begin{quote}
``Kind of disappointment, or kind of feeling of failure.. ? I always want to finish what I start.. So I do not like if things are not finished..''[Interviewee 5]
\end{quote}
\begin{quote}
``In addition, it [the experiment] cannot be too big, as it easily inflates.. so I think it is better to start, no matter how great idea there was, to slightly narrow it in some certain idea and then experiment that one. As I think many times it is a major hindrance that employees have no energy when the experiment inflates too much..'' [Interviewee 8]
\end{quote}
As described in quotation above from interviewee 8, the size of an experiment has an effect on the energy level of an employee. When planned experiment is too large or complicated and tasks needed to perform it too challenging or numerous, lots of resources are consumed and the energy level of an individual is lowered causing fatigue. 

\subsubsection{Engagement and motivation towards work}
The experience of experimentation and seeing the result of experimentation is likely to encourage employees in their work.  Especially when the employees performing an experiment are satisfied with the outcomes of the experimentation, so that they consider it successful, experimenting improves the engagement and motivation of an employee towards his work. Furthermore, this encourages employees to be more creative and say out loud one?s ideas as well as gives a boost to energy level.

 As mentioned in the factors affecting experimenting, receiving feedback is important part of experimentation. In the field studied, employees work in a very close interface with customers and stakeholders. Thus, positive feedback from them raised positive emotions and encouraged to continue developing. Interviewee 10 states how important the feedback from work is for engagement.
\begin{quote}
``But every time there is a good idea and it works when we try it, and we notice that it helps, of course it improves my performing in the work also in mental level. And there comes moments, with customers, when something works with them and we get positive feedback --- of course it is very important. It then makes me happy and motivates, and encourages further on. Or feedback from relatives, if we get positive feedback from them, it again encourages.'' [Interviewee 10]
\end{quote}
Overall, interviewees described they felt more engaged to their work when they were able to perform ?quick and dirty? experiments from which they get instant feedback. Ideating and new ways of performing work-related tasks through experimenting increased the meaningfulness of work and made it more interesting. In addition, through ideating and experimenting interviewees felt they have more influence on their own work.

\subsection{Learning}
Experimenting seems to have an impact on learning skills of an individual. Learning occurs in various levels, and three subcategories where learning was especially noticed were formed from the data. Reflection of work means that experimenting helps an employee to reflect ways of working and the work overall. Secondly, through experimenting an employee gains deeper understanding of experimentation process, which is here called Process know-how.  In addition, it seems that an experimentation process helps and individual to overcome anxious and insecure emotions described above, thus improving the Resilience towards work of an employee. 

\subsubsection{Reflection of work}
According to the study experimentations helped to question conventional ways of working and offered wider, more objective perspective that helped to improve the work. Some interviewees noticed the same as interviewee 2; that stopping to reflecting one's work is not and ordinary practice for an organisation and its employees. Experimentation process helped employees to reflect and make the purpose of their own work clearer, as interviewee 8 emphasises. 
\begin{quote}
``Many things are done without actually stopping to think about them more, like would there be something to improve. It might be kind of quite typical way to act for an organisation, to forget further evaluation.''[Interviewee 2]
\end{quote}
\begin{quote}
``Maybe it differs when.. you have to think, or you get to think, but let?s say that you have to think some issue in a deeper level and maybe make your thoughts more structural and that you have some understanding from what you view your own work and work of others ? So it does give in a certain way a deeper understanding on how one is doing his work.'' [Interviewee 8]
\end{quote}
As mentioned in the factors that affect experimentation, giving and receiving feedback is relevant in experimentation-driven approach. Receiving instant feedback improves learning process of an employee and the ability to iterate, meaning that an employee can reflect the outcomes of an experiment and improve his idea for another experiment and work overall. 

Furthermore, interviewees described surprising outcomes of small experimentations they performed and said talking about experimentations as well as performing them led to new information and ideas from different parts, including colleagues, customers and stakeholders. The interviewees described learning things they most likely would not have learnt without experimenting something new. Through experiments they also got deeper contact with stakeholders such as customer?s parents. Interviewee 14 describes below his relation to customers and relatives.
\begin{quote}
``The working becomes more interesting and I have clearer targets [through experimenting]. I get better in contact with customer and find new aspects as I told.. With customer?s relatives we talk in a different way when I tell that I?ve been planning of this kind of experiment ?-- and then the relatives begin to tell the history of a customer..'' [Interviewee 14]
\end{quote}
In addition to experimenting making the actual core of the work clearer, in this field of work where employees work very close to customers it helped the employees, like interviewee 6, to understand and listen better their needs and become more customer-oriented.  
\begin{quote}
``Through that [an experiment] we learnt to listen more and be even more customer-oriented ''[Interviewee 6]
\end{quote}
\begin{quote}
``Seems to me that my own prejudices are best repealed by just starting to do and act. Through that also abilities are found.'' [Interviewee 3]
\end{quote}
Furthermore, according to interviewee 3, employees learnt to overcome prejudices through rapid experimenting and found hidden capabilities.

\subsubsection{Process know-how}
Factor called process know-how was recognised from the data. In this instance, process know-how means understanding of experimentation process, including the ability to ideate, plan and perform small experiments, the ability to reflect and learn from experiments and do iterations. 

Seems that process know-how of an individual improves through the experience of experimentation. Interviewees who were more familiar with vocabulary and the process of experimentation and who had done at least one experiment during the experimentation challenge were likely to reflect experiments in a deeper level than those who were not that familiar with the process and vocabulary. 

Those who had deeper understanding of experimenting process emphasized that one can learn from each experiment whether the actual goal set for the experiment is achieved or not, and those teachings are relatively important. As interviewee 12, some said the more the experiment fails, the more could be learned from it. 
\begin{quote}
``One always learns [from experiments]. It is kind of like the worse the experimentation is, the better one learns from it.''[Interviewee 12]
\end{quote}
\begin{quote}
``Failure is also a result, it leads to something. You can improve or try once more.'' [Interviewee 3]
\end{quote}
As interviewee 3 states, failure was rather seen as a result or learning point than totally failing. Those employees who had process know-how on experimenting were able to continue and learn better from each experiment, understand failures as learning objectives, process ideas and truly develop their work and challenge conventions. They reported feedback being an essential part of the work and developing, as it teaches what has to be done differently and what was successful. Interviewee 10 shows process know-how by stating the role of feedback, whether it is positive or negative. 
\begin{quote}
``Feedback is important in that essence so that one knows is it worth to continue to other experiments. So it [feedback] is always good, whether it was positive or negative, but it is always needed.''[Interviewee 10]
\end{quote}
When talking about feedback helping reflection of the experimentations, interviewees also reported discussion with colleagues being essential. Through discussion and feedback important information is exchanged and new aspects can be found and learnt. Yet, according to the study, receiving and listening to feedback requires humility and willingness to admit own faults and receive help from others.

Furthermore, essential part of process know-how is starting to perform with small experiments, prototyping with small group of customers, making a prototype as simple as possible and learning from the iterative process. Interviewees described being surprised by how fast a rough prototype can be done and how helpful it can be for work, compared to usual way of working meaning years of developing before some tool is launched for use. 
\begin{quote}
``This thing [a prototype], was welcomed so well, and actually it helped right away.. when usually these kind of tools are developed for many years before they are valid. So actually this kind of very simple system built this fast.. So it helped right away and that was a happy surprise..'' [Interviewee 1]
\end{quote}
Furthermore, a deeper perspective of process know-how affects bigger changes in an organisation. One interviewee described experimentation as a way to manage and change complex systems while smaller experiments can assist bigger changes to happen. 
\begin{quote}
``But it can be, that this kind of small experiment can help bigger changes to happen.. ? And I guess that?s the trick in this whole thing and behind, that large things consist of several small ones and if those small ones can be fixed in several ways, it can have great impacts..'' [Interviewee 1]
\end{quote}

\subsubsection{Resilience towards work}
Even though in experimentation process an employee goes through negative emotions described earlier such as frustration, fear of failure or disappointment, experimenting helps to overcome those anxious feelings and improves resilience of an employee.  Performing experiments forces employees to turn abstract ideas into concrete, smaller and lighter steps that are easier to approach, thus making the gap between planning and experimenting smaller.  

When experimentation is experienced as important among participants or there is a real problem to be solved, employees may turn all the disappointment and frustration rising from previous experiments or abatement of the idea from colleagues or a leader into passion of performing better. This improves the capability of resilience, as interviewee 13 comments. 
\begin{quote}
``So first comes the frustration, but after that like next year we?ll show them--. It may turn upside down when I get to process it in my head, like it?s a bummer how badly this went, but it can be we will try it again in a bit different way and we will do it better then. This also encourages to continue..''[Interviewee 13]
\end{quote}
Choosing the right terms may have a major impact on behaviour and resilience. Experiment as a word in a way consists of failing, and according to the study, compared to failing in daily routine work, failing in an experiment is experienced rather acceptable. This leads an employee feeling less pressure for succeeding with the first try. In addition, when the effort put on the first experiment is bearable, it is easier to persistently try another way. Interviewee 11 describes below his resistance. 
\begin{quote}
``Sometimes you really feel like giving up when you no longer come up with solutions how to make something work. But still you just.. You have a small break and then you get back to business.''[Interviewee 11]
\end{quote}
Even though during ideating and experimenting negative feelings are likely to occur, the characteristics, experience and know-how of experimenting helps to overcome these emotions and turn them to resilience. Iterative process of developing, dividing a task or a problem into smaller steps and learning by doing all support the emotional struggle with self-criticism, fear of failure, insecurity and uncertainty. 


\chapter{Discussion and conclusions}
This chapter concludes the findings of the thesis. Theoretical perspective is discussed and compared with the findings of the case study.  Factors affecting experimentation behaviour are discussed in section \ref{fae}. Then, relationship between learning and experimenting is discussed. Managerial implications describe guidelines for organisations based on the study, following the future research topics and discussion of reliability of the thesis. 

A need for novel approaches for development that allows employees to improve their work in more iterative and creative ways that support learning exists. The aim of this thesis was to study experimentation-driven development as such an approach, shed the light on factors affecting experimentation behaviour and provide guidelines for organisations to support its employees in experimenting. 

\subsection*{Review of research objectives}
This section briefly reviews the research questions set in the beginning of the thesis. These research questions are reflected through the discussion. 

\begin{enumerate}
 \item What kinds of factors affect on experimenting behaviour of an employee? 
 \item How experimenting affects an individual? 
  \item How can experimenting behaviour be supported in organisations?
  \item How can experimentation support organisational learning?
\end{enumerate}

\section{Factors affecting experimentation behaviour} \label{fae}
From the data rose various factors affecting experimentation behaviour, and the literature review revealed how similar factors are related to organisation's ability to innovate and employee's creativity. Studies show creativity has enhancing impact on business profit and growth \citep{nystrom1990organizational} According to \citet{vincent2002divergent} creative work consists of creative and innovation processes, and as experimenting stands as significant part of innovation process, similar factors that foster innovation are likely to foster experimenting. In this section synthesis on factors affecting experimentation is formed.

Conversation
Support from the team 
Team engagement
Relevant feedback 
Building on ideas 
License to do experiments
Encouraging people to use their individual strengths  

Setting frames to experimentation is important; defining when the experimentation begins and when is a moment for closure. In many reported experiments experimenting has been understood as fast prototyping but no reflection. These leave a person easily with an unfinished feeling that is too easily related to failure. 

Prior knowledge and experience of an employee \newline

In the empirical data insight of unwillingness to change was recognised, as few interviewees described preferring and resisting change and novel ideas. However, some interviewees described experience assisting in bringing out opinions, being confident enough to conduct experiments and being able to get others along. So according to the study previous experience can encourage performing experiments and improve believing in their ideas, yet it can also lead to routine hard to change after many years. Thus, employee's attitude towards experimenting seems to depend on employee's attitude, motivation and engagement towards work rather than only on work experience and prior knowledge.

However, acquiring a first experience of experimentation narrows the gap between an idea and actually performing an experiment. As according to the literature, familiarity of a subject is likely to assist in adopting novel methods, an implication could be drawn how making the first experiment an essential part for the longer-term experimentation approach. The interviewees described realising only after the first experiment that it actually brought energy and resources and not only deprived them. Thus, the sooner the first experiment is executed after the idea has rose, the better it seems to be for the whole developing process.

Leadership behaviour \newline
When affecting and changing only one organisational factor, organisation management need to be alert for occurring inconsistencies that may result to decrease in willingness to engage to experimental behaviour \citep{lee2004mixed}

Interviewees described vividly how essential role their leaders and immediate superior have on their willingness to conduct experiments. 
Example of the leaders was also recognised from both from the empirical data and as an important factor affecting experimentation behaviour, and \citet{garvin2008yours} emphasises how through own example leaders can encourage employees to offer new ideas and options.

Even though almost every of 14 interviewed person described how the immediate superior they had have succeeded in establishing a climate where experimenting is possible to happen, all units did not perform experiments during the experimentation challenge. Thus leadership behaviour only is not sufficient for experimentation behaviour in organisations.

Resources \newline
\citet{thomke2001enlightened} emphasise organisation should allow and manage the work for the employees so that fast experimentation is possible. This usually requires challenging routine ways of working and shaping the routines, yet fast experimenting is essential in order to get rapid feedback for shaping the ideas. According to empirical data, interviewees described lacking the time to develop their work, discuss about ideas and generate them, not to mention actual experimenting. This implicates the structure of the work and time allocated for developing, whether through conventional planning-based or experimentation-driven approach, is not sufficient. 

Psychologically safe environment
\newline
In empirical data many characteristics interviewees described can be related to psychologically safe environment. 

Clear process and goals\newline
Setting a clear goal for an experiment makes measuring and evaluating the experiment easier. Interviewees described usually having an eligible result for an experiment, and if that result is achieved, the experiment is considered successful. Goal assists in learning of experiment and further developing. 

This thesis brought novel perspective on organisational learning and development by combining and presenting an approach for fostering creativity and innovation of employees in an organisation through experimentation-driven process. 
Experimentation-driven development has not yet been widely studied, thus this thesis provided important theoretical data and insights on experimentation process as well as sheds the light on the important issue of employee engagement and learning in order to create new value and competitive advantage in organisations. Furthermore, emotional experience of experimenting was not the focus of this study. Yet several factors from empirical data and theory support the early hypothesis of experience of experimenting being different from planning-oriented developing. 

Considering the complex character of organisational change, learning and behaviour affected by various factors in individual, organisational and team levels, comprehensive literature review was gathered. It combined literature on various fields of research aiming to form holistic picture of factors affecting experimentation in changing business environment. However, more focused research should be conducted on various topics in order to gain proof on the relations and factors found in this study. 

\citet{edmondson1999psychological} suggested, studies in real work teams are required, and this study aims to add on this aspect studying factoring affecting experimentation behaviour in real work context. 

\section{The relation between learning and experimentation} \label{relation}
In this thesis, learning is considered as a process of continuous trial and error \citep{argyris1978organizational,edmondson1999psychological} that includes individual growth and improved performance. According to the experimentation process of \citet{thomke1998managing}, learning glues together all the four phases of the process. Setting a hypothesis, planning an experiment, executing it and analysing the results are all reflected throughout the process in order to learn about the fundamental idea and develop it further. In addition, according to the Execution Innovation Model, novelties are only generated through a learning process of iterative experimentation. \citep{tuulenmaki2011art} 

According to the data and experience of experimentation, interviewees described how experimenting assists in learning about problem at hand, and at provides information whether trial is malfunctioning. In addition, interviewees described that experimenting stands as an excellent method for learning, major factors being both the amount of experimentations done and the reflection on them. Empirical data thus supports the hypothesis that experimenting can serve as a method for learning. 

Few interviewees described using this method in their daily life and that 'experimentation' is a new word in their vocabulary while few told realising through the experimentation challenge that their work is actually about trying out new ways of doing things and finding the best way to help customers in their daily lives.Thus, through experimenting interviewees gained deeper understanding of their work.

In the planning-oriented developing process learning is likely to happen when the product or service is launched or a first, large-scale pilot is tested. Thus, the results of this study support the statement how experimenting could be used as a tool for learning about development idea with lower resources. As failure is very likely outcome of experimentation, every experiment are opportunities for growth and learning. 

According to \citet{buijs2007innovation}, innovation consists of coming up with novel ideas and implementing them. Ideating begins with exploring, developing and implementing the ideas, following introducing the ideas, which have turned into products or services, into the marketplace. Innovation process is a series of stages for processing the idea, and in the end of every stage the idea is reflected and evaluated before further processing. Evaluation points stands for usable tool for measuring the quality of idea but gives also understanding of how the evaluation process is going. In addition, while evaluating, team members also need to reflect the process and the idea, through which learning occurs. Also \citep{runco1994problem} emphasises how only after evaluation of ideas implementation can be discussed and performed and several studies show the essence of evaluation \citep{mumford2002leading,vincent2002divergent}. Useful questions in evaluation process could be "What went well?", "What can be improved?" and "What has been learned?" \citep{buijs2007innovation}. These were factors revealing also from empirical data, supporting the relationship between learning and evaluation of experiments. 

\section{Practical implications} \label{framework}
This section assembles the findings of the study to set of guidelines consisting of factors organisation should consider in order experimentation to occur.

\subsubsection*{Safe and supporting environment}
In the heart of every change and development project are employees, the group of individuals who are touched by the change. In order to proceed to meaningful and efficient changes for both the company and its employees, individuals has to be onboard. 

In the very heart of willingness to conduct experiments seems to be individuals sense of safe and supporting environment towards creativity, idea generation and experimentation. This includes team engagement, positive attitude towards failing, environment tolerating uncertainty and fostering risk-taking. Furthermore, brainstorming and saying out loud problems and ideas should be encouraged. 

\subsubsection*{Support from leaders}
Leaders should show their support towards experimenting by acting as role-models, encouraging employees to work on their own expertise and interests, reward from successful experiments and ideas and showing support by taking results of experiments to upper management. When providing sufficient level of autonomy to employees, leaders are likely to encourage their employee's intrinsic motivation leading to more satisfied, efficient and creative employees. 

\subsubsection*{Allocating resources}
Allocating resources refers to established truth that developing one's work requires time, as well as creativity process. Experimentations themselves should be designed to consume little resources, yet reasonable amount of resources should be reserved for executing experiments. Professional conversations among colleagues, visiting peer units and meeting peer colleagues in the country are likely to foster the expertise, creativity and willingness to develop one's work. No time for idea generation or experimenting is likely to decrease willingness to conduct develop one's job and will be considered as extra. Thus, time for develop one's work is necessary. 

\subsubsection*{Careful experimentation design}
Experimentation design consists of planning experiments carefully, defining the learning goal of each experiment and how it will be evaluated. Identifying the schedule for experiment and appointing a responsible person for the experiment. Transparent communication and documentation of ideas and the results of experiments. 

Setting a clear goal for an experiment makes measuring and evaluating the experiment easier. Interviewees described usually having an eligible result for an experiment, and if that result is achieved, the experiment is considered successful. Goal assists in learning of experiment and further developing. 

Careful experimentation design considers individual characteristics and experience of employees conducting experiments. Experiments should be designed to not consume steep amount of resources of an employee. Experiments should be easy to approach, conduct and even bring resources and energy instead of consuming them. In addition, experimentations should be designed by the employees themselves as they are experts of their own work. When working in close context of customer interface, customer insights and ideas could be considered for experimenting when possible. 

Furthermore, as developing requires adapting novel ways of working and challenging status quo, great experimentation design considers the implementation process of successful ideas and results of experiments. Implementation is a key issue when adapting experimentation-driven approach, and organisational and work structures need to support implementation of experimentations. 
This thesis provided framework through which experimenting can be considered as a tool to foster learning of employees. Furthermore, it provides various managerial implications for fostering organisational learning, innovation and creativity of employees. Additionally, experimentation process was presented as a tool for developing and learning. Requirements for organisational environment in which employee's are willing to conduct experiments were outlined, and several practical recommendations were presented for top management to adapt in their work.

The case study setting, experimentation challenge, can be implemented with ease, and can be considered as a method to adapt experimentation in the organisational development culture. 

\section{Future research topics}
This study focused on identifying factors affecting experimentation behaviour and creating a framework for supporting environment for experimenting. However, interesting findings from the data consisted of affects experimenting has on individual. Experimenting is highly different experience for an individual than planning-based development. Further study of the experience of experimenting should be done in order to form deeper understanding on how to support experimentation in organisational context. Accordingly, as \citet{edmondson1999psychological} argues, psychological safety as a means to promote team performance is increasingly relevant both in future work and research. 

This thesis was studied in a case company of specific service business area, where communication with customer is constant; employees being daily in tight contact with customers, the gap between an idea and conducting experiments, receiving feedback and learning from them can be lower than in other fields of business. This especially, when the aim is to learn of customer needs and ways to serve them better. This is not necessarily an obstacle, as every work life has their own challenges and demands for development. Experiments can be conducted regardless of the business field, the art of experimenting being the ability to design low-cost and low-resource experiments that teach about the fundamental idea or assumption. Future research topic could be to study the experimentation design and how to design experiments that can be best learned from and suit best the occasion. 

In this thesis framework for organisational support experimenting was created, yet further studies are required in order to learn more of the factors and about transferring an experimentation-driven culture in organisations. 

Experimentation challenge was a method MIND team invented in order to study experimentation behaviour in organisations. Interviewees described challenge being encouraging and positive way to put thoughts on improving work. Interesting future research would be to study further experimentation-challenge as a way to implement experimenting to organisational culture. For instance, few interviewees hoped experimentation challenge became an annual tradition, which could support the adaptation of new way of working, learning and reflecting one's work. However, these are hypothesis too early to confirm without further research. 

\section{Reliability of the thesis}
In order to assess the reliability of the thesis, approach of \citet{lincoln1985naturalistic} on reliability is used. According to this approach, reliability is assessed through trustworthiness, which consists of four aspects: credibility, transferability, dependability and confirmability. 

Credibility means that the interpretations made of the original data maintain credible \citep{lincoln1985naturalistic} (page 301-316). In this thesis conclusions are drawn after describing the data collection carefully, so the reader is able to follow the process of interpretations, and by using direct quotations the data behind interpretations is revealed for the reader. In addition, discussions with co-researchers and professors about the interpretations have aimed to maintain credibility. However, interviews were conducted good time after experimentation challenge was over. Few interviews described being difficult to recall the feelings and experiments back when conducting experiments then in detail. Thus, better and more credible data would be gained to have interviewed employees right after the experimentation challenge, while experiments are actively in mind. In this study this was not possible due to the holiday season employees had right after the experimentation challenge. 

Transferability refers to possibilities to transfer the results and findings to another context \citep{lincoln1985naturalistic} (page 316). In the thesis experimentation-driven process was presented and brought to organisational context in a case study. The study revealed several factors in organisational level which can be affected in order to foster experimentation behaviour from individual, team, management and organisational structures perspective. Even though the case study was conducted in specific field, the themes and environmental factors together with managerial implications are transferable to other organisations, as they can be considered as guidelines for good practices and development. 

Dependability refers to the consistency of the research process \citep{lincoln1985naturalistic} (page 316-327). Throughout the thesis the research design and process is described clearly. The research questions are presented in the beginning of the thesis and further revisited in the conclusions, and the results are evaluated through the research questions. 

Confirmability refers to objectivity and neutrality of the thesis \citep{lincoln1985naturalistic} (page 316-327). The writer of the thesis has never been working on studied industry field and was not involved in the empirical case other than in a role of interviewer and observer. In the data analysis process other researchers were involved and the results were discussed among three researchers. The interviews were recorded and transcribed. The theoretical part formed a broad synthesis on factors affecting experimentation, building on the theories from organisational management, organisational learning, development and innovation as well as creativity and leadership. 

\chapter{Summary}
Current and future business environment main features are complexity, uncertainty and unpredictable customer needs. To survive in this complex environment, organisations need to learn faster than rivals. Thus, organisational learning needs to be understood and supported. When creating novel approaches and business opportunities in complex environment, creativity and innovation abilities are required. Therefore, it is crucial to understand how creativity and innovation abilities in organisations and individuals can be supported. 

This thesis presented an experimentation-driven approach as a method to develop and learn in organisational circumstances under high complexity and uncertainty. Experimentation is a significant part of any innovation process and requires creative abilities of individuals. When organisational environment supports creative actions of employees and fosters innovation, the threshold for trying out novel approaches is lower and the culture is more likely to encourage development by experimentation. 

Organisational and business environment being a complex system, various factors affect behaviour of employees and willingness to conduct experiments. This thesis shed light on how experimentation behaviour is likely to be fostered by assuring safe environment for experimentation, supportive leadership behaviour, allocating resources and careful design of experiments. 

%Menisk� t�m� discussioniin johonkin? 
Static friction describes the phenomenon when the attitude towards developing, experimenting and ideating is positive, but when time for action, nothing happens.

One reason to this phenomenon to happen is probably that people don?t want to use lot of energy in new things, and if they sense developing is taking resources and energy, they don?t want to take responsibility of that. Furthermore, employees could see their job as a stabile routine in which developing doesn?t belong, and they tend to think developing, ideating or experimenting a task of someone else, for instance developing department, trainees, substitutes or students. 

I call this the paradox of getting of light. Employees are not eager to develop things that are supposed to help their workload and form a more efficient routine. So instead they stick to the same routines, that actually consume good amount of their energy. 

When resources are effectively used, experimenting happens by itself like in the following situation. 

?Easy thing was probably that when we got the idea, right people at the right time were there. So we were able to execute the idea right away. So we started working on the idea right away and we didn?t have to be like ?ok we have to wait for someone to come and organize when we are all together at the same time? or anything like that..?


\chapter{Conclusions}
Final part of the thesis concludes the findings and consists of x chapters. The results presented in previous part are further discussed in the first chapter. In addition, the theoretical framework presented in the theoretical part is supplemented with the findings of the empirical part. The research questions are revisited in order to determine how they were answered in the thesis. Both theoretical and managerial implications of the thesis are discussed in the second chapter. In the final chapter future research topics are discussed and the reliability of the thesis is evaluated.

\section{Reliability of the thesis}
Lincoln and cuba, l\"ahde taijan dipasta
Common criteria for evaluating the quality of a quantitative study are reliability and validity, where reliability refers to repeatability and validity to accuracy in means of measurement. In this thesis, however, quantitative approach is not used and in order to assess the reliability of the thesis, Lincoln and Guba?s (1985) approach on reliability is used. According to this approach, reliability is assessed through trustworthiness, which consists of four aspects: credibility, transferability, dependability and confirmability. 

Credibility refers to the interpretations made of the original data and their credibility (Lincoln and Guba, pp. 301-316). In this thesis? 

In addition, direct quotations are used in this thesis in order to reveal the data behind the interpretations. Co-researchers and professors have also been discussing about the interpretations thus adding credibility. 

Transferability means the possibilities to transfer the results and findings to another context. In the thesis?

Dependability refers to the consistency of the research process. Throughout the thesis the research design and process is described clearly. The research questions are presented in the beginning of the thesis and further revisited in the conclusions, and the results are evaluated through the research questions. 
Theory building process followed the principles of chosen research method, case study. 

Confirmability refers to objectivity and neutrality of the thesis. The writer of the thesis has never been working on studied industry field and was not involved in the empirical case other than in a role of interviewer and observer. In the data analysis process other researchers were involved and the results were discussed at least among three different researchers. In the theoretical research? 



% Load the bibliographic references
% ------------------------------------------------------------------

% Start bibliography from a new page
\clearpage

% removes section header from bibliography
% if you want to change bibliography to, say, references, add [/refname] after {bibliography}
\defbibheading{bibliography}{%
  \section*{#1}%
  \markboth{}{}
}
% to make the reference right in toc, adding phantom section
\phantomsection

% manual page-ref to last page of the content
% now where bibliography starts. If want to where bib graphy ends, move after printbibliography!
\label{pages-end}

\addcontentsline{toc}{chapter}{Bibliography} % Add bibliography to table of contents
\printbibliography

% Start appendices from a new page
\clearpage

% Appendices go here
% ------------------------------------------------------------------
% If you do not have appendices, comment out the following lines
%\appendix

\thispagestyle{myheadings}

% to make the reference right in toc, adding phantom section
\phantomsection

% Adds appendices to toc
\addcontentsline{toc}{chapter}{Appendices} 

\begin{appendices}
%removes listed appendices from toc
%\addtocontents{toc}{\protect\setcounter{tocdepth}{0}}

\chapter*{Appendices}
\thispagestyle{headings}

%\addtocontents{toc}{\protect\setcounter{tocdepth}{2}}
\listofappendices{}
\clearpage

%removes listed appendices from toc
%\addtocontents{toc}{\protect\setcounter{tocdepth}{0}}

\renewcommand\thesection{\Alph{section}}

\section{Interview questions}\appcaption{Appendix A ~ Interview questions}
\label{sec:appA}
The interview questions are presented below.
\bigskip

\noindent\emph{Background}
\vspace{-3mm} 
\begin{list}{*}{}
\setlength{\itemsep}{-3pt}
 \item Work description, and how long has the interviewee been in the position?
\end{list}

\noindent \emph{Know-how}
\vspace{-3mm} 
\begin{list}{*}{}
\setlength{\itemsep}{-3pt}
 \item What experiments did you do during the experimentation challenge?
 \item What idea did you work on further?
 \item How did you progress? What did you do?
 \item Did you develop the idea and conducted an experiment alone or together with colleagues? Did this deviate from your conventional way of working? 
\newline
 \item What did you find easy? (What made it easy?)
 \item What did you find difficult? (What made it difficult?)
 \item Did something surprising or unexpected happen?
 \item How did you act in this situation?
 \item What do you personally consider as critical incidents during the experimentation challenge ? eg. What excited you or discouraged?
 \item Where do you think you succeeded? (Why?)
 \item What made an experiment successful? How do you know that an experiment was successful?
 \item What affected to the success of the experiment? What were the conditions?
 \item What went wrong from your perspective? Where did you consider failing? (Why?) 
 \item What made an experiment unsuccessful? 
 \item What affected or caused an experiment to fail? 
 \item Can you describe some idea that you experimented during the experimentation challenge. 
 \item How would you continue developing this idea?
 \item What would you do this time differently than in the first experiment?
 
\end{list}

\noindent\emph{Supporting structures and practices}
\vspace{-3mm} 
\begin{list}{*}{}
\setlength{\itemsep}{-3pt}
    \item How did the experimentation challenge differ from the conventional way of improving ideas?
    \item Have you developed through experimenting before? Is it part of daily routine?
    \item How did experiments affect normal working day and routines?
    \item To whom did you tell about experiments?
    \item How did you document the experiments?
    \item How do you collect feedback from experiments?
    \medskip
\end{list}

\noindent\emph{Climate}
\vspace{-3mm} 
\begin{list}{*}{}
\setlength{\itemsep}{-3pt}
    \item How was the climate of your work unit during the experimentation challenge?
    \item What affected?
    \item Were everybody equally involved?
    \item Did everybody speak up about their ideas? 
    \item Were there conflicts? What were the effects of conflicts?
    \item What kind of support did you get in experimenting from your organisation/your colleagues? 
    \item What kind of support you would have wished?
    \item Is there some specific thing preventing experimenting generally in your work?
    \item What usually happens after telling out an idea? (Do you get support and encouragement and start acting?)
    \item How failed experiments are dealt with in your team? 
\end{list}

\noindent\emph{Leadership behaviour}
\vspace{-3mm} 
\begin{list}{*}{}
\setlength{\itemsep}{-3pt}
    \item How immediate superiors react on new ideas and experimenting?
    \item Is time allocated for ideating and experimenting in your work?
    \end{list}
    
\noindent\emph{Managing experimentation}
\vspace{-3mm} 
\begin{list}{*}{}
\setlength{\itemsep}{-3pt}
     \item  Do you feel that through experimenting you have more autonomy and you can affect better on your own work? Is experimenting one way to affect your work and improve it? 
    \item During the experimentation challenge, did you get more ideas than usually? How did they emerge?
    
\end{list}

\noindent\emph{Psychological factors}
\vspace{-3mm} 
\begin{list}{*}{}
\setlength{\itemsep}{-3pt}
\item What kinds of emotions rose during experimenting? (Did you for instance feel frustrated, insecure etc.)
\item How did it feel to tell an idea out loud among a team? (Do you get support or was your idea refected?)
\item How do you face a failed or unfinished experiment? (If there were any experiments like that)
\item What did you get from the experience of experimentation challenge?
\item What kind of factors brought good feeling? 
\item What kind of factors brought you down or caused anxiety somehow? 

\item How the amount and quality of feedback differs from when developing through experimenting?
\item How do you consider feedback? (Does it encourage to develop an idea further? Did it bring you down?)
\item  How does experimenting affect your own learning and developing your work? 
\end{list}

\noindent\emph{Wrap-up} 
\vspace{-3mm} 
\begin{list}{*}{}
\setlength{\itemsep}{-3pt}
\item Do you have any questions or comments?
\end{list}
% start next appendix on new page
\clearpage



% manual page-ref to last page of appendices
\label{appendices-end}
\end{appendices}

% End of document!
% ------------------------------------------------------------------
% The LastPage package automatically places a label on the last page.
% That works better than placing a label here manually, because the
% label might not go to the actual last page, if LaTeX needs to place
% floats (that is, figures, tables, and such) to the end of the
% document.
\end{document}