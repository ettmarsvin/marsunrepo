\chapter{Discussion and conclusions}
This chapter concludes the findings of the thesis. Theoretical perspective is discussed and compared with the findings of the case study.  Factors affecting experimentation behaviour are discussed in section \ref{fae}. Then, relationship between learning and experimenting is discussed. Managerial implications describe guidelines for organisations based on the study, following the future research topics and discussion of reliability of the thesis. 

A need for novel approaches for development that allows employees to improve their work in more iterative and creative ways that support learning exists. The aim of this thesis was to study experimentation-driven development as such an approach, shed the light on factors affecting experimentation behaviour and provide guidelines for organisations to support its employees in experimenting. 

\subsection*{Review of research objectives}
This section briefly reviews the research questions set in the beginning of the thesis. These research questions are reflected through the discussion. 

\begin{enumerate}
 \item What kinds of factors affect on experimenting behaviour of an employee? 
 \item How experimenting affects an individual? 
  \item How can experimenting behaviour be supported in organisations?
  \item How can experimentation support organisational learning?
\end{enumerate}

\section{Factors affecting experimentation behaviour} \label{fae}
From the data rose various factors affecting experimentation behaviour, and the literature review revealed how similar factors are related to organisation's ability to innovate and employee's creativity. Studies show creativity has enhancing impact on business profit and growth \citep{nystrom1990organizational} According to \citet{vincent2002divergent} creative work consists of creative and innovation processes, and as experimenting stands as significant part of innovation process, similar factors that foster innovation are likely to foster experimenting. In this section synthesis on factors affecting experimentation is formed.

Conversation
Support from the team 
Team engagement
Relevant feedback 
Building on ideas 
License to do experiments
Encouraging people to use their individual strengths  

Setting frames to experimentation is important; defining when the experimentation begins and when is a moment for closure. In many reported experiments experimenting has been understood as fast prototyping but no reflection. These leave a person easily with an unfinished feeling that is too easily related to failure. 

Prior knowledge and experience of an employee \newline

In the empirical data insight of unwillingness to change was recognised, as few interviewees described preferring and resisting change and novel ideas. However, some interviewees described experience assisting in bringing out opinions, being confident enough to conduct experiments and being able to get others along. So according to the study previous experience can encourage performing experiments and improve believing in their ideas, yet it can also lead to routine hard to change after many years. Thus, employee's attitude towards experimenting seems to depend on employee's attitude, motivation and engagement towards work rather than only on work experience and prior knowledge.

However, acquiring a first experience of experimentation narrows the gap between an idea and actually performing an experiment. As according to the literature, familiarity of a subject is likely to assist in adopting novel methods, an implication could be drawn how making the first experiment an essential part for the longer-term experimentation approach. The interviewees described realising only after the first experiment that it actually brought energy and resources and not only deprived them. Thus, the sooner the first experiment is executed after the idea has rose, the better it seems to be for the whole developing process.

Leadership behaviour \newline
When affecting and changing only one organisational factor, organisation management need to be alert for occurring inconsistencies that may result to decrease in willingness to engage to experimental behaviour \citep{lee2004mixed}

Interviewees described vividly how essential role their leaders and immediate superior have on their willingness to conduct experiments. 
Example of the leaders was also recognised from both from the empirical data and as an important factor affecting experimentation behaviour, and \citet{garvin2008yours} emphasises how through own example leaders can encourage employees to offer new ideas and options.

Even though almost every of 14 interviewed person described how the immediate superior they had have succeeded in establishing a climate where experimenting is possible to happen, all units did not perform experiments during the experimentation challenge. Thus leadership behaviour only is not sufficient for experimentation behaviour in organisations.

Resources \newline
\citet{thomke2001enlightened} emphasise organisation should allow and manage the work for the employees so that fast experimentation is possible. This usually requires challenging routine ways of working and shaping the routines, yet fast experimenting is essential in order to get rapid feedback for shaping the ideas. According to empirical data, interviewees described lacking the time to develop their work, discuss about ideas and generate them, not to mention actual experimenting. This implicates the structure of the work and time allocated for developing, whether through conventional planning-based or experimentation-driven approach, is not sufficient. 

Psychologically safe environment
\newline
In empirical data many characteristics interviewees described can be related to psychologically safe environment. 

Clear process and goals\newline
Setting a clear goal for an experiment makes measuring and evaluating the experiment easier. Interviewees described usually having an eligible result for an experiment, and if that result is achieved, the experiment is considered successful. Goal assists in learning of experiment and further developing. 

This thesis brought novel perspective on organisational learning and development by combining and presenting an approach for fostering creativity and innovation of employees in an organisation through experimentation-driven process. 
Experimentation-driven development has not yet been widely studied, thus this thesis provided important theoretical data and insights on experimentation process as well as sheds the light on the important issue of employee engagement and learning in order to create new value and competitive advantage in organisations. Furthermore, emotional experience of experimenting was not the focus of this study. Yet several factors from empirical data and theory support the early hypothesis of experience of experimenting being different from planning-oriented developing. 

Considering the complex character of organisational change, learning and behaviour affected by various factors in individual, organisational and team levels, comprehensive literature review was gathered. It combined literature on various fields of research aiming to form holistic picture of factors affecting experimentation in changing business environment. However, more focused research should be conducted on various topics in order to gain proof on the relations and factors found in this study. 

\citet{edmondson1999psychological} suggested, studies in real work teams are required, and this study aims to add on this aspect studying factoring affecting experimentation behaviour in real work context. 

\section{The relation between learning and experimentation} \label{relation}
In this thesis, learning is considered as a process of continuous trial and error \citep{argyris1978organizational,edmondson1999psychological} that includes individual growth and improved performance. According to the experimentation process of \citet{thomke1998managing}, learning glues together all the four phases of the process. Setting a hypothesis, planning an experiment, executing it and analysing the results are all reflected throughout the process in order to learn about the fundamental idea and develop it further. In addition, according to the Execution Innovation Model, novelties are only generated through a learning process of iterative experimentation. \citep{tuulenmaki2011art} 

According to the data and experience of experimentation, interviewees described how experimenting assists in learning about problem at hand, and at provides information whether trial is malfunctioning. In addition, interviewees described that experimenting stands as an excellent method for learning, major factors being both the amount of experimentations done and the reflection on them. Empirical data thus supports the hypothesis that experimenting can serve as a method for learning. 

Few interviewees described using this method in their daily life and that 'experimentation' is a new word in their vocabulary while few told realising through the experimentation challenge that their work is actually about trying out new ways of doing things and finding the best way to help customers in their daily lives.Thus, through experimenting interviewees gained deeper understanding of their work.

In the planning-oriented developing process learning is likely to happen when the product or service is launched or a first, large-scale pilot is tested. Thus, the results of this study support the statement how experimenting could be used as a tool for learning about development idea with lower resources. As failure is very likely outcome of experimentation, every experiment are opportunities for growth and learning. 

According to \citet{buijs2007innovation}, innovation consists of coming up with novel ideas and implementing them. Ideating begins with exploring, developing and implementing the ideas, following introducing the ideas, which have turned into products or services, into the marketplace. Innovation process is a series of stages for processing the idea, and in the end of every stage the idea is reflected and evaluated before further processing. Evaluation points stands for usable tool for measuring the quality of idea but gives also understanding of how the evaluation process is going. In addition, while evaluating, team members also need to reflect the process and the idea, through which learning occurs. Also \citep{runco1994problem} emphasises how only after evaluation of ideas implementation can be discussed and performed and several studies show the essence of evaluation \citep{mumford2002leading,vincent2002divergent}. Useful questions in evaluation process could be "What went well?", "What can be improved?" and "What has been learned?" \citep{buijs2007innovation}. These were factors revealing also from empirical data, supporting the relationship between learning and evaluation of experiments. 

\section{Practical implications} \label{framework}
This section assembles the findings of the study to set of guidelines consisting of factors organisation should consider in order experimentation to occur.

\subsubsection*{Safe and supporting environment}
In the heart of every change and development project are employees, the group of individuals who are touched by the change. In order to proceed to meaningful and efficient changes for both the company and its employees, individuals has to be onboard. 

In the very heart of willingness to conduct experiments seems to be individuals sense of safe and supporting environment towards creativity, idea generation and experimentation. This includes team engagement, positive attitude towards failing, environment tolerating uncertainty and fostering risk-taking. Furthermore, brainstorming and saying out loud problems and ideas should be encouraged. 

\subsubsection*{Support from leaders}
Leaders should show their support towards experimenting by acting as role-models, encouraging employees to work on their own expertise and interests, reward from successful experiments and ideas and showing support by taking results of experiments to upper management. When providing sufficient level of autonomy to employees, leaders are likely to encourage their employee's intrinsic motivation leading to more satisfied, efficient and creative employees. 

\subsubsection*{Allocating resources}
Allocating resources refers to established truth that developing one's work requires time, as well as creativity process. Experimentations themselves should be designed to consume little resources, yet reasonable amount of resources should be reserved for executing experiments. Professional conversations among colleagues, visiting peer units and meeting peer colleagues in the country are likely to foster the expertise, creativity and willingness to develop one's work. No time for idea generation or experimenting is likely to decrease willingness to conduct develop one's job and will be considered as extra. Thus, time for develop one's work is necessary. 

\subsubsection*{Careful experimentation design}
Experimentation design consists of planning experiments carefully, defining the learning goal of each experiment and how it will be evaluated. Identifying the schedule for experiment and appointing a responsible person for the experiment. Transparent communication and documentation of ideas and the results of experiments. 

Setting a clear goal for an experiment makes measuring and evaluating the experiment easier. Interviewees described usually having an eligible result for an experiment, and if that result is achieved, the experiment is considered successful. Goal assists in learning of experiment and further developing. 

Careful experimentation design considers individual characteristics and experience of employees conducting experiments. Experiments should be designed to not consume steep amount of resources of an employee. Experiments should be easy to approach, conduct and even bring resources and energy instead of consuming them. In addition, experimentations should be designed by the employees themselves as they are experts of their own work. When working in close context of customer interface, customer insights and ideas could be considered for experimenting when possible. 

Furthermore, as developing requires adapting novel ways of working and challenging status quo, great experimentation design considers the implementation process of successful ideas and results of experiments. Implementation is a key issue when adapting experimentation-driven approach, and organisational and work structures need to support implementation of experimentations. 
This thesis provided framework through which experimenting can be considered as a tool to foster learning of employees. Furthermore, it provides various managerial implications for fostering organisational learning, innovation and creativity of employees. Additionally, experimentation process was presented as a tool for developing and learning. Requirements for organisational environment in which employee's are willing to conduct experiments were outlined, and several practical recommendations were presented for top management to adapt in their work.

The case study setting, experimentation challenge, can be implemented with ease, and can be considered as a method to adapt experimentation in the organisational development culture. 

\section{Future research topics}
This study focused on identifying factors affecting experimentation behaviour and creating a framework for supporting environment for experimenting. However, interesting findings from the data consisted of affects experimenting has on individual. Experimenting is highly different experience for an individual than planning-based development. Further study of the experience of experimenting should be done in order to form deeper understanding on how to support experimentation in organisational context. Accordingly, as \citet{edmondson1999psychological} argues, psychological safety as a means to promote team performance is increasingly relevant both in future work and research. 

This thesis was studied in a case company of specific service business area, where communication with customer is constant; employees being daily in tight contact with customers, the gap between an idea and conducting experiments, receiving feedback and learning from them can be lower than in other fields of business. This especially, when the aim is to learn of customer needs and ways to serve them better. This is not necessarily an obstacle, as every work life has their own challenges and demands for development. Experiments can be conducted regardless of the business field, the art of experimenting being the ability to design low-cost and low-resource experiments that teach about the fundamental idea or assumption. Future research topic could be to study the experimentation design and how to design experiments that can be best learned from and suit best the occasion. 

In this thesis framework for organisational support experimenting was created, yet further studies are required in order to learn more of the factors and about transferring an experimentation-driven culture in organisations. 

Experimentation challenge was a method MIND team invented in order to study experimentation behaviour in organisations. Interviewees described challenge being encouraging and positive way to put thoughts on improving work. Interesting future research would be to study further experimentation-challenge as a way to implement experimenting to organisational culture. For instance, few interviewees hoped experimentation challenge became an annual tradition, which could support the adaptation of new way of working, learning and reflecting one's work. However, these are hypothesis too early to confirm without further research. 

\section{Reliability of the thesis}
In order to assess the reliability of the thesis, approach of \citet{lincoln1985naturalistic} on reliability is used. According to this approach, reliability is assessed through trustworthiness, which consists of four aspects: credibility, transferability, dependability and confirmability. 

Credibility means that the interpretations made of the original data maintain credible \citep{lincoln1985naturalistic} (page 301-316). In this thesis conclusions are drawn after describing the data collection carefully, so the reader is able to follow the process of interpretations, and by using direct quotations the data behind interpretations is revealed for the reader. In addition, discussions with co-researchers and professors about the interpretations have aimed to maintain credibility. However, interviews were conducted good time after experimentation challenge was over. Few interviews described being difficult to recall the feelings and experiments back when conducting experiments then in detail. Thus, better and more credible data would be gained to have interviewed employees right after the experimentation challenge, while experiments are actively in mind. In this study this was not possible due to the holiday season employees had right after the experimentation challenge. 

Transferability refers to possibilities to transfer the results and findings to another context \citep{lincoln1985naturalistic} (page 316). In the thesis experimentation-driven process was presented and brought to organisational context in a case study. The study revealed several factors in organisational level which can be affected in order to foster experimentation behaviour from individual, team, management and organisational structures perspective. Even though the case study was conducted in specific field, the themes and environmental factors together with managerial implications are transferable to other organisations, as they can be considered as guidelines for good practices and development. 

Dependability refers to the consistency of the research process \citep{lincoln1985naturalistic} (page 316-327). Throughout the thesis the research design and process is described clearly. The research questions are presented in the beginning of the thesis and further revisited in the conclusions, and the results are evaluated through the research questions. 

Confirmability refers to objectivity and neutrality of the thesis \citep{lincoln1985naturalistic} (page 316-327). The writer of the thesis has never been working on studied industry field and was not involved in the empirical case other than in a role of interviewer and observer. In the data analysis process other researchers were involved and the results were discussed among three researchers. The interviews were recorded and transcribed. The theoretical part formed a broad synthesis on factors affecting experimentation, building on the theories from organisational management, organisational learning, development and innovation as well as creativity and leadership. 

\chapter{Summary}
Current and future business environment main features are complexity, uncertainty and unpredictable customer needs. To survive in this complex environment, organisations need to learn faster than rivals. Thus, organisational learning needs to be understood and supported. When creating novel approaches and business opportunities in complex environment, creativity and innovation abilities are required. Therefore, it is crucial to understand how creativity and innovation abilities in organisations and individuals can be supported. 

This thesis presented an experimentation-driven approach as a method to develop and learn in organisational circumstances under high complexity and uncertainty. Experimentation is a significant part of any innovation process and requires creative abilities of individuals. When organisational environment supports creative actions of employees and fosters innovation, the threshold for trying out novel approaches is lower and the culture is more likely to encourage development by experimentation. 

Organisational and business environment being a complex system, various factors affect behaviour of employees and willingness to conduct experiments. This thesis shed light on how experimentation behaviour is likely to be fostered by assuring safe environment for experimentation, supportive leadership behaviour, allocating resources and careful design of experiments. 

%Menisk� t�m� discussioniin johonkin? 
Static friction describes the phenomenon when the attitude towards developing, experimenting and ideating is positive, but when time for action, nothing happens.

One reason to this phenomenon to happen is probably that people don?t want to use lot of energy in new things, and if they sense developing is taking resources and energy, they don?t want to take responsibility of that. Furthermore, employees could see their job as a stabile routine in which developing doesn?t belong, and they tend to think developing, ideating or experimenting a task of someone else, for instance developing department, trainees, substitutes or students. 

I call this the paradox of getting of light. Employees are not eager to develop things that are supposed to help their workload and form a more efficient routine. So instead they stick to the same routines, that actually consume good amount of their energy. 

When resources are effectively used, experimenting happens by itself like in the following situation. 

?Easy thing was probably that when we got the idea, right people at the right time were there. So we were able to execute the idea right away. So we started working on the idea right away and we didn?t have to be like ?ok we have to wait for someone to come and organize when we are all together at the same time? or anything like that..?
