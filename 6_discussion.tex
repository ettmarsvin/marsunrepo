\chapter{Discussion}
Even though almost every of 14 interviewed person described how the immediate superior they had have succeeded in establishing a climate where experimenting is possible to happen, all units did not perform experiments during the experimentation challenge. Thus leadership behavior is only one factor affecting experimentation behavior in an organization. 

Interviewees described that experimenting stands as an excellent method for learning, major factors being both the amount of experimentations done and the reflection on them. Few interviewees described using this method in their daily life and that ?experimentation? is a new word in their vocabulary while few told realizing through the experimentation challenge that their work is actually about trying out new ways of doing things and finding the best way to help customers in their daily lives.

Some of the interviewees described even liking the routine and not being that open for change and new ideas. So according to the study previous experience can encourage performing experiments and improve believing in their ideas, yet it can also lead to routine hard to change after many years. 


At present the informal structure for developing daily work can roughly be divided as follows: the idea emerges from work experience of an employee, a problem at hand, from customer or stakeholder need or by accident during a conversation with colleagues or friends. According to the interviewees, coincidence and problem-based approach play major roles at the moment. After the idea has emerged, support for the idea is asked from a trustworthy colleague and only after that it is democratically discussed in a team meeting. In order an idea to turn into experiment or a new routine, essential arguments and major part of employees ready to engage to the experiment are needed. When an idea is agreed to turn into an experiment, responsibility has to be divided and responsible employees chosen. After that a project team that continues with turning the idea into experimentation is collected. 

(an idea: draw a figure how the process idea to new way of working is done at the moment. Then in the discussion part, suggest a new one) 


As Edmondson (1999)\citeasperson{edmondson1999pscychological} pointed out, psychologicl safety as a means to promote team performance is increasingly relevant both in future work and research. 

As Edmondson \cite{edmondson1999psychological} suggested, studies in real work teams are required, and this study aims to add on this aspect studying factoring affecting experimentation behaviour in real work context. 

"Of course, in a complex social situation, where many causes operate, not all of which are controllable \cite{katz1978social}, innovation may ultimately depend on boiling things down to their essential and focusing on those essential elements of the situation you can do something about. \cite{mumford2002social} Voisko t�m� toimia pointtina sille, ett� kokeileminen auttaa keskittym��n sellaiseen, ja vain sellaiseen, johon voi vaikuttaa itse. Eli discussioniin 

From the data rose various factors affecting experimentation behaviour, and the literature review revealed how similar factors are related to learning in an organisation and organisational learning. 

Example of the leaders was also recocgnized from the data as an important factor affecting experimentation behaviour, and \citet{garvin2008yours}emphasises how through own example leaders can encourage employees to offer new ideas and options.

When affecting and changing only one organisational factor, organisation management need to be alert for occurring inconsistencies that may result to decrease in willingness to engage to experimental behaviour (lee2004mixed)
