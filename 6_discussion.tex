\chapter{Discussion and conclusions} \label{conclusions}
This chapter concludes the findings of the thesis. Theoretical perspective is discussed and compared with the findings of the case study.  Factors affecting experimentation behaviour are discussed in section \ref{fae}. Then, relationship between learning and experimenting is discussed. Managerial implications describe guidelines for organisations based on the study, following the future research topics and discussion of reliability of the thesis. 

A need for novel approaches for development that allows employees to improve their work in more iterative and creative ways that support learning exists. The aim of this thesis was to study experimentation-driven development as such an approach, shed the light on factors affecting experimentation behaviour and provide guidelines for organisations to support its employees in experimenting. Furthermore, \citet{edmondson1999psychological} suggested, studies in real work teams are required, and this study aimed to add on this aspect studying factoring affecting experimentation behaviour in real work context. 

\subsubsection*{Review of research objectives}
The research questions set in the beginning of the thesis are presented below. They are reflected through the discussion. 

\begin{enumerate}
 \item What kinds of factors affect on experimenting behaviour of an employee? 
 \item How experimenting affects an individual? 
  \item How can experimenting behaviour be supported in organisations?
  \item How can experimentation support organisational learning?
\end{enumerate}

\section{Factors affecting experimentation behaviour} \label{fae}
From the data rose various factors affecting experimentation behaviour, and the literature review revealed how similar factors are related to organisation's ability to innovate and employee's creativity. Studies show creativity has enhancing impact on business profit and growth \citep{nystrom1990organizational} According to \citet{vincent2002divergent} creative work consists of creative and innovation processes, and as experimenting stands as significant part of innovation process, similar factors that foster innovation are likely to foster experimenting. In this section most interesting findings from the data are discussed. 

Creativity and innovation in an organisation require integrated organisational approach, right climate, appropriate incentives for innovators, and a systematic way and resources to transform an idea into an innovation. \citep{roffe1999innovation} Experimentation-driven approach enables a systematic way to foster creativity and development. 

\subsubsection*{Role of the immediate superior}
Interviewees described vividly how essential role their leaders and immediate superior have on their willingness to conduct experiments. First of all, leaders acting as role models was considered essential. Example of the leaders was also recognised from literature review as \citet{garvin2008yours} emphasise how through own example leaders can encourage employees to offer new ideas and options. 

Secondly, interviewees described how leaders should support ideation and experimentation by providing feedback and encouraging employees to try out novel approaches and utilise their professionalism. According to \citep{redmond1993putting} leaders should encourage employees to find alternative solutions, approach problems from different perspectives and overall support several alternative problem-solving strategies in order to enhance the creativity and experimentation behaviour of employees. 

Leadership has a great role in ensuring that the climate and culture, structure and practises of work and work environment together with human resource practices are supportive for creative endeavours to occur \citep{shalley2004leaders,oldham1996employee,mumford2002leading}. Furthermore, speaking out loud ideas or making mistakes should not result in punishment or humiliation, and leaders should act in supporting and coach-oriented manner \citep{edmondson1999psychological}. According to \citet{sosik1999leadership} leaders should find ways to inspire employees. 

Thirdly, license to conduct experiments was considered important among interviewees.  Some interviewees described their immediate superiors even requesting for novel approaches, which was considered fair and encouraging, as it showed support for experimenting. According to \citet{barczak1989leadership} leader's task is to provide clear focus for the work of employees. In addition, they can even request creative and innovative solutions form the team, which may lead to better results in creativity of individuals \citep{amabile2002creativity}. Thus, seems that requesting to conduct experiments supports clear focus of the work and is likely to encourage employees.

However, leaders need to stay consistent when requesting for novel approaches and allocating resources for them. It was clear from the empirical data how little time is allocated for developing, and according to \citet{lee2004mixed} when affecting and changing only one organisational factor, organisation management need to be alert for occurring inconsistencies that may result to decrease in willingness to engage to experimental behaviour \citep{lee2004mixed}. Through the green light given and their own example leaders can change the focus from success and failure into thinking in terms of learning and experience \citep{farson2002failuretolerantleader}.

Even though almost every of 14 interviewed person described how the immediate superior they had have succeeded in establishing a climate where experimenting is possible to happen, all units did not perform experiments during the experimentation challenge. Thus leadership behaviour alone is not sufficient for experimentation behaviour in organisations.

\subsubsection*{Role of the team}
Interviewees described four main characteristics belonging to the role of the team. First of all, team should base its decisions and actions on low hierarchy and democracy. Secondly, supportive climate and team practices refers tightly to the concept of psychological safety; an atmosphere where employees feel safe to tell their ideas out loud, where they get support for the ideas and experiments. According to \citet{amabile1996assessing} team can support and improve individuals' ability and willingness to aim for creative actions. 

Thirdly, attitude towards failing is an essential factor for interviewees when considering willingness to conduct experiments. Interviewees described experimenting being more likely when the whole team shares the perspective of failing being allowed. Talking about failures and accepting them as part of developing was in most positive units towards experimenting considered useful. Safe environment does not humiliate or punish employees for failing or coming up with novel ideas or doubts. \citep{garvin2008yours, de2001minority,amabile2008creativity, amabile1996assessing} Also according to \citet{edmondson1996learning} employees are less hesitant to discuss mistakes when normative values of the organisation and work group assure failures being allowed and even expected part of developing and learning. In addition, \citet{garvin2008yours} argues when knowing that well-intentioned interpersonal risks are not punished is a shared belief of a team, team members are more likely to take proactive actions essential for experimenting. Failure can disclose important information and reveal gaps in knowledge, and is thus important in as early phase of the development process as possible. \citep{buijs2007innovation,thomke2001enlightened} Many interviewees, indeed, described failure being essential part of experimenting and developing. 

Lastly, overall team engagement affects positively on the willingness to conduct experiments and is essential for experimenting: interviewees described how experiments and ideas are more likely tail off or be rejected when the whole team is not involved and engaged to the experiment. Also according to \citet{agrell1994team} including team members in ideation assists in idea implementation and through participation new ideas are not that likely to be rejected or abandoned.

\subsubsection*{Structures and practices of developing}
According to empirical data, interviewees described lacking the time to develop their work, discuss about ideas and generate them, not to mention actual experimenting. This implicates the structure of the work and time allocated for developing, whether through conventional planning-based or experimentation-driven approach, is not sufficient. 

\citet{thomke2001enlightened} emphasises an organisation should allow and manage the work for the employees so that fast experimentation is possible. This usually requires challenging routine ways of working and shaping the routines, yet fast experimenting is essential in order to get rapid feedback for shaping the ideas.

According to \citet{amabile2002creativity} clear time should be allocated for developing especially when the aim is to flourish idea generation, creativity, learning and experimentation of new concepts. \citet{redmond1993putting} state that leaders should allow enough time for problem solving and creative actions, and according to \citet{amabile1987creativity} also playing with ideas and exploring multiple perspectives.

\subsubsection*{Characteristics and know-how of an employee}
Substance know-how and expertise of an employee rose from the empirical data as essential for experimenting. Importance of various types of people in a team was considered important. For instance, few interviewees described how interns or students are warmly welcome to the workplace as they have fresh opinions and they can observe the workplace from novel perspectives. However, some interviewees claimed how only through gained experience they felt courageous enough to suggest novel approaches. According to the literature review, homogeneity in teams easily leads to groupthink, routine work and repeating traditional daily practices, while even one or two different individuals can stimulate the innovativeness of a team \citep{sternberg1997creativity}. This perspective supports the findings in empirical data. According to \citet{sternberg1997creativity} the outcasts and those who stand out from the group are required in order to think outside the box, challenge the status quo and present alternative solutions and ideas that would be missing without the participations of these individuals. The outcasts can be considered as interns or students mentioned in interviews.

In the empirical data insight of unwillingness to change was recognised, as interviewees described recognising resistance to change and novel ideas. However, some interviewees described professional expertise and experience assisting in bringing out opinions, being confident enough to conduct experiments and being able to get others along. Thus, according to the study previous experience can encourage performing experiments and improve believing in one's ideas, yet it can also lead to routine hard to change after many years. Meaning of prior knowledge and experience of an employee of area of work before demanding or anticipating creative actions from them is related to creativity \citep{mumford1988creativity,redmond1993putting,shalley2004leaders}. According to \citet{mumford1988creativity} and \citet{redmond1993putting} without previous experience of the job routine and substance knowledge and expertise on the field creative endeavours are more rare. Also \citet{jung2003role} argues for technical knowledge of an individual for fostering creativity.

Also according to \citet{kolb1984experiential} all learning situations should take into account people arriving from all different experiential backgrounds to what they build their new experiences and knowledge on. This partly explains resistance to new ideas, as when new information and experiences are in contradiction to old beliefs and experiences, new ideas and information is more difficult to adapt. In the education process learner's old beliefs and theories should be brought out, examined and tested, following integration of the new models and refined ideas into learner's belief systems. Experimentation seems to assist in this process, as feedback from them should be instant. 

Furthermore, employee's attitude towards experimenting seems to depend on employee's attitude, motivation and engagement towards work rather than only on work experience and prior knowledge. Together with the attitude and motivation towards developing, individual tolerance for uncertainty, self-criticism and confidence affect on his willingness to conduct experiments. Few interviews recognised experimenting assisting in lowering the threshold for acting, while usually being perfectionistic and fearing failures. This resonates with \citet{mumford2002leading} who claim that level of uncertainty can be reduced for instance through goal-setting and fast prototyping. Furthermore, experiments are allowed to fail and they are encouraged to be designed as low-risky as possible. 

\subsubsection*{The gap between an idea an experiment}
Characteristic of an idea and experiment have effects on whether the idea is evolved into an experiment. Interviewees for instance described planning experiments so large, that when actual time to conduct the experiment and leave the ideation mode, experiment was left undone. Employees did not want to take additional responsibility while fearing it would take too much time or did not consider the experiment important enough.

Static friction refers to a moment when the attitude towards developing, experimenting and ideating is positive, but when time for action, nothing happens. This could be called as the paradox of getting of light. Employees are not eager to develop things that are supposed to help their workload and form a more efficient routine. So instead they stick to the same routines, that actually consume good amount of their energy.

Interviewees described situation that was lacking static friction as smooth and clear, when sufficient resources were available right away, experiment was light enough to conduct with ease, necessary people were engaged to experiment and thus time from idea to actual experiment and reflection was short. However, this is rarely the case when considering hectic working environments with changing situations and no allocated time for developing.
Thus, when resources are effectively used and participants engaged, experimenting happens by itself like in the situation described above. 

Setting a clear goal for an experiment makes measuring and evaluating the experiment easier. Interviewees described usually having an eligible result for an experiment, and if that result is achieved, the experiment is considered successful. Goal assists in learning of experiment and further developing, and the experimentation overall occurring. Furthermore, team members were more likely to be engaged when sharing a transparent goal. According to \citet{thomke2001enlightened} the whole team understanding the meaning of experimenting and developing forms a basis for team engagement. In addition, clear purpose for the team in order it to function and exist was supported in literature review \citep{katzenbach1993wisdom}.

Additionally, team members have supplementary knowledge and abilities compared to each other, and they share a goal, targets and ways of working and approach \citep{edmondson1999psychological}. According to \citet{katzenbach1993wisdom} great team performance consists of continuous work of shaping a common purpose, agreeing on performance goals, defining a common working approach, developing high level complementary skills and being transparent on the results. He emphasises that through disciplined actions groups transform to teams and argues how demanding schedules, long-standing habits and unwarranted assumptions tend to threaten team efficiency and performance. Even though these aspects were not all straight recognised from the empirical data, many parts of it were, and further studies should be done in order to study the affects experimenting has on team level. 

Some interviewees described being able to ideate and conduct experiments freely, getting support and being encouraged by the leader and the team. However, few interviewees described being important to design experiments as safe and harmless for participants, and that experiments always need to be designed considering customers and their safety. Also \citep{farson2002failuretolerantleader} point out how issues about safety and health of people participating experiments should not be overruled by the insights gained from an experiment. 

In addition, empirical data revealed how ideas emerging from customers and when experiments conducted in close distance to customer and stakeholder interface, best learnings were gained. According to \citet{thomke2003r} experiments concerning service development remain most useful when conducted in real life circumstances, as the feedback is instant and customer transactions real. Also \citet{quinn1985managing} emphasise engaging lead customers in the interactive development process in order to elucidate more relevant information about customer's demands. 

\subsubsection*{Emotional experience and engagement}
Experience of experimenting consists of both positive and negative emotions resulting from the process of experimentation as well as engagement and motivation towards work. Interviewees described how successful experiments, ideation and planning conducting experiments brought them happiness, excitement, inspiration and boost to self-esteem. Interviewees also described experimenting nourishing one's creativity and ability to throw oneself to try novel approaches. 

In turn, failing is likely to cause frustration and disappointment which can also result from disengagement of all team members. Interviewees told failing does not feel good, and sometimes it is frustrating. Yet oftentimes failing is not considered too serious and after some time only gives boost to try again, with better knowledge. 
Failing as a personal matter remains a difficult subject, as failing never feels exceptionally great, and often employees still consider failed work as failing personally \citep{farson2002failuretolerantleader}. 

Interviewees described how experimenting assisted in braking the routines and challenging conventional ways of working. \citet{dewey1956human} separates learning from the conventional way for humans to behave and follow the routines. 

Interviewees also described feeling even tired when their idea did not succeed in an experiment, did not go through or team members were not engaged. Through literature review straight correlations to fatigue was not found, yet correlations between lack of time for development and willingness to conduct experiments were found. Employees should be empowered through motivating their intrinsic motivation and allowing them autonomously conduct experiments and learn from them. \citep{amabile1998kill,jung2001transformational} 

According to \citet{quinn1985managing}, fast multiple-idea prototyping leads to more innovative outcomes, offers essential information about ideas or product's quality, motivates employees, and helps the company and the team to cope with anxiety and uncertainty in development. Thus, fast prototyping serves an essential way for learning from the iterative process. Market analysis, however, remain valuable when dealing with familiar products, yet with radical innovations they may easily offer misleading information. \citep{quinn1985managing}

\subsubsection*{Learning}
Empirical data brought to discussion learning aspects of experimenting through reflection of work, process know-how of experimenting and grown resilience towards work. This learning aspect of experimenting is further discussed in next section. 

However, seems that acquiring a first experience of experimentation narrows the gap between an idea and actually performing an experiment. As according to the literature, familiarity of a subject is likely to assist in adopting novel methods, an implication could be drawn how making the first experiment an essential part for the longer-term experimentation approach. The interviewees described realising only after the first experiment that it actually brought energy and resources and not only deprived them. Thus, the sooner the first experiment is executed after the idea has rose, the better it seems to be for the whole developing process.

\section{The relation between learning and experimentation} \label{relation}
In this thesis, learning is considered as a process of continuous trial and error \citep{argyris1978organizational,edmondson1999psychological} that includes individual growth and improved performance. According to the experimentation process of \citet{thomke1998managing}, learning glues together all the four phases of the process. Setting a hypothesis, planning an experiment, executing it and analysing the results are all reflected throughout the process in order to learn about the fundamental idea and develop it further. In addition, according to the Execution Innovation Model, novelties are only generated through a learning process of iterative experimentation. \citep{tuulenmaki2011art} 

According to the data on experience of experimentation, interviewees described how experimenting assists in learning about problem at hand, and provides information whether trial is malfunctioning. In addition, interviewees described that experimenting stands as an excellent method for learning, major factors being both the amount of experimentations conducted and the reflection on them. Empirical data with literature review thus supports the perspective that experimenting can serve as a method for learning. 

Through experimenting interviewees described gaining insights on their own work, on experimentation process and on team and individual behaviour. This however required communication and discussion among colleagues, shared trust and engagement towards experiments and positive attitude towards failure. In the theory of \citet{edmondson1999psychological} on team learning, factors essential for learning are similar to essential factors for experimenting. These include transparent information sharing, asking for help, receiving and giving feedback, tolerating failures and discussing about them in order to reflect experiments and improve work. 

Few interviewees described using experimentation-driven approach in their daily life and that through the experimentation challenge 'experimentation' is a new word in their vocabulary while few told realising through the experimentation challenge that their work is actually about trying out new ways of doing things and finding the best way to help customers in their daily lives.Thus, through experimenting interviewees gained deeper understanding of their work and it assisted in reflection of work. 

In the planning-oriented developing process learning is likely to happen when the product or service is launched or a first, large-scale pilot is tested. Thus, the results of this study support the statement how experimenting could be used as a tool for learning about development idea with lower resources. As failure is very likely outcome of experimentation, every experiment are opportunities for growth and learning. 

According to \citet{buijs2007innovation}, valuation points stand for usable tool for measuring the quality of idea but gives also understanding of how the evaluation process is going. In addition, while evaluating, team members also need to reflect the process and the idea, through which learning occurs. 

Also \citep{runco1994problem} emphasises how only after evaluation of ideas implementation can be discussed and performed and several studies show the essence of evaluation \citep{mumford2002leading,vincent2002divergent}. Useful questions in evaluation process could be "What went well?", "What can be improved?" and "What has been learned?" \citep{buijs2007innovation}. These were factors revealing also from empirical data, supporting the relationship between learning and evaluation of experiments. 

The essence of learning from experiments is to figure out what works and what does not in an experiment or idea. Thus, experiments should be designed and planned keeping in mind how to maximise the amount of learning and valuable insights, not focus on wrong details and successful experiment. Through defining accurate measures one can actually know whether the experimentation was useful and essential was learned or not \citep{thomke2003r}. 

In group level, learning is enabled through testing assumptions and discussion of opinion differences transparently in order to improve team performance. \citep{edmondson1999psychological} 

According to the data experimenting allows to test assumptions together with a team providing a transparent tool to learn together. This resonates with \citet{edmondson1999psychological}, who argues how through testing assumptions and discussion of opinions transparently learning is enabled. 

Accordingly, in management literature learning is considered relating and even being dependent on receiving feedback \citep{schon1983reflective}, discussion and failure \citep{sitkin1992learning} and experimenting \citep{henderson1990architectural}. As relevant information about performance is acquired through errors, discussion about them has been related with organisational effectiveness \citep{sitkin1992learning}. According to \citet{huy2003rhythm} organisations learn best through small experiments and trying out new things, and the closer and more related experimentations are to customers and customer interfaces, the more can be learned. 

Setting frames to experimentation is important; defining when the experimentation begins and when is a moment for closure. In many reported experiments experimenting has been understood as fast prototyping but no reflection. These leave a person easily with an unfinished feeling that is too easily related to failure. 

\section{Practical implications} \label{framework}
This thesis brought novel perspective on organisational learning and development by combining and presenting an approach for fostering creativity and innovation of employees in an organisation through experimentation-driven process. 

Experimentation-driven development has not yet been widely studied, thus this thesis provided important theoretical data and insights on experimentation process as well as sheds the light on the important issue of employee engagement and learning in order to create new value and competitive advantage in organisations. Furthermore, insights were gained on the emotional experience of experimenting.

This thesis provided perspective through which experimenting can be considered as a tool to foster learning of employees. Requirements for organisational environment in which employee's are willing to conduct experiments were outlined.

This section assembles the findings of the study to set of guidelines of factors organisation should consider in order to support experimentation.

\subsubsection*{Safe and supporting environment}
In the heart of every change and development project are employees, the group of individuals who are touched by the change. In order to proceed to meaningful and efficient changes for both the company and its employees, individuals has to be onboard. 

In the very heart of willingness to conduct experiments seems to be individuals sense of safe and supporting environment towards creativity, idea generation and experimentation. This includes team engagement, positive attitude towards failing, environment tolerating uncertainty and fostering risk-taking. Furthermore, brainstorming and saying out loud problems and ideas should be encouraged. 

\subsubsection*{Support from leaders}
Leaders should show their support towards experimenting by acting as role-models, encouraging employees to work on their own expertise and interests, reward from successful experiments and ideas and showing support by taking results of experiments to upper management. When providing sufficient level of autonomy to employees, leaders are likely to encourage their employee's intrinsic motivation leading to more satisfied, efficient and creative employees. 

\subsubsection*{Allocating resources}
Allocating resources refers to established truth that developing one's work requires time, as well as creative process. Experimentations themselves should be designed to consume little resources, yet reasonable amount of resources should be reserved for executing experiments. Professional conversations among colleagues, visiting peer units and meeting peer colleagues in the country are likely to foster the expertise, creativity and willingness to develop one's work. No time for idea generation or experimenting is likely to decrease willingness to conduct develop one's job and will be considered as extra. Thus, time for develop one's work is necessary. 

\subsubsection*{Careful experimentation design}
Experimentation design consists of planning experiments carefully, defining the learning goal of each experiment and how it will be evaluated. Furthermore, identifying the schedule for experiment and appointing a responsible person for the experiment. In addition, transparent communication and documentation of ideas and the results of experiments are all equally essential for successful experiments.  

Setting a clear goal for an experiment makes measuring and evaluating the experiment easier. Interviewees described usually having an eligible result for an experiment, and if that result is achieved, the experiment is considered successful. Goal assists in learning of experiment and further developing. 

Careful experimentation design considers individual characteristics and experience of employees conducting experiments. Experiments should be designed to not consume steep amount of resources of an employee. Experiments should be easy to approach, conduct and even bring resources and energy instead of consuming them. In addition, experimentations should be designed by the employees themselves as they are experts of their own work. When working in close context of customer interface, customer insights and ideas could be considered for experimenting when possible. 

Furthermore, as developing requires adapting novel ways of working and challenging status quo, great experimentation design considers the implementation process of successful ideas and results of experiments. Implementation is a key issue when adapting experimentation-driven approach, and organisational and work structures need to support implementation of experimentations. 

\section{Future research topics}
This study focused on identifying factors affecting experimentation behaviour and creating a framework for supporting environment for experimenting. Additionally, interesting findings from the data consisted of affects experimenting has on individual. Experimenting is highly different experience for an individual than planning-based development. Further study of the experience of experimenting should be done in order to form deeper understanding on how to support experimentation in organisational context. Accordingly, as \citet{edmondson1999psychological} argues, psychological safety as a means to promote team performance is increasingly relevant both in future work and research. 

This thesis was made in a case company of specific field of service business, where communication with customers is constant; employees being daily in tight contact with customers, the gap between an idea and conducting experiments, receiving feedback and learning from them can be lower than in other fields of business. This especially, when the aim is to learn of customer needs and ways to serve them better. This is not necessarily an obstacle, as every work life has their own challenges and demands for development. Experiments can be conducted regardless of the business field, the art of experimenting being the ability to design low-cost and low-resource experiments that teach about the fundamental idea or assumption. Future research topic could be to study the experimentation design and how to design experiments that can be best learned from and suit best the occasion. 

Considering the complex character of organisational change, learning and behaviour affected by various factors in individual, organisational and team levels, comprehensive literature review was gathered. It combined literature on various fields of research aiming to form holistic picture of factors affecting experimentation in changing business environment. However, more focused research should be conducted on various topics in order to gain proof on the relations and factors found in this study. 

In this thesis guidelines for organisations to support experimenting was provided, yet further studies are required in order to learn more about the factors and about transferring an experimentation-driven culture in organisations. 

Experimentation challenge was a method MIND team invented in order to study experimentation behaviour in organisations. Interviewees described challenge being encouraging and positive way to put thoughts on improving work. Interesting future research would be to study further experimentation-challenge as a way to implement experimenting to organisations way of developing. For instance, few interviewees hoped experimentation challenge to became an annual tradition, which could support the adaptation of new way of working, learning and reflecting one's work. However, these are hypothesis too early to confirm without further research. 

\section{Reliability of the thesis}
In order to assess the reliability of the thesis, approach of \citet{lincoln1985naturalistic} on reliability is used. According to this approach, reliability is assessed through trustworthiness, which consists of four aspects: credibility, transferability, dependability and confirmability. 

Credibility means that the interpretations made of the original data maintain credible \citep{lincoln1985naturalistic} (page 301-316). In this thesis conclusions are drawn after describing the data collection carefully, so the reader is able to follow the process of interpretations, and by using direct quotations the data behind interpretations is revealed for the reader. In addition, discussions with co-researchers and professors about the interpretations have aimed to maintain credibility. However, interviews were conducted good time after experimentation challenge was over. Few interviews described being difficult to recall the feelings and experiments back when conducting experiments then in detail. Thus, better and more credible data would be gained to have interviewed employees right after the experimentation challenge, while experiments are actively in mind. In this study this was not possible due to the holiday season employees had right after the experimentation challenge. 

Transferability refers to possibilities to transfer the results and findings to another context \citep{lincoln1985naturalistic} (page 316). In the thesis experimentation-driven process was presented and brought to organisational context in a case study. The study revealed several factors in organisational level which can be affected in order to foster experimentation behaviour from individual, team, management and organisational structures perspective. Even though the case study was conducted in specific field, the themes and environmental factors together with managerial implications are transferable to other organisations, as they can be considered as guidelines for good practices and development. 

Dependability refers to the consistency of the research process \citep{lincoln1985naturalistic} (page 316-327). Throughout the thesis the research design and process is described clearly. The research questions are presented in the beginning of the thesis and further revisited in the conclusions, and the results are evaluated through the research questions. 

Confirmability refers to objectivity and neutrality of the thesis \citep{lincoln1985naturalistic} (page 316-327). The writer of the thesis has never been working on studied industry field and was not involved in the empirical case other than in a role of interviewer and observer. In the data analysis process other researchers were involved and the results were discussed among three researchers. The interviews were recorded and transcribed. The theoretical part formed a broad review on factors affecting experimentation, building on the theories from organisational management, organisational learning, development and innovation as well as creativity and leadership. 


