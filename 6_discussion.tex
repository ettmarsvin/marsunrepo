\chapter{Discussion}
This chapter concludes the findings of the thesis. Results are discussed and the theoretical framework is
First, research objectives are revisited and answers to them concluded. Then, theoretical perspective is discussed and compared with the findings of the case study.  Both theoretical and managerial implications of the thesis are discussed in section x, following the future research topics and discussion of reliability and limitations of the thesis. 

\section{Research objectives}

How can experimenting support organisational learning? 
What kinds of factors affect on experimenting behaviour of an individual? 
How can experimenting behaviour be supported in an organisation? 

\section{Framework for experimentation-driven developing and learning}
According to \citet{buijs2007innovation}, innovation consists of coming up with novel ideas and implementing them. Ideating begins with exploring, developing and implementing the ideas, following introducing the ideas, which have turned into products or services, into the marketplace. Innovation process is a series of stages for processing the idea, and in the end of every stage the idea is reflected and evaluated before further processing. Evaluation points stands for usable tool for measuring the quality of idea but gives also understanding of how the evaluation process is going. In addition, while evaluating, team members also need to reflect the process and the idea, through which learning occurs. Also \citep{runco1994problem} emphasises how only after evaluation of ideas implementation can be discussed and performed and several studies show the essence of evaluation \citep{mumford2002leading,vincent2002divergent}. Useful questions in evaluation process could be "What went well?", "What can be improved?" and "What has been learned?" \citep{buijs2007innovation}. t�m� suoraan nyt innovation-kappaleesta, mutta t�m�n voisi yhdist�� experimentationiin, ett� paljon samosa piirteit� ja siksikin olennaista ottaa innovaatiopointit huomioon experimentationissa. 

From the data rose various factors affecting experimentation behaviour, and the literature review revealed how similar factors are related to organisation's ability to innovate and employee's creativity. 

\section{Requirements for experimentation-driven development}
In the heart of every change and development project there are employees, the group of individuals who are touched by the change. In order to proceed to meaningful and efficient changes for both the company and its employees, individuals has to be onboard. 

Even though almost every of 14 interviewed person described how the immediate superior they had have succeeded in establishing a climate where experimenting is possible to happen, all units did not perform experiments during the experimentation challenge. Thus leadership behaviour only is not sufficient for experimentation behaviour in organisations.

Interviewees described that experimenting stands as an excellent method for learning, major factors being both the amount of experimentations done and the reflection on them. Few interviewees described using this method in their daily life and that 'experimentation' is a new word in their vocabulary while few told realising through the experimentation challenge that their work is actually about trying out new ways of doing things and finding the best way to help customers in their daily lives.

Many studies relate experimenting to innovation and creative abilities of an individual. Thus, in this chapter, creativity and intrinsic motivation are outlined. In the empirical part developing and learning through experimenting were in focus, yet understanding experimenting in broader perspective may assist in seeing its advantages for value-creation. Therefore innovation perspective is presented in chapter x. 

Prior knowledge
Some of the interviewees described even liking the routine and not being that open for change and new ideas. So according to the study previous experience can encourage performing experiments and improve believing in their ideas, yet it can also lead to routine hard to change after many years. 

Leadership behaviour 
Interviewees described vividly how essential role their leaders and immediate superior have on their willingness to conduct experiments. 
As ... claims, inconsistency remains as a risk when aiming to foster novel approaches. Even though 
 
\section{Theoretical implications}
This thesis brought novel perspective on organisational learning and development by combining and presenting an approach for fostering creativity and innovation of employees in an organisation through experimentation-driven process. 
Experimentation-driven development has not yet been widely studied, thus this thesis provided important theoretical data and insights on experimentation process as well as sheds the light on the important issue of employee engagement and learning in order to create new value and competitive advantage for organisations. 

\citet{edmondson1999psychological} suggested, studies in real work teams are required, and this study aims to add on this aspect studying factoring affecting experimentation behaviour in real work context. 

\section{Managerial implications}
This thesis provides framework through which experimenting can be considered as a tool to foster innovation, creativity and learning of employees. Furthermore, it provides various managerial implications for fostering organisational learning, innovation and creativity of employees. Additionally, experimentation process was presented as a tool for developing and learning. Requirements for organisational environment in which employee's are willing to conduct experiments were outlined, and several practical recommendations were presented for top management to adapt in their work.

The case study setting, experimentation challenge, can be implemented with ease, and can be considered as a method to adapt experimentation in the organisational development culture. 

\section{Future research topics}
As \citet{edmondson1999pscychological} pointed out, psychological safety as a means to promote team performance is increasingly relevant both in future work and research. 

\section{Reliability of the thesis}
Common criteria for evaluating the quality of a quantitative study are reliability and validity, where reliability refers to repeatability and validity to accuracy in means of measurement. In this thesis, however, quantitative approach is not used and in order to assess the reliability of the thesis, approach of \citet{lincoln1985naturalistic} on reliability is used. According to this approach, reliability is assessed through trustworthiness, which consists of four aspects: credibility, transferability, dependability and confirmability. 

Credibility means that the interpretations made of the original data maintain credible \citep{lincoln1985naturalistic} (page 301-316). In this thesis conclusions are drawn after describing the data collection carefully, so the reader is able to follow the process of interpretations, and by using direct quotations the data behind interpretations is revealed for the reader. In addition, discussions with co-researchers and professors about the interpretations have aimed to maintain credibility. 

Transferability refers to possibilities to transfer the results and findings to another context \citep{lincoln1985naturalistic} (page 316). In the thesis experimentation-driven process was presented and brought to organisational context in a case study. The study revealed several factors in organisational level which can be affected in order to foster experimentation behaviour from individual, team, management and organisational structures perspective. Even though the case study was conducted in specific field, the themes and environmental factors together with managerial implications are transferable to other organisations, as they can be considered as guidelines for good practices and development. 

Dependability refers to the consistency of the research process \citep{lincoln1985naturalistic} (page 316-327). Throughout the thesis the research design and process is described clearly. The research questions are presented in the beginning of the thesis and further revisited in the conclusions, and the results are evaluated through the research questions. 
Theory building process followed the principles of chosen research method, case study. 

Confirmability refers to objectivity and neutrality of the thesis \citep{lincoln1985naturalistic} (page 316-327). The writer of the thesis has never been working on studied industry field and was not involved in the empirical case other than in a role of interviewer and observer. In the data analysis process other researchers were involved and the results were discussed among three researchers. The interviews were recorded and transcribed. The theoretical part formed a broad synthesis on factors affecting experimentation, building on the theories from organisational management, organisational learning, development and innovation as well as creativity and leadership. Considering these several aspects increase the objectivity of the thesis. 

----------------------



"Of course, in a complex social situation, where many causes operate, not all of which are controllable \citet{katz1978social}, innovation may ultimately depend on boiling things down to their essential and focusing on those essential elements of the situation you can do something about. \citet{mumford2002social} Voisko t�m� toimia pointtina sille, ett� kokeileminen auttaa keskittym��n sellaiseen, ja vain sellaiseen, johon voi vaikuttaa itse. Eli discussioniin 

Example of the leaders was also recognised from the data as an important factor affecting experimentation behaviour, and \citet{garvin2008yours} emphasises how through own example leaders can encourage employees to offer new ideas and options.

When affecting and changing only one organisational factor, organisation management need to be alert for occurring inconsistencies that may result to decrease in willingness to engage to experimental behaviour (lee2004mixed)


Acquiring a first experience of experimentation narrows the gap between an idea and actually performing an experiment, making the first experiment an essential part for the longer-term experimentation approach. The interviewees described realising only after the first experiment that it actually brought energy and resources and not only deprived them. Thus, the sooner the first experiment is done after the idea has rose, the better it seems to be for the whole developing process.

Setting a clear goal for an experiment makes measuring and evaluating the experiment easier. Interviewees described usually having an eligible result for an experiment, and if that result is achieved, the experiment is considered successful. 

?Well usually when we have some idea, some certain result is wanted, and if that result was achieved, it [experiment] was quite successful.?

Interviewees described that experimenting stands as an excellent method for learning, major factors being both the amount of experimentations done and the reflection on them. Few interviewees described using this method in their daily life and that ?experimentation? is a new word in their vocabulary while few told realizing through the experimentation challenge that their work is actually about trying out new ways of doing things and finding the best way to help customers in their daily lives.

Some of the interviewees described even liking the routine and not being that open for change and new ideas. So according to the study previous experience can encourage performing experiments and improve believing in their ideas, yet it can also lead to routine hard to change after many years. 

At present the informal structure for developing daily work can roughly be divided as follows: the idea emerges from work experience of an employee, a problem at hand, from customer or stakeholder need or by accident during a conversation with colleagues or friends. According to the interviewees, coincidence and problem-based approach play major roles at the moment. After the idea has emerged, support for the idea is asked from a trustworthy colleague and only after that it is democratically discussed in a team meeting. In order an idea to turn into experiment or a new routine, essential arguments and major part of employees ready to engage to the experiment are needed. When an idea is agreed to turn into an experiment, responsibility has to be divided and responsible employees chosen. After that a project team that continues with turning the idea into experimentation is collected. 

(an idea: draw a figure how the process idea to new way of working is done at the moment. Then in the discussion part, suggest a new one) 
