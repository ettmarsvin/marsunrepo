\chapter{Discussion}
The discussion of results is divided into x sub-chapters. 

Even though almost every of 14 interviewed person described how the immediate superior they had have succeeded in establishing a climate where experimenting is possible to happen, all units did not perform experiments during the experimentation challenge. Thus leadership behavior is only one factor affecting experimentation behavior in an organization. 

Interviewees described that experimenting stands as an excellent method for learning, major factors being both the amount of experimentations done and the reflection on them. Few interviewees described using this method in their daily life and that ?experimentation? is a new word in their vocabulary while few told realizing through the experimentation challenge that their work is actually about trying out new ways of doing things and finding the best way to help customers in their daily lives.

Some of the interviewees described even liking the routine and not being that open for change and new ideas. So according to the study previous experience can encourage performing experiments and improve believing in their ideas, yet it can also lead to routine hard to change after many years. 


At present the informal structure for developing daily work can roughly be divided as follows: the idea emerges from work experience of an employee, a problem at hand, from customer or stakeholder need or by accident during a conversation with colleagues or friends. According to the interviewees, coincidence and problem-based approach play major roles at the moment. After the idea has emerged, support for the idea is asked from a trustworthy colleague and only after that it is democratically discussed in a team meeting. In order an idea to turn into experiment or a new routine, essential arguments and major part of employees ready to engage to the experiment are needed. When an idea is agreed to turn into an experiment, responsibility has to be divided and responsible employees chosen. After that a project team that continues with turning the idea into experimentation is collected. 

(an idea: draw a figure how the process idea to new way of working is done at the moment. Then in the discussion part, suggest a new one) 


As \citet{edmondson1999pscychological} pointed out, psychologicl safety as a means to promote team performance is increasingly relevant both in future work and research. 

\citet{edmondson1999psychological} suggested, studies in real work teams are required, and this study aims to add on this aspect studying factoring affecting experimentation behaviour in real work context. 

"Of course, in a complex social situation, where many causes operate, not all of which are controllable \citet{katz1978social}, innovation may ultimately depend on boiling things down to their essential and focusing on those essential elements of the situation you can do something about. \citet{mumford2002social} Voisko t�m� toimia pointtina sille, ett� kokeileminen auttaa keskittym��n sellaiseen, ja vain sellaiseen, johon voi vaikuttaa itse. Eli discussioniin 

From the data rose various factors affecting experimentation behaviour, and the literature review revealed how similar factors are related to learning in an organisation and organisational learning. 

Example of the leaders was also recocgnized from the data as an important factor affecting experimentation behaviour, and \citet{garvin2008yours}emphasises how through own example leaders can encourage employees to offer new ideas and options.

When affecting and changing only one organisational factor, organisation management need to be alert for occurring inconsistencies that may result to decrease in willingness to engage to experimental behaviour (lee2004mixed)

Conclusions
Final part of the thesis concludes the findings and consists of x chapters. The results presented in previous part are further discussed in the first chapter. In addition, the theoretical framework presented in the theoretical part is supplemented with the findings of the empirical part. The research questions are revisited in order to determine how they were answered in the thesis. Both theoretical and managerial implications of the thesis are discussed in the second chapter. In the final chapter future research topics are discussed and the reliability of the thesis is evaluated. 

Acquiring a first experience of experimentation narrows the gap between an idea and actually performing an experiment, making the first experiment an essential part for the longer-term experimentation approach. The interviewees described realizing only after the first experiment that it actually brought energy and resources and not only deprived them. Thus, the sooner the first experiment is done after the idea has rose, the better it seems to be for the whole developing process.

Setting a clear goal for an experiment makes measuring and evaluating the experiment easier. Interviewees described usually having an eligible result for an experiment, and if that result is achieved, the experiment is considered successful. 

?Well usually when we have some idea, some certain result is wanted, and if that result was achieved, it [experiment] was quite successful.?

Interviewees described that experimenting stands as an excellent method for learning, major factors being both the amount of experimentations done and the reflection on them. Few interviewees described using this method in their daily life and that ?experimentation? is a new word in their vocabulary while few told realizing through the experimentation challenge that their work is actually about trying out new ways of doing things and finding the best way to help customers in their daily lives.

Some of the interviewees described even liking the routine and not being that open for change and new ideas. So according to the study previous experience can encourage performing experiments and improve believing in their ideas, yet it can also lead to routine hard to change after many years. 


At present the informal structure for developing daily work can roughly be divided as follows: the idea emerges from work experience of an employee, a problem at hand, from customer or stakeholder need or by accident during a conversation with colleagues or friends. According to the interviewees, coincidence and problem-based approach play major roles at the moment. After the idea has emerged, support for the idea is asked from a trustworthy colleague and only after that it is democratically discussed in a team meeting. In order an idea to turn into experiment or a new routine, essential arguments and major part of employees ready to engage to the experiment are needed. When an idea is agreed to turn into an experiment, responsibility has to be divided and responsible employees chosen. After that a project team that continues with turning the idea into experimentation is collected. 

(an idea: draw a figure how the process idea to new way of working is done at the moment. Then in the discussion part, suggest a new one) 

Reliability of the thesis
Lincoln and Guba ?l�hde Taijan dipasta

Common criteria for evaluating the quality of a quantitative study are reliability and validity, where reliability refers to repeatability and validity to accuracy in means of measurement. In this thesis, however, quantitative approach is not used and in order to assess the reliability of the thesis, Lincoln and Guba?s (1985) approach on reliability is used. According to this approach, reliability is assessed through trustworthiness, which consists of four aspects: credibility, transferability, dependability and confirmability. 

Credibility refers to the interpretations made of the original data and their credibility (Lincoln and Guba, pp. 301-316). In this thesis? 

In addition, direct quotations are used in this thesis in order to reveal the data behind the interpretations. Co-researchers and professors have also been discussing about the interpretations thus adding credibility. 

Transferability means the possibilities to transfer the results and findings to another context. In the thesis?

Dependability refers to the consistency of the research process. Throughout the thesis the research design and process is described clearly. The research questions are presented in the beginning of the thesis and further revisited in the conclusions, and the results are evaluated through the research questions. 
Theory building process followed the principles of chosen research method, case study. 

Confirmability refers to objectivity and neutrality of the thesis. The writer of the thesis has never been working on studied industry field and was not involved in the empirical case other than in a role of interviewer and observer. In the data analysis process other researchers were involved and the results were discussed at least among three different researchers. In the theoretical research? 
