% ==================================================================
% This template has been used and modified by
% Sanna Suoranta, Pekka Kanerva, Lauri Karppinen, Linda Staffans,
% Laura Takkinen, Jukka Larja, Samuli Sorvakko, 
% and numerous unknown authors.
% ------------------------------------------------------------------
% 9.2.2011 Tuukka Haapasalo made a near-complete rewrite of the
% package, cleaning up stuff and updating the layout to the
% new Aalto specifications, and adding a standard thesis layout
% example.
% ------------------------------------------------------------------
\NeedsTeXFormat{LaTeX2e}
\ProvidesPackage{aalto-thesis}[2011/02/25 v1.1 Aalto Thesis style]

\selectlanguage{english}

% Graphics are required for including PDF files into the document
\RequirePackage{graphicx}
% The url package provides commands for formatting URLs
\RequirePackage{url}
% Multirow allows table cells to span multiple rows
\RequirePackage{multirow}
% The tabularx environment is an extension of tabular that allows 
% paragraph-type column widths to be calculated automatically 
\RequirePackage{tabularx}

% Improvements for mathematical formatting
\RequirePackage{amsmath}

% The lastpage package provides a label that can be used to get the 
% page number of the last page
\RequirePackage{lastpage}

% The etoolbox command provides some advanced commands 
\RequirePackage{etoolbox}

% The array package fixes some things in the default LaTeX arrays and 
% tabulars. This package is required by many other packages. 
\RequirePackage{array}
% The ifthen package provides conditional evaluation of statements
\RequirePackage{ifthen}
% Textpos can be used to position text absolutely (at an absolute 
% position in the page, instead of a relative position)
\RequirePackage[absolute]{textpos}

% This package allows to transform pagerefs to numbers
\RequirePackage{refcount}

\newcommand{\startcoverpage}{\pagenumbering{arabic}}
\newcommand{\startfirstchapter}{}
% The number of pages is counted based on the labels.
% The lastpage package calculates the position of the LastPage label. 
% A plain label may not work if last pages contain floats that are placed 
% after text.
\newcommand{\PAGES}{\pageref{LastPage}}


% From Samwise, edited by JLarja. Fixme can be used to wrote something that should be fixed later. 
% It is better than just writing something to comments, since it is shown in the output file. If 
% mydraft is declared (in documentclass), text will be shown bold between two FIXMEs. If not, text 
% will be shown normally.
\newcommand{\fixme}[1]{#1}
% Draft command is used in the cover page to show the date of the
% draft (assumed to be today)
\newcommand{\DRAFT}{}
\DeclareOption{mydraft}{
\renewcommand{\fixme}[1]{\textsc{\small !Fixme} \textbf{#1}
\textsc{\small Fixme!}} 
\renewcommand{\DRAFT}{
\begin{center}
\textbf{DRAFT! --- \today\ --- DRAFT!}
\end{center}}
}

\newcommand{\NUMINSTRUCTORS}{1}
\DeclareOption{twoinstructors}{
\renewcommand{\NUMINSTRUCTORS}{2}
}
\newcommand{\NUMSUPERVISORS}{1}
\DeclareOption{twosupervisors}{
\renewcommand{\NUMSUPERVISORS}{2}
}

\DeclareOption{doublenumbering}{
\renewcommand{\startcoverpage}{\pagenumbering{roman}}
\renewcommand{\startfirstchapter}{\pagenumbering{arabic}}
% The number of pages is counted based on the labels.
% Pages-prelude is placed manually, and the lastpage package calculates
% the position of the LastPage label. A plain label may not work if last
% pages contain floats that are placed after text.
%\renewcommand{\PAGES}{\pageref{pages-prelude}~$+$~\pageref{LastPage}}
}


% this creates a counter that can count the number of pages between two pagerefs
\newcommand{\pagedifference}[2]{%
  \number\numexpr\getpagerefnumber{#2}-\getpagerefnumber{#1}\relax}
  
% add 4/9/2013: pager-app laskee vika l�hdeluettelosivu
%\renewcommand{\PAGES}{\pageref{pages-prelude}$+$\pageref{pages-end}$+$12}
\renewcommand{\PAGES}{\pageref{pages-end}$+$\pagedifference{pages-end}{appendices-end}}
\ProcessOptions

% A shortcut command for using ifthenelse when you don't need the
% else clause.
\newcommand{\ifthen}[2]{\ifthenelse{#1}{#2}{}}

% By default, only up to 3rd level headers are numbered (e.g. 1.2.2.)
% Specify which headers are numbered
\setcounter{secnumdepth}{4} 
% Specify which headers are put in the TOC
\setcounter{tocdepth}{3} 

% These set the units of the textpos positioning parameters. 
% Now they are set to be multiples of centimeters.
\setlength{\TPHorizModule}{1cm}
\setlength{\TPVertModule}{1cm}

% This command can be used in tables to allow linebreaks within a table
% cell. Takes two parameters: the width of the text area, and the text 
% to print. Unfortunately the width seems to be required; although at
% some situations you can use \textwidth (but probably not inside tables).
\newcommand{\allowlinebreaks}[2]{
% Top align makes sure that there is enough padding/margins at the top
\begin{minipage}[t]{#1}%
#2
% Strut here makes sure that there is enough padding/margins at the bottom
\strut
\end{minipage}
}


% Text options for rendering the cover page 
% ------------------------------------------------------------------
% These get overwritten when a language is specified
\newcommand{\ATUNIV}{}
\newcommand{\ATSCHOOL}{}
\newcommand{\ATDEGREEPROG}{}
\newcommand{\ATTITLE}{}
\newcommand{\ATTITLENAME}{}
\newcommand{\ATSUBTITLE}{}
\newcommand{\ATTHESIS}{}
\newcommand{\ATSUPERVISORNAME}{}
\newcommand{\ATINSTRUCTORNAME}{}
\newcommand{\ATSUPERVISOR}{}
\newcommand{\ATINSTRUCTOR}{}
\newcommand{\ATDATENAME}{}
\newcommand{\ATDATE}{}
\newcommand{\ATCITY}{}
\newcommand{\ATABSTRACTNAME}{}
\newcommand{\ATLOGO}{}
\newcommand{\ATAUTHORNAME}{}
\newcommand{\ATKEYWORDSNAME}{}
\newcommand{\ATKEYWORDS}{}
\newcommand{\ATPAGESNAME}{}
\newcommand{\ATLANGUAGENAME}{}
\newcommand{\ATLANGUAGE}{}
\newcommand{\ATPROFSHIPNAME}{}
\newcommand{\ATPROFCODENAME}{}
\newcommand{\ATPROFSHIP}{}

% Default language terms (in English)
% ------------------------------------------------------------------
% You may change these if you wish
\newcommand{\selectterms}[1]{
\ifthen{\equal{#1}{english}}{
  \renewcommand{\ATTITLENAME}{Title}
  \renewcommand{\ATTITLE}{\TITLE}
  \renewcommand{\ATSUBTITLE}{\SUBTITLE}
  \renewcommand{\ATTHESIS}{Master's Thesis}
  \ifthenelse{\NUMSUPERVISORS = 1}%
    {\renewcommand{\ATSUPERVISORNAME}{Supervisor}}%
    {\renewcommand{\ATSUPERVISORNAME}{Supervisors}}
  \ifthenelse{\NUMINSTRUCTORS = 1}%
    {\renewcommand{\ATINSTRUCTORNAME}{Advisor}}%
    {\renewcommand{\ATINSTRUCTORNAME}{Advisors}}
  \renewcommand{\ATSUPERVISOR}{\SUPERVISOR}
  \renewcommand{\ATINSTRUCTOR}{\INSTRUCTOR}
  \renewcommand{\ATDATENAME}{Date}
  \renewcommand{\ATDATE}{\DATE}
  \renewcommand{\ATCITY}{Espoo}
  \renewcommand{\ATUNIV}{Aalto University}
  \renewcommand{\ATSCHOOL}{School of Science}
  \renewcommand{\ATDEGREEPROG}{Degree Programme in Information Networks} 
  \renewcommand{\ATABSTRACTNAME}{ABSTRACT OF MASTER'S THESIS}
  \renewcommand{\ATLOGO}{aalto-logo-en}
  \renewcommand{\ATAUTHORNAME}{Author}
  \renewcommand{\ATKEYWORDSNAME}{Keywords}
  \renewcommand{\ATKEYWORDS}{\KEYWORDS}
  \renewcommand{\ATPAGESNAME}{Pages}
  \renewcommand{\ATLANGUAGENAME}{Language}
  \renewcommand{\ATLANGUAGE}{\LANGUAGE}
  \renewcommand{\ATPROFSHIPNAME}{Professorship}
  \renewcommand{\ATPROFCODENAME}{Code}
  \renewcommand{\ATPROFSHIP}{\PROFESSORSHIP}
}
% Terms in Finnish
% ------------------------------------------------------------------
% You may change these if you wish
\ifthen{\equal{#1}{finnish}}{
  \renewcommand{\ATTITLENAME}{Ty�n nimi}
  \renewcommand{\ATTITLE}{\FTITLE}
  \renewcommand{\ATSUBTITLE}{\FSUBTITLE}
  \ifthenelse{\NUMSUPERVISORS = 1}%
    {\renewcommand{\ATSUPERVISORNAME}{Valvoja}}%
    {\renewcommand{\ATSUPERVISORNAME}{Valvojat}}
  \ifthenelse{\NUMINSTRUCTORS = 1}%
    {\renewcommand{\ATINSTRUCTORNAME}{Ohjaaja}}%
    {\renewcommand{\ATINSTRUCTORNAME}{Ohjaajat}}
  \renewcommand{\ATSUPERVISOR}{\FSUPERVISOR}
  \renewcommand{\ATINSTRUCTOR}{\FINSTRUCTOR}
  \renewcommand{\ATDATENAME}{P�iv�ys}
  \renewcommand{\ATDATE}{\FDATE}
  \renewcommand{\ATTHESIS}{Diplomity�}
  \renewcommand{\ATCITY}{Espoo}
  \renewcommand{\ATUNIV}{Aalto-yliopisto}
  \renewcommand{\ATSCHOOL}{Perustieteiden korkeakoulu}
  \renewcommand{\ATDEGREEPROG}{Informaatioverkostojen koulutusohjelma} 
  \renewcommand{\ATABSTRACTNAME}{DIPLOMITY�N TIIVISTELM�}
  \renewcommand{\ATLOGO}{aalto-logo-fi}
  \renewcommand{\ATAUTHORNAME}{Tekij�}
  \renewcommand{\ATKEYWORDSNAME}{Asiasanat}
  \renewcommand{\ATKEYWORDS}{\FKEYWORDS}
  \renewcommand{\ATPAGESNAME}{Sivum��r�}  
  \renewcommand{\ATLANGUAGE}{\FLANGUAGE}
  \renewcommand{\ATLANGUAGENAME}{Kieli}
  \renewcommand{\ATPROFSHIPNAME}{Professuuri}
  \renewcommand{\ATPROFCODENAME}{Koodi}
  \renewcommand{\ATPROFSHIP}{\FPROFESSORSHIP}
}
% Terms in Swedish
% ------------------------------------------------------------------
% You may change these if you wish
\ifthen{\equal{#1}{swedish}}{
    \renewcommand{\ATTITLENAME}{Ty�n nimi}
  \renewcommand{\ATTITLE}{\FTITLE}
  \renewcommand{\ATSUBTITLE}{\FSUBTITLE}
  \ifthenelse{\NUMSUPERVISORS = 1}%
    {\renewcommand{\ATSUPERVISORNAME}{Valvoja}}%
    {\renewcommand{\ATSUPERVISORNAME}{Valvojat}}
  \ifthenelse{\NUMINSTRUCTORS = 1}%
    {\renewcommand{\ATINSTRUCTORNAME}{Ohjaaja}}%
    {\renewcommand{\ATINSTRUCTORNAME}{Ohjaajat}}
  \renewcommand{\ATSUPERVISOR}{\FSUPERVISOR}
  \renewcommand{\ATINSTRUCTOR}{\FINSTRUCTOR}
  \renewcommand{\ATDATENAME}{P�iv�ys}
  \renewcommand{\ATDATE}{\FDATE}
  \renewcommand{\ATTHESIS}{Diplomity�}
  \renewcommand{\ATCITY}{Espoo}
  \renewcommand{\ATUNIV}{Aalto-yliopisto}
  \renewcommand{\ATSCHOOL}{Perustieteiden korkeakoulu}
  \renewcommand{\ATDEGREEPROG}{Informaatioverkostojen koulutusohjelma} 
  \renewcommand{\ATABSTRACTNAME}{DIPLOMITY�N TIIVISTELM�}
  \renewcommand{\ATLOGO}{aalto-logo-fi}
  \renewcommand{\ATAUTHORNAME}{Tekij�}
  \renewcommand{\ATKEYWORDSNAME}{Asiasanat}
  \renewcommand{\ATKEYWORDS}{\FKEYWORDS}
  \renewcommand{\ATPAGESNAME}{Sivum��r�}  
  \renewcommand{\ATLANGUAGE}{\FLANGUAGE}
  \renewcommand{\ATLANGUAGENAME}{Kieli}
  \renewcommand{\ATPROFSHIPNAME}{Professuuri}
  \renewcommand{\ATPROFCODENAME}{Koodi}
  \renewcommand{\ATPROFSHIP}{\FPROFESSORSHIP}
}
}


% Command to render the cover page 
% ------------------------------------------------------------------
% Takes one parameter, options: finnish, swedish, and (default): english
% These control in which language the cover-page information is shown
\newcommand{\coverpage}[1]{
\thispagestyle{empty}
\selectlanguage{#1}
\selectterms{#1}
\noindent{\small\ATUNIV\\
\ATSCHOOL\\
\ATDEGREEPROG}
\vskip 30mm

\noindent{\small \AUTHOR}
\vskip 10mm

% The title is printed here. Change the formatting if you like! 
\noindent{\LARGE\textbf{\ATTITLE}}
\vskip 2ex

% The subtitle is printed here. Change the formatting if you like! 
\noindent{\Large\textbf{\ATSUBTITLE}}

\vfill
{\noindent
\small 
\ATTHESIS \\
\ATCITY, \ATDATE
\vskip 2em 

\DRAFT

\small
\noindent
\begin{tabular}{@{}l@{\hspace{3em}}l}
\ATSUPERVISORNAME: & \allowlinebreaks{10cm}{%
\ifthenelse{\isundefined{\COVERSUPERVISOR}}%
{\ATSUPERVISOR}{\COVERSUPERVISOR}}\\
\ATINSTRUCTORNAME: & \allowlinebreaks{10cm}{%
\ifthenelse{\isundefined{\COVERINSTRUCTOR}}%
{\ATINSTRUCTOR}{\COVERINSTRUCTOR}}\\
\end{tabular}
}
\newpage%
}


% Command to render abstracts
% ------------------------------------------------------------------
% Takes two parameters: language and the abstract text
% Language can be either english, finnish, or swedish
% Text and logos are selected accordingly
\newcommand{\thesisabstract}[2]{
\selectlanguage{#1}
\selectterms{#1}
{%
% First block sets the logo. Must be placed manually, because
% the logo itself has huge blank areas around it.
\begin{textblock}{10}(9.07,1)
\raggedleft\noindent\includegraphics[height=2.7cm]{\ATLOGO}
\end{textblock}
% Before the main abstract table there is the school name
% and abstract title (ABSTRACT OF THE MASTER'S THESIS etc.) 
\begin{textblock}{15}(3.2,3)
\setlength{\extrarowheight}{0.5ex}
\noindent
\begin{tabularx}{\textwidth}{@{}X@{}r@{}}
\begin{minipage}[b]{11cm}
\ATUNIV\\
\ATSCHOOL\\
\ATDEGREEPROG
\end{minipage} & 
\begin{minipage}[b]{4cm}
\raggedleft
\noindent\ATABSTRACTNAME
\end{minipage}\\
\end{tabularx}\\
% Now we start the main abstract table
\noindent
\begin{tabularx}{\textwidth}{|p{3cm}Xlp{2cm}|}
\hline
\textbf{\ATAUTHORNAME:} & \multicolumn{3}{l|}{\AUTHOR} \\
\multicolumn{4}{|l|}{\allowlinebreaks{14.5cm}{\textbf{\ATTITLENAME:}\\
\ATTITLE\ \ATSUBTITLE}}\\
\hline
\textbf{\ATDATENAME:} & \ATDATE & \textbf{\ATPAGESNAME:} & \PAGES\\
\hline
\textbf{\ATPROFSHIPNAME:} & \ATPROFSHIP & 
\textbf{\ATPROFCODENAME:} & \PROFCODE\\
\hline
\textbf{\ATSUPERVISORNAME:} &
\multicolumn{3}{l|}{\allowlinebreaks{11cm}{\ATSUPERVISOR}}\\
\textbf{\ATINSTRUCTORNAME:} & 
\multicolumn{3}{l|}{\allowlinebreaks{11cm}{\ATINSTRUCTOR}}\\
\hline
\multicolumn{4}{|l|}{\allowlinebreaks{14.5cm}{%
% The abstract text is here
\setlength{\parskip}{1ex}%
#2%
}}\\
\hline
\textbf{\ATKEYWORDSNAME:} & \multicolumn{3}{p{11cm}|}{\ATKEYWORDS}\\
\hline
\textbf{\ATLANGUAGENAME:} & \multicolumn{3}{l|}{\ATLANGUAGE}\\
\hline
\end{tabularx}
\end{textblock}}
\null\newpage
}


% Document setup
% ------------------------------------------------------------------


% The following command we instruct LaTeX to not 
% complain about underfull hboxes in the bibliography:
\apptocmd{\thebibliography}{\hbadness 10000\relax}{}{}
% Alternatively, we can force the bibliography to be raggedright 
% (not justified), this often works as well:
% \apptocmd{\thebibliography}{\raggedright}{}{}


% The standard teletype font makes URLs look a bit too big. These commands
% here make the URLs a bit smaller, so that they look better in text. 
\makeatletter
\def\url@smallurlstyle{%
  \@ifundefined{selectfont}{\def\UrlFont{\sf}}{\def\UrlFont{\small\ttfamily}}}
\makeatother
%% Now actually use the newly defined style.
\urlstyle{smallurl}