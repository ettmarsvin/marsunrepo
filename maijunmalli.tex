% General information
% ==================================================================
% ------------------------------------------------------------------
% Document class for the thesis is report
% ------------------------------------------------------------------
% You can change this but do so at your own risk - it may break other things.
% Note that the option pdftext is used for pdflatex; there is no
% pdflatex option.
% ------------------------------------------------------------------
\documentclass[12pt,a4paper,oneside,pdftex]{report}
% The input files (tex files) are encoded with the latin-1 encoding
% (ISO-8859-1 works). Change the latin1-option if you use UTF8
% (at some point LaTeX did not work with UTF8, but I'm not sure
% what the current situation is)
\usepackage[utf8]{inputenc}
%\DeclareUnicodeCharacter{2265}{\ensuremath{\ge}}
% OT1 font encoding seems to work better than T1. Check the rendered
% PDF file to see if the fonts are encoded properly as vectors (instead
% of rendered bitmaps). You can do this by zooming very close to any letter
% - if the letter is shown pixelated, you should change this setting
% (try commenting out the entire line, for example).

\usepackage[OT1]{fontenc}
% The babel package provides hyphenating instructions for LaTeX. Give
% the languages you wish to use in your thesis as options to the babel
% package (as shown below). You can remove any language you are not
% going to use.
% Examples of valid language codes: english (or USenglish), british,
% finnish, swedish; and so on.
\usepackage[finnish,swedish,english]{babel}
\usepackage{csquotes}
% Font selection
% ------------------------------------------------------------------
% you can try out these font variants by uncommenting one of the following lines
% Changing the font causes the layouts to shift a bit; you many
% need to manually adjust some layouts. 
% ------------------------------------------------------------------
%\usepackage{palatino}
%\usepackage{tgpagella}

% Changing citation style to biblatex
% ------------------------------------
% Options: natbib allows to use \citep & \citet. 
% isbn removes isbn from ref. list, firstinits truncates first and middle names, 
% maxbibname lists all authors in ref. list, , hyperref: citation marks are hyperlinks
\usepackage[natbib=true, style=authoryear, isbn=false, doi=false, firstinits=true, 
maxbibnames=99, maxcitenames=2, hyperref=true, urldate=long, dateabbrev=true, 
dashed=false, uniquename=init, backend=bibtex]{biblatex}
%, bibencoding=utf8
\bibliography{ref.bib}% ONLY selects .bib file; syntax for version <= 1.1b
\addbibresource{ref.bib}% Syntax for version >= 1.2

% Customized bibliography list:
% ---------------------------
% Order names: "Last, F. M."
\DeclareNameAlias{sortname}{last-first}
% title of these formats is printed as normal text
\DeclareFieldFormat[article, inbook, thesis, incollection, inproceedings, techreport, misc]{title}{#1}

% Change "visited on" to "accessed on" on urls
% \DefineBibliographyStrings{english}{urlseen = {Accessed:}} 

% Customizing tables:
% -------------------
% Longtable provides a tabular environment that can span multiple pages. This is used in the example acronyms file.
\usepackage{longtable}

% For tables: allow left-aligned/centered, multirow and to use \newline or \\ for manual page breaks
\usepackage{array}
\newcolumntype{L}[1]{>{\raggedright\let\newline\\\arraybackslash\hspace{0pt}}m{#1}}
\newcolumntype{C}[1]{>{\centering\let\newline\\\arraybackslash\hspace{0pt}}m{#1}}
% to allow multirow columns in tables
\usepackage{multirow}

% customizing tables
\usepackage{booktabs}
% more space between rows (or 1.3)
\renewcommand{\arraystretch}{1.2} 

% Other:
% -----------------------------------------------------------

% The eurosym package provides a euro symbol. Use with \euro{}
\usepackage{eurosym}

% Loading calcualtion packege
\usepackage{calc}

% Verbatim provides a standard teletype environment that renderes
% the text exactly as written in the tex file. Useful for code
% snippets (although you can also use the listings package to get
% automatic code formatting).
\usepackage{verbatim}

% Subfigure package allows you to use subfigures (i.e. many subfigures within one figure environment). 
% These can have different labels and they are numbered automatically. Check the package documentation.
\usepackage{subfigure}

% Defining a red-color macro for highlighing drafts
\usepackage{color}
\newcommand{\red}[1]{\textcolor{red}{#1}}

% Adding appendix package:
% ------------------------
% appendices environment gives you, is that once the environment ends, you can carry on with sections or chapters as before ? numbering isn't affected by the intervening appendixes. 
\usepackage[titletoc]{appendix}

% code for list of appendices
\newcommand\listappendixname{List of Appendices}
\newcommand\appcaption[1]{%
  \addcontentsline{app}{section}{#1}}
\makeatletter
\newcommand\listofappendices{%
  \section*{\listappendixname}\@starttoc{app}}
\makeatother

% The titlesec package can be used to alter the look of the titles
% of sections, chapters, and so on. This example uses the ``medium''
% package option which sets the titles to a medium size, making them
% a bit smaller than what is the default. You can fine-tune the
% title fonts and sizes by using the package options. See the package
% documentation.
\usepackage[medium]{titlesec}

% Customization chapter headers:
% -----------------------------
% no chapter name in the beginning
\titleformat{\chapter}[hang]{\bfseries\huge}{\thechapter}{2pc}{}{}
% Spacing before and after headings
\titlespacing*{\chapter}{0pt}{12pt}{18pt}{}

% The aalto-thesis package provides typesetting instructions for the
% standard master's thesis parts (abstracts, front page, and so on)
% Load this package second-to-last, just before the hyperref package.
% Options that you can use:
%   mydraft - renders the thesis in draft mode.
%             Do not use for the final version.
%   doublenumbering - [optional] number the first pages of the thesis
%                     with roman numerals (i, ii, iii, ...); and start
%                     arabic numbering (1, 2, 3, ...) only on the
%                     first page of the first chapter
%   twoinstructors  - changes the title of instructors to plural form
%   twosupervisors  - changes the title of supervisors to plural form
%\usepackage[mydraft,twosupervisors]{aalto-thesis}
%\DeclareUnicodeCharacter{00A0}{~}
\usepackage[doublenumbering, twoinstructors]{aalto-thesis}
%\usepackage{aalto-thesis}

% Hyperref
% ------------------------------------------------------------------
% Hyperref creates links from URLs, for references, and creates a TOC in the PDF file.
% This package must be the last one you include, because it has compatibility issues 
% with many other packages and it fixes those issues when it is loaded.
\RequirePackage[pdftex]{hyperref}
% Setup hyperref so that links are clickable but do not look different
\hypersetup{colorlinks=false,raiselinks=false,breaklinks=true}
\hypersetup{pdfborder={0 0 0}}
\hypersetup{bookmarksnumbered=true}
% The following line suggests the PDF reader that it should show the
% first level of bookmarks opened in the hierarchical bookmark view.
\hypersetup{bookmarksopen=true,bookmarksopenlevel=1}
% Hyperref can also set up the PDF metadata fields. These are
% set a bit later on, after the thesis setup.

% Add hyperref working if citing only the year (for ISO standards)
\DeclareCiteCommand{\citeyear}
    {}
    {\bibhyperref{\printdate}}
    {, }
    {}

% Thesis setup
% ==================================================================
% Change these to fit your own thesis.
% \COMMAND always refers to the English version;
% \FCOMMAND refers to the Finnish version
% ------------------------------------------------------------------
% If you do not find the command for a text that is shown in the cover page or
% in the abstract texts, check the aalto-thesis.sty file and locate the text
% from there.
% All the texts are configured in language-specific blocks (lots of commands
% that look like this: \renewcommand{\ATCITY}{Espoo}.
% You can just fix the texts there. Just remember to check all the language
% variants you use (they are all there in the same place).
% ------------------------------------------------------------------
\newcommand{\TITLE}{Improving Project Management \hbox{System} in R\&D with Human-Centered Design}
\newcommand{\FTITLE}{Projektienhallintaj�rjestelm�n parantaminen k�ytt�j�keskeisell� suunnittelulla T\&K-organisaatiossa}
\newcommand{\SUBTITLE}{}
\newcommand{\FSUBTITLE}{}
\newcommand{\DATE}{January 8, 2014}
\newcommand{\FDATE}{8. tammikuuta 2014}

% Supervisors and instructors
% ------------------------------------------------------------------
% If you have two supervisors, write both names here, separate them with a
% double-backslash (see below for an example)
% Also remember to add the package option ``twoinstructors'' to the aalto-thesis package 
% so that the titles are in plural.
\newcommand{\SUPERVISOR}{Prof. Marko Nieminen}
\newcommand{\FSUPERVISOR}{Prof. Marko Nieminen}

% If you have two instructors, separate them with \\ to create linefeeds
\newcommand{\INSTRUCTOR}{Juha Lindfors D.Sc. (Tech.) \\ Mari Rosengren M.Sc. (Tech.)}
\newcommand{\FINSTRUCTOR}{Tekniikan tohtori Juha Lindfors \\ Diplomi-insin��ri Mari Rosengren}

% Other stuff
% ------------------------------------------------------------------
\newcommand{\PROFESSORSHIP}{Usability and user interfaces}
\newcommand{\FPROFESSORSHIP}{K�ytett�vyys ja k�ytt�liittym�t}
% Professorship code is the same in all languages
\newcommand{\PROFCODE}{T-121}
\newcommand{\KEYWORDS}{project management system, human-centered design, R\&D, project manager}
\newcommand{\FKEYWORDS}{projektien hallintaj�rjestelm�, k�ytt�j�l�ht�inen suunnittelu, T\&K, projektip��llikk�}
\newcommand{\LANGUAGE}{English}
\newcommand{\FLANGUAGE}{englanti}

% Author is the same for all languages
\newcommand{\AUTHOR}{Maiju Tompuri}

% Currently the English versions are used for the PDF file metadata
% Set the PDF title
\hypersetup{pdftitle={\TITLE\ \SUBTITLE}}
% Set the PDF author
\hypersetup{pdfauthor={\AUTHOR}}
% Set the PDF keywords
\hypersetup{pdfkeywords={\KEYWORDS}}
% Set the PDF subject
\hypersetup{pdfsubject={Master's Thesis}}

% Layout settings
% ------------------------------------------------------------------

% When you write in English, you should use the standard LaTeX
% paragraph formatting: paragraphs are indented, and there is no
% space between paragraphs.

% Use this to control how much space there is between each line of text.
% 1 is normal (no extra space), 1.3 is about one-half more space, and
% 1.6 is about double line spacing.
% \linespread{1} % This is the default
% \linespread{1.3}

% Extra hyphenation settings
% ------------------------------------------------------------------
% You can list here all the files that are not hyphenated correctly.
% You can provide many \hyphenation commands and/or separate each word
% with a space inside a single command. Put hyphens in the places where
% a word can be hyphenated.
% Note that (by default) LaTeX will not hyphenate words that already
% have a hyphen in them (for example, if you write ``structure-modification
% operation'', the word structure-modification will never be hyphenated).
% You need a special package to hyphenate those words.
\hyphenation{re-searcher stud-ies con-tex-tual}

% Table of contents levels
\setcounter{tocdepth}{2}

% The preamble ends here, and the document begins.
% Place all formatting commands and such before this line.
% ------------------------------------------------------------------
\begin{document}
% This command adds a PDF bookmark to the cover page. You may leave
% it out if you don't like it...
\pdfbookmark[0]{Cover page}{bookmark.0.cover}
% This command is defined in aalto-thesis.sty. It controls the page
% numbering based on whether the doublenumbering option is specified
\startcoverpage

% Cover page
% ------------------------------------------------------------------
% Options: finnish, english, and swedish
% These control in which language the cover-page information is shown
\coverpage{english}

% Abstracts
% ------------------------------------------------------------------
% Include an abstract in the language that the thesis is written in and in Finnish

% Abstract in English
% ------------------------------------------------------------------
\thesisabstract{english}{
Project management software aims to minimize the effort of planning and monitoring projects but these management systems can be difficult to master. 

}

% Abstract in Finnish
% ------------------------------------------------------------------
\thesisabstract{swedish}{

}

% ------------------------------------------------------------------
% Select the language you use in your acknowledgements
\selectlanguage{english}

% Uncomment this line if you wish acknoledgements to appear in the
% table of contents
%\addcontentsline{toc}{chapter}{Acknowledgements}
\chapter*{Acknowledgements}

Thank you! 

\vskip 10mm
\noindent Espoo, January 7, 2014 %\DATE
\vskip 20mm
\noindent\AUTHOR

% Acronyms
% ------------------------------------------------------------------
% Use \cleardoublepage so that IF two-sided printing is used
% (which is not often for masters theses), then the pages will still
% start correctly on the right-hand side.
\cleardoublepage

\input{0_acronyms.tex}

% Table of contents
% ------------------------------------------------------------------
\cleardoublepage
% This command adds a PDF bookmark that links to the contents.
% You can use \addcontentsline{} as well, but that also adds contents
% entry to the table of contents, which is kind of redundant.
% The text ``Contents'' is shown in the PDF bookmark.
\pdfbookmark[0]{Contents}{bookmark.0.contents}
\tableofcontents

% The following label is used for counting the prelude pages
\label{pages-prelude}
\cleardoublepage

%%%%%%%%%%%%%%%%% The main content starts here %%%%%%%%%%%%%%%%%%%%%
% ------------------------------------------------------------------
% This command is defined in aalto-thesis.sty. It controls the page
% numbering based on whether the doublenumbering option is specified
\startfirstchapter

% Add headings to pages (the chapter title is shown)
% Page number is top right, and it is possible to control the rest of the header. No chapter title on header.
\pagestyle{myheadings}

\chapter{Introduction}
\citet{hammer1993reengineering} have summarised aspects of change in organisational environment, beginning from the change in organisational structure; from functional departments to process teams. Work tasks change from simple and detailed tasks to multi-dimensional knowledge work while employees are becoming more autonomous instead of strict control. Furthermore, instead of educating, focus shifts to learning of an employee, and evaluation of work will change from operations to outcomes. Knowledge and capability are preferred over single performance and values change to more productive behaviour than over-protective. Superiors turn from leaders of the work to coaches and hierarchical organisational structures turn lower while managers focus on leadership instead of task management. \citep{hammer1993reengineering} 

Fierce competition for market share and urge for technological innovations have increased the pace of change leading organisations in high pressure to adapt new business environment, rearrange resources, understand and meet new customer and business environment demands. \citep{andriopoulos2000enhancing} Wide access to the information has put tremendous pressure on today's business and companies to increase their efficiency and effectiveness and to develop novel products and processes. Simultaneously, budgets are squeezed and margins of profit grow smaller. \citep{andriopoulos2000enhancing,oldham1996employee} Short time horizons require companies to stay in continuous stream of quarterly profits, oftentimes at the cost of long time benefits. Especially large companies easily favour narrow-minded actions such as quick marketing fixes, cost cutting and acquisition strategies over systemic thinking and process, product or quality innovations. \citep{quinn1985managing}

However, current economy is driven by innovation and innovativeness, requiring new understanding and abilities to generate great ideas \citep{amabile2008creativity} as well as new way of leadership \citep{shalley2004leaders}. Conventional business consists of repetition, risk-avoidance and focusing on business outcomes \citep{buijs2007innovation}, while innovation requires novel solutions, thinking out of the box, risk-taking, breaking the rules, challenging the status quo and questioning the future \citep{burns1961management,kanter1984change,march1991exploration}. 

Understanding change analytically and from systems perspective in the turbulent world appears challenging, with the need of different skills and strategies than before. However, adapting to change and tolerating uncertainty are keys to successful organisation. \citep{senge1990fifth} According to \citet{edmondson1999psychological}, reflection and learning are critical in order to understand the circumstances of increased uncertainty and complexity, pace of change and decreased job security in future organisations. According to \citet{geus1997living} to maintain company's competitive advantage company needs to to learn faster than rivals. Current and future business environment requires continuous learning from organisations, meaning deploying the collective knowledge, skills and creative efforts of their employees \citep{dess2001changing}. 

In addition to company's ability to learn, organisations have begun to value great ideas and demand creative endeavours of employees. Employees who are able to produce those competitive ideas are precious for the current business environment and organisations which strive for innovativeness. \citep{andriopoulos2000enhancing,oldham1996employee} \citet{shalley2004leaders} argues creative employees create competitive advantage in the business field, and various studies recognise creativity influencing on performance and survival of the company across variety of tasks, occupations and industries\citep{hennessey19881,shalley2004leaders,amabile2008creativity}.

Furthermore, studies have well established the positive relation between creativity and innovation skills of an organisation and organisational performance \citep{jung2003role,mumford2002leading}. According to \citet{hennessey19881}, individual creativity stands for an essential building block for organisational innovation \citep{hennessey19881} and is essential in new idea generation and design processes that aim for innovative solutions\citep{sethi2001cross}. The significance of creativity lays in its first step in creating something novel, whereas innovation refers to the implementation phase of the novel ideas in individual, team or organisational level \citep{shalley2004leaders,amabile1996assessing,mumford1988creativity}. 

However, not all jobs require same amount of creativity, yet all organisations benefit from understanding where creativity is required and how it can be fostered and managed \citep{shalley2004leaders}. Likewise, creative actions of an employee are not worthwhile for an organisation when not coordinated or harnessed to yield organisational-level outcomes \citep{jung2003role}. Thus, the future focus should be in organisations' ability to mobilise creative actions of employees to create novel, socially valued products or services and more efficient ways of working \citep{mumford1988creativity}. 

\section{Scope of the study}
Unpredictable, complex an uncertain environments require ability from both organisation and its employees to learn faster than rivals and adapt to changes in creative ways that foster innovation. Need for non-predictive approaches that support learning and growth in organisational and individual level occurs. In this thesis, experimentation-driven approach for development is presented as a method for learning and building competitive advantage in an organisation. Furthermore, factors affecting experimentation behaviour are examined and framework for organisational support for experimenting is created. 

Theoretical part of the thesis forms a synthesis of organisational and individual learning and presents experimenting as a method for developing and learning. Furthermore, thesis presents factors that affect experimentation behaviour in an organisation, forming a picture of optimal environment for experimentation-driven development. When talking about new-value creation, innovation, creativity comes to the topic constantly. Thus, in this study, perspectives of creativity are also presented together with arguments of innovation. 

The thesis was written as a part of a two-year research project called MindExpe studying experimentation-driven innovation at MIND research group, 
Aalto University. MIND operates under the Business, Innovation and Technology (BIT) research centre, which is part of Department of Industrial Engineering 
in Aalto University School of Science. MIND research group is based on Aalto Design Factory. Tekes-funded MINDexpe project studies innovation and development 
in established organisations through experimentation-driven approach. 

The research team instructed a client organisation on using experimentation-driven approach by organising an experimentation challenge where the units of the 
client organisation were tasked to create, develop and report new ideas to develop their work during a six-week time period. Instructions were given before the challenge; during the challenge further instructions were not provided. 

\section{Research objectives}
In MINDexpe client organisations are tasked to use the experimentation-driven approach instead of more traditional planning-based approaches to development. 
This thesis aims to discover how various organisational conditions may affect the experimentation behaviour. The larger aim of the MINDexpe project is to 
widen the understanding of experimentation-driven innovation itself.

The motivation for this study was to reveal factors affecting experimentation behaviour in organisations. In addition the aim was to study
the experimentation-driven approach as a method for learning in organisations and identify how this approach could be supported in an organisation culture. More specifically, research questions that this thesis aims to answer are as follow. 

\begin{enumerate}
  \item How can experimenting support organisational learning?
  \item What kinds of factors affect on experimenting behaviour of an employee? 
  \item How can experimenting behaviour be encouraged in organisations?
\end{enumerate}

The first research question is mainly answered through theoretical research and complemented with empirical findings. Second question is answered through theoretical research and empirical findings. As experimentation-driven development has not been widely studied, important findings from interviews on experimenting in an organisation are gained. To support these empirical findings, literature on creativity and innovation are researched in order to form understanding of the third research question.  

\section{Motivation for the study}
The key motivation for Mind research group is to study how and why some business ideas or businesses work better than others, how new-value can be created and strategic innovations emerged. Mind approaches these broad questions through three agendas. First of all, in order extraordinary innovations to emerge, great ideas are needed. Thus, in the interest of Mind is to find methods and tools for improving the quality of ideas. 

Leader of the research group Anssi Tuulenm�ki states how new value cannot be planned, it needs to be developed through experimenting. Thus, second agenda of Mind is to study experimentation-driven development and its impacts on organisational and individual level. Experimenting is mainly described and used as 
a tool for developing, creating something new. This thesis focuses on this second agenda of Mind group, and deepens the understanding of how experimenting can be used as tool for learning and creates a synthesis on organisational conditions in which experimenting is likely to happen. 

Third agenda of Mind relates to organisational structures and networks, aiming to understand the essence in structures and utilise that to create the most simple organisational structures that support business. 

\section{Methodology}
In this thesis research design is based on case study method. Client organisation Service Foundation for People with Intellectual Disabilities (KVPS) were interested in applying novel approach towards developing in their organisation, and through participating in the MINDexpe research project KVPS experienced experimentation-driven approach in action while MINDexpe benefitted from the real life research context. Case study refers to a method dealing with contemporary phenomenon in real life context and is ideal in studies where boundaries between the phenomenon and its context cannot be clearly delimited. \citep{yin2014case} When studying factors affecting experimentation behaviour in organisations, real life context is significant, yet boundaries of the work and the context of developing remain complex and unclear. 

Researchers of the Mind group organised together with the management of KVPS a six-week experimentation challenge to employees of KVPS in order to generate novel ideas to improve work and test them in action. Researchers gave very brief introduction poster on experimentation and instructions for the challenge, further instructions during the challenge were not provided. 

In the analysis process thematic analysis was used \citep{braun2006using} as a method for analysing the data. The data consists of 14 semi-structured interviews of client organisations members from five different units. The analysis of the data demonstrates various requirements for supporting experimentation behaviour in developing in an organisation. Furthermore, the data revealed that experimentation behaviour has various affects on an individual's performance and the way an employee experiences his work.

Research methodology and study surrounding are introduced further in \ref{reserachdesign}. 

\section{Structure of the thesis}
First chapter briefly introduces background, research objectives and motives for the thesis as well as methodology used. 

This thesis consists of theoretical and empirical part. Chapters 2, 3 and 4 form the theoretical basis for the thesis, and chapter 5 and 6 present the empirical part of the thesis. In current and future organisations in order to create competitive advantage, focus will be on organisations who learn faster than rivals. Additionally, creative ideas of employees has been related to improve competitive advantage for companies. Thus, in the second chapter learning as a tool for an organisation to be better than rivals is presented, together with introduction to creative aspects. 

In the chapter 2, current change in business environment is described in order to form understanding of the need for novel approaches towards development and new-value creation. Through innovations new business and competitive advantage is created, and thus aspects for innovation are outlined. Furthermore, organisational learning and supporting conditions for learning are described. Behind every successful innovation, product or service stands an individual employee or a team with a great idea, so individual and team perspectives on learning are described. Furthermore, in order great ideas to emerge, organisational conditions need to support individual learning, creativity and innovation processes. These aspects are presented in the end of the first chapter. 

Chapter 3 presents experimentation-driven approach for development and learning. Understanding of experimenting and experimentation process is formed, which is the focus at Mind research group. Furthermore, this chapter provides insights on occasions when experimentation-driven approach should be adapted as a way of developing and creating something new. Experimentation-driven developing works best when uncertainty is high and under development is a process with many unfamiliar factors. Experimenting stands as a method for learn on the way of the development process and through iterative experiments and reflections better products, services or ways of working are formed. 

Chapter 4 outlines factors affecting experimentation behaviour in organisations based on literature on innovation, creativity, and organisational organisational management and behaviour. It provides understanding how through organisational conditions creative actions of employees, willingness to conduct experiments and courage to say out ideas can be fostered. 

Chapter 5 presents the research design, including surroundings, case company description and methodology used in the study. It clarifies the experimentation 
challenge organised for the case company, explains data gathering methods and sheds light on the analysis process. 

After this, in chapter 6, the results of the data are presented. Two main concepts were recognised from the data: factors affecting experimentation behaviour and effects 
experimenting has on individual. 

Chapter 7 consists of the discussion, where implications of the results are analysed. Furthermore, both theoretical and managerial
implications of the study are being evaluated as well as suggestions for future research. Lastly, reliability of the thesis is analysed. 

The final chapter of the thesis consists of the brief conclusion drawn from both the theoretical framework, empirical results and discussion. 


\input{2_background.tex}

\input{3_hcd.tex}

\input{4_surroundings.tex}

\input{5_methods.tex}

\input{6_results.tex}

\input{7_discussion.tex}

\input{8_conclusions.tex}

% Load the bibliographic references
% ------------------------------------------------------------------

% Start bibliography from a new page
\clearpage

% removes section header from bibliography
% if you want to change bibliography to, say, references, add [/refname] after {bibliography}
\defbibheading{bibliography}{%
  \section*{#1}%
  \markboth{}{}
}
% to make the reference right in toc, adding phantom section
\phantomsection

% manual page-ref to last page of the content
% now where bibliography starts. If want to where bib graphy ends, move after printbibliography!
\label{pages-end}

\addcontentsline{toc}{chapter}{Bibliography} % Add bibliography to table of contents
\printbibliography

% Start appendices from a new page
\clearpage

% Appendices go here
% ------------------------------------------------------------------
% If you do not have appendices, comment out the following lines
%\appendix

\thispagestyle{myheadings}

% to make the reference right in toc, adding phantom section
\phantomsection

% Adds appendices to toc
\addcontentsline{toc}{chapter}{Appendices} 

\begin{appendices}
%removes listed appendices from toc
%\addtocontents{toc}{\protect\setcounter{tocdepth}{0}}

\chapter*{Appendices}
\thispagestyle{headings}

%\addtocontents{toc}{\protect\setcounter{tocdepth}{2}}
\listofappendices{}
\clearpage

%removes listed appendices from toc
%\addtocontents{toc}{\protect\setcounter{tocdepth}{0}}

\renewcommand\thesection{\Alph{section}}

\section{Interview questions}\appcaption{Appendix A ~ Interview questions}
\label{sec:appA}
The interview questions are presented below.
\bigskip

\noindent\emph{Background}
\vspace{-3mm} 
\begin{list}{*}{}
\setlength{\itemsep}{-3pt}
 \item Work description, and how long has the interviewee been in the position?
\end{list}

\noindent \emph{Know-how}
\vspace{-3mm} 
\begin{list}{*}{}
\setlength{\itemsep}{-3pt}
 \item What experiments did you do during the experimentation challenge?
 \item What idea did you work on further?
 \item How did you progress? What did you do?
 \item Did you develop the idea and conducted an experiment alone or together with colleagues? Did this deviate from your conventional way of working? 
\newline
 \item What did you find easy? (What made it easy?)
 \item What did you find difficult? (What made it difficult?)
 \item Did something surprising or unexpected happen?
 \item How did you act in this situation?
 \item What do you personally consider as critical incidents during the experimentation challenge ? eg. What excited you or discouraged?
 \item Where do you think you succeeded? (Why?)
 \item What made an experiment successful? How do you know that an experiment was successful?
 \item What affected to the success of the experiment? What were the conditions?
 \item What went wrong from your perspective? Where did you consider failing? (Why?) 
 \item What made an experiment unsuccessful? 
 \item What affected or caused an experiment to fail? 
 \item Can you describe some idea that you experimented during the experimentation challenge. 
 \item How would you continue developing this idea?
 \item What would you do this time differently than in the first experiment?
 
\end{list}

\noindent\emph{Supporting structures and practices}
\vspace{-3mm} 
\begin{list}{*}{}
\setlength{\itemsep}{-3pt}
    \item How did the experimentation challenge differ from the conventional way of improving ideas?
    \item Have you developed through experimenting before? Is it part of daily routine?
    \item How did experiments affect normal working day and routines?
    \item To whom did you tell about experiments?
    \item How did you document the experiments?
    \item How do you collect feedback from experiments?
    \medskip
\end{list}

\noindent\emph{Climate}
\vspace{-3mm} 
\begin{list}{*}{}
\setlength{\itemsep}{-3pt}
    \item How was the climate of your work unit during the experimentation challenge?
    \item What affected?
    \item Were everybody equally involved?
    \item Did everybody speak up about their ideas? 
    \item Were there conflicts? What were the effects of conflicts?
    \item What kind of support did you get in experimenting from your organisation/your colleagues? 
    \item What kind of support you would have wished?
    \item Is there some specific thing preventing experimenting generally in your work?
    \item What usually happens after telling out an idea? (Do you get support and encouragement and start acting?)
    \item How failed experiments are dealt with in your team? 
\end{list}

\noindent\emph{Leadership behaviour}
\vspace{-3mm} 
\begin{list}{*}{}
\setlength{\itemsep}{-3pt}
    \item How immediate superiors react on new ideas and experimenting?
    \item Is time allocated for ideating and experimenting in your work?
    \end{list}
    
\noindent\emph{Managing experimentation}
\vspace{-3mm} 
\begin{list}{*}{}
\setlength{\itemsep}{-3pt}
     \item  Do you feel that through experimenting you have more autonomy and you can affect better on your own work? Is experimenting one way to affect your work and improve it? 
    \item During the experimentation challenge, did you get more ideas than usually? How did they emerge?
    
\end{list}

\noindent\emph{Psychological factors}
\vspace{-3mm} 
\begin{list}{*}{}
\setlength{\itemsep}{-3pt}
\item What kinds of emotions rose during experimenting? (Did you for instance feel frustrated, insecure etc.)
\item How did it feel to tell an idea out loud among a team? (Do you get support or was your idea refected?)
\item How do you face a failed or unfinished experiment? (If there were any experiments like that)
\item What did you get from the experience of experimentation challenge?
\item What kind of factors brought good feeling? 
\item What kind of factors brought you down or caused anxiety somehow? 

\item How the amount and quality of feedback differs from when developing through experimenting?
\item How do you consider feedback? (Does it encourage to develop an idea further? Did it bring you down?)
\item  How does experimenting affect your own learning and developing your work? 
\end{list}

\noindent\emph{Wrap-up} 
\vspace{-3mm} 
\begin{list}{*}{}
\setlength{\itemsep}{-3pt}
\item Do you have any questions or comments?
\end{list}
% start next appendix on new page
\clearpage



% manual page-ref to last page of appendices
\label{appendices-end}
\end{appendices}

% End of document!
% ------------------------------------------------------------------
% The LastPage package automatically places a label on the last page.
% That works better than placing a label here manually, because the
% label might not go to the actual last page, if LaTeX needs to place
% floats (that is, figures, tables, and such) to the end of the
% document.
\end{document}