\chapter{Introduction}
\citet{hammer1993reengineering} have summarised aspects of change in organisational environment, beginning from the change in organisational structure; from functional departments to process teams. Work tasks change from simple and detailed tasks to multi-dimensional knowledge work while employees are becoming more autonomous instead of strict control. Furthermore, instead of educating, focus shifts to learning of an employee, and evaluation of work will change from operations to outcomes. Knowledge and capability are preferred over single performance and values change to more productive behaviour than over-protective. Superiors turn from leaders of the work to coaches and hierarchical organisational structures turn lower while managers focus on leadership instead of task management. \citep{hammer1993reengineering} 

Fierce competition for market share and urge for technological innovations have increased the pace of change leading organisations in high pressure to adapt new business environment, rearrange resources, understand and meet new customer and business environment demands. \citep{andriopoulos2000enhancing} Wide access to the information has put tremendous pressure on today's business and companies to increase their efficiency and effectiveness and to develop novel products and processes. Simultaneously, budgets are squeezed and margins of profit grow smaller. \citep{andriopoulos2000enhancing,oldham1996employee} Short time horizons require companies to stay in continuous stream of quarterly profits, oftentimes at the cost of long time benefits. Especially large companies easily favour narrow-minded actions such as quick marketing fixes, cost cutting and acquisition strategies over systemic thinking and process, product or quality innovations. \citep{quinn1985managing}

However, current economy is driven by innovation and innovativeness, requiring new understanding and abilities to generate great ideas \citep{amabile2008creativity} as well as new way of leadership \citep{shalley2004leaders}. Conventional business consists of repetition, risk-avoidance and focusing on business outcomes \citep{buijs2007innovation}, while innovation requires novel solutions, thinking out of the box, risk-taking, breaking the rules, challenging the status quo and questioning the future \citep{burns1961management,kanter1984change,march1991exploration}. 

Understanding change analytically and from systems perspective in the turbulent world appears challenging, with the need of different skills and strategies than before. However, adapting to change and tolerating uncertainty are keys to successful organisation. \citep{senge1990fifth} According to \citet{edmondson1999psychological}, reflection and learning are critical in order to understand the circumstances of increased uncertainty and complexity, pace of change and decreased job security in future organisations. According to \citet{geus1997living} to maintain company's competitive advantage company needs to to learn faster than rivals. Current and future business environment requires continuous learning from organisations, meaning deploying the collective knowledge, skills and creative efforts of their employees \citep{dess2001changing}. 

In addition to company's ability to learn, organisations have begun to value great ideas and demand creative endeavours of employees. Employees who are able to produce those competitive ideas are precious for the current business environment and organisations which strive for innovativeness. \citep{andriopoulos2000enhancing,oldham1996employee} \citet{shalley2004leaders} argues creative employees create competitive advantage in the business field, and various studies recognise creativity influencing on performance and survival of the company across variety of tasks, occupations and industries\citep{hennessey19881,shalley2004leaders,amabile2008creativity}.

Furthermore, studies have well established the positive relation between creativity and innovation skills of an organisation and organisational performance \citep{jung2003role,mumford2002leading}. According to \citet{hennessey19881}, individual creativity stands for an essential building block for organisational innovation \citep{hennessey19881} and is essential in new idea generation and design processes that aim for innovative solutions\citep{sethi2001cross}. The significance of creativity lays in its first step in creating something novel, whereas innovation refers to the implementation phase of the novel ideas in individual, team or organisational level \citep{shalley2004leaders,amabile1996assessing,mumford1988creativity}. 

However, not all jobs require same amount of creativity, yet all organisations benefit from understanding where creativity is required and how it can be fostered and managed \citep{shalley2004leaders}. Likewise, creative actions of an employee are not worthwhile for an organisation when not coordinated or harnessed to yield organisational-level outcomes \citep{jung2003role}. Thus, the future focus should be in organisations' ability to mobilise creative actions of employees to create novel, socially valued products or services and more efficient ways of working \citep{mumford1988creativity}. 

\section{Scope of the study}
Unpredictable, complex an uncertain environments require ability from both organisation and its employees to learn faster than rivals and adapt to changes in creative ways that foster innovation. Need for non-predictive approaches that support learning and growth in organisational and individual level occurs. In this thesis, experimentation-driven approach for development is presented as a method for learning and building competitive advantage in an organisation. Furthermore, factors affecting experimentation behaviour are examined and framework for supporting environment for experimenting is created. 

Theoretical part of the thesis forms a synthesis of organisational and individual learning and presents experimenting as a method for developing and learning. Furthermore, thesis presents factors that affect experimentation behaviour in an organisation, forming a picture of optimal environment for experimentation-driven development. When talking about new-value creation, innovation, creativity comes to the topic constantly. Thus, in this study, perspectives of creativity are also presented together with arguments of innovation. 

The thesis was written as a part of a two-year research project called MindExpe studying experimentation-driven innovation at MIND research group, 
Aalto University. MIND operates under the Business, Innovation and Technology (BIT) research centre, which is part of Department of Industrial Engineering 
in Aalto University School of Science. MIND research group is based on Aalto Design Factory. Tekes-funded MINDexpe project studies innovation and development 
in established organisations through experimentation-driven approach. 

The research team instructed a client organisation on using experimentation-driven approach by organising an experimentation challenge where the units of the 
client organisation were tasked to create, develop and report new ideas to develop their work during a six-week time period. Instructions were given before the challenge; during the challenge further instructions were not provided. 

\section{Research objectives}
In MINDexpe client organisations are tasked to use the experimentation-driven approach instead of more traditional planning-based approaches to development. 
This thesis aims to discover how various organisational conditions may affect the experimentation behaviour. The larger aim of the MINDexpe project is to 
widen the understanding of experimentation-driven innovation itself.

The motivation for this study was to reveal factors affecting experimentation behaviour in organisations. In addition the aim was to study
the experimentation-driven approach as a method for learning in organisations and identify how this approach could be supported in an organisation culture. More specifically, research questions that this thesis aims to answer are as follow. 

\begin{enumerate}
  \item How can experimenting support organisational learning?
  \item What kinds of factors affect on experimenting behaviour of an individual? 
  \item How can experimenting behaviour be supported in an organisation?
\end{enumerate}

The first research question is mainly answered through theoretical research and complemented with empirical findings. Second question is answered through theoretical research and empirical findings. As experimentation-driven development has not been widely studied, important findings from interviews on experimenting in an organisation are gained. To support these empirical findings, literature on creativity and innovation are researched in order to form understanding of the third research question.  

\section{Motivation for the study}
The key motivation for Mind research group is to study how and why some business ideas or businesses work better than others, how new-value can be created and strategic innovations emerged. Mind approaches these broad questions through three agendas. First of all, in order extraordinary innovations to emerge, great ideas are needed. Thus, in the interest of Mind is to find methods and tools for improving the quality of ideas. 

Leader of the research group Anssi Tuulenm�ki states how new value cannot be planned, it needs to be developed through experimenting. Thus, second agenda of Mind is to study experimentation-driven development and its impacts on organisational and individual level. Experimenting is mainly described and used as 
a tool for developing, creating something new. This thesis focuses on this second agenda of Mind group, and deepens the understanding of how experimenting can be used as tool for learning and creates a synthesis on organisational conditions in which experimenting is likely to happen. 

Third agenda of Mind relates to organisational structures and networks, aiming to understand the essence in structures and utilise that to create the most simple organisational structures that support business. 

\section{Methodology}
In this thesis research design is based on case study method. Client organisation Service Foundation for People with Intellectual Disabilities (KVPS) were interested in applying novel approach towards developing in their organisation, and through participating in the MINDexpe research project KVPS experienced experimentation-driven approach in action while MINDexpe benefitted from the real life research context. Case study refers to a method dealing with contemporary phenomenon in real life context and is ideal in studies where boundaries between the phenomenon and its context cannot be clearly delimited. \citep{yin2014case} When studying factors affecting experimentation behaviour in organisations, real life context is significant, yet boundaries of the work and the context of developing remain complex and unclear. 

Researchers of the Mind group organised together with the management of KVPS a six-week experimentation challenge to employees of KVPS in order to generate novel ideas to improve work and test them in action. Researchers gave very brief introduction poster on experimentation and instructions for the challenge, further instructions during the challenge were not provided. 

In the analysis process thematic analysis was used \citep{braun2006using} as a method for analysing the data. The data consists of 14 semi-structured interviews of client organisations members from five different units. The analysis of the data demonstrates various requirements for supporting experimentation behaviour in developing in an organisation. Furthermore, the data revealed that experimentation behaviour has various affects on an individual's performance and the way an employee experiences his work.

Research methodology and study surrounding are introduced further in chapter 5. 

\section{Structure of the thesis}
First chapter briefly introduces background, research objectives and motives for the thesis as well as methodology used. 

This thesis consists of theoretical and empirical part. Chapters 2, 3 and 4 form the theoretical basis for the thesis, and chapter 5 and 6 present the empirical part of the thesis. In current and future organisations in order to create competitive advantage, focus will be on organisations who learn faster than rivals. Additionally, creative ideas of employees has been related to improve competitive advantage for companies. Thus, in the second chapter learning as a tool for an organisation to be better than rivals is presented, together with introduction to creative aspects. 

In the chapter 2, current change in business environment is described in order to form understanding of the need for novel approaches towards development and new-value creation. Through innovations new business and competitive advantage is created, and thus aspects for innovation are outlined. Furthermore, organisational learning and supporting conditions for learning are described. Behind every successful innovation, product or service stands an individual employee or a team with a great idea, so individual and team perspectives on learning are described. Furthermore, in order great ideas to emerge, organisational conditions need to support individual learning, creativity and innovation processes. These aspects are presented in the end of the first chapter. 

Chapter 3 presents experimentation-driven approach for development and learning. Understanding of experimenting and experimentation process is formed, which is the focus at Mind research group. Furthermore, this chapter provides insights on occasions when experimentation-driven approach should be adapted as a way of developing and creating something new. Experimentation-driven developing works best when uncertainty is high and under development is a process with many unfamiliar factors. Experimenting stands as a method for learn on the way of the development process and through iterative experiments and reflections better products, services or ways of working are formed. 

Chapter 4 outlines factors affecting experimentation behaviour in organisations based on literature on innovation, creativity, and organisational organisational management and behaviour. It provides understanding how through organisational conditions creative actions of employees, willingness to conduct experiments and courage to say out ideas can be fostered. 

Chapter 5 presents the research design, including surroundings, case company description and methodology used in the study. It clarifies the experimentation 
challenge organised for the case company, explains data gathering methods and sheds light on the analysis process. 

After this, in chapter 6, the results of the data are presented. Two main concepts were recognised from the data: factors affecting experimentation behaviour and effects 
experimenting has on individual. 

Chapter 7 consists of the discussion, where implications of the results are analysed. Furthermore, both theoretical and managerial
implications of the study are being evaluated as well as suggestions for future research. Lastly, reliability of the thesis is analysed. 

The final chapter of the thesis consists of the brief conclusion drawn from both the theoretical framework, empirical results and discussion. 
