\chapter{Introduction}
This chapter is an introduction for the thesis. It first outlines the background for the study, then presents research objectives and motivation for the study. In the last section, structure of the thesis is presented. 

\section{Background}
Fierce competition for market share and urge for technological innovations have increased the pace of change leading organisations in high pressure to adapt to new business environment, rearrange resources, understand and meet new customer and business environment demands. \citep{andriopoulos2000enhancing} Wide access to the information has put tremendous pressure on today's business and companies to increase their efficiency and effectiveness and to develop novel products and processes. Simultaneously, budgets are squeezed and margins of profit grow smaller. \citep{andriopoulos2000enhancing,oldham1996employee} Short time horizons require companies to stay in continuous stream of quarterly profits, oftentimes at the cost of long time benefits. Especially large companies easily favour narrow-minded actions such as quick marketing fixes, cost cutting and acquisition strategies over systemic thinking and process, product or quality innovations. \citep{quinn1985managing}

Current economy is, however, driven by innovation and innovativeness, requiring new understanding and abilities to generate great ideas \citep{amabile2008creativity}. Conventional business consists of repetition, avoiding risks and focusing on business outcomes \citep{buijs2007innovation}, whereas innovation requires novel solutions, thinking out of the box, risk-taking, breaking the rules, challenging the status quo and questioning the future \citep{burns1961management,kanter1984change,march1991exploration}. Employees who are able to produce competitive ideas are precious for organisations striving for innovativeness \citep{andriopoulos2000enhancing,oldham1996employee}. Various studies recognise creativity influencing on performance and survival of the company across variety of tasks, occupations and industries \citep{hennessey19881,amabile2008creativity,jung2003role,mumford2002leading} and \citet{shalley2004leaders} argues creative employees create competitive advantage in the business field. According to \citet{hennessey19881}, individual creativity stands for an essential building block for organisational innovation \citep{hennessey19881} and is essential in new idea generation and design processes that aim for innovative solutions \citep{sethi2001cross}. The significance of creativity lays in its first step in creating something novel, whereas innovation refers to the implementation phase of the novel ideas in individual, team or organisational level \citep{shalley2004leaders,amabile1996assessing,mumford1988creativity}. 

Understanding change analytically and from systems perspective in the turbulent world appears challenging, with the need of different skills and strategies than before. However, adapting to change and tolerating uncertainty are keys to successful organisation. \citep{senge1990fifth} According to \citet{edmondson1999psychological}, reflection and learning are critical in order to understand the circumstances of increased uncertainty and complexity, pace of change and decreased job security in future organisations. According to \citet{geus1997living} to maintain company's competitive advantage company needs to to learn faster than rivals. Current and future business environment requires continuous learning from organisations, meaning deploying the collective knowledge, skills and creative efforts of their employees \citep{dess2001changing}. 

When dealing in this unpredictable, complex and uncertain environments, traditional ways for developing and innovation are not efficient, tend to take lot of resources and are too specification-driven, where specifications of the product or serviced are locked in the beginning of the project. Need for non-predictive approaches for development that support learning and growth in organisational and individual level occurs. \citep{thomke1998managing,tuulenmaki2011art}

Concurrently current business is remarkably dependent on services, yet innovation techniques and processes still focuses on products. Systematic learning methods are needed in order to avoid occasional successes and provide more stable base for consistency and productivity of service development. \citep{thomke2003r} 

In recent years the centre of innovation discussion in management and business literature have shed light on the concept of early, rough and iterative experimentation process models on innovation \citep{thomke1998managing,tuulenmaki2011art}.  In this thesis, experimentation-driven approach for development is presented as a method for learning and building competitive advantage in an organisation. It refers to an iterative trial-and-error process, where final product is developed through test and feedback -loop. Through experimenting essential factors concerning the final product are revealed before too much resources are spent, and through iterations success can be reached both earlier and faster. In addition to product development, experimentation can be applied to service design and development. \citep{thomke2003r} 

So far little research has been made about factors that support experimentation from organisational level. However, various studies consider behaviours such as learning, creativity, information seeking and other interpersonally risky yet organisationally favourable behaviours as predictors for experimentation behaviour \citep{lee2004mixed,amabile1996assessing,argyris1994good,edmondson1996learning,edmondson2003speaking}. For instance, \citet{amabile1996assessing} found relation between creativity and organisational culture, reward system, encouragement from leaders, trust and resources. Likewise, feedback, asking for help and information as well as solution-oriented behaviour can all be supported through organisational norms, open leadership and shared trust\citep{ashford1992conveying,ashford1998out,lee1997going,morrison1993newcomer}.

\section{Research objectives}
The objective for this study was to reveal factors affecting experimentation behaviour in organisations and shed the light on how experimenting affects an individual. In addition the aim was to study the experimentation-driven approach as a method for learning in organisations and identify how this approach could be supported in organisations. 

Research questions this thesis aims to answer are presented below. 

\begin{enumerate}
 \item What kinds of factors affect on experimenting behaviour of an employee? 
 \item How experimenting affects an individual? 
  \item How can experimenting behaviour be supported in organisations?
  \item How can experimentation support organisational learning?
\end{enumerate}

First research question is answered through theoretical research and empirical findings. When talking about new-value creation and innovation, creativity comes to the topic constantly. Thus, in this study, perspectives of creativity are also presented together with arguments of innovation. As experimentation-driven development has not been widely studied, important findings from interviews on experimenting in an organisation are gained.

The objective for the second research question is to shed light on the experience of experimenting in order to better understand how experimentation could be better supported in organisations. Experimenting seems to affect an individual differently than planning-based developing, and requires different skills, attitude and motivation. This question is to study how experimenting actually affects an individual. 

Third research question aims to reveal factors for organisations to support experimenting behaviour of employees. This is the most practical research question of the thesis aiming to provide clear guidelines for organisations, based on theoretical and empirical findings of the study. 

As organisational learning and organisations capabilities to quickly adapt to changing business environments and customer needs are essential in current and future organisations, need for new tools and methods to support learning are required. Last research question studies whether experimenting can be seen as such a tool and how experimenting could help organisational learning.

\section{Motivation for the study}
The motivation for this study raises from author's enthusiasm to learn how to support employees to become more autonomous, find excitement in their own work and assist in learning and acquiring novel perspectives. As employees know their work and customer interface oftentimes a lot better than top-management, novel approaches are needed to make the best out of employees professionalism in their own work. Author finds remarkable value in the experimentation approach, and through this study she was able to learn more about its effects on employees as well as how to support experimentation in organisations. 

The thesis was written as a part of a two-year research project called MINDexpe studying experimentation-driven innovation at MIND research group, Aalto University. MIND operates under the Business, Innovation and Technology (BIT) research centre, which is a part of Department of Industrial Engineering in Aalto University School of Science. MIND research group is based on Aalto Design Factory. Tekes-funded MINDexpe project studies innovation and development 
in established organisations through experimentation-driven approach. In MINDexpe client organisations are tasked to use the experimentation-driven approach instead of more traditional planning-based approaches to development. The larger aim of the MINDexpe project is to widen the understanding of experimentation-driven innovation itself.

The key motivation for Mind research group is to study how and why some business ideas or businesses work better than others, how new-value can be created and strategic innovations emerged. Mind approaches these broad questions through three agendas. First of all, in order extraordinary innovations to emerge, great ideas are needed. Thus, in the interest of MIND is to find methods and tools for improving the quality of ideas. 

Leader of the research group Anssi Tuulenm�ki states how new value cannot be planned, it needs to be developed through experimenting. Thus, second agenda of MIND is to study experimentation-driven development and its impacts on organisational and individual level. Experimenting is mainly described and used as a tool for developing, creating something new. This thesis focuses on this second agenda of the MIND group, and deepens the understanding of how experimenting can be used as tool for learning and creates a synthesis on organisational conditions in which experimenting is likely to happen. 

Third agenda of MIND relates to organisational structures and networks, aiming to understand the essence in structures and utilise that to create the most simple organisational structures supporting business. 

\section{Structure of the thesis}
First chapter briefly introduces background, research objectives and motives for the thesis. 

Chapters \ref{learning}, \ref{expe} and \ref{faetheory} form the theoretical basis for the thesis, and chapter \ref{researchdesign} with chapter \ref{results} present the empirical part of the thesis. 

In current and future organisations in order to create competitive advantage, focus will be on organisations who learn faster than rivals. Additionally, creative ideas of employees has been related to improve competitive advantage for companies. Thus, in the chapter \ref{learning} organisational learning is presented, together with introduction to creative aspects. Behind every successful innovation, product or service stands an individual employee or a team with a great idea, so individual and team perspectives on learning are described, as well as conditions that support it.

Chapter \ref{expe} presents experimentation-driven approach for development. Understanding of experimenting and experimentation process is formed, which is the focus of MIND research group. Furthermore, this chapter provides insights on occasions when experimentation-driven approach should be adapted as a way of developing and creating something new. Experimentation-driven developing works best when uncertainty is high and under development is a process with many unfamiliar factors. Experimenting stands as a method to learn on the way of the development process; through iterative experiments and reflection better products, services and ways of working are formed. 

Chapter \ref{faetheory} outlines factors affecting experimentation behaviour in organisations based on literature on innovation, creativity, and organisational management and behaviour. It provides understanding how through organisational conditions creative actions of employees, willingness to conduct experiments and courage to say out ideas can be fostered. 

Chapter \ref{researchdesign} presents the research design, including surroundings, case company description and methodology used in the study. It clarifies the experimentation challenge organised for the case company, explains data gathering methods and sheds light on the data analysis process. 

In chapter \ref{results}, the results of the data are presented. Two main classes were recognised from the data: factors affecting experimentation behaviour and how experimenting affects an individual. 

Chapter \ref{conclusions} consists of the discussion, where implications of the results are analysed. Furthermore, practical implications and suggestions for future research are presented. In addition, reliability of the thesis is analysed. 