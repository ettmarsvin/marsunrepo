\chapter{Introduction}
This thesis combines and reflects knowledge based on organisational learning and behaviour, introduces process of experimentation-driven development. It suggests experimentation driven approach as a method for learning and developing, and outlines factors affecting experimentation behaviour in organisations together with optimal environment for developing. 

Theoretical part of the thesis forms a synthesis of organisational and individual learning and presents experimenting as a method for developing and learning. Furthermore, thesis presents factors that affect experimentation behaviour in an organisation, forming a picture of optimal environment for experimentation-driven development. 

Additionally, when talking about new-value creation, innovation, creativity comes to the topic constantly. Thus, in this study, perspectives of creativity are also presented together with arguments of innovation. 

The thesis was written as a part of a two-year research project called MindExpe studying experimentation-driven innovation at MIND research group, 
Aalto University. MIND operates under the Business, Innovation and Technology (BIT) research centre, which is part of Department of Industrial Engineering 
in Aalto University School of Science. MIND research group is based on Aalto Design Factory. Tekes-funded MINDexpe project studies innovation and development 
in established organisations through experimentation-driven approach. 

The research team instructed a client organisation on using experimentation-driven approach by organising an experimentation challenge where the units of the 
client organisation were tasked to create, develop and report new ideas during a six-week time period. Instructions were given before the challenge; during 
the challenge further instructions were not given. 

In MINDexpe client organisations are tasked to use the experimentation-driven approach instead of more traditional planning-based approaches to development. 
This thesis aims to discover how various organisational conditions may affect the experimentation behaviour. The larger aim of the MINDexpe project is to 
widen the understanding of experimentation-driven innovation itself.

To study the factors affecting experimentation behaviour, an interpretative approach was used with thematic analysis as a method for analysing the data. 
The data consists of 14 semi-structured interviews of client organisations members from five different units. The analysis of the data demonstrates various 
requirements for supporting experimentation behaviour in developing in an organisation. Furthermore, the data revealed that experimentation behaviour has various 
affects on an individual's performance and the way an employee experiences his work. 

\section{Background}
Samin dipasta
"In a marketplace that is constantly changing it is practically impossible to know beforehand which innovations will become successes, and it is not necessarily always the most economically efficient solutions that win (Arthur, 1989; Alvarez and Barney, 2007; Dew, Sarasvathy, Read, and Wiltbank, 2009). This presents a challenge also to strategic management, which is fundamentally concerned with how firms can achieve and sustain competitive advantage (Teece, Pisano, and Shuen, 1997). Simply put, such a thing does not exist in an environment that is continuously being created and shaped by the numerous systems interacting with each other.
The key motivation for Mind research group is to study how and why some business ideas or businesses work better than others, how new-value can be created and strategic innovations emerged. Mind approaches these broad questions through three agendas. First of all, in order extraordinary innovations to emerge, great ideas are needed. Thus, in the interest of Mind is to find methods and tools for improving the quality of ideas. "

Why innovation and creativity should be a matter for an organisation and concerns experimenting? Studies have well established the positive relation between creativity and innovation skills of an organisation and organisational performance \citep{jung2003role,mumford2002leading}. According to \citet{hennessey19881}, individual creativity stands for an essential building block for organisational innovation and also \citet{sethi2001cross} argue creativity being essential in new idea generation and design processes that aim for innovative solutions. Other studies also emphasise the role of creativity as first step in creating something novel, whereas innovation refers to the implementation phase of the novel ideas in individual, team or organisational level \citep{shalley2004leaders,amabile1996assessing,mumford1988creativity}. As experimenting is an essential part of innovation, and experimenting requires creativity of an employee, in this thesis innovation and creativity literature is referred in order to form synthesis on factors affecting experimentation behaviour. 

Leader of the research group Anssi Tuulenm�ki states how new value cannot be planned, it needs to be developed through experimenting. Thus, second agenda of Mind is to study experimentation-driven development and its impacts on organisational and individual level. Experimenting is mainly described and used as 
a tool for developing, creating something new. This thesis focuses on this second agenda of Mind. 

Third agenda of Mind relates to organisational structures and networks, aiming to understand the essence in structures and use that to create the most simple organisational structures that support organisations learning and business. 

In this thesis experimentation-driven approach is presented as one possible approach to development and learning and this thesis aims to deepen the understanding of the requirements for an organisational unit to take experimentation driven approach to developing instead of planning. 

A combination of case study and action research was used 

\section{Research objectives}
In the essence of the motivation for this study was to reveal factors affecting experimentation behaviour in organisations. In addition the aim was to establish 
the experimentation-driven approach and identify how this approach could be supported in an organisation culture. 

This thesis aims to define how organisations can foster experimentation culture. 

Research objectives are described as below. 

How can experimenting support organisational learning? 
What kinds of factors affect on experimenting behaviour of an individual? 
How can experimenting behaviour be supported in an organisation? 

\section{Methodology}

\section{Structure of the thesis}
First chapter briefly introduces background, research objectives and motives for the thesis as well as methodology used. 

This thesis consists of theoretical and empirical part. Chapters 2, 3 and 4 form the theoretical basis for the thesis, and chapter 5 and 6 present the empirical part of the thesis. In current and future organisations in order to create competitive advantage, focus will be on organisations who learn faster than rivals.Thus, in the second chapter learning as a tool for an organisation to be better than rivals is presented. In the chapter 2, current change in business environment is described in order to form understanding of the need for novel approaches towards development and new-value creation. Through innovations new business and competitive advantage is created, and thus aspects for innovation are outlined. Furthermore, organisational learning and supporting conditions for learning are described. Behind every successful innovation, product or service stands an individual employee or a team with a great idea, so individual and team perspectives on learning are described. Furthermore, in order great ideas to emerge, organisational conditions need to support individual learning, creativity and innovation processes. These aspects are presented in the end of the first chapter. 

Chapter 3 presents experimentation-driven approach for development and learning. Understanding of experimenting and experimentation process is formed, which is the focus at Mind research group. Furthermore, this chapter provides insights on occasions when experimentation-driven approach should be adapted as a way of developing and creating something new. Experimentation-driven developing works best when uncertainty is high and under development is a process with many unfamiliar factors. Experimenting stands as a method for learn on the way of the development process and through iterative experiments and reflections better products, services or ways of working are formed. 

Chapter 4 outlines factors affecting experimentation behaviour in organisations. It provides understanding how through organisational conditions creative actions of employees, willingness to conduct experiments and courage to say out ideas can be fostered. 

Chapter 5 presents the research design, including surroundings, case company description and methodology used in the study. It clarifies the experimentation 
challenge organised for the case company, explains data gathering methods and sheds light on the analysis process. 

After this, in chapter 6, the results of the data are presented. Two main concepts were recognised from the data: factors affecting experimentation behaviour and effects 
experimenting has on individual. 

Chapter 7 consists of the discussion, where implications of the results are analysed. Furthermore, both theoretical and managerial
implications of the study are being evaluated as well as suggestions for future research. Lastly, reliability of the thesis is analysed. 

The final chapter of the thesis consists of the brief conclusion drawn from both the theoretical framework, empirical data results and discussion. 
