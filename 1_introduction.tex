\chapter{Introduction}
This thesis aims to combine and reflect the knowledge based on.. It suggests experimentation driven innovation as a method for efficient change management. 

The thesis was written as a part of a two-year research project called MindExpe studying experimentation-driven innovation at MIND research group, 
Aalto University. MIND operates under the Business, Innovation and Technology (BIT) research centre, which is part of Department of Industrial Engineering 
in Aalto University School of Science. MIND research group is based on Aalto Design Factory. Tekes-funded MINDexpe project studies innovation and development 
in established organizations through experimentation-driven approach. 

The research team instructed a client organization on using experimentation-driven approach by organizing an experimentation challenge where the units of the 
client organization were tasked to create, develop and report new ideas during a six-week time period. Instructions were given before the challenge; during 
the challenge further instructions were not given. 

In MINDexpe client organizations are tasked to use the experimentation-driven approach instead of more traditional planning-based approaches to development. 
This thesis aims to discover how various organizational conditions may affect the experimentation behavior. The larger aim of the MINDexpe project is to 
widen the understanding of experimentation-driven innovation itself.

To study the factors affecting experimentation behavior, an interpretative approach was used with thematic analysis as a method for analyzing the data. 
The data consists of 14 semi-structured interviews of client organizations members from five different units. The analysis of the data demonstrates various 
requirements for supporting experimentation behavior in developing in an organization. Furthermore, the data revealed that experimentation behavior has various 
affects on an individual?s performance and personality and emotional level. 

\section{Background}
Mind research group focuses on studying experimentation and its impacts on organisational and individual level. Experimenting is mainly described and used as 
a tool for developing, creating something new. 

In this thesis experimentation-driven approach is presented as one possible approach to development and this thesis aims to deepen the understanding of the 
requirements for an organizational unit to take experimentation driven approach to developing instead of planning. 

A combination of case study and action research was used 



\section{Research objectives}
- About organisations competitiveness, how nowadays it is lot about how fast organisation learns and adapts new things. Systems thinking? 
- About learning and organisational well-being in organisational, team and individual level

Furthermore, the motivation for this study was to find factors that affect on experimenting in supportive or preventing way. In addition the aim was to establish 
the experimentation-driven approach and identify how this approach could be supported in an organization culture. 

This thesis aims to define what can be done in organisation in order to allow experimentation culture to happen. 

Research objectives: 

What is organisational learning? 
How can organisational learning be supported? 
Can experimentation behaviour be seen as a method for organisational learning? 
How can experimenting support organisational learning? 
How an organisation can support individual learning and growth? 

\section{Methodology}

\section{Structure of the thesis}
This thesis consists of theoretical and empirical part. After the introduction, the theoretical part of the thesis begins. First of all, it forms an understanding 
of what can be understood by experimenting and experimentation-driven innovation, which is the focus of study at Mind research group. Secondly, in the heart of 
every change and development project there are the employees, the group of individuals who are touched by the change. In order to proceed to meaningful and efficient 
changes for both the company and its employees, individuals has to be onboard.  One reason for resisting change is the threat that an individual encounters, which 
goes to very primitive parts of the brain. Thus, the second part, chapter x, of the theoretical framework presents the threat-reward response in brain, and binds 
that together with our everyday life, especially work life. 

Additionally, when talking about innovation, creativity comes to the topic constantly. Thus, in this study, perspectives of creativity are also presented together 
with arguments of innovation. 

Chapter y presents the research design, including surroundings,  case company description and methodology used in the study.   It clarifies the experimentation 
challenge organised for the case company, explains data gathering methods and sheds light on the analysis process done. 

After this, in chapter r, the results of the data are presented. Two main concepts were recognised from the data: factors affecting experimentation behaviour and effects 
experimenting has on individual. These results are further reflected to the theory framework provided in chapter xx (Scarf). Following chapter consists of the discussion, 
where implications of the results are analysed. 

The final chapter of the thesis consists of the brief conclusion drawn from both the theoretical framework, empirical data results and discussion. Furthermore, practical 
implications of the study are being evaluated as well as suggestions for future research. Lastly, reliability of the thesis is being analysed. 
