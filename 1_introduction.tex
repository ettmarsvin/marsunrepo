\chapter{Introduction}
This chapter is an introduction for the thesis. It first outlines the background for the study, then presents research objectives and motivation for the study. In the last section, structure of the thesis is presented. 

\section{Background}
%tiivist� t�t� ja tuo experimentation n�kyville aiemmin ja paremmin!
Fierce competition for market share and urge for technological innovations have increased the pace of change leading organisations in high pressure to adapt new business environment, rearrange resources, understand and meet new customer and business environment demands. \citep{andriopoulos2000enhancing} Wide access to the information has put tremendous pressure on today's business and companies to increase their efficiency and effectiveness and to develop novel products and processes. Simultaneously, budgets are squeezed and margins of profit grow smaller. \citep{andriopoulos2000enhancing,oldham1996employee} Short time horizons require companies to stay in continuous stream of quarterly profits, oftentimes at the cost of long time benefits. Especially large companies easily favour narrow-minded actions such as quick marketing fixes, cost cutting and acquisition strategies over systemic thinking and process, product or quality innovations. \citep{quinn1985managing}

Current economy is driven by innovation and innovativeness, requiring new understanding and abilities to generate great ideas \citep{amabile2008creativity}. Role of employees is changing to more autonomous members performing multi-dimensional knowledge work rather than simple and detailed tasks under strict control \citep{hammer1993reengineering}. This requires new way of leadership \citep{shalley2004leaders}, and instead of leading the work, role of the leaders is turning to more coaching-oriented \citep{hammer1993reengineering}.

While organisational structures are changing from functional departments into process teams \citep{hammer1993reengineering}, 

Conventional business consists of repetition, risk-avoidance and focusing on business outcomes \citep{buijs2007innovation}, while innovation requires novel solutions, thinking out of the box, risk-taking, breaking the rules, challenging the status quo and questioning the future \citep{burns1961management,kanter1984change,march1991exploration}. Employees who are able to produce those competitive ideas are precious for the current business environment and organisations which strive for innovativeness. \citep{andriopoulos2000enhancing,oldham1996employee} \citet{shalley2004leaders} argues creative employees create competitive advantage in the business field, and various studies recognise creativity influencing on performance and survival of the company across variety of tasks, occupations and industries\citep{hennessey19881,shalley2004leaders,amabile2008creativity}.

Understanding change analytically and from systems perspective in the turbulent world appears challenging, with the need of different skills and strategies than before. However, adapting to change and tolerating uncertainty are keys to successful organisation. \citep{senge1990fifth} According to \citet{edmondson1999psychological}, reflection and learning are critical in order to understand the circumstances of increased uncertainty and complexity, pace of change and decreased job security in future organisations. According to \citet{geus1997living} to maintain company's competitive advantage company needs to to learn faster than rivals. Current and future business environment requires continuous learning from organisations, meaning deploying the collective knowledge, skills and creative efforts of their employees \citep{dess2001changing}. 

Furthermore, studies have well established the positive relation between creativity and innovation skills of an organisation and organisational performance \citep{jung2003role,mumford2002leading}. According to \citet{hennessey19881}, individual creativity stands for an essential building block for organisational innovation \citep{hennessey19881} and is essential in new idea generation and design processes that aim for innovative solutions\citep{sethi2001cross}. The significance of creativity lays in its first step in creating something novel, whereas innovation refers to the implementation phase of the novel ideas in individual, team or organisational level \citep{shalley2004leaders,amabile1996assessing,mumford1988creativity}. 

However, not all jobs require same amount of creativity, yet all organisations benefit from understanding where creativity is required and how it can be fostered and managed \citep{shalley2004leaders}. Likewise, creative actions of an employee are not worthwhile for an organisation when not coordinated or harnessed to yield organisational-level outcomes \citep{jung2003role}. Thus, the future focus should be in organisations' ability to mobilise creative actions of employees to create novel, socially valued products or services and more efficient ways of working \citep{mumford1988creativity}. 

%t�h�n expe-sidos
Unpredictable, complex an uncertain environments require ability from both organisation and its employees to learn faster than rivals and adapt to changes in creative ways that foster innovation. Need for non-predictive approaches that support learning and growth in organisational and individual level occurs. In this thesis, experimentation-driven approach for development is presented as a method for learning and building competitive advantage in an organisation. Furthermore, factors affecting experimentation behaviour are examined and framework for organisational support for experimenting is created. 

When dealing in unpredictable, complex and uncertain environments, traditional ways for developing and innovation are not efficient, tend to take lot of resources and are too specification-driven, where specifications of the product or serviced are locked in the beginning of the project. In recent years the centre of innovation discussion in management and business literature have shed light on concept of early, rough and iterative experimentation process models on innovation \citep{thomke1998managing,tuulenmaki2011art}. 

Through experimenting essential factors concerning the final product are revealed before too much resources have been spent, and through failure success can be reached both earlier and faster.
\section{Research objectives}
The objective for this study was to reveal factors affecting experimentation behaviour in organisations and shed the light on how experimenting affects an individual. In addition the aim was to study the experimentation-driven approach as a method for learning in organisations and identify how this approach could be supported in an organisation. Research questions this thesis aims to answer are presented below. 

\begin{enumerate}
 \item What kinds of factors affect on experimenting behaviour of an employee? 
 \item How experimenting affects an individual? 
  \item How can experimenting behaviour be supported in organisations?
  \item How can experimentation support organisational learning?
\end{enumerate}

%puhu t�ss� osiossa vain sun ty�n tavoitteesta! HUOM uusi research question! 
Theoretical part reviews organisational and individual learning and presents experimenting as a method for developing and learning. Furthermore, thesis presents factors that affect experimentation behaviour in an organisation, forming a picture of optimal environment for experimentation-driven development. When talking about new-value creation, innovation, creativity comes to the topic constantly. Thus, in this study, perspectives of creativity are also presented together with arguments of innovation. 

First research question is answered through theoretical research and empirical findings. As experimentation-driven development has not been widely studied, important findings from interviews on experimenting in an organisation are gained. To support these empirical findings, literature on creativity, innovation and organisational behaviour are researched in order to form understanding of the second research question.The third research question is mainly answered through theoretical research and complemented with empirical findings. 

\section{Motivation for the study}
While the author has always been driven by enthusiasm, interested in creativity and challenging the status quo, the opportunity to work in the MIND research group studying strategic innovations and experimentation-driven development could not have been better fit for her. During the writing and research process the author learned about innovation and organisational strategies, change management, organisational behaviour, learning, creativity and even neuropsychology. Comes without saying the hardest part of this study was to narrow the focus sufficiently. 

The motivation for this study raises from author's enthusiasm to learn how to support employees to become more autonomous and excited about their own work. As employees know their work and customer interface oftentimes a lot better than top-management, novel approaches are needed to make the best out of employees professionalism in their own work. Author finds remarkable value in the experimentation approach, and through this study she was able to learn more about its effects on employees as well as how to support experimentation in organisations. 

The thesis was written as a part of a two-year research project called MINDexpe studying experimentation-driven innovation at MIND research group, Aalto University. MIND operates under the Business, Innovation and Technology (BIT) research centre, which is part of Department of Industrial Engineering in Aalto University School of Science. MIND research group is based on Aalto Design Factory. Tekes-funded MINDexpe project studies innovation and development 
in established organisations through experimentation-driven approach. In MINDexpe client organisations are tasked to use the experimentation-driven approach instead of more traditional planning-based approaches to development. The larger aim of the MINDexpe project is to widen the understanding of experimentation-driven innovation itself.

The key motivation for Mind research group is to study how and why some business ideas or businesses work better than others, how new-value can be created and strategic innovations emerged. Mind approaches these broad questions through three agendas. First of all, in order extraordinary innovations to emerge, great ideas are needed. Thus, in the interest of Mind is to find methods and tools for improving the quality of ideas. 

Leader of the research group Anssi Tuulenm�ki states how new value cannot be planned, it needs to be developed through experimenting. Thus, second agenda of Mind is to study experimentation-driven development and its impacts on organisational and individual level. Experimenting is mainly described and used as a tool for developing, creating something new. This thesis focuses on this second agenda of Mind group, and deepens the understanding of how experimenting can be used as tool for learning and creates a synthesis on organisational conditions in which experimenting is likely to happen. 

Third agenda of Mind relates to organisational structures and networks, aiming to understand the essence in structures and utilise that to create the most simple organisational structures that support business. 

\section{Structure of the thesis}
First chapter briefly introduces background, research objectives and motives for the thesis. 

This thesis consists of theoretical and empirical part. Chapters 2, 3 and 4 form the theoretical basis for the thesis, and chapter 5 and 6 present the empirical part of the thesis. In current and future organisations in order to create competitive advantage, focus will be on organisations who learn faster than rivals. Additionally, creative ideas of employees has been related to improve competitive advantage for companies. Thus, in the second chapter learning as a tool for an organisation to be better than rivals is presented, together with introduction to creative aspects. 

In the chapter 2, organisational learning and supporting conditions for learning are described. Behind every successful innovation, product or service stands an individual employee or a team with a great idea, so individual and team perspectives on learning are described. Furthermore, in order great ideas to emerge, organisational conditions need to support individual learning, creativity and innovation processes. These aspects are included. 

Chapter 3 presents experimentation-driven approach for development and learning. Understanding of experimenting and experimentation process is formed, which is the focus of MIND research group. Furthermore, this chapter provides insights on occasions when experimentation-driven approach should be adapted as a way of developing and creating something new. Experimentation-driven developing works best when uncertainty is high and under development is a process with many unfamiliar factors. Experimenting stands as a method for learn on the way of the development process and through iterative experiments and reflections better products, services or ways of working are formed. 

Chapter 4 outlines factors affecting experimentation behaviour in organisations based on literature on innovation, creativity, and organisational management and behaviour. It provides understanding how through organisational conditions creative actions of employees, willingness to conduct experiments and courage to say out ideas can be fostered. 

Chapter 5 presents the research design, including surroundings, case company description and methodology used in the study. It clarifies the experimentation challenge organised for the case company, explains data gathering methods and sheds light on the data analysis process. 

In chapter 6, the results of the data are presented. Two main concepts were recognised from the data: factors affecting experimentation behaviour and effects experimenting has on individual. 

Chapter 7 consists of the discussion, where implications of the results are analysed. Furthermore, both theoretical and managerialimplications of the study are being evaluated as well as suggestions for future research. Lastly, reliability of the thesis is analysed. 

The final chapter of the thesis consists of the brief summary of the thesis.  
