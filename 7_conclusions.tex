\chapter{Conclusions}
Final part of the thesis concludes the findings and consists of x chapters. The results presented in previous part are further discussed in the first chapter. In addition, the theoretical framework presented in the theoretical part is supplemented with the findings of the empirical part. The research questions are revisited in order to determine how they were answered in the thesis. Both theoretical and managerial implications of the thesis are discussed in the second chapter. In the final chapter future research topics are discussed and the reliability of the thesis is evaluated.

\section{Reliability of the thesis}
Lincoln and cuba, l\"ahde taijan dipasta
Common criteria for evaluating the quality of a quantitative study are reliability and validity, where reliability refers to repeatability and validity to accuracy in means of measurement. In this thesis, however, quantitative approach is not used and in order to assess the reliability of the thesis, Lincoln and Guba?s (1985) approach on reliability is used. According to this approach, reliability is assessed through trustworthiness, which consists of four aspects: credibility, transferability, dependability and confirmability. 

Credibility refers to the interpretations made of the original data and their credibility (Lincoln and Guba, pp. 301-316). In this thesis? 

In addition, direct quotations are used in this thesis in order to reveal the data behind the interpretations. Co-researchers and professors have also been discussing about the interpretations thus adding credibility. 

Transferability means the possibilities to transfer the results and findings to another context. In the thesis?

Dependability refers to the consistency of the research process. Throughout the thesis the research design and process is described clearly. The research questions are presented in the beginning of the thesis and further revisited in the conclusions, and the results are evaluated through the research questions. 
Theory building process followed the principles of chosen research method, case study. 

Confirmability refers to objectivity and neutrality of the thesis. The writer of the thesis has never been working on studied industry field and was not involved in the empirical case other than in a role of interviewer and observer. In the data analysis process other researchers were involved and the results were discussed at least among three different researchers. In the theoretical research? 
