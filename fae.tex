\chapter{Factors affecting experimentation}
Even though novel experimentation techniques are useful and may reduce the cost and time used for product development, adopting experimentation techniques may require changes in organisational culture or way of doing work \citep{thomke1998modes}. However, adapting changes may lead to increase in productivity and affect the overall competitive positioning among companies. 

This chapter provides consensus of factors that affect experimentation behaviour. Various factors have been recognised to foster or hinder creativity and experimentation behaviour of employees, and through understanding the factors, experimenting can be supported. Creativity and learning skills of an individual have been related to willingness and ability to conduct experiments, thus this chapter outlines from organisational behaviour, leadership, creativity and innovation as well as experimentation research. However, experimentation as a method for learning and developing has not yet been widely studied. Thus, in this chapter, theories from research fields such as organisational behaviour, leadership and management, creativity, innovation and prototyping are combined. Overall, employees should be encouraged to ask questions, seek for help and tolerate mistakes and uncertainty \citep{edmondson1999psychological}.

So far little research has been made about factors that support experimentation from organisational level. However, various studies attach behaviours such as learning, creativity, information seeking and other interpersonally risky yet organisationally favourable behaviours as predictors for experimentation behaviour \citep{lee2004mixed,amabile1996,argyris1994good,edmondson1996learning,edmondson2003speaking}. For instance, \citet{amabile1996assessing} found relation between creativity and organisational culture, reward system, encouragement from leaders, trust and resources. Likewise, feedback, asking for help and information as well as solution-oriented behaviour can all be supported through organisational norms, open leadership and shared trust \citep{ashford1992conveying,ashford1998out,lee1997going,morrison1993newcomer}. Supportive organisational context consists of access to sufficient amount of resources, information, training, rewards and management coaching \citep{edmondson1996learning}. Routine work should encourage information exchange, allow feedback and through trustworthy culture decrease defensiveness \citep{argyris1994good}. Proactive learning behaviour is related to all above mentioned \citep{edmondson2003speaking}

The chapter is divided into three sections. First of all, supporting environment for experimenting is presented, outlining how psychological safety, tolerating risk and learning from failure are all essential for experimentation behaviour. Second section outlines team perspective towards development, including low hierarchy, clear and fair communication and team engagement. Current trend in research shows leaders and their behaviour have great influence on the creativity and innovation ability of employees (eg. \citep{mumford2002leading,jung2001transformational,amabile1998kill}) as well as willingness to conduct experiments. Thus, the last section presents how leader can support employees in experimentation. 

\section{Supporting environment for experimenting}
Experimenting requires safe and supportive environment. According to \citet{edmondson1999psychological} team psychological safety should be the first essential building block of learning behaviour in work teams. Supporting learning environment consists of four characteristics: psychological safety, appreciation of differences, openness to new ideas and time for reflection \citep{garvin2008yours}. Likewise, \citet{mumford1988creativity} emphasise the meaning of environmental variables as a means to support employee's creativity by providing resources to stimulate fresh ideas of employees. Furthermore, strong positive relations between organisational environmental variables have been found; organisational encouragement as well as support for innovation and creativity from team improve employee's creativity \citep{amabile1996assessing}

Experimenting requires creative actions and willingness to solve problems, think out of the box and challenge status quo. According to \citet{mumford1988creativity} through environmental variables employee's creativity can be fostered. New solutions may be achieved through problem solving and challenging the routine ways of thinking, and environment should be designed to encourage and facilitate these skills. Environmental factors may affect on employee's intrinsic motivation and willingness to generate novel ideas, when social and physical environments work as a source for support and resources in idea generation and implementation \citep{amabile1998kill,mumford1988creativity}. \citet{mumford1988creativity} have studied the gap between an idea an action, and revealed it depending on various attributes related to individual and organisational circumstances.

Essentially, interaction between leaders and employees, team members and outside members, sufficient resources, employees clear expectations on their evaluation and rewards and environment which is perceived fair all have affect on employees behaviour at work. Through these variables, employees may feel they are working in supporting working environment, and foster creativity and ability find novel approaches and try out new things. \citep{shalley2004leaders}

\subsection{Psychological safety}
However, in some organisational environments people do ask help, admit errors and discuss about problems and are willing to experiment novel approaches to problems. In these environments employees seem to perceive interpersonal threat low enough to perform despite of the threat. Some studies argue how familiarity among group members is likely to encourage openness towards new information and ideas \citep{sanna1990valence}, yet this factor alone is not sufficient to explain when group members find it safe to act instead of feeling threatened \citep{edmondson1999psychological}. \citet{edmondson1999psychological} has studied working environments and realised in environments employees act despite the threat, they feel safe and supported for their actions. She refers to this as psychological safety, which serves as a mechanism that assists in explaining how structural and interpersonal characteristics both have effects on learning and performance in teams \citep{edmondson1999psychological}.  

Team psychological safety consists of team members shared belief team being safe for interpersonal risk-taking, coming up with new ideas and breaking the status-quo. Integrative perspective suggests that team performance and outcomes can be shaped through both team structure and shared beliefs, in contrast to previous studies that separate structural and interpersonal factors from each other. For instance, employee's willingness to take interpersonal risks depends highly on the experience of team safety and person's beliefs how others will respond in ideas or situations involving uncertainty. \citep{edmondson1999psychological}

Trust, indeed, has been widely noted in research as essential factor in organisational teams and groups \citep{golembiewski1975centrality,kramer1999trust,shalley2004leaders,edmondson1999psychological}. Trust refers to one's willingness to be vulnerable in his actions as he expects his actions will not be judged and will be favourable to one's interests \citep{robinson1997corporate}. Interpersonal trust is involved in psychological safety, yet it also includes perception of mutual respect and overall climate where team members feel free to be themselves \citep{edmondson1999psychological}. 

Psychological safety can be fostered for instance through structural factors such as context support and team leader coaching \citep{hackman1987design,edmondson1999psychological}. Context support refers for instance to access to information and resources needed. Safe environment that fosters creativity also takes into account employees' perceptions of just and transparent decision-making as well as applied actions \citep{shalley2004leaders}. When knowing that well-intentioned interpersonal risks are not punished is a shared belief of a team, team members are more likely to take proactive actions essential for experimenting \citep{garvin2008yours}.Climate of safety and supportiveness encourages employees to seek for feedback and ask for help in addition to admit and reflect mistakes. \citep{edmondson1999psychological} In psychologically safe environment uncertainty is not totally avoided but managed and tolerated \citep{shalley2004leaders}. 

"For example, differences across organi- zations in psychological safety have been shown to affect the level of anxiety people feel when confronting ambiguity and uncertainty (Schein 1985). Organizational differences in psychological safety can be created by supportive structures such as information and reward systems (Edmondson and Mogelof forthcoming) and by the words and actions of high-level management;"

Accordingly, \citet{amabile1998kill} suggested that creative thinking can be encouraged by shaping organisational culture such that employees feel encouraged to tell their ideas out loud freely and without judging, increasing idea exchange and discussion about them. In addition, study of \citet{oldham1996employee} show how creative individuals may only produce more creative outputs than less creative individuals when the context is supporting and encouraging towards creativity.  As physical work environment affects on creativity, information sharing and innovation in an organisation, it should be designed to support the natural flow of traffic through the building so that informal conversations between different functional areas are enabled \citep{shalley2004leaders}. 

All human resource practices should be in line and systematically linked together in order to create a clear picture for employees of what is expected of them. Perceived fairness and sense of loyalty towards a company of an employee is higher when employees understand what is expected of them, how, when and what for they are being rewarded, promoted or fired. These same attitudes are essential for fostering creativity. Committed and loyal employees are more likely to exceed what is required of them, be more motivated and committed to working towards specific goal and find novel approaches in order to succeed. \citep{shalley2004leaders} 

According to \citet{lee2004mixed} in order to foster innovation, organisational conditions should be regarded from a broad, holistic perspective. Inconsistency among organisational conditions stands a as threat for employee's behaviour. For instance, willingness to conduct experiments reduces when some part of the organisation or managers encourage experimentation and others do not. In their study \citet{lee2004mixed} found that in uncertain and unpredictable situations, employees under high evaluative pressure are likely to become uncertain, rigid and narrowly focused, leading decrease in psychological safety. In turn, employees under less evaluative pressure turned out more tolerant for taking risks, thinking optimistically and working more proactively. 

Also \citet{lee2004mixed} argues experimentation behaviour being essential for innovation and failures being inevitable in the process. Thus, conditions supporting psychological safety are likely to reduce fear of failure and encourage experimentation. Furthermore, inconsistent organisational conditions tend to prevent experimentation behaviour. According to their study \citet{lee2004mixed} individuals under low evaluative pressure were more likely to experiment in inconsistent organisational conditions (such as normative values and instrumental rewards) in contrast to those who were under high evaluative pressure. The latter group were not that likely to tolerate uncertainty but to become anxious and feel decrease in psychological safety and were less willing to conduct experiments. 

\subsection{Tolerating risk and uncertainty}
When dealing with novel solutions and challenging status quo, we are dealing with innovations. In order for company and its employees to be innovative, they need to take risks. Yet, at the same time usual management processes avoid risk-taking and focus on managing daily routine business. As \citet{quinn1985managing} stated it in his Harvard Business Review article: we love innovation and we urge for innovation, but we can tolerate it only if it is controllable and results everything remaining the same. \citep{quinn1985managing}

Individuals are likely to avoid risks and uncertainty, stick to routine and prefer more certain outcomes and ways of performing \citep{bazerman2012judgment}. This does not encourage, however, creative actions that actually require several trial-and-error, iterative, experimentation processes. Employees fearing risk-taking tend to perform the routine way instead of taking chance with new approaches. \citep{shalley2004leaders} However, in his study \citet{nystrom1990organizational} found that organisational culture reflecting challenge and risk taking lead to more innovative actions of employees and the whole organisation. 

The phenomenon of threat and embarrassment of employees in organisations is widely studied and consensus is rising how threat effects on cognitive and behavioural flexibility and responsibility in reducing manner. \citep{argyris1982reasoning,edmondson1999psychological,staw1989tradeoff} An employee is likely to inhibit trying novel approaches as a result of feeling the fear of being rejected, under pressure or feeling they are placing themselves at risk \citep{edmondson1999psychological}. This is likely even though their transparency and honesty would be highly important for the behaviour of the team \citep{argyris1982reasoning, edmondson1999psychological}. This may occur in a situation where an employee should ask for help, yet is afraid of appearing incompetent as or giving unfavourable impressions on people who have the power to give promotions, raises or who assigns projects. \citep{edmondson1999psychological,brown1990politeness}. 

Admitting mistakes, asking for help and seeking feedback are all relevant abilities for experimenting, yet threatening for an individual's image of himself and his skills \citep{brown1990politeness}. Especially in psychologically unsafe environments, interpersonal costs of failure may easily be exaggerated: employees are afraid how their need for help or gaps in knowledge become salient to colleagues or managers \citep{lee1997going}. Ambiguity is often perceived by individuals when lacking sufficient cues to structure a situation, and usually arises from novelty, complexity or unsolvability of situation at hand,\citep{budner1962intolerance}. These are all characteristics of experimentation. 

As predicting the future is impossible, focus should be in managing risks involved in playing with creative ideas in both the company and individual level. Uncertainty should not be considered only as a threat or inconvenience occurring in organisations, rather should appropriate level of messiness let exist, and develop opportunities where uncertainty can be exploited. Overall, uncertainty in creative processes should not be overly controlled. \citep{sternberg1997creativity} Level of uncertainty can be reduced for instance through goal-setting and fast prototyping \citep{mumford2002leading}. \citet{kanter1983change} continues that, actually, opportunities grow from uncertainty and creative endeavours rise when struggling with uncertainty and mess, as individuals impose order where it does not exist, and thus individuals are forced to form new connections. 

Feeling of self-efficacy may affect individual's willingness to provide unique and novel ideas even when some degree of risk is involved \citep{mumford1988creativity}. In creative work risk concerns both the need to do experiments and tolerate failure \citep{andriopoulos2000enhancing,quinn1985managing}, as failing is widely considered as essential part of learning \citep{farson2002failuretolerantleader}. Thus, employees should feel being allowed to conduct experiments and despite the outcome of the experimentation \citep{jung2003role}.

According to \citet{andriopoulos2000enhancing} facing and dealing with risk serves also as positive boost to creativity, as employees' learn new skills and strengthen their capabilities constantly and adapt new knowledge to already known. This, however, requires for safe environment which \citet{andriopoulos2000enhancing} refers as safety net: environment that tolerates failure. Bank of America also got to notice how staff turnover in experimentation laboratories dropped considerably during the experimenting period. Even though employees faced difficulties and had to tolerate risks and uncertainties, they felt engaged and enthusiastic during the test period.  \citep{thomke2003r}

\subsection{Failures are opportunities for growth}
Failing early and often is important for experimenting. According to various studies, failing and negative consequences are natural part of creative, innovation and learning processes \citep{hennessey19881,shalley2004leaders,andriopoulos2000enhancing}. For instance, \citet{hennessey19881} emphasise that process is likely to have negative consequences, and in the concept of perpetual challenging of \citet{andriopoulos2000enhancing}, adventuring phase includes making mistakes. 

\citet{garvin2008yours} divide organisational failure into three categories: unsuccessful trials, system break-downs and process deviations. In this thesis, unsuccessful trials refer to experimenting. In order to learn and develop, these all types of failures need to be recognised and their special characteristics analysed. For instance, experimenting may foster creative learning, but as important is to overcome failures resulting from deeply ingrained norms that inhibit experimenting. \citep{garvin2008yours}

Failure can disclose important information and reveal gaps in knowledge, and is thus important in as early phase of the development process as possible. \citep{buijs2007innovation,thomke2001enlightened} Also \citet{sitkin1992learning} emphasises how failures facilitate innovation and performance through new knowledge, which narrows the scope of following experiments. According to \citet{thomke2001enlightened}, this is not an usual way for an organisation to think about failure, thus building the capacity for rapid experimentation as well as tolerating and learning from failure is essential and often requires overcoming ingrained attitudes. However, according to \citet{farson2002failuretolerantleader} failure as a part of innovation and development process begins to be generally recognised and approved. Succeeding companies even thrive for failure in order to learn fast and find the best practices and business models. For instance, credit company Capital One conducts continually large amount of market experiments. They know most of the tests will not pay off, yet they also know how much can be learned about customers and markets from failed tests in early phase of development. \citep{farson2002failuretolerantleader}

Especially when brought to innovation perspective, innovation processes always include mistakes and failure, from which organisations, teams and leaders need to learn, the faster the better. Organisation that learns fastest is likely to take the lead, which is the essence of innovation according to \citet{buijs2007innovation}. Furthermore, according to \citet{thomke2001enlightened}, mistakes and failures produce most value, when the experiment is well planned and the goal or hypothesis that needs to be tested is clear. Thus, failed experiments should not be considered as failing, instead they offer valuable opportunities for growth, while of course issues about safety and health of people participating experiments need to be taken into account \citep{farson2002failuretolerantleader}. 

According to \citet{thomke2003} in any experimentation, the most radical experiments provides the greatest learning. At the same time, the most radical experiments tend to have the highest level of risk and thus high probability to fail. When brought to real life, employees tend to fear alienating customers, damaging the bottom line or alienating top management who have the power to prevent further development. While these risks are real, they have to be weighted carefully and remember how important benefits and insights experiments and failures can provide. In the example of Bank of America, in only few years had they gained essential benefits that had real impact on business. \citep{thomke2003}

According to \citet{amabile2008creativity} constant experimenting should be the goal of working, learning by doing and iterating until sufficiently is learnt from the process. Furthermore, when company grows, it usually leads to more conservative actions and increase in fear of failure. When fearing failure managers tend to deny failure and erase it from the memory instead of learning from it. \citep{amabile2008creativity}. In their study \citet{lee2004mixed} found support for the notion that individuals are more likely to conduct experiments when being rewarded, when rewards did not penalise for failures. 

Also \citet{edmondson1999psychological} relates experimenting tightly to failing, and emphasises team learning, creative problem solving, reflection and overall organisational performance rising from the failure that occurs. However, willingness to interpersonal risk-taking is tightly related to how employees perceive and believe team members or leaders would react and response in uncertain actions or ideas \citep{farson2002failuretolerantleader}. When failures are being punished through reward systems, the cost of experimentation increases and makes employees less willing to conduct experiments \citep{thomke2001enlightened}. Failing as a personal matter remains a difficult subject, as failing never feels exceptionally great, and often employees still consider failed work as failing personally \citep{farson2002failuretolerantleader}. Studies show how nominal groups perform remarkably better in ideation and brainstorming processes by producing greater amount of ideas than real groups \citep{jung2001transformational,sosik1998transformational}. This may be due to the learnt practices and norms of a real work group, fear of failure that prevents free idea exchange and fear of evaluation and others judgement when suggesting creative solutions. Overall, oftentimes it is way more difficult to take a different role and actions in group with familiar members and routines. \citep{jung2001transformational}

Thus, team's tolerance for imperfection and error should be increased \citep{edmondson1999psychological}. According to \citet{thomke2001enlightened} this can be done for instance through brainstorming sessions where judgement is not allowed. Furthermore, employees are more likely to contribute ideas when failing is considered safe. Leaders should emphasise that constant experimenting requires failing early and often and through these iterations learning is possible. Equally important is that employees should feel not being punished nor humiliated for negatives outcomes, if mistakes occur or whatever ideas are spoken up. \citep{amabile2008creativity, amabile1996assessing} Also \citet{de2001minority} found a positive relation between employee's creativity and participative safety, referring to employee's perception of generating ideas without being judged. Organisational environment that allows failing is likely to assist in employees acquiring diverse perspectives and questioning the status quo and habitual way of performing. \citep{amabile1996assessing}

\citet{garvin2008yours} states by creating an environment that serves psychological safety for employees, organisations may capitalise on failure. Safe environment does not humiliate or punish employees for failing or coming up with novel ideas or doubts. \citet{garvin2008yours} Also according to \citet{edmondson1996learning} employees are less hesitant to discuss mistakes when normative values of the organisation and work group assure that failures are allowed and even expected part of learning. 

\section{Organisational structures and practices}

\subsection{Team perspective}
In order to create new value and competitive advantage in rapidly changing and uncertain organisational environments, new managerial imperative is growing focusing on teams. Supporting teams in their work and understanding the aspects of learning is also required in experimenting \citep{edmondson1999psychological}. 

As presented in chapter "Organizational and team learning", teams that are able to learn are better at solving problems, confronting challenging situations, observe changes in environment and customer requirements. In the theory of\citet{edmondson1999psychological} of team learning, factors essential for learning are similar to essential factors for experimenting. These include transparent information sharing, asking for help, receiving and giving feedback, tolerating failures and discussing about them in order to reflect experiments and improve work. \citep{edmondson1999psychological}

As individuals oftentimes requires support and input from several individuals who help to challenge ideas in constructive ways, teams are essential in generating and implementing ideas \citep{mumford2002social}. Stimulating those constructive individuals for creative actions may be valuable \citep{robinson1997corporate}. In addition, including team members in ideation assists in idea implementation and through participation new ideas are not that likely to be rejected or abandoned \citep{agrell1994team}. Through brainstorming activities focus on non-traditional thinking and fantasising intellectual skills of employees can be enhanced \citep{sosik1998transformational}.

The composition of the team matters. Studies have shown how team performance, especially related to innovation, is improved when team consists of individuals with various and different set of skills and characteristics\citep{buijs2007innovation}. Homogeneity in teams easily leads to groupthink, routine work and repeating traditional daily practices, while even one or two different individuals can stimulate the innovativeness of a team.Actually, the outcasts and those who stand out from the group are required in order to think outside the box, challenge the status quo and present alternative solutions and ideas that would be missing without the participations of these individuals. \citep{sternberg1997creativity} However, according to \citet{quinn1985managing} especially larger companies tend to hold tight on their conventional opinion how enthusiastic employees who challenge the status quo are likely to cause embarrassment and troubles for organisation.

In addition, in order to function team needs a clear purpose and vision what makes it a team and why it exists. Teams get energy from significant performance challenges regardless of where they are in the organisation. Set of shared, demanding performance goals usually form a team, and personal chemistry or willingness to form a team may boost that. Thus, in order to receive great results teams should focus on performance regardless of the organisational hierarchy or what team does. Team performance may exceed the results of what could be achieved if employees were acting alone as individuals without the team effort. \citep{katzenbach1993wisdom}

The whole team understanding the meaning of experimenting and developing forms a basis for team engagement . Experimenting and sharing ideas and information in as early stage of development process as possible and throughout the process remain essential. \citep{thomke2001enlightened} \citet{thomke2001enlightened} argues how small project teams together with parallel experimenting serves efficient especially when time is the most critical factor. In addition, the amount of experiments at the same time and place should be considered. Simultaneous experimenting keeps the learning speed high, whereas sequential experimenting is likely to delay the overall process. However, too many experiments conducted at the same place may increase the amount of surrounding noise and affect the results of experiments. Thus, capacity and amount of experiments need to be managed. \citep{thomke2003r}

Organizational and team structures and hierarchies affect on innovation and experimenting. Relationship between formal reporting and responsibility levels are essential: highly bureaucratic organisation do not tend to encourage employees to reach for novel approaches and experiments, whereas organisation with flatter structure may enhance employees' autonomy and creativity. Employees may be likely to perceive presentations of organisations structure and hierarchy as discouraging and only highlighting how employees are not allowed or encouraged to make decisions on their own. This leads to less enthusiasm for trying out new ways of working and developing. In addition, heavy bureaucracy demanding lot of time and effort from employees to get novel ideas forward in the organisation is likely to destroy the enthusiasm of employees. \citep{shalley2004leaders} According to \citet{thomke2003r} experimenting requires openness for constant changes in practices and processes.

\subsection{Communication of ideas}
Although goals are important for successful planning of experiments they should, however, be kept broad, in order not to create undue oppositions to new ideas. Flexibility should be maintained by not defining intermediate steps in detail and by trying alternate options and routes. Identifying and solving problems at early phase fosters momentum, confidence and identity towards novel approaches. Furthermore, sufficient amount of information about the project and progress should be offered in order managers to follow and realise the work performed. \citep{quinn1985managing}

Furthermore, small teams tend to handle communication and commitment among team members better, while as few management layers as possible decreases the jeopardy of rejection \citep{quinn1985managing}. According to \citet{quinn1985managing} "Since it takes a chain of yeses and only one no to kill a project, jeopardy multiplies as management layers increase."By promoting open communication, idea and ongoing information exchange with internal and external team members as well as encouraging information seeking from different perspectives and sources is likely to enhance creativity \citep{ancona1992demography,dougherty1996sustained}. 

Study of \citet{sethi2001cross} showed how good interaction in a team and high level of commitment to the success of the team lead to more radical innovation abilities. In the study team members were highly encouraged to take risks, which lead to more motivated members in suggesting novel ideas from their perspectives. In addition, team members identified themselves strongly as part of the team, which again lead to higher commitment level. \citep{sethi2001cross}

Communication about ideas among team has been widely recognized being related to idea generation, creativity and innovation (e.g.\citep{robinson1997corporate,mumford2002social,monge1992communication,amabile1996assessing}). According to \citet{staw1989tradeoff} social influence of others plays a major role for individuals' beliefs; attitudes towards job, for instance, rise from the social labelling of work by others.  Also \citet{salancik1978social} argue the essential role opinions of others may have on individual: individual's perception of her work and organisation can be greatly influenced by opinions of others. Additionally, team member's collective view of support they get from their leader has been related to the team's creative endeavours and success in them (e.g., \citep{amabile1998kill,amabile1996assessing}). 
  
\citet{andriopoulos2000enhancing} define in their study a concept of perpetual challenging - a way to enhance creativity and innovation in an organisation. According to the concept adventuring occurs when goal is idea generation and through that process individuals are encouraged to face uncertainty in order to generate novel solutions. One tool for idea generation is scenario making, which purpose is to develop possible ways to tackle situation at hand. Through scenario making employees scans what is both known and not known about current problem or situation.  Experimenting process then consists of testing different scenarios generated in ideation phase and evaluating the outcomes in order to decide and develop the scenarios further to meet the needs of clients and industry. This calls for individuals skills to tolerate risks and uncertainties, as well as skills to constructively challenge and question colleagues ideas in order to use their full potential. \citet{andriopoulos2000enhancing}

\section{Leadership behaviour}
According to studies leaders have a strong direct impact on employee's behaviour and way of performing at workplace \citep{katz1978social,redmond1993putting}. Being a key person in organising group work and processes a leader may encourage employees to achieve shared goals \citep{amabile1998kill}. \citet{avolio1988transformational} list several mechanisms through which leaders can affect employees behaviour. These include role modelling, goal definition, reward allocation, resource distribution, defining norms and values of the company, showing the way to interact as a group, condition employees' perceptions of work environment and being the lead decision maker on organisational structure and procedures.

 Studies also suggest leaders have a significant effect on employees' creativity and ability to innovate \citep{mumford2002leading,jung2001transformational,amabile1998kill,hennessey19881}, and according to \citet{redmond1993putting} a leader can have an affect on employee's level of creativity through leadership behaviours such as problem construction, learning goals and feelings of self-efficacy. As leaders play a major role in establishing, influencing and shaping organisational culture and climate through their communicated values and beliefs, they are able to shape the organisational culture into more innovative direction and foster creativity and experimentation in an organisation \citep{jung2003role,schein2010organizational} for instance by nurturing organisational climate that supports creative efforts and learning \citep{yukl2002leadership}. 

Following subsections describe in more detail how leaders can affect experimentation behaviour of employees by creating safe and supporting environment, acting as a role model and allowing sufficient resources for experimentation. 

\subsection{Creating safe and supporting environment}
Employees are more likely to conduct experiments in a psychologically safe environment described in chapter "psychological safety" and leaders have a great role in creating this safe and supporting environment for experimenting. Top executives have significant role in encouraging creative and innovative endeavours and creating an atmosphere and value system that supports innovation \citep{quinn1985managing}. While individual characteristics affect on the creativity of an individual, an environment fostering creativity is likely to assist in producing novel ideas during the routine work of employees \citep{amabile1996assessing}. In addition, engaging employees to creative activities is likely to lead better and more qualified decisions \citep{shalley2004leaders}. Thus, leaders should create an environment where idea generation is possible \citep{andrews1970social}.

As described in chapter "fear of failure", employees are not likely to take interpersonal risks essential for learning if a leader acts in authoritarian or punitive ways, and responses to questions or challenges in defensive manner \citep{edmondson1999psychological}. In order to decrease the fear of failure and risk-taking, \citet{amabile2008creativity} suggest leaders should put emphasis on creating an environment where an employee feels safe to fail. Furthermore, speaking out loud ideas or making mistakes should not result in punishment or humiliation, and leaders should act in supporting and coach-oriented manner\citep{edmondson1999psychological}.

Leaders often fail in showing their employees the support and tools for failing fast and early enough, even though through encouraging risk-taking and making mistakes innovation is fostered.\citep{farson2002failuretolerantleader} Bank of America's top management serves an example how management level can show their support and commitment towards experimentation process. When setting up an experimentation laboratory in some of their banks in order to test ideas in real environment, some employees were afraid that their rewards and bonus scores would get harmed through failed experiments, which decreased the willingness to try out novel approaches. Senior management decided to abandon the conventional bonus system in the test branches, and instead reward employees based on team performance. This lead employees feel they were special and being supported in experimenting. \citep{thomke2003r}

Leadership plays a major role in defining group goals, controlling resources and providing rewards through interactive leadership process, making leadership behaviour an essential environmental variable in stimulating creative behaviour as a means for achieving goals \citep{redmond1993putting}. According to componential theory of organisational creativity \citep{hennessey19881,amabile1996assessing}, employee's perception of the work environment influences individual and team creativity and emphasises the role of local leader support for creating an creativity supporting environment. Local leaders are in essential role in directing and evaluating work of employees, facilitating and allowing resources and information as well as encouraging employees to engage with the tasks and team members. \citep{amabile2004leader}
 
Likewise, according to \citet{mumford2002leading} creative leadership highlights three key elements: encouraging employee's idea generation, creating safe environment for ideas to emerge and improving idea promotion and implementation. By idea stimulation, education of various problem solving techniques, support for novel ideas, involving employees in developing ideas and allowing them freely pursue ideas, idea generation can be enhanced by the leader. Essential elements of safe environment include diverse teams, transparent and good communication, leader acting as a role model and being in charge of conflict management. In addition to idea generation, idea structuring phase consists of creating action or project frameworks so that employee's have as much autonomy to perform the task as needed. Idea promotion, in turn, refers to leaders task to transfer ideas to broader levels of an organisation, achieve support and assist with implementation of chosen ideas. \citep{mumford2002leading} 

\citet{amabile2008creativity} draw a poetical picture how leader cannot manage creativity, but manages for creativity. Furthermore, they suggest that culture that fosters creativity includes leadership that enables collaboration, enhances diversity, encourages ideation, maps the stages of creativity to different needs, accepts inability and utility of failure and motivates employees with intellectual challenges. According to \citet{sosik1999leadership} leaders should concentrate on vision of work and its outcomes that is meaningful and motivational enough to inspire employees.

Also Buijs (2007) \citet{buijs2007innovation} states how leaders dealing with uncertain and new innovations should stay certain about uncertainties and provide a safe environment and encourage employees to work on current task comfortably. Thus, high level of tolerance for dealing with different states of minds and various personal feelings is required from a leader. \citep{buijs2007innovation} 

Furthermore, fear of failure is likely to be decreased through this transformational leadership that fosters the culture of intrinsic motivation, rewards from creative endeavours, idea exchange and open discussion \citet{amabile1998kill}. Creativity, exchanging ideas and turning them into action requires intrinsic motivation from employees \citep{jung2001transformational}, and according to \citet{amabile1998kill} leaders should foster organisational culture in which individuals find their motivation in divergent thinking and experimenting new ways of performing tasks. 

Also \citet{amabile2004leader} emphasise in their componential theory on creativity the support of immediate supervisors as a way to enhance employee's creativity and intrinsic motivation, which also affects employees willingness to conduct experiments. Supporting actions include defining and setting appropriate goals and tasks \citet{amabile1998kill}, showing the work group support and confidence within the organisation, showing appreciation of individuals contributions to the project, focusing on efficient and good communication, and listening novel ideas with open mind. \citep{amabile2004leader} According to \citet{hackman1987design} setting directions and goals to employees also influences positively on team effectiveness.

Leaders can affect employees' creativity and innovation skills both directly and indirectly \citep{jung2003role}. By stimulating employee's intrinsic motivation and higher level needs leaders are able to affect directly on employees' creativity \citep{tierney1999examination}, where indirect way may be through establishing a work environment where new ways of doing are encouraged and failure is not being punished \citep{amabile1996assessing}. Leaders who create and support a reward-system that values creative performance, provides both intrinsic and extrinsic rewards for employee's efforts to learn new skills and to challenge status quo by experimenting new approaches, increase employees' willingness to constantly engage in creative endeavours \citep{jung2001transformational,mumford1988creativity}.

According to the study of \citet{amabile2004leader}, leaders are likely to influence employees feelings, perceptions and performance as well as overall creativity do their own behaviour. By acting fairly, consulting with employees on essential decisions, offering emotional support and rewarding and recognising them for performing well leaders can enhance creativity of employees. In turn, leaders may play as hindrance for creativity by not offering support and clear task assignments, preventing autonomy of employees, treating employees unfairly and not trying to resolve important problems. 

Offering and receiving valuable feedback serves as a key function of leaders and one of the most challenging tasks they have, that has its affects on employees willingness to conduct experiments \citep{amabile2004leader,amabile1998kill}. According to \citet{shalley2004leaders} giving performance feedback is essential for creativity and accordingly difficult: creativity often involves approaching problems from new approaches, concurrently experimenting novel things includes risk-taking. 

In addition, training and coaching seem to be useful approaches for leaders, who pursue to contribute employee's self-efficacy and team effectiveness essential for experimenting \citep{amabile1998kill}. \citet{edmondson1999psychological} also states that team leader coaching influences positively on team psychological safety. 

Idea generation is an essential phase of experimentation process. Leaders need to acquire resources, encourage idea generation \citep{mcgourty1996managing} and support employees to break routines by stating how essential experimenting, iterating and failing is for learning and developing \citep{amabile2008creativity,shalley2004leaders}. Leaders supporting new ideas and idea exchange has been related to enhancing creativity especially among those employees who showed disposition towards creativity \citep{oldham1996employee}. In addition to supporting idea generation, leaders need to commit to the experimentation process by evaluating employees' ideas and integrating them to organisational needs\citep{mumford2002leading}.

In experimentation process, employees need to contribute imagination, and this may require new kind of encouragement for creativity from the leaders. Much success rises from employee's own initiatives, which results from wide amount of autonomy at work. \citep{amabile2008creativity} A culture of creativity can be fostered in an organisation through opening the organisation to diverse perspectives and openness to various ideas. This calls for safe environment for employees to share their thinking from different fields of expertise. Furthermore, encouraging passion and knowledge of an employee is likely to result in more creative action at work. \citep{amabile2008creativity}

Empowering employees is an essential tasks of leaders, through which a work environment is created where employees desire to seek innovative approaches to perform their work tasks \citep{jung2003role}. Transformational leaders encourage employees to participate in developing by highlighting the importance of cooperation, providing the opportunity to learn from shared experience and allowing employees to perform necessary actions in order to be more effective\citep{bass1990implications}. Thus, organisational climate has to support and encourage innovation \citep{mumford1988creativity} by valuing initiative and innovative approaches that support employees in risk-taking, accepting challenging assignments and stimulate intrinsic motivation towards work \citep{jung2003role}.

\citep{mumford2002leading} argue organisational climate and culture being a collective social construction where the role of the leader on control and influence is remarkable. \citep{schein2010organizational} also presents a view where leaders communicated personal values and beliefs become essentially part of organisation's culture and climate. Furthermore,\citep{jung2001transformational} considers managers essential for shaping organisational culture, whether the concern is in developing, transforming or institutionalising. The way employees perceive their work environment created by their leaders, and especially the way they perceive the instrumental and socioemotional support both have influence on employees' creativity. \citep{oldham1996employee}

As undertaking novel approaches to work oftentimes involves risk-concerned decision-making, employees should be offered decent level of guidance, goals and some measure of structure \citep{jung2003role}. Leader not taking an active role in supporting and guiding the work of his employees may lead to organisational units working at cross-purpose. Thus, leadership is about maintaining a balance between empowering employees and providing guidance and structure through setting goals and agenda. However, according to \citet{mumford2002leading} leaders' planning and guidance should focus on progress, projects on general level and implementation of the results of projects instead of focusing on offering detailed guidance on piece of work. 

\citet{lee2004mixed} have studied the inconsistencies that are likely to inhibit innovation behaviour. Where current research claims how affecting on one organisational condition is likely to foster behaviour essential for innovation, this approach reveals how changing only one organisational condition may lead to inconsistency between work tasks and expectations and lead to decrease in willingness to act towards innovation. Example of inconsistency in organisational conditions can be found from the behaviour of Bank of America's management levels. In order to show support for essential failure and to communicate how novel ideas are not able to rise without it, senior management set the failure rate in 30 per cent. This was supposed to indicate sufficient risk taking and novelty. Concurrently, rewards and employee compensation remained to be based on traditional measures of performance. This lead to inconsistency: the aim to increase innovation and novel ideas was contradictory to the reward system. If an employee spent a lot of time and effort on experimenting and faced failures, his rewards were likely to suffer. \citep{lee2004mixed} 

\subsubsection{Leadership style}
Actually, leadership behaviour is only recently recognised as essential part of enhancing creativity and innovation skills of employees \citep{mumford2002leading}. This may be due to our romantic perception of creative behaviour, which defines creativity as an heroic act of an individual, where leaders only hinder this creative force. Furthermore, conventional models of leadership are not likely to encourage employees to challenge the status quo but to achieve required goals.\citep{mumford2002leading} \citet{jung2003role} argue that several aspects of leadership behaviour can be learned and practiced. Thus, organisations should foster and improve innovativeness by offering managers training and mentoring processes that develop transformational leadership. 

Experimentation requires special leadership style. General leadership styles and practices are set for industrial management, whereas at present the focus should be on leading the people in a collaborative way, from authoritarian style to increased autonomy and trust. This change from authority-based leadership to collaboration with employees has occurred in literature and in practice \citep{amabile2008creativity,farson2002failuretolerantleader}. As \citet{mumford2002leading} state, "organisations may now need jazz group leaders rather than orchestra directors". 

Conventional leadership behaviour focuses on internal activities within the team, whereas innovative team leader needs various set of skills and approaches in order to encourage developing and growing of teams and individuals. For instance, according to \citet{barczak1989leadership} leaders of innovative teams utilise wide range of familiar and unfamiliar techniques in order to accomplish the team objectives, whereas leaders of operating teams use only a few familiar techniques. Even though in this study innovation teams were not studied, similar elements of developing by experimenting and encouragement for that may be recognised, when dealing with new tasks and developing something which result is uncertain. 

Leadership style has great impact on organisational innovation and creativity. Transformational leadership refers to leadership style and processes which emphasises longer-term and vision-based motivational processes \citep{bass1997full}, encourages divergent thinking and looking at problems from unconventional perspectives, while providing and explaining clearly defined goals and facilitating the innovation process of employees \citep{bass1990implications}. Furthermore, through offering an explanation of the importance and value of the work, leaders encourage their employees' to think beyond self-interest \citep{yukl2002leadership}. Leaders shape and define the goals and working context \citep{amabile1998kill, redmond1993putting}. Through a long-term vision (separated from short-term business outcomes, which usually focuses on quarterly profit), leader's are able to direct employee's efforts towards creativity and innovative work processes leading to likeminded outcomes\citep{amabile1996assessing}. 

Transformational leaders can build environments that support creative actions \citep{sosik1998transformational,avolio1988transformational}. According to \citet{sosik1998transformational} key characteristic of transformational leader is the intellectual stimulation, which is likely to encourage creativity and divergent thinking leading to unconventional solutions to problems at hand. 

Failure-tolerant leaders put effort on explaining to employees how important part failure is to the development process as a whole, and how failing actually refers to a point where surprising, failed outcomes are not reflected and further analysed in order to learn. Performing accordingly, admitting own failures and not chasing anyone to blame, failure-tolerant leaders encourage failure, lower the threshold and ease the fear of failing of employees. \citep{farson2002failuretolerantleader}

Furthermore, failure-tolerant leaders treat success and failure similarly, analysing and reflecting the outcomes in order to grow the intellectual capital of the team, including experience, knowledge and creativity. Other characteristics of failure-tolerant leaders are being rather collaborative than controlling, listening carefully, seeing the bigger picture, asking questions and focusing on the development and future rather than blaming on mistakes. In addition, in order to gain empathy and trust among employees, leader should admit their own mistakes, as it shows self-confidence and honesty, assisting in forming closer ties with employees. Vulnerability and transparency play a major role in trustworthy relationship between leader and employees.  \citep{farson2002failuretolerantleader}

In contrast to transformational leadership, transactional leadership refers to focus on employees ability to fulfil and achieve clearly defined goals \citep{hollander1978leadership,house1971path} and successful goal achievement is rewarded \citep{waldman1990adding}. This exchange relationship between leader and employee is based on a contract of specified goals and emphasises on the process of achievement of objectives (Avolio and Bass 1988) but does not encourage employee's to develop their creativity and innovation skills \citep{jung2001transformational}. Instead, employees are rather motivated extrinsically to perform their job under transactional leader but not expected to question and change the status quo in creative ways \citep{amabile1998kill}.  

Innovation leaders, indeed, in managing innovation processes need to have several contradictory skills, roles and attitudes used smoothly during the day with employees; and they need to tolerate these competing and conflicting aspects within the team. Innovation team trusts their leader to be in charge and in control, yet allow them support, autonomy and enthusiasm.  At the same time, innovation leader should be few steps ahead thinking of uncertain future scenarios and support his innovation team in current step without showing doubts too strongly about team's work. \citep{buijs2007innovation} Resulting from this contradictory and challenging role of an innovation leader, according to \citet{buijs2007innovation} they should act and have characteristic of a controlled schizophrenic. 

As \citet{garvin2008yours} bring out in their article, reasonable question to ask for the fresh leadership approach is, can managers actually be excited about being a facilitator of creative process, and where to find those managers who feel engaged and aspired to that role and want to do it? \citet{lingo2010nexus} has offered one perspective to this question in her study with production of music. She claims that producer is the one bringing it all together; it is actually hard leadership exercise, where people from different fields and teams need to work together for one production, where there are no clear rules for who is controlling the output nor yardstick how good or bad the production is. Through creating a shared purpose and common goal in production team, and while still letting "other apply their distinctive expertise", a producer actually operates at the centre of the storm without being at the focus of attention as well as aims for productivity without being over controlling. According to this example, glory comes from being able to help others to find and realise their unique talents at the same time with achieving a collective goal. \citep{lingo2010nexus}

\subsection{Role-modelling}
In order to encourage creativity and experimenting in teams, leaders should lead by example and act as role models. Leaders should consider their own behaviour and actions in a way that stimulates employees to new and innovative, creative approaches to problems. \citep{mumford2002leading,amabile2008creativity,waldman1990adding} \citet{katz1978social} refer to role of the leader in a sense where leader defines by his example the reality of workplace; norms, practices and culture. According to \citet{barczak1989leadership} leader's task is also to provide clear focus for the work of employees. In addition, they can even request creative and innovative solutions form the team, which may lead to better results in creativity of individuals \citep{amabile2002creativity}. 

Through the green light given and their own example leaders can change the focus from success and failure into thinking in terms of learning and experience. \citep{farson2002failuretolerantleader} Also \citet{amabile2004leader} consider leader being a role model for employees essential for enhancing employee's creativity and intrinsic motivation. 

By defining organisational culture, climate and group norms leaders shape the way of working of employees. Through such role-modelling and mentoring process leaders also show employees in practise how tasks are performed. Employees, in turn, follow the example of leader in order to achieve high level of performance. \citep{redmond1993putting} Role-modeling stands also as powerful tool for opening employee's eyes and attitudes to new perspectives, thinking 'out of the box' and adopting generative and exploratory thinking processes \citep{jung2003role,sternberg1997creativity}, influencing creativity of an employee \citep{shalley2004leaders}.

Changing overall organisational climate is challenging, yet various components are reasonably manageable and should foster creativity. For instance, risk-taking, constructive feedback can be supported through role modelling of management. \citep{shalley2004leaders} Team members observe and reflect other members responses and actions and attend to them, yet behaviour of the leader is often their particular concern \citep{tyler1992relational}.

\subsection{Allowing resources}
Top management and immediate superiors can affect and support employees' experimentation behaviour by allowing different resources experimentation requires. The most urgent resources are time for creative thinking and experimenting, material resources and autonomy over employee's own work\citet{amabile2008creativity,katz1985project}. 

In order to find fast and easy ways to test hypothesis and conduct experiments, creativity is required. Prior studies show how creative efforts of employees require sufficient amount of time and energy \citep{gardner1988creativity,getzels1975problem}, big ideas do not hatch overnight. Furthermore, trying out novel approaches and conducting experiments require more energy and is overall more difficult for employees than performing and sticking to the routine tasks. As it takes more cognitive resources to generate several alternative solutions, practice divergent thinking and approach problems from different perspectives, allowing time for creative work and thinking is essential \citep{amabile2002creativity,shalley2004leaders}. 

According to \citet{amabile2002creativity} clear time should be allocated for developing especially when the aim is to flourish idea generation, creativity, learning and experimentation of new concepts. \citet{redmond1993putting} state that leaders should allow enough time for problem solving and creative actions, and according to \citet{amabile1987creativity} also playing with ideas and exploring multiple perspectives. However, no sense of urgency leads employees easily to auto-pilot mode, in which routine tasks are performed without further thinking and analysing \citep{amabile2002creativity}. Shared goals are once more essential in engaging team members to play with ideas and feel more motivated in developing their work. \citep{amabile2002creativity} 

According to a study of \citet{amabile2002creativity}, employees working on high time pressure affects negatively on ability to engage in creative cognitive processing. In addition, analytic and creative thinking are prevented under stress, heavy workload and too tight schedule and ability to recognise and react to problems and learn from experiences deteriorates \citep{garvin2008yours}.  \citet{katz1985project} found in their study how uninterrupted time was considered critical for engineers working on novel technologies. Indeed, lack of time and resources may serve as a hindrance to employee's willingness to take risks and perform experiments \citep{jung2003role}. Through leaders who allow their employees to participate in developing and ideating, reserve budget for it and set it as a part of performance standard, the hindrance for risk-taking may be lowered \citep{jung2003role}. Instead of looking and judging by numbers of hours of work or results employees should be given enough time to reflect their work. 
 
Furthermore, leaders can assist their employees by recognising times with high pressure, and allowing employees to focus on certain thing at a time, leaving the expectations of creativity and new ideas into the future endeavours, when time pressure has decreased. On the other hand, if creativity is required under stress, leader should transparently explain the importance and reasons behind the strict schedule and required goals. Thus an employee may relate to the problem at hand and engage better at his work. Indeed, helping people to understand the importance of work is essential especially under high time pressure. \citep{amabile2002creativity}

\citet{shalley2004leaders} also state how employees should be given appropriate level of autonomy. However, too much autonomy, meaning full control over planning and conducting the work and experimentations, may lead to negative consequences and contradictory goals between employee and organisation. Thus, setting appropriate goals and understandable requirements that inspire employees is essential. Furthermore, leaders need to realise whether the goals require creativity or lead to creative outcomes, and not anticipate creativity or creative outcomes and instead accept employees being less creative where it is not needed. \citep{shalley2004leaders} 

Leaders task to promote ideas and results of experiments to upper levels of organisation serve as a major way to insure sufficient resources and support for the idea implementation. \citep{mumford2002leading} 

In order to be creative and conduct experiments sufficient access to material resources should be allowed for employees \citep{katz1985project}. However, even though material resources are essential for creativity, studies have suggested a contradictory perspective: when employees have access to wide range of material resources, their creativity tendencies may decrease. This may happen due to the creative actions and thoughts an employee needs to perform when needing certain resources to finish his task but not having them at hand. This, in a way, stretches employees' skills to think differently and achieve goals. \citep{csikszentmihalyi199916} Thus, \citet{csikszentmihalyi199916} states how resources are likely to make employees feel too comfortable and lead to decrease in creativity. 

Furthermore, autonomy and freedom to perform essential tasks has major effects on organisational creativity, as individuals are more likely to produce creative work when having the feeling of personal control over how to approach given tasks \citep{amabile1996assessing}. Yet, in order to maintain organisational innovation and risk-taking, autonomy given to an employee can not be in contradiction with fear of failure or discouragement towards challenging status quo or trying out novel solutions \citep{yukl2002leadership}. Furthermore, allowing employees freedom to act actually arouses desire to act.\citep{kanter1983change}

Organizational structures affect in the traditional roles of leadership as a means of direct responsibility given to employees. The trend of flatter organisations provides more autonomy to employees, whereas leaders' role transforms to more involved in external resource acquisition and managing the interfaces. \citep{shalley2004leaders}

Creative work environment is likely to be created through leaders who support and encourage employees, provide them autonomy in decision-making and everyday tasks, and communicate openly with employees \citep{oldham1996employee,tierney1999examination}. However, in addition to contextual factors and environment, studies show level of support, control and assist an employee needs depends on personal characteristics. Thus, leaders knowing and understanding their employees is essential in order to provide employees individual support needed. \citep{shalley2004leaders}

Under some circumstances, according to Monge et al. (1992) \citet{monge1992communication} group communication is likely to increase innovation. Thus, leaders should consider managing wide range of formal and informal meetings and facilitated discussions in order to create opportunities for ideation. Furthermore, innovation occurs over time and is a dynamic process. Leaders should be sensitive in which pace more managerial impact is needed, and in which pace of the process more freedom and autonomy should be allowed for employees. \citep{monge1992communication} 

When brought to wider scale of innovations, a company desiring for innovation should allocate resources and define long-term goals and actions accordingly. Even though companies urge to invest most resources in current lines, sufficient resources should be allocated for long-term growth and innovation. This includes providing an environment strong enough to seize surprising opportunities and tolerate unforeseen threats in all organisational, technical and external relations levels. \citep{quinn1985managing} 

\section{Summary}
The aim for this chapter was to from a consensus on current research concerning factors affecting experimentation behaviour. As experimentation has not yet been widely studied, research from innovation and creativity were included. These aspects are also essential for experimenting.  

In psychologically safe environment employees do not fear being rejected, ask naive questions, make mistakes or present viewpoint of minority. In contrast, psychologically safe environment enables employees comfortably express their thoughts at work. Appreciation of differences is important, as opening minds for different ideas and world views increases both energy and motivation, brings out fresh thinking and are all essential for experimentation behaviour. \citep{garvin2008yours} 

When developing novel products and processes, iteration and failure are included, employees need to feel safe to try various approaches and fail \citep{shalley2004leaders}. As discussed earlier, organisational culture has great influence on employee's perception of safe environment for failing. Employees should be encouraged in risk-taking and exploring and testing uncertain approaches \citep{garvin2008yours}.

In order to foster experimentation behaviour of employees, failing should be considered as opportunity for learning and growth \citep{farson2002failuretolerantleader}. Actually, failure should be encouraged in as early phase of the developing process as possible, as it brings the most value to the process \citep{thomke2001enlightened}.

In order truly novel things to emerge through experimenting and employees to learn and engage to their work, top management and leaders should allocate creative time for playing with ideas, brainstorming, learning and experimenting \citep{amabile2002creativity}. Additionally, employees should be provided with sufficient access to material resources \citep{katz1985project} and autonomy on experimentation \citep{shalley2004leaders}. 

Creative thinking and actions require time, and contradictory, in fast-paced and rapidly changing world and working environment managers should allocate employees sufficient time for creative thinking and experimenting novel approaches \citep{shalley2004leaders}.

Organizational and team structures and hierarchies affect on innovation and experimenting. Flat organisations and small project teams foster innovation performance in a company, as well as team engagement, clear purpose and transparent communication. 

When combining routine work and innovative experimenting, needs to be noticed that the threat for inconsistencies and causing confusion to employees rises. Thus, creativity, learning and experimentation and their dynamics should be understood as well as the consequences inconsistent signals and requirements may lead to \citep{lee2004mixed}.

