\chapter{Factors affecting experimentation}
This chapter provides consensus of factors that affect experimentation behaviour. Various factors have been recognised to foster or hinder creativity and experimentation behaviour of employees, and through understanding the factors, experimenting and learning can be supported. Creativity and learning skills of an individual have been related to willingness and ability to conduct experiments, thus this chapter outlines from organisational behaviour, leadership, creativity and innovation as well as experimentation research. However, experimentation as a method for learning and developing has not yet been widely studied. Thus, in this chapter, theories from research fields such as organisational behaviour, leadership and management, creativity, innovation and prototyping are combined. Overall, employees should be engouraged to ask questions, seek for help and tolerate mistakes and uncertainty \citep{edmondson1999psychological}.

The chapter is divided into three sections. First of all, supporting environment for experimentation is presented, outlining how psychological safety, tolerating risk and learning from failure are all essential for experimentation behaviour. Secondly, organisational practicalities towards development are presented, including low hierarchy, clear and fair communication as well as team engagement. Current trend in research shows leaders and their behaviour have great influence on the creativity and innovation ability of employees (eg. \citep{mumford2002leading,jung2001transformational,amabile1998kill}) as well as willingness to conduct experiments, thus the last section presents how leader can support employees in experimentation. 

\section{Leadership behaviour}
According to studies leaders have a strong direct impact on employee's behaviour and way of performing at workplace \citep{katz1978social,redmond1993putting}.  \citet{avolio1988transformational} list several mechanisms through which leaders can affect employees behaviour. These include role modelling, goal definition, reward allocation, resource distribution, defining norms and values of the company, showing the way to interact as a group, condition employees' perceptions of work environment and being the lead decision maker on organisational structure and procedures. Studies also suggest leaders have a significant effect on employees' creativity and ability to innovate (\citep{mumford2002leading,jung2001transformational,amabile1998kill,hennessey19881}), and according to \citet{redmond1993putting} a leader can have an affect on employee's level of creativity through leadership behaviours such as problem construction, learning goals and feelings of self-efficacy. 

Actually, leadership behaviour is only recently recognised as essential part of enhancing creativity and innovation skills of employees \citep{mumford2002leading}. This may be due to our romantic perception of creative act, which defines creativity as an heroic act of an individual and leaders only hindering this creative force. Furthermore, conventional models of leadership are not likely to encourage employees to challenge the status quo but to achieve required goals.\citep{mumford2002leading} 

Experimentation requires special leadership style. General leadership styles and practices are set for industrial management, whereas at present the focus should be on leading the people in a collaborative way, from authoritarian style to increased autonomy and trust. This change from authority-based leadership to collaboration with employees has occurred in literature and in practice \citep{amabile2008creativity,farson2002failuretolerantleader}. As \citet{mumford2002leading} state, "organisations may now need jazz group leaders rather than orchestra directors". 

As leaders play a major role in establishing, influencing and shaping organisational culture and climate through their communicated values and beliefs, they are able to shape the organisational culture into more innovative direction and foster creativity in an organisation \citep{jung2003role,schein2010organizational} for instance by nurturing organisational climate that supports creative efforts and learning \citep{yukl2002leadership}. 

Role of leaders and managers is essential; being a key person in organising group work and processes a leader may encourage employees to achieve shared goals \citep{amabile1998kill}. Collective organisational achievements can, in turn, be affected through affecting working environment and organisational culture and leaders influence on employee's attitudes and motivation towards work \citep{amabile1998kill}.

\subsection{Offering safe environment for experimenting}
In order to decrease the fear of failure, \citet{amabile2008creativity} suggest leaders should put emphasis on creating an environment where an employee feels safe to fail and speaking out loud ideas nor making mistakes does not result in punishment or humiliation. Leaders should motivate and encourage employees to ideate, break routines and learn by stating how essential experimenting, iterating and failing is for learning and developing \citep{amabile2008creativity,shalley2004leaders}. Furthermore, fear of failure can be decreased through transformational leaders who foster the culture of intrinsic motivation and rewards from creative endeavours, idea exchange and discussion \citep{amabile1998kill}. 

Training, coaching, giving feedback and assigning tasks seem to be useful approaches for leaders, who pursue to contribute empoloyee's self-efficacy and team effectiveness\citep{amabile1998kill}. According to \citet{hackman1987design}, leader behaviour plays a major role in enhancing team effectiveness, and can be facilitated for instance through coaching and setting directions to employees. 

While individual characteristics affect on the creativity of an individual, creating an environment fostering creativity is likely to assist in producing novel ideas during the routine work of employees \citep{amabile1996assessing}. 

For instance, leadership behaviour helps in creating supportive learning environment, which supports managers and employees in creating and defining concrete learning processes and practices. Furthermore, concrete processes support leaders behaviour in a way that fosters learning and through own example cultivates that behaviour to others. \citep{garvin2008yours}

Likewise, according to \citet{mumford2002leading} creative leadership highlights three key elements: encouraging employee's idea generation, creating safe environment for ideas to emerge and improving idea promotion and implementation. By idea stimulation, education of various problem solving techniques, support for novel ideas, involving employees in developing ideas and allowing them freely pursue ideas, idea generation can be enhanced by the leader. Essential elements of safe environment include diverse teams, transparent and good communication, leader acting as a role model and being in charge of conflict management. In addition to idea generation, idea structuring phase consists of creating action or project frameworks so that employee's have as much autonomy to perform the task as needed. Idea promotion, in turn, refers to leaders task to transfer ideas to broader levels of an organisation, achieve support and assist with implementation of chosen ideas. Promotional activities to upper levels of organisation serve as a major way to insure sufficient resources and support for the idea implementation. \citep{mumford2002leading} 

Creativity, exchanging ideas and turning them into action requires intrinsic motivation from employees \citep{jung2001transformational}. Thus, in order to increase creativity and innovation at workplace, leaders should foster organisational culture in which individuals find their motivation in divergent thinking and trying out new ways of performing tasks \citep{amabile1998kill}. 

Conventional leadership behaviour focuses on internal activities within the team, whereas innovative team leader needs various set of skills and approaches in order to encourage developing and growing of teams and individuals. For instance, according to \citet{barczak1989leadership} leaders of innovative teams utilise wide range of familiar and unfamiliar techniques in order to accomplish the team objectives, whereas leaders of operating teams use only a few familiar techniques. Even though in this study innovation teams were not studied, similar elements of developing by experimenting and encouragement for that may be recognised, when dealing with new tasks and developing something which result is uncertain. 

Leadership style has great impact on organisational innovation and creativity. Transformational leadership refers to leadership style and processes which emphasises longer-term and vision-based motivational processes \citep{bass1997full}. Furthermore, through offering an explanation of the importance and value of the work, leaders encourage their employees' to think beyond self-interest \citep{yukl2002leadership}. Leaders shape and define the goals and working context \citep{amabile1998kill, redmond1993putting}. Through a long-term vision (separated from short-term business outcomes, which usually focuses on quarterly profit), leader's are able to direct employee's efforts towards creativity and innovative work processes leading to likeminded outcomes\citep{amabile1996assessing}.

Leaders can affect employees' creativity and innovation skills both directly and indirectly \citep{jung2003role}. By stimulating employee's intrinsic motivation and higher level needs leaders are able to affect directly on employees' creativity \citep{tierney1999examination}, where indirect way may be through establishing a work environment where new ways of doing are encouraged and failure is not being punished \citep{amabile1996assessing}. Creating and supporting a reward-system that values creative performance, provides both intrinsic and extrinsic rewards for employee's efforts to learn new skills and to challenge status quo by experimenting new approaches, employees are constantly willing to engage in creative endeavours \citep{jung2001transformational,mumford1988creativity}.

\citep{mumford2002leading} argue organisational climate and culture being a collective social construction where the role of the leader on control and influence is remarkable. \citep{schein2010organizational} also presents a view where leaders communicated personal values and beliefs become essentially part of organisation's culture and climate. Furthermore,\citep{jung2001transformational} considers managers essential for shaping organisational culture, whether the concern is in developing, transforming or institutionalising. The way employees perceive their work environment created by their leaders, and especially the way they perceive the instrumental and socioemotional support both have influence on employees' creativity. \citep{oldham1996employee}

Also Buijs (2007) \citet{buijs2007innovation} states how leaders dealing with uncertain and new innovations should stay certain about uncertainties and provide a safe environment and encourage employees to work on current task comfortably. Thus, high level of tolerance for dealing with different states of minds and various personal feelings is required from a leader. \citep{buijs2007innovation} 

Failure as a part of innovation and development process begins to be generally recognised and approved. Succeeding companies even thrive for failure in order to learn fast and find the best practices and business models. Through encouraging employees in risk-taking and making mistakes, leaders are likely to boost innovation. For instance, credit company Capital One conducts continually large amount of market experiments. They now most of the tests will not pay off, yet they also know how much can be learned about customers and markets from failed tests in early phase of development. Yet leaders fail in showing their employees the support and tools for failing fast and early enough. Failing in a personal matter remains a difficult subject, as failing never feels exceptionally great, and often employees still consider failed work as failing personally. \citep{farson2002failuretolerantleader}

Failure-tolerant leaders put effort on explaining to employees how important part failure is to the development process as a whole, and how failing actually refers to a point where surprising, failed outcomes are not reflected and further analysed in order to learn. Performing accordingly, admitting own failures and not chasing anyone to blame, failure-tolerant leaders encourage failure, lower the threshold and ease the fear of failing of employees. \citep{farson2002failuretolerantleader}

Naturally management need to take seriously issues about safety and health, yet most of the failures should be seen as opportunities for growth. Furthermore, failure-tolerant leaders treat success and failure similarly, analysing and reflecting the outcomes in order to grow the intellectual capital of the team, including experience, knowledge and creativity. Other characteristics of failure-tolerant leaders are being rather collaborative than controlling, listening carefully, seeing the bigger picture, asking questions and focusing on the development and future rather than blaming on mistakes. In addition, in order to gain empathy and trust among employees, leader should admit their own mistakes, as it shows self-confidence and honesty, assisting in forming closer ties with employees. Vulnerability and transparency play a major role in trustworthy relationship between leader and employees.  \citep{farson2002failuretolerantleader}

Through the green light given and their own example leaders can change the focus from success and failure into thinking in terms of learning and experience. \citep{farson2002failuretolerantleader}

\citet{amabile2008creativity} draw a poetical picture how leader cannot manage creativity, but manages for creativity. Furthermore, they suggest that culture that fosters creativity includes leadership that enables collaboration, enhances diversity, encourages ideation, maps the stages of creativity to different needs, accepts inability and utility of failure and motivates employees with intellectual challenges. According to \citet{sosik1999leadership} leaders should concentrate on vision of work and its outcomes that is meaningful and motivational enough to inspire employees. 

In experimentation process, employees need to contribute imagination, and this may require new kind of encouragement for creativity from the leaders. Much success rises from employee's own initiatives, which results from wide amount of autonomy at work. \citep{amabile2008creativity}

A culture of creativity can be fostered in an organisation through opening the organisation to diverse perspectives and openness to various ideas. This calls for safe environment for employees to share their thinking from different fields of expertise. Furthermore, encouraging passion and knowledge of an employee is likely to result in more creative action at work. \citep{amabile2008creativity}

Empowering employees is an essential tasks of leaders, through which a work environment is created where employees desire to seek innovative approaches to perform their work tasks \citep{jung2003role}. Transformational leaders encourage employees to participate in developing by highlighting the importance of cooperation, providing the opportunity to learn from shared experience and allowing employees to perform necessary actions in order to be more effective\citep{bass1990implications}. Furthermore, autonomy and freedom to perform essential tasks has major effects on organisational creativity, as individuals are more likely to produce creative work when having the feeling of personal control over how to approach given tasks \citep{amabile1996assessing}.

Yet, in order to maintain organisational innovation and risk-taking, autonomy given to an employee can not be in contradiction with fear of failure or discouragement towards challenging status quo or trying out novel solutions \citep{yukl2002leadership}. Thus, organisational climate has to support and encourage innovation \citep{mumford1988creativity} by valuing initiative and innovative approaches that support employees in risk-taking, accepting challenging assignments and stimulate intrinsic motivation towards work \citep{jung2003role}.

\citep{jung2001transformational} has studied how leadership style affects group's creativity and performance by comparing transactional and transformational leadership styles. Transformational leader refers to a leader who encourages divergent thinking and looking at problems from unconventional perspectives, while providing and explaining clearly defined goals and facilitating the innovation process of employees \citep{bass1990implications}. Furthermore, development of clear long-term vision and practises supporting the way to achieve it is essential characteristic of transformational leaders \citep{avolio1988transformational}. The relationship between transformational leader and an employee is active and emotionally attached \citep{avolio1988transformational} and through the strong attachment resulting from tight relationship leaders can better support employees in using their personal values and self-concepts in the way that employees can pursue higher level performance and fulfil personal needs through the work. This focus of transformational leadership on value alignment is likely to lead to the root of intrinsic motivation of an employee \citep{gardner1998charismatic}, which is considered as one of the key elements in creative thinking and innovation skills of an employee (eg. \citep{jung2001transformational,amabile1998kill,deciintrinsic}).

According to \citet{bass1997full} transformational leadership consists of four unique yet interrelated behavioural components: inspirational motivation (articulating long-term vision), intellectual stimulation (promoting creativity and innovation), idealised influenced (meaning charismatic role modelling) and individualized consideration referring to coaching and mentoring leadership style. 

Transformational leaders can build environments that support creative actions (\citep{sosik1998transformational,avolio1988transformational}). According to \citet{sosik1998transformational} key characteristic of transformational leader is the intellectual stimulation, which is likely to encourage creativity and divergent thinking leading to unconventional solutions to problems at hand. 

In contrast to transformational leadership, transactional leadership refers to focus on employees ability to fulfil and achieve clearly defined goals \citep{hollander1978leadership,house1971path} and successful goal achievement is rewarded \citep{waldman1990adding}. This exchange relationship between leader and employee is based on a contract of specified goals and emphasises on the process of achievement of objectives (Avolio and Bass 1988) but does not encourage employee's to develop their creativity and innovation skills \citep{jung2001transformational}. Instead, employees are rather motivated extrinsically to perform their job under transactional leader but not expected to question and change the status quo in creative ways \citep{amabile1998kill}.  

Few studies have been made linking the transformational leadership and positive outcomes of employees' creativity in organisational level and outcomes \citep{jung2003role}, even though several studies have been made revealing the positive relation between these factors. in their study \citet{jung2003role} draw this link clearer and suggest that while leaders define the context and goals of their employes, transformational leadership can be extrapolated to an organisational level.  

According to \citet{jung2003role} transformational leadership is positively related to organisational innovation, employee's perception of empowerment and support for innovation. Furthermore, the perception of empowerment and is positively related to organisational innovation, and when perception being strong,  the relationship between transformational leadership and organisational innovation tends to be stronger. Results of the study conducted on 32 Taiwanese companies suggest that through transformational leadership by top managers organisational innovation can be affected directly or indirectly, latter referring to creating an organisational culture in innovation, discussion, novel approaches and experimenting is encouraged. \citep{jung2003role}

As undertaking novel approaches to work oftentimes involves risk-concerned decision-making, employees should be offered decent level of guidance, goals and some measure of structure \citep{jung2003role}. Leader not taking an active role in supporting and guiding the work of his employees may lead to organisational units working at cross-purpose. Thus, leadership is about maintaining a balance between empowering employees and providing guidance and structure through setting goals and agenda. However, according to \citet{mumford2002leading} leaders' planning and guidance should focus on progress, projects on general level and implementation of the results of projects instead of focusing on offering detailed guidance on piece of work. 

Study of \citet{sethi2001cross} showed how good interaction in a team and high level of commitment to the success of the team lead to more radical innovation abilities. In the study team members were highly encouraged to take risks, which lead to more motivated members in suggesting novel ideas from their perspectives. In addition, team members identified themselves strongly as part of the team, which again higher commitment level. \citep{sethi2001cross} 

Under some circumstances, according to Monge et al. (1992) \citet{monge1992communication} group communication is likely to increase innovation. Thus, leaders should consider managing wide range of formal and informal meetings and facilitated discussions in order to create opportunities for ideation. Furthermore, innovation occurs over time and is a dynamic process. Leaders should be sensitive in which pace more managerial impact is needed, and in which pace of the process more freedom and autonomy should be allowed for employees. \citep{monge1992communication} 

Organizational leaders play a great role in establishing strong team performance culture. This can be achieved through addressing and demanding performance that meets the need of customers, employees and shareholders. Teams should not be fostered by the sake of the team only, rather should leaders clearly state how the team performance affects to customers and through that foster clearer performance ethics and cultures. In addition, even though people tend to have great sense of individualism, it does not have to bias the teamwork performance, as real teams find ways to support individual strengths and performance for shared goal. Furthermore, in order to team function properly and efficiently, discipline across the team and organisation is needed, focusing again on performance.  \citep{katzenbach1993wisdom}
 
Leadership plays a major role in defining group goals, controlling resources and providing rewards through interactive leadership process, making leadership behaviour an essential environmental variable in stimulating creative behaviour as a means for achieving goals \citep{redmond1993putting}. \citet{katz1978social} even refer to role of the leader in a sense where leader defines by his example the reality of workplace; norms, practices and culture. According to \citet{barczak1989leadership} leader's task is also to provide clear focus for the work of employees. 

Although different leadership styles and their effect on employee's creativity behaviour has not yet been studied widely, some studies show, how transformational leadership behaviour encourages employees look problems from different perspectives and thus widen their intellectual and creativity skills \citep{jung2001transformational,sosik1998transformational}. \citet{jung2001transformational} has studied the relation between leadership style and group creativity finding that transformational leadership is most likely to stimulate creative effort of employees. 
 
In his study, \citet{jung2001transformational} emphasises that transformational leadership skills can be practised in order to foster creativity and intellectual skills of employees and shape organisational culture. His study showed how transformational leadership; encouraging divergent thinking and solving problems at hand from unconventional perspectives, is likely to increase intrinsic motivation of employees leading to more creative problem solving and behaviour.  Through brainstorming activities that focus on non-traditional thinking and fantasising intellectual skills of employees can be enhanced \citep{sosik1998transformational}. Furthermore, \citet{jung2003role} argue that several aspects of leadership behaviour can be learned and practiced. Thus, organisations should foster and improve innovativeness by offering managers training and mentoring processes that develop transformational leadership. 

Innovation leaders, indeed, in managing innovation processes need to have several contradictory skills, roles and attitudes used smoothly during the day with employees; and they need to tolerate these competing and conflicting aspects within the team. Innovation team trusts their leader to be in charge and in control, yet allow them support, autonomy and enthusiasm.  At the same time, innovation leader should be few steps ahead thinking of uncertain future scenarios and support his innovation team in current step without showing doubts too strongly about team's work. \citep{buijs2007innovation} Resulting from this contradictory and challenging role of an innovation leader, according to \citet{buijs2007innovation} they should act and have characteristic of a controlled schizophrenic. 

As \citet{buijs2007innovation} argues, leaders who are to lead employees and work handling innovations need to understand the paradox and natural conflicts between routine processes (exploitation) in order to earn money in the present and the innovation processes (exploration) in order to earn money in the future. \citet{buijs2007innovation} four aspects for innovation which leader should be able to master all providing a secure environment for a team to perform in novel and creative ways. These consist of innovation process, psychological process of innovation team, creativity process, and leading and playing. 

Goals, however, should be kept broad, in order not to create undue oppositions to new ideas. Flexibility should be maintained by not defining intermediate steps in detail and by trying alternate options and routes. Identifying and solving problems at early phase fosters momentum, confidence and identity towards new approach. Furthermore, sufficient amount of information about the project and progress should be offered in order managers to follow and realise the work performed.  \citep{quinn1985managing}

Local leaders are in essential role in directing and evaluating work of employees, facilitating and allowing resources and information as well as encouraging employees to engage with the tasks and team members. \citep{amabile2004leader}

According to the study of \citet{amabile2004leader}, leaders are likely to influence employees feelings, perceptions and performance as well as overall creativity do their own behaviour. By acting fairly, consulting with employees on essential decisions, offering emotional support and rewarding and recognising them for performing well leaders can enhance creativity of employees. In turn, leaders may play as hindrance for creativity by not offering support and clear task assignments, preventing autonomy of employees, treating employees unfairly and not trying to resolve important problems. 

According to componential theory of organisational creativity \citep{hennessey19881,amabile1996assessing}, employee's perception of the work environment influences individual and team creativity and emphasises the role of local leader support for creating an creativity supporting environment. 

In their study, \citet{shalley2004leaders} present how leaders should use human resource practices in order to develop work context which improves the creativity skills of employees. 

Organizational structures affect in the traditional roles of leadership as a means of direct responsibility given to employees. The trend of flatter organisations provides more autonomy to employees, whereas leaders' role transforms to more involved in external resource acquisition and managing the interfaces. \citep{shalley2004leaders}

Creative work environment is likely to be created through leaders who support and encourage employees, provide them autonomy in decision-making and everyday tasks, and communicate openly with employees \citep{oldham1996employee,tierney1999examination}. However, in addition to contextual factors and environment, studies show level of support, control and assist an employee needs depends on personal characteristics. Thus, leaders knowing and understanding their employees is essential in order to provide employees individual support needed. \citep{shalley2004leaders}

 As \citet{garvin2008yours} bring out in their article, reasonable question to ask for this fresh leadership approach is, can managers actually be excited about being a facilitator of creative process, and where to find those managers who feel engaged and aspired to that role and want to do it? \citet{lingo2010nexus} has offered one perspective to this question in her study with production of music. She claims that producer is the one bringing it all together; it is actually hard leadership exercise, where people from different fields and teams need to work together for one production, where there are no clear rules for who is controlling the output nor yardstick how good or bad the production is. Through creating a shared purpose and common goal in production team, and while still letting "other apply their distinctive expertise", a producer actually operates at the centre of the storm without being at the focus of attention as well as aims for productivity without being over controlling. According to this example, glory comes from being able to help others to find and realise their unique talents at the same time with achieving a collective goal. 
 
\subsection{Role-modelling}
In order to encourage creativity and experimenting in teams, leaders should lead by example and act as role models. Leaders should consider their own behaviour and actions in a way that stimulates employees to new and innovative, creative approaches to problems. In addition, they can even request creative and innovative solutions form the team, which may lead to better results in creativity of individuals \citep{amabile2002creativity}. \citep{mumford2002leading,amabile2008creativity,waldman1990adding}
By defining organisational culture, climate and group norms leaders shape the way of working of employees. Through such role-modelling and mentoring process leaders also show employees in practise how tasks are performed. Employees, in turn, follow the example of leader in order to achieve high level of performance. \citep{redmond1993putting} Role-modeling stands also as powerful tool for opening employee's eyes and attitudes to new perspectives, thinking 'out of the box' and adopting generative and exploratory thinking processes \citep{jung2003role,sternberg1997creativity}, influencing creativity of an employee \citep{shalley2004leaders}

Changing overall organisational climate is challenging, yet various components are reasonably manageable and should foster creativity. For instance, risk-taking, constructive feedback can be supported through role modelling of management. \citep{shalley2004leaders} Team members observe and reflect other members responses and actions and attend to them, yet behaviour of the leader is often their particular concern \citep{tyler1992relational}.

\subsection{Showing support and encouraging experimenting}
Leaders need to acquire resources and encourage idea generation \citep{mcgourty1996managing}, and overall create environment where idea generation is possible \citep{andrews1970social} as well as evaluate the ideas and integrate them to organisational needs \citep{mumford2002leading}. 

Furthermore, engaging employees to creative activities is likely to lead better and more qualified decisions. \citep{shalley2004leaders}
Also \citep{amabile2004leader} emphasise in their componential theory on creativity the support of immediate supervisors as a way to enhance employee's creativity and intrinsic motivation. Supporting actions include being a role model, defining and setting appropriate goals, showing the work group support and confidence within the organisation, showing appreciation of individuals contributions to the project, focusing on efficient and good communication, offering valuable feedback, and listening openly novel ideas. Accordingly \citep{amabile2004leader} divide required behaviours of leaders for providing support into two categories: instrumental or task-oriented and socio-emotional or relationship-oriented actions.

Giving and receiving feedback remain simultaneously a key function of leaders and one of the most challenging tasks they have. According to \citet{shalley2004leaders} giving performance feedback is essential for creativity and accordingly difficult as creativity often involves approaching problems from new approaches and trying out novel things as well as taking risks. 

For instance, \citet{edmondson1999psychological} states team leader coaching influencing positively on team psychological safety. Psychologically safe environment includes team leaders being supportive, coaching-oriented, who doesn't response to questions or challenges in defensive manner. Employees are not likely to take interpersonal risks that might lead to learning if leader tends to act in authoritarian of punitive ways. 

\citet{quinn1985managing} emphasis top executives role over particular management. Innovation is likely to occur when top executives encourage creative and innovative endeavours and create an atmosphere and value system that supports innovation. Leaders supporting new ideas and idea exchange has been related to enhancing creativity especially among those employees who showed disposition towards creativity \citep{oldham1996employee}.  

\subsection{Allowing resources}
Top management and immediate superiors can affect and support employees' experimentation behaviour by allowing different resources experimentation requires. The most urgent resources are time for creative thinking and experimenting, material resources and autonomy over employee's own work\citet{amabile2008creativity,katz1985project}. 

In order to find fast and easy ways to test hypothesis and conduct experiments, creativity is required. Prior studies show how creative efforts of employees require sufficient amount of time and energy \citep{gardner1988creativity,getzels1975problem}, big ideas do not hatch overnight. Furthermore, trying out novel approaches and conducting experiments require more energy and is overall more difficult for employees than performing and sticking to the routine tasks. As it takes more cognitive resources to generate several alternative solutions, practice divergent thinking and approach problems from different perspectives, allowing time for creative work and thinking is essential \citep{amabile2002creativity,shalley2004leaders}. 

According to \citet{amabile2002creativity} clear time should be allocated for developing especially when the aim is to flourish idea generation, creativity, learning and experimentation of new concepts. \citet{redmond1993putting} state that leaders should allow enough time for problem solving and creative actions, and according to \citet{amabile1987creativity} also playing with ideas and exploring multiple perspectives. However, no sense of urgency leads employees easily to auto-pilot mode, in which routine tasks are performed without further thinking and analysing \citep{amabile2002creativity}. Shared goals are once more essential in engaging team members to play with ideas and feel more motivated in developing their work. \citep{amabile2002creativity}

According to a study of \citet{amabile2002creativity}, employees working on high time pressure affects negatively on ability to engage in creative cognitive processing.  \citet{katz1985project} found in their study how uninterrupted time was considered critical for engineers working on novel technologies. Indeed, lack of time and resources may serve as a hindrance to employee's willingness to take risks and perform experiments \citep{jung2003role}. Through leaders who allow their employees to participate in developing and ideating, reserve budget for it and set it as a part of performance standard, the hindrance for risk-taking may be lowered \citep{jung2003role}. 
 
Furthermore, leaders can assist their employees by recognising times with high pressure, and allowing employees to focus on certain thing at a time, leaving the expectations of creativity and new ideas into the future endeavours, when time pressure has decreased. On the other hand, if creativity is required under stress, leader should transparently explain the importance and reasons behind the strict schedule and required goals. Thus an employee may relate to the problem at hand and engage better at his work. Indeed, helping people to understand the importance of work is essential especially under high time pressure. \citep{amabile2002creativity}

\citet{shalley2004leaders} also state how employees should be given appropriate level of autonomy. However, too much autonomy, meaning full control over planning and conducting the work and experimentations, may lead to negative consequences and contradictory goals between employee and organisation. Thus, setting appropriate goals and understandable requirements that inspire employees is essential. Furthermore, leaders need to realise whether the goals require creativity or lead to creative outcomes, and not anticipate creativity or creative outcomes and instead accept employees being less creative where it is not needed. \citep{shalley2004leaders} 

In order to be creative and conduct experiments sufficient access to material resources should be allowed for employees \citep{katz1985project}. However, even though material resources are essential for creativity, studies have suggested a contradictory perspective: when employees have access to wide range of material resources, their creativity tendencies may decrease. This may happen due to the creative actions and thoughts an employee needs to perform when needing certain resources to finish his task but not having them at hand. This, in a way, stretches employees' skills to think differently and achieve goals. \citep{csikszentmihalyi199916} Thus, \citet{csikszentmihalyi199916} states how resources are likely to make employees feel too comfortable and lead to decrease in creativity. 

In summary, in order truly novel things to emerge through experimenting and employees to learn and engage to their work, top management and leaders should allocate creative time for playing with ideas, brainstorming, learning and experimenting \citep{amabile2002creativity}. Additionally, employees should be provided with sufficient access to material resources \citep{katz1985project} and autonomy on experimentation \citep{shalley2004leaders}. 

\section{Team perspective and practicalities}
In order to create new value and competitive advantage in rapidly changing and uncertain organisational environments, new managerial imperative is growing focusing on teams. Thus, supporting teams in their work and understanding the aspects of learning is required \citep{edmondson1999psychological}. Recent studies has moved the focus from individual learning to team learning. Edmondson's definition of group learning stems with definition of \citet{argote2001group}, who emphasises that knowledge is acquired, shared and combined through processes and outcomes of group interaction, focus being on processes. 

Work team refers to small group of people that exist within the context of a larger organisation, members share understanding of being member of the team and its tasks, responsibility for product or service team is working on \citep{hackman1987design,alderfer1983intergroup} as well as its performance \citep{edmondson1999psychological}. Additionally, team members have supplementary knowledge and abilities compared to each other, and they share a goal, targets and way of working and approach \citep{edmondson1999psychological}. According to  \citet{katzenbach1993wisdom} great team performance consists of continuos work of shaping a common purpose, agreeing on performance goals, defining a common working approach, developing high level complementary skills and being transparent on the results. He emphasises that through disciplined action groups transform to teams and argues how demanding schedules, long-standing habits and unwarranted assumptions tend to threaten team efficiency and performance.

\citet{edmondson1999psychological} have studied factors that affect and influence learning behaviour in teams by studying in which conditions and to what extent learning occurs naturally. Learning behaviour of teams refers to activities that team members carry out and through which team is able to obtain, adapt and reflect data and outcomes of actions which further shapes and improves team behaviour. Such activities consist of reflection and improvement-aiming factors such as asking for feedback, transparent information sharing, asking for help, admitting and discussing about failures and errors as well as experimenting. Through such activities teams may observe changes in environment, customer requirements and improve collective understanding. In addition, team's ability to discover and react to unexpected situations and consequences of their actions is likely to improve through learning behaviour. Consequently, compared to low-learning teams that tend to get stuck and be unable to solve problems, teams who master in learning are greater in confronting difficult situation and improve their work. \citep{edmondson1999psychological}

The composition of the team matters. Studies have shown how team performance, especially related to innovation, is improved when team consists of individuals with various and different set of skills and characteristics\citep{buijs2007innovation}. Homogeneity in teams easily leads to groupthink, routine work and repeating traditional daily practices, while even one or two different individuals can stimulate the innovativeness of a team, and actually, the outcasts and those who stand out from the group are required in order to think outside the box, challenge the status quo and present alternative solutions and ideas that would be missing without the participations of these individuals. \citep{sternberg1997creativity}

In addition, in order to function team needs a clear purpose and vision what makes it a team and why it exists. Teams get energy from significant performance challenges regardless of where they are in the organisation. Set of shared, demanding performance goals usually form a team, and personal chemistry or willingness to form a team may boost that. Thus, in order to receive great results teams should focus on performance regardless of the organisational hierarchy or what team does. Thus, team performance may exceed the results of what could be achieved if employees were acting alone as individuals without the team effort. \citep{katzenbach1993wisdom}

In order an organisation to be more innovative, according to \citet{thomke2001enlightened}, team engagement is essential, as the whole team need to understand the meaning of experimenting and developing and it should be encouraged to sharing information and ideas in as early stage of development process as possible and throughout the process. \citet{thomke2001enlightened} suggests using small teams and parallel experiments especially when the time is the most critical factor. 

Accordingly, \citet{amabile1998kill} suggested that creative thinking can be encouraged by shaping organisational culture such that employees feel encouraged to tell their ideas out loud freely and without judging, increasing idea exchange and discussion about them. In addition, studies show how creative individuals may only produce more creative outputs than less creative individuals when the context is supporting and encouraging towards creativity \citep{oldham1996employee}. 

Structures have their influences on creativity of employees. Relationship between formal reporting and responsibility levels, referring to bureaucracy levels are essential: highly bureaucratic organisation do not tend to encourage employees to reach for novel approaches and experiments, whereas organisation with flatter structure may enhance organisations autonomy and creativity \citep{shalley2004leaders}

Employees may be likely to perceive presentations of organisations structure and hierarchy as discouraging and highlighting how employees are not allowed or encouraged to make decisions on their own leading to less enthusiasm towards trying out new ways of working and developing. In addition, heavy bureaucracy demanding lot of time and effort from employees to get novel ideas forward in the organisation is likely to destroy the enthusiasm and willingness of employees towards developing and new approaches. \citep{shalley2004leaders}


\section{Supporting environment for experimenting}
Experimenting requires safe and supportive environment. According to \citet{edmondson1999psychological} team psychological safety should be the first essential building block of learning behaviour in work teams. Supporting learning environment consists of four characteristics: psychological safety, appreciation of differences, openness to new ideas and time for reflection \citep{garvin2008yours}. Likewise, \citet{mumford1988creativity} emphasise the meaning of environmental variables as a means to support employee's creativity by providing resources to stimulate fresh ideas of employees. Furthermore, strong positive relations between organisational environmental variables have been found; organisational encouragement as well as support for innovation and creativity from team improve employee's creativity \citep{amabile1996assessing}

According to \citet{mumford1988creativity} through environmental variables employee's creativity can be fostered, which is important for experimenting. New solutions may be achieved through problem solving and challenging the routine ways of thinking, and environment should be designed to encourage and facilitate these skills. Environmental factors may, furthermore, affect on employee's intrinsic motivation and willingness to generate novel ideas, when social and physical environments work as a source for support and resources in idea generation and implementation \citep{amabile1998kill,mumford1988creativity}. \citet{mumford1988creativity} have studied the gap between an idea an action, and revealed it depending on various attributes related to individual and organisational circumstances. As physical work environment affects on creativity, information sharing and innovation in an organisation, it should be designed to support the natural flow of traffic through the building so that informal conversations between different functional areas are enabled \citep{shalley2004leaders}. 

A company desiring for innovation should allocate resources and define long-term goals and actions accordingly. Even though companies urge to invest most resources in current lines, sufficient resources should be allocated for long-term growth and innovation. This includes providing an environment strong enough to seize surprising opportunities and tolerate unforeseen threats in all organisational, technical and external relations levels. \citep{quinn1985managing} 

Organizational and team structures and hierarchies affect on innovation and experimenting. Flat organisations and small project teams foster innovation performance in a company. Smaller team handles communication and commitment better, while as few management layers as possible decreases the jeopardy of rejection. "Since it takes a chain of yesses and only one no to kill a project, jeopardy multiplies as management layers increase."\citep{quinn1985managing}

Organizational structures can influence in many ways creativity of a team and individual. For instance, by promoting open communication, idea and ongoing information exchange with internal and external team members as well as encouraging information seeking from different perspectives and sources is likely to enhance creativity (e.g. \citep{ancona1992demography,dougherty1996sustained}). 

Also \citet{shalley2004leaders} emphasises the meaning of prior knowledge and experience of an employee of area of work before demanding or anticipating creative actions from them. Naturally, job rotation and employees from different areas works as a great source for new perspectives and development, yet creativity requires sufficient level of familiarity of target area. \citep{shalley2004leaders}

All human resource practices should be in line and systematically linked together in order to create a clear picture for employees of what is expected of them. Perceived fairness and sense of loyalty towards a company of an employee is higher when an employees understands what is expected of them, how, when and what for they are being rewarded, promoted or fired. These same attitudes are essential for fostering creativity. Committed and loyal employees are more likely to exceed what is required of them, be more motivated and committed to working towards specific goal and find novel approaches in order to succeed. \citep{shalley2004leaders}

Essentially, interaction between leaders and employees, team members and outside members, sufficient resources, employees clear expectations on their evaluation and rewards and environment which is perceived fair all have affect on employees behaviour at work. Through these variables, employees may feel they are working in supportive working environment, and foster creativity and ability find novel approaches and try out new things. \citep{shalley2004leaders}

\subsection{Psychological safety}
However, in some organisational environments people do ask help, admit errors and discuss about problems. In these environments employees seem to perceive interpersonal threat low enough to perform despite of the threat. \citet{edmondson1999psychological} has studied working environments and realised in environments employees act despite the threat, they felt safe and supported for their actions. She refers to this as psychological safety. Some studies argue how familiarity among group members is likely to encourage openness towards new information and ideas \citep{sanna1990valence}, yet this factor alone is not sufficient to explain when group members find it safe to act instead of feeling threatened \citep{edmondson1999psychological}.

Psychological safe environment refers to an environment where new ideas and breaking with the status quo are supported \citep{edmondson1999psychological} and uncertainty is not totally avoided but managed and tolerated \citep{shalley2004leaders}.  Psychological safety serves as a mechanism that assists in explaining how structural and interpersonal characteristics both have effects on learning and performance in teams \citep{edmondson1999psychological}. Psychological safety can be boosted for instance through structural factors such as context support and team leader coaching affecting behavioural and performance outcomes \citep{hackman1987design,edmondson1999psychological}. Furthermore, climate of safety and supportiveness encourages employees to seek for feedback and ask for help in addition to admit and reflect mistakes. \citep{edmondson1999psychological}

\citet{edmondson1999psychological} defines a concept of team psychological safety, that fosters learning behaviour in work teams by reducing the risks of embarrassment or threat and increasing mutual trust between team members. Oftentimes in work teams employees are not willing to tell their ideas or errors out loud as they are afraid of being labelled as incompetent. Thus, they prefer staying silent ignoring how it may lead to negative consequences for the team performance. When team members share feeling of respect and trust of others, and stay confident on other member's not using the errors against them, they are more likely to put more weight on the benefits of telling concerns out loud. When knowing that well-intentioned interpersonal risks are not punished is a shared belief of a team, team members are more likely to take proactive actions that foster learning leading to more effective performance. 

Even though building mutual trust may not lead to mutual respect and caring among team members, it is essential for creating psychologically safe environment and through building trust a foundation for further development of team psychological safety is built. \citep{edmondson1999psychological} 

\citet{edmondson1999psychological} lists factors affecting psychological safety in teams, including context support and team leader coaching. Context support refers for instance to access to information and resources needed. Safe environment that fosters creativity also takes into account employees' perceptions of just and transparent decision-making as well as applied actions \citep{shalley2004leaders}.

\citet{garvin2008yours} states by creating an environment that serves psychological safety for employees, organisations may capitalise on failure. Safe environment does not humiliate or punish employees for failing or coming up with novel ideas or doubts. 

Trust, indeed, has been widely noted in research as essential factor in organisational teams and groups \citep{golembiewski1975centrality,kramer1999trust,shalley2004leaders,edmondson1999psychological}. Trust refers to one's willingness to be vulnerable in his actions as he expects his actions will not be judged and will be favourable to one's interests \citep{robinson1997corporate}. Interpersonal trust is involved in psychological safety, yet it also includes perception of mutual respect and overall climate where team members feel free to be themselves \citep{edmondson1999psychological}. 

Team psychological safety refers to Amy Edmondson's concept of team members shared belief team being safe for interpersonal risk-taking. Together with team efficacy these have great affect on team performance and learning in an organisational work. \citep{edmondson1999psychological} Integrative perspective suggests that team performance and outcomes can be shaped through both team structure and shared beliefs, in contrast to previous studies that separate structural and interpersonal factors from each other. For instance, employee's willingness to take interpersonal risks depends highly on the experience of team safety and person's beliefs how others will respond in ideas or situations involving uncertainty. Team psychological safety refers to interpersonal trust among team but beyond that mutual respect and caring. \citep{edmondson1999psychological}

Communication about ideas among team has been widely recognized being related to idea generation, creativity and innovation (e.g.\citep{robinson1997corporate,mumford2002social,monge1992communication,amabile1996assessing}). According to \citet{staw1989tradeoff} social influence of others plays a major role for individuals' beliefs; attitudes towards job, for instance, rise from the social labelling of work by others.  Also \citet{salancik1978social} argue the essential role opinions of others may have on individual: individual's perception of her work and organisation can be greatly influenced by opinions of others. Additionally, team member's collective view of support they get from their leader has been related to the team's creative endeavours and success in them (e.g., \citep{amabile1998kill,amabile1996assessing}). 
 
Thus, as individuals oftentimes requires support and input from several individuals who help to challenge ideas in constructive ways, teams are essential in generating and implementing ideas \citep{mumford2002social}. Stimulating those constructive individuals for creative actions may be valuable \citep{robinson1997corporate}. In addition, including team members in ideation assists in idea implementation and through participation new ideas are not that likely to be rejected or abandoned \citep{agrell1994team}.

Learning of employees occurs when employees do not fear being rejected, ask naive questions, make mistakes or present viewpoint of minority. Psychologically safe environment enables employees comfortably express their thoughts at work. Appreciation of differences is important, as opening minds for different ideas and world views increases both energy and motivation, brings out fresh thinking. Novel approaches are relevant for learning, thus employees should be encouraged in risk-taking and exploring and testing uncertain things. Lastly, providing time for reflection is likely to foster learning in safe environment. Instead of looking and judging by numbers of hours of work or results employees should be given enough time to reflect their work. Analytic and creative thinking are prevented under stress, heavy workload and too tight schedule. Under stress ability to recognise and react to problems and learn from experiences deteriorates. In supportive learning environment time for reflection is allowed. \citep{garvin2008yours} 

Organizational structures are also likely to enhance or hinder creativity in organisational, team or individual levels \citep{shalley2004leaders}. 

Furthermore, \citet{quinn1985managing} argues enthusiasm is not yet widely accepted and tolerated characteristic of employees and refers to them as entrepreneurial fanatics. Larger companies may perceive them as causing embarrassment by challenging status quo and causing troubles. 

\subsection{Tolerating risk}
Individuals are likely to avoid risks and uncertainty, stick to routine and prefer more certain outcomes and ways of performing \citep{bazerman2012judgment}. This does not encourage, however, creative actions that actually require several trial-and-error, iterative, experimentation processes where risk is involved. Employees fearing risk-taking tend to perform the routine way instead of taking chance with new approaches. \citep{shalley2004leaders} However, in his study \citet{nystrom1990organizational} found that organisational culture reflecting challenge and risk taking lead to more innovative actions of employees and the whole organisation. 

Ambiguity is often perceived by individuals when lacking sufficient cues to structure a situation, and usually arises from novelty, complexity or unsolvability of situation at hand \citep{budner1962intolerance}. 

Several factors may affect on the gap between idea and action in employees of an organisation. The phenomenon of threat of employees in organisations is widely studied and consensus is rising how threat effects on cognitive and behavioural flexibility and responsibility in reducing manner. \citep{argyris1982reasoning,edmondson1999psychological}

One essential factor is the beliefs, emotions and actions of an employee. An employee is likely to inhibit learning as a result of feeling the fear of being rejected, under pressure or feeling they are placing themselves at risk \citep{edmondson1999psychological}, or when facing the potential for embarrassment of threat, even though their transparency and honesty would be highly important for the behaviour of the team \citep{argyris1982reasoning}. This may occur in a situation where an employee should ask for help, yet is afraid of admitting he lacks abilities, skills or knowledge. \citep{edmondson1999psychological}. In addition, admitting mistakes, asking for help and seeking feedback are all relevant abilities in the recent organisational world, yet threatening for an individual's image of himself and his skills \citep{brown1990politeness}.

\citep{sosik1998transformational} suggested that anonymous ideating through nominal groups leads to better results and greater amount of ideas than brainstorming activities in real groups. When ideating in daily working groups, members may fear failing, being ashamed or measured by their performance. Overall, oftentimes it is way more difficult to take a different role and actions in group with familiar members and routines. \citep{jung2001transformational}

When individuals ask for help, admit errors or seek feedback they place themselves under risk, and perceive a threat, fear of being judged and appearing incompetent as well as fear of giving unfavourable impressions on people who have the power to give promotions, raises or who assigns projects\citep{edmondson1999psychological,brown1990politeness}. Even though knowing a team would benefit of this kind of behaviour, perceived threat appears strong \citep{edmondson1999psychological}. Feeling of threat and embarrassment is linked to reduce cognitive and behavioural flexibility and responsiveness \citep{staw1989tradeoff} and this leads easily individuals acting ways that rather inhibit than fosters learning \citep{argyris1982reasoning}. 

Uncertainty, actually, should not be considered only as a threat or inconvenience occurring in organisations, rather should appropriate level of messiness let exist, and develop opportunities where uncertainty can be exploited. Overall, uncertainty in creative processes should not be overly controlled. \citep{sternberg1997creativity} Level of uncertainty can be reduced for instance through goal-setting and fast prototyping \citep{mumford2002leading}. 

Predicting the future being impossible, focus should be in managing risks involved in playing with creative ideas in both the company and individual level. As \citet{sternberg1997creativity} state, "as uncomfortable as it is, while not being able to predict and control uncertainty in creative projects, the messiness does have to let exist". \citet{kanter1983change}continues that, actually, opportunities grow from uncertainty and creative endeavours rise when struggling with uncertainty and mess, as individuals impose order where it does not exist, and thus individuals are forced to form new connections. Furthermore, allowing employees freedom to act actually arouses desire to act.

Feeling of self-efficacy may affect individual's willingness to provide unique and novel ideas even when some degree of risk is involved \citep{mumford1988creativity}. In creative work risk concerns both the need to do experiments and tolerate failure \citep{andriopoulos2000enhancing,quinn1985managing}, as failing is widely considered as essential part of learning \citep{farson2002failuretolerantleader}. Thus, employees should feel being allowed to conduct experiments and despite the outcome of the experimentation \citep{jung2003role}.

\citet{garvin2008yours} divide organisational failure into three categories: unsuccessful trials, system break-downs and process deviations. In this thesis, unsuccessful trials refer to experimenting. In order to learn and develop, these all types of failures need to be recognised and their special characteristics analysed. For instance, experimenting may foster creative learning, but as important is to overcome failures resulting from deeply ingrained norms that inhibit experimenting. \citep{garvin2008yours}

According to various studies, failing and negative consequences are natural part of creative, innovation and learning processes \citep{hennessey19881,shalley2004leaders,andriopoulos2000enhancing} . For instance, \citet{hennessey19881} emphasise that process is likely to have negative consequences, and in the concept of perpetual challenging of \citet{andriopoulos2000enhancing}, adventuring phase includes making mistakes. Thus, when developing novel products and processes, iteration and failure is included, employees need to feel safe to try various approaches and fail \citep{shalley2004leaders}. As discussed earlier, organisational culture has great influence on employee's perception of safe environment for failing.

\subsection{Failure as a benefit}
Interestingly, studies show how nominal groups perform remarkably better in ideation and brainstorming processes by producing greater amount of ideas than real groups. This may be due to the learnt practices and norms of a real work group, fear of failure that prevents free idea exchange and fear of evaluation and others judgement when suggesting creative solutions. \citep{jung2001transformational}

However, \citet{thomke2001enlightened} does not suggest failing and making mistakes as a result of poorly planned experiments. Mistakes and failures produce most value, when the experiment is well planned and the goal or hypothesis that needs to be tested is clear. Thus, failed experiments should not be considered as failing, instead they offer valuable learning points.

Failing early and often, yet avoiding mistakes is important for experimenting. Failure can disclose important information and reveal gaps in knowledge, and is thus important as early phase of the development process as possible. However, according to \citet{thomke2001enlightened}, this is not an usual way for an organisation to think about failure, thus building the capacity for rapid experimentation as well as tolerating and learning from failure is essential and often requires overcoming ingrained attitudes. Encouraging and creating a culture where failing is allowed and not being afraid of, for instance brainstorming sessions where judgement is not allowed are important. \citep{thomke2001enlightened}

Furthermore, employees are more likely to contribute ideas when failing is considered safe. Leaders should emphasise that constant experimenting requires failing early and often and through these iterations learning is possible. Equally important is that employees should feel not being punished nor humiliated for negatives outcomes, if mistakes occur or whatever ideas are spoken up. \citep{amabile2008creativity, amabile1996assessing} Also \citet{de2001minority} have found a positive relation between employee's creativity and participative safety, referring to employee's perception of generating ideas without being judged. According to \citet{amabile1996assessing} creative solutions in an organisation can be achieved by encouraging employees to reach and experiment new perspectives and ways of performing. Organizational environment that allows failing is likely to assist in employees acquiring diverse perspectives and questioning the status quo and habitual way of performing. 

According to \citet{amabile2008creativity} essential part of creating a safe environment for creativity is managers to decrease the fear of failure. Instead, constant experimenting should be the goal of working, learning by doing and iterating until sufficiently is learnt from the process. Furthermore, when company grows, it usually leads to more conservative actions and increase in fear of failure. When fearing failure managers tend to deny failure and erase it from the memory instead of learning from it. \citep{amabile2008creativity} 

Also \citet{edmondson1999psychological} relates experimenting tightly to failing, and emphasises team learning, creative problem solving, reflection and overall organisational performance rising from the failure that occurs. However, willingness to interpersonal risk-taking is tightly related to how employees perceive and believe team members or leaders would react and response in uncertain actions or ideas \citet{edmondson1999psychological}. Thus, team's tolerance for imperfection and error should be increased. 

Especially when brought to innovation perspective, however, innovation processes always include mistakes and failure, from which organisations, teams and leaders need to learn, preferably rather fast. Organisation that learns fastest is likely to take the lead, which is the essence of innovation according to \citet{buijs2007innovation}.
