\chapter{Experimentation-driven innovation}
Tsekkaa t�m�! 
How to Design Smart Business Experiments
by: Thomas H. Davenport
Harvard Business Review, Vol. 87, No. 2. (February 2009), pp. 68-76  Key: citeulike:4008535

poisj��nytt� tavaraa, ehk� k�yttist� my�hemmin

In order teams to function under ambiguous circumstances, they need to feel safe to ask questions, seek help and tolerate mistakes. Team members, with the help and support of managers and leaders can assist each other in providing required safety. \citep{edmondson1999psychological}

Work needs to meet the skills and interests of employees while offering sufficient level of challenges in order to increase the motivation of employees towards work \citep{amabile1998kill}.

Question of team composition comes relevant especially when forming teams for innovation. According to\citet{buijs2007innovation} innovation teams are the heart and the engine of innovation process and essential for the rest of the organisation to accept changes and innovation results. As he suggests right people in the team is the premise for innovation, all the members should be chosen carefully starting from the leader. Furthermore, the leader should be allowed to affect on the formation of the rest of the team in order to ensure positive base for teamwork and innovation process. Accordingly, team membership should be based on voluntary. \citep{buijs2007innovation}


According to \citet{schein2010organizational}, concept of culture refers to and helps to explain some seemingly incomprehensible and irrational aspects of what is going on in groups, organisations and other kinds of social units, that share history. \citet{schein2010organizational} divides culture into three levels: artefacts (visible and feelable structures and processes and observed behaviour), espoused beliefs and values (ideals, goals, aspirations, ideologies and rationalisations) and basic underlying assumptions (unconscious, taking-for-granted beliefs and values). Climate of the group should not be mixed with culture of the group, it should rather be considered among artefacts. However, essential point of view \citet{schein2010organizational} provides is how culture in organisation or group level is easy to observe yet very difficult to decipher. Put in other words: researchers are able to observe and make remarks on what they see and feel, yet they are unable to reconstruct the deeper meaning of those observations to the group. Cultural analysis and understanding of dynamics of a group should begin in observing and asking members the norms, values and rules that shape practicalities of work in day-to-day level.

The approach of leaders oftentimes divides into task-oriented or relationship-oriented. Task-oriented leaders value performing the job, focusing on clarifying roles and responsibilities, monitoring work while managing time and resources. In turn, relationship-oriented leaders value socioemotional aspects of work through empathetic actions, showing consideration for employees, being friendly and supporting the team personally. Should be noted that in literature concerning leader behaviour, term support refers to relationship-oriented leadership behaviour, wheres in creativity literature same term refers to both task- and relationship-oriented behaviours and actions - all that are to foster creativity. In this thesis, latter and broader usage of term support is used. \citep{amabile2004leader}

Rather than managing the inevitable chaos of innovation productively, these managers soon drive out the very things that lead to innovation in order to prove their announced plans. In the name of efficiency, bureaucratic structures require many approvals and cause delays at every turn. Experiments that a small company can perform in hours may take days or weeks in large organizations. The interactive feedback that fosters innovation is lost, important time windows can be missed, and real costs and risks rise for the corporation. Inappropriate incentives. Reward and control systems in most big companies are designed to minimize surprises. Yet innovation, by definition, is full of surprises. It often disrupts well-laid plans, accepted power patterns, and entrenched organizational behavior at high costs to many. Few large companies make millionaires of those who create such disruptions, however profitable the innovations may turn out to be. When control systems neither penalize opportunities missed nor reward risks taken, the results are predictable." \citep{quinn1985managing}

 
Particularly when large new, complicated systems at hand, meaning of co-operation in production, development and communication rises exponentially. Especially in large organisations innovation can be inhibited by the errors increasing as a result of complexity of the system and inability to control, understand or make intelligent decisions. Challenging as it is for one department, faculty or company to survive on its own without communication and help of others in design, production and other business-related decisions, with management that takes the complex environment into account, the disastrous effects resulting from lack of communication can be lessen. \citep{quinn1985managing} Yet, at the same time, as a result of the difficulty of managing complex situations, innovation may denote finding the core, boiling things down and focusing on the most essential elements \citep{katz1978social}.

\citet{quinn1985managing} list several barriers to innovation, including intolerance of fanatics, short time horizons, accounting practices, excessive rationalism and bureaucracy and inappropriate incentives. \citet{hayes1982managing} supplements the list with concern of top management isolation, arguing how top management oftentimes has too little contact and understanding of the environment and conditions at factory floor or customer requirements for innovative solutions. Top managers who tend to be financially-driven and are not familiar nor have experience with current technology and its possibilities, may fear technological innovations and perceive them as too risky. Thus, more familiar traditions remain with ease. \citep{hayes1982managing} 

Amabilen (1997) tekstiss pdf:n sivulla 17 on hyva kiteytys Amabilen osalta, mita innovointi vaatii organisaatiolta. 
Expertise 
- Expert performance and its affects on implementing ideas \citep{ericsson1994expert} (tsekkaa artsu viel�)


Several types of teams function in organisations, type depending on various dimensions such as cross-functional versus single-function, time-limited versus enduring and manager-led versus self-led. These dimensions should be recognized and team learning fostered depending on the type. \citep{edmondson1999psychological}

Thus, leaders should allow team members time to think creatively \citep{amabile2002creativity}.

In previous research, structural and design-related factors have been combined to have influence on work teams's effectiveness and team performance. Well-designed tasks and goals, suitable and functional team composition, as well as physical environment and practices ensuring transparent communication and information exchange, sufficient materials, resources and motivating rewards all affect team efficiency. \citep{hackman1987design,goodman1988groups,campion1993relations}. 

 \citet{quinn1985managing} offers an explanation why it seems easier for engineer and scientific leaders to create atmosphere supporting innovation: understanding and psychological comfort are related to familiarity, and engineers, for instance, have wider understanding and knowledge of technology, which makes newer technological innovations easier to accept and adapt.
 
 Accordingly \citep{amabile2004leader} divide required behaviours of leaders for providing support into two categories: instrumental or task-oriented and socio-emotional or relationship-oriented actions.
 
  Likewise, collective organisational achievements can be affected through affecting working environment and organisational culture and leaders influence on employee's attitudes and motivation towards work \citep{amabile1998kill}.
  
  According to \citet{bass1997full} transformational leadership consists of four unique yet interrelated behavioural components: inspirational motivation (articulating long-term vision), intellectual stimulation (promoting creativity and innovation), idealised influenced (meaning charismatic role modelling) and individualized consideration referring to coaching and mentoring leadership style. 
  
  \citep{jung2001transformational} has studied how leadership style affects group's creativity and performance by comparing transactional and transformational leadership styles. Furthermore, development of clear long-term vision and practises supporting the way to achieve it is essential characteristic of transformational leaders \citep{avolio1988transformational}. The relationship between transformational leader and an employee is active and emotionally attached \citep{avolio1988transformational} and through the strong attachment resulting from tight relationship leaders can better support employees in using their personal values and self-concepts in the way that employees can pursue higher level performance and fulfil personal needs through the work. This focus of transformational leadership on value alignment is likely to lead to the root of intrinsic motivation of an employee \citep{gardner1998charismatic}, which is considered as one of the key elements in creative thinking and innovation skills of an employee (eg. \citep{jung2001transformational,amabile1998kill,deciintrinsic}).
  
  Few studies have been made linking the transformational leadership and positive outcomes of employees' creativity in organisational level and outcomes \citep{jung2003role}, even though several studies have been made revealing the positive relation between these factors. in their study \citet{jung2003role} draw this link clearer and suggest that while leaders define the context and goals of their employes, transformational leadership can be extrapolated to an organisational level.  
  
  In their study, \citet{shalley2004leaders} present how leaders should use human resource practices in order to develop work context which improves the creativity skills of employees. 
  
  In his study, \citet{jung2001transformational} emphasises that transformational leadership skills can be practised in order to foster creativity and intellectual skills of employees and shape organisational culture. His study showed how transformational leadership; encouraging divergent thinking and solving problems at hand from unconventional perspectives, is likely to increase intrinsic motivation of employees leading to more creative problem solving and behaviour.  
  
  According to \citet{jung2003role} transformational leadership is positively related to organisational innovation, employee's perception of empowerment and support for innovation. Furthermore, the perception of empowerment is positively related to organisational innovation, and when perception being strong,  the relationship between transformational leadership and organisational innovation tends to be stronger. Results of the study conducted on 32 Taiwanese companies suggest that through transformational leadership by top managers organisational innovation can be affected directly or indirectly, latter referring to creating an organisational culture in innovation, discussion, novel approaches and experimenting is encouraged. \citep{jung2003role}
  
  Organizational leaders play a great role in establishing strong team performance culture. This can be achieved through addressing and demanding performance that meets the need of customers, employees and shareholders. Teams should not be fostered by the sake of the team only, rather should leaders clearly state how the team performance affects to customers and through that foster clearer performance ethics and cultures. In addition, even though people tend to have great sense of individualism, it does not have to bias the teamwork performance, as real teams find ways to support individual strengths and performance for shared goal. Furthermore, in order to team function properly and efficiently, discipline across the team and organisation is needed, focusing again on performance.  \citep{katzenbach1993wisdom}
  
  Although different leadership styles and their effect on employee's creativity behaviour has not yet been studied widely, some studies show, how transformational leadership behaviour encourages employees look problems from different perspectives and thus widen their intellectual and creativity skills \citep{jung2001transformational,sosik1998transformational}. \citet{jung2001transformational} has studied the relation between leadership style and group creativity finding that transformational leadership is most likely to stimulate creative effort of employees. 
  
  As \citet{buijs2007innovation} argues, leaders who are to lead employees and work handling innovations need to understand the paradox and natural conflicts between routine processes (exploitation) in order to earn money in the present and the innovation processes (exploration) in order to earn money in the future. \citet{buijs2007innovation} four aspects for innovation which leader should be able to master all providing a secure environment for a team to perform in novel and creative ways. These consist of innovation process, psychological process of innovation team, creativity process, and leading and playing. 

Organizational structures are also likely to enhance or hinder creativity in organisational, team or individual levels \citep{shalley2004leaders}. 

Recent studies has moved the focus from individual learning to team learning. Edmondson's definition of group learning stems with definition of \citet{argote2001group}, who emphasises that knowledge is acquired, shared and combined through processes and outcomes of group interaction, focus being on processes. 

Organizational learning differs from individual or team learning. Organizational learning occurs through the shared knowledge, insights and approaches of the employees of an organisation. Secondly, organisational learning is based on prior knowledge and experience, the memory of organisation, which consists of the ways of working, processes and instructions of an organisation. Even though individual and team learning are highly related to organisational learning, it is not the sum of the previously mentioned. \citep{sydanmaanlakka2007}


\citet{thomke1998agile} defines a term development flexibility, which refers to 
It is an alternative for forecasting the future and works as a powerful method in risk-managing of development. As forecasting the future has become increasingly difficult, emphasis should be put on managing the risks of failed decisions. Shorter development cycles
Thus, development flexibility is critical as companies are no longer (if they ever actually were) possible to accurately forecast unknowable future. Furthermore, unknown, unstable and changing customer needs 
Through development flexibility entire product changes can be avoided, as design commitment and decisions can be made late phase in the process. Furthermore, significant reduce can be seen in time and cost used for development. 

"Development flexiblllty can be expressed as a function of the incremental economic cost of modifying a product as a response to changes thai are external (e.g., a change Ui customer needs) or internal (e.g., discovering a better technical solu- tion) to the development process. The higher the economic cost of modifying a produa, ihe lower the development flexibility."\citep{thomke1998agile}

"While this argument leads to the somewhat intuitive idea that consistently encouraging organizational conditions would lead to more experi- mentation behaviors than inconsistent conditions, it also suggests a less intuitive scenario. Specifically, it is possi- ble that individuals will engage in more experimentation behavior when organizational conditions consistently discourage experimentation than when some conditions encourage experimentation and others do not. In the con- sistently ?discouraging? situation, individuals are clear about the rules and constraints they encounter, and there- fore may experience more psychological safety than they would when facing the uncertainty created by inconsis- tent conditions. If so, they may experiment more.
Further, in consistently discouraging conditions, peo- ple working closely together can experience a sense of solidarity based on shared perceptions of negative work conditions (Edmondson 1999, George and Zhou 2002), whereas inconsistent conditions may lead to mistrust and suspicion that undermine psychological safety. " (lee2004mixed)

"The combinational perspective has been used to examine measures of organizational performance such as productivity, manufacturing quality, and efficiency, but it has not been applied to the study of individ- ual behaviors within the organization. Meyer et al. (1993) found that current theories of individual and group behavior in organizations, ranging from per- sonality, motivation, task design, work group design, and organizational demography, have largely adopted a componential rather than combinational approach." (lee2004mixed, t�h�n systeemiajattelupointti) 

"The componential perspective suggests that organizational conditions (such as normative values, instrumental rewards, or evaluative pressure) indepen- dently affect innovation behaviors. Thus normative val- ues that encourage experimentation will lead to higher levels of experimentation behavior, regardless of instru- mental rewards and evaluative pressure; instrumental rewards that do not punish failures will lead to higher levels of experimentation behavior, regardless of nor- mative values and evaluative pressure; and individuals under high evaluative pressure will experiment less than individuals under low evaluative pressure, regardless of normative values and instrumental rewards.
The combinational perspective suggests that the inter- action between organizational conditions is also impor- tant. This perspective suggests that when organizational conditions are consistent?for example, when norma- tive values, instrumental rewards, and evaluative pres- sure all encourage experimentation?there will be more experimentation than when organizational conditions are inconsistent?when some encourage experimentation and some discourage experimentation. This perspective also allows the possibility that experimentation will be greater when organizational conditions consistently dis- courage it than when they are inconsistent." (lee2004mixed)

"Evaluative pressure is distinct from coaching, in which close attention or monitoring is provided to facilitate rather than evaluate performance. Indeed, monitoring in the context of supportive coaching can actually enable interpersonal risk taking (Edmondson 1999, 2002), while close and constant evaluation intended to identify and expose failures has been shown to inhibit creativity (Amabile et al. forthcoming) and make novel or unfa- miliar tasks more difficult (Zajonc 1965). " (lee2004mixed)

For enlightened experimentation \citet{thomke2001enlightened} puts emphasis on combining new and traditional technologies. 

According to \citet{sternberg1997creativity} a company can enhance its creative skills by focusing on six resources: knowledge, intellectual abilities, thinking styles, motivation, personality and environment. \citet{sternberg1997creativity} argues that too much information may hinder change and be seen as rigidity in thinking. Therefore, one should not over-weight the criticism of senior people in an organisation, and at least consider the chance for rigid thinking and intolerance for change. 

\citet{kasof1997creativity} argued in his study that breadth of attention affects on creative performance of an individual: wide spread of attention is usually related to creative ability. By breadth of attention Kasof refers to "number and range of stimuli attended to at any time." Breadth of attention being narrow, individuals are able to focus on narrow range of stimuli and are better at filtering redundant stimuli from awareness. However, those individuals with wide breadth of attention tend to be more aware of irrelevant or extraneous stimuli, these individuals are strongly affected by their environment and are highly arousable. \citep{kasof1997creativity}

Study of creativity is a combination of two different disciplines and research approach: sociological and historical lenses study the conditions in which creative actions and processes are likely to occur, whereas neurobiological approach presents neural structures and processes that are active and associated with creative outcomes. \citep{gardner1988creativity}

Meaning of environment in learning should be emphasised. Learning concerns of transaction between an individual and the environment, learning does not happen only inside of individual's thoughts, experiences and processes but is dependent on the real world environment. Understanding of different learning styles and modes assists in supporting individuals in learning and problem-solving.

Even though novel experimentation techniques are useful and may reduce the cost and time used for product development, adopting experimentation techniques may require changes in organisational culture or way of doing work \citep{thomke1998modes}. However, adapting changes may lead to increase in productivity and affect the overall competitive positioning among companies. In this thesis experimentation-driven approach is presented as one possible approach to development and learning and this thesis aims to deepen the understanding of the requirements for an organisational unit to take experimentation driven approach to developing instead of planning. 


"Of course, in a complex social situation, where many causes operate, not all of which are controllable \citet{katz1978social}, innovation may ultimately depend on boiling things down to their essential and focusing on those essential elements of the situation you can do something about. \citet{mumford2002social} Experimenting Voisko t�m� toimia pointtina sille, ett� kokeileminen auttaa keskittym��n sellaiseen, ja vain sellaiseen, johon voi vaikuttaa itse. Eli discussioniin 

At present the informal structure for developing daily work can roughly be divided as follows: the idea emerges from work experience of an employee, a problem at hand, from customer or stakeholder need or by accident during a conversation with colleagues or friends. According to the interviewees, coincidence and problem-based approach play major roles at the moment. After the idea has emerged, support for the idea is asked from a trustworthy colleague and only after that it is democratically discussed in a team meeting. In order an idea to turn into experiment or a new routine, essential arguments and major part of employees ready to engage to the experiment are needed. When an idea is agreed to turn into an experiment, responsibility has to be divided and responsible employees chosen. After that a project team that continues with turning the idea into experimentation is collected. 

According to \citet{mumford2002social} characteristics associated with innovation are integration of work units, decentralisation of control and professionalisation are likely to effect innovation in a way that through these suitable environment for innovation, dynamic idea exchange and implementation is created. Playful attitude should be allowed and encouraged in innovation process \citep{bujis2007innovation}. 

Thus, in order to drive for social innovations, opportunism and showmanship of an individual or team may be required. \citep{mumford2002social}  However, \citet{amabile2004leader} emphasise how, ultimately, truly novel ideas raise from individuals, making them the ultimate source of any new idea or solution to a problem.

Social innovation refers to the generation and implementation of novel ideas concerning people in demand to organise their interpersonal or social activities and interactions in new ways in order to achieve common goals. Results and products of social innovation, like other types of innovation, are likely to vary depending on the breadth and impact of the innovation. \citep{mumford1988creativity} \citet{mumford2002social} presents four factors affecting social innovation: active exchange of ideas and information in supportive climate, tangible and low-cost ideas that can be at the fewer guessed to be beneficial, support from upper level management, and effective communication through the innovation process in order to proceed from the idea to implementation.

Even though literature on innovation focuses on organisational-level structures and processes, the innovation process and organisation's ability to launch a new product or service, create new value and processes as well as leverage novel technologies begins with individual employees presenting their ideas out loud and trying out novel approaches. \citep{argote2000knowledge} Thus, forming understanding of conditions that foster experimentation behaviour of individuals is important in order to understand organisational innovation \citep{thomke2003r}.

Furthermore, \citet{edmondson1999psychological} apply the term learning behaviour to separate it from learning outcomes, and states how set of several activities form the basis of learning behaviour. 

Class 1: Factors affecting experimentation 

Category 1: Role of the immediate superior
Refers to the significance of leadership behaviour in fostering experimentation behaviour, consisting of leading by example, giving licenses to conduct experiments and showing support for experimenting and ideation.

Category 2: Role of the team 
Category 3: Structures and practices of developing 
Category 4: Characteristics and know-how of an employee 
Category 5: The gap between an idea and experiment

Class 2: Effects of experimenting on individual

Category 6: Emotional experience and engagement
Category 7: Learning 
