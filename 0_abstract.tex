% Abstracts
% ------------------------------------------------------------------
% Include an abstract in the language that the thesis is written in and in Finnish

% Abstract in English
% ------------------------------------------------------------------
\thesisabstract{english}{
Significance of organisation's ability to learn has been recently recognised in both academic and business world, especially in the context of new business ventures and innovation. Organisations that are able to learn faster than rivals and are thus better at adapting to changes in business environments are claimed to gain better competitive advantage. Recently, creativity and innovation abilities are related to be essential for employee's performance at work. Unpredictable, complex an uncertain environments require ability from both organisation and its employees to adapt to changes in creative ways that foster innovation. Thus, need for non-predictive approaches that support learning and growth in organisational and individual level exists. 

In this thesis, experimentation-driven approach for development is presented as a method for learning and building competitive advantage in an organisation. Furthermore, factors affecting experimentation behaviour are examined and framework for supporting environment for experimenting is created. 

Case study method was used in the study where client organisation was instructed to apply experimentation-driven approach during a six-week experimentation challenge aiming for employees to create novel ideas to develop their work and rapidly experiment those ideas. To study the factors affecting experimentation behaviour, an interpretive approach together with thematic analysis was used. The data consisted on 14 semi-structured interviews. 

As a result from this study, factors affecting experimentation behaviour in an organisation are presented. Additionally, light is shed on on how experience of experimenting affects on individual. Furthermore, the findings of the research form a synthesis of a supporting environment for experimentation-driven development. 

Organisational and business environment being a complex system, various factors affect on the behaviour of employees and willingness to conduct experiments. Ability for holistic thinking and sensitivity for interdependencies is required. Experimentation behaviour is likely to be fostered by assuring safe environment for experimenting, supporting leadership behaviour, and organisational structures and practices supporting experimenting. 

This thesis was done as a part of the two-year MINDexpe research project, undertaken by the MIND research group of Aalto University and funded by Tekes.
}

% Abstract in Finnish
% ------------------------------------------------------------------
\thesisabstract{finnish}{

}
