\documentclass[11pt, oneside]{article}   	% use "amsart" instead of "article" for AMSLaTeX format
\usepackage{geometry}                		% See geometry.pdf to learn the layout options. There are lots.
\geometry{letterpaper}                   		% ... or a4paper or a5paper or ... 
%\geometry{landscape}                		% Activate for for rotated page geometry
\usepackage[parfill]{parskip}    		% Activate to begin paragraphs with an empty line rather than an indent
\usepackage{graphicx}				% Use pdf, png, jpg, or eps§ with pdflatex; use eps in DVI mode
								% TeX will automatically convert eps --> pdf in pdflatex		
\usepackage{amssymb}
\setcounter{secnumdepth}{5}

\title{Experimenting with the brain in mind - how to avoid threat responses while creating something new?}
\author{Marianne Tenhula}
%\date{}							% Activate to display a given date or no date

\begin{document}
\maketitle

%:Acknowledgements
\section{Introduction}
This thesis aims to combine and reflect the knowledge based on both neuropsychology and management. It suggests experimentation driven innovation as a method for efficient change management. 

The thesis was written as a part of a two-year research project called MindExpe studying experimentation-driven innovation at MIND research group, Aalto University. MIND operates under the Business, Innovation and Technology (BIT) research centre, which is part of Department of Industrial Engineering in Aalto University School of Science. MIND research group is based on Aalto Design Factory. Tekes-funded MINDexpe project studies innovation and development in established organizations through experimentation-driven approach. 

The research team instructed a client organization on using experimentation-driven approach by organizing an experimentation challenge where the units of the client organization were tasked to create, develop and report new ideas during a six-week time period. Instructions were given before the challenge; during the challenge further instructions were not given. 

In MINDexpe client organizations are tasked to use the experimentation-driven approach instead of more traditional planning-based approaches to development. This thesis aims to discover how various organizational conditions may affect the experimentation behavior. The larger aim of the MINDexpe project is to widen the understanding of experimentation-driven innovation itself.

To study the factors affecting experimentation behavior, an interpretative approach was used with thematic analysis as a method for analyzing the data. The data consists of 14 semi-structured interviews of client organizations members from five different units. The analysis of the data demonstrates various requirements for supporting experimentation behavior in developing in an organization. Furthermore, the data revealed that experimentation behavior has various affects on an individual?s performance and personality and emotional level. 

\subsection{Background}
Mind research group focuses on studying experimentation and its impacts on organisational and individual level. Experimenting is mainly described and used as a tool for developing, creating something new. 

In this thesis experimentation-driven approach is presented as one possible approach to development and this thesis aims to deepen the understanding of the requirements for an organizational unit to take experimentation driven approach to developing instead of planning. 

A combination of case study and action research was used 



\subsection{Research objectives}
- About organisations competitiveness, how nowadays it is lot about how fast organisation learns and adapts new things. Systems thinking? 
- About learning and organisational well-being in organisational, team and individual level

Furthermore, the motivation for this study was to find factors that affect on experimenting in supportive or preventing way. In addition the aim was to establish the experimentation-driven approach and identify how this approach could be supported in an organization culture. 

This thesis aims to define what can be done in organisation in order to allow experimentation culture to happen. 

Research objectives: 

What is organisational learning? 
How can organisational learning be supported? 
Can experimentation behaviour be seen as a method for organisational learning? 
How can experimenting support organisational learning? 
How an organisation can support individual learning and growth? 

\subsection{Methodology}

\subsection{Structure of the thesis}
This thesis consists of theoretical and empirical part. After the introduction, the theoretical part of the thesis begins. First of all, it forms an understanding of what can be understood by experimenting and experimentation-driven innovation, which is the focus of study at Mind research group. Secondly, in the heart of every change and development project there are the employees, the group of individuals who are touched by the change. In order to proceed to meaningful and efficient changes for both the company and its employees, individuals has to be onboard.  One reason for resisting change is the threat that an individual encounters, which goes to very primitive parts of the brain. Thus, the second part, chapter x, of the theoretical framework presents the threat-reward response in brain, and binds that together with our everyday life, especially work life. 

Chapter y presents the research design, including surroundings,  case company description and methodology used in the study.   It clarifies the experimentation challenge organised for the case company, explains data gathering methods and sheds light on the analysis process done. 

After this, in chapter r, the results of the data are presented. Two main concepts were recognised from the data: factors affecting experimentation behaviour and effects experimenting has on individual. These results are further reflected to the theory framework provided in chapter xx (Scarf). Following chapter consists of the discussion, where implications of the results are analysed. 

The final chapter of the thesis consists of the brief conclusion drawn from both the theoretical framework, empirical data results and discussion. Furthermore, practical implications of the study are being evaluated as well as suggestions for future research. Lastly, reliability of the thesis is being analysed. 


%\section{Literature review: organisational development - learning organisations}
%\subsection{Supporting structures}
%\subsection{Brainstorming and idea implementation}
%\subsection{Social support}
%\subsection{Experience and characteristics of an individual}
%
%\subsection{Learning organiazations}
%According to Garvin et al. (2008) a learning organisation is built of supportive learning environment, concrete learning processes and practices and leadership behaviour that reinforces learning. 
%
%
%Keskusteluta aineistoa siten, ett� lukija pystyy ymm�rt�m��n, miten t�m� liittyy aiheeseen. 
\section{Change and learning in an organization}

Current and future business environment requires continuous innovation from organisations, meaning deploying the collective knowledge, skills and creative efforts of their employees \cite{dess2001changing}(Dess and Picken, 2000). 

"As budgets are squeezed tighter and margins
of profit grow smaller, ideas are a precious
commodity and employees who produce them
become sought-after resources. Within such a
competitive business environment
companies nowadays increasingly strive to
become innovative organisations." \cite{andriopoulos2000enhancing}

"Change. The rapid technological
advancements and the fierce competition
for market share have contributed to the
unprecedented increasing pace of change.
Therefore, organisations should be ready
to rearrange their resources to meet the
new demands." \cite{andriopoulos2000enhancing}

Wide access to the information has put tremendous pressure on today's business and companies to increase their efficiency and effectiveness. Concurrently, demand for creative endeavours has rose in order to improve and develop products and processes. (Andriopoulos and Lowe, 2000; Cummings and Oldham, 1996)

Today, economy is driven by innovation and innovativeness, requiring new understanding and abilities to generate great ideas in order to survive in every business and business level. innovativeness calls for creativity, which again calls for new managing skills of leaders. In contrast to many leaders beliefs, creativity and creative individuals can be managed and encouraged. Amabile and Khaire (2008) \citeasperson{amabile2008creativity} 

Although currently creativity and creative processes of an individual at work are rather well recognised and essential, even more focus should be put on organisations' ability to mobilise creative actions of employees to create novel, socially valued products or services and more efficient ways of working (Mumford and Gustafson, 1988). In other words, creative actions of an employee are not worthwhile to an organisation when not coordinated or harnessed to yield organisational-level outcomes (Jung et al. 2003) 

"Increasingly, creativity has become valued across a variety of tasks, occupations, and industries. In today's fast-paced dynamic work environment, managers continue to realize that to remain competitive they need their employees to be actively involved in their work and trying to generate novel and appropriate products, processes, and approaches. Although the level of creativity required and the importance of creativity can differ depending on the tasks or job in question, most managers would agree that there is room, in almost every job, for employees to be more creative. Further, because individual creativity provides the foundation for organizational creativity and innovation (Amabile, 1988), and these have been linked to firm performance and survival (Nystrom, 1990), it is important, if not critical, that employees are creative in their work." (Shalley and Gilson 2004) 


According to Schein (2010), concept of culture refers to and helps to explain some seemingly incomprehensible and irrational aspects of what is going on in groups, organisations and other kinds of social units, that share history. Schein divides culture into three levels: artefacts (visible and feelable structures and processes and observed behaviour), espoused beliefs and values (ideals, goals, aspirations, ideologies and rationalisations) and basic underlying assumptions (unconscious, taking-for-granted beliefs and values). Climate of the group should not be mixed with culture of the group, it should rather be considered among artefacts. However, essential point of view Schein (2010) provides is how culture in organisation or group level is easy to observe yet very difficult to decipher. Put in other words: researchers are able to observe and make remarks on what they see and feel, yet they are unable to reconstruct the deeper meaning of those observations to the group. Cultural analysis and understanding of dynamics of a group should begin in observing and asking members the norms, values and rules that shape practicalities of work in day-to-day level.


\subsection{Learning as competitive advantage and way to survive}

Understanding change analytically in the turbulent world appears challenging. Change being hectic and fast calls for different skills and strategy than before. Only when change is understood can it be managed, and in order to survive new perspective and understanding towards change is required from an organisation. 

In the changing environment tolerance for uncertainty is needed, and while future can not be predicted, forecasting is a usable method in order to cope with the anxiousness resulting from the uncertainty. Furthermore, even more emphasis should be put on ability to learn and adapt to changes. 

"In this study, my focus on learning behavior and its accom- 
panying risk made the interpersonal context especially sa- 
lient; however, the need for learning in work teams is likely 
to become increasingly critical as organizational change and 
complexity intensify. Fast-paced work environments require 
learning behavior to make sense of what is happening as 
well as to take action. With the promise of more uncertainty, 
more change, and less job security in future organizations, 
teams are in a position to provide an important source of 
psychological safety for individuals at work. The need to ask 
questions, seek help, and tolerate mistakes in the face of 
uncertainty-while team members and other colleagues 
watch-is probably more prevalent in companies today than 
in those in which earlier team studies were conducted" (Edmondson 1999)

\subsection{Change in organisational level}
Hammer and Champy (1993) have summarised aspects of change in organisational environment, beginning from the change in organisational structure; from functional departments to process teams. Work tasks change from simple and detailed tasks to multi-dimensional knowledge work while employees are becoming more autonomous instead of strict control. Furthermore, instead of educating, focus is in the learning of an employee, and evaluation of work will change from operations to outcomes. Knowledge and capability are preferred over single performance and values change to more productive behaviour than over-protective. Superiors turn from leaders of the work to coaches and hierarchical organisational structures turn lower while managers focus on leadership instead of task management. 

According to Arie de Geus (1997) the only one can maintain company's competitive advantage is to make sure the company is able to learn faster than rivals. Generally organisations are considered as machines, yet recently more emphasis has been put on organisations as living organisms. When considered as machine, organisational model is mechanic and simple, which purpose is to gain profit. Whereas, organisation as a living organism is a whole-systemic model, and organisations are considered as place which has deeper, permanent meaning offering people the opportunity to grow and fulfil themselves while earning money.

Liable vision of the future focuses on the latter perspective of organisations, where learning and renewal form the essence of being. 

According to Huy and Minztberg (2003), organisations learn best through small experiments and trying out new things, and the closer and more related experimentations are to customers and customer interfaces, the more can be learned. 

\subsection{Organizational learning}

"Organizational learning is presented in the literature in two 
different ways: some discuss learning as an outcome; others 
focus on a process they define as learning. For example, 
Levitt and March (1 988: 320) conceptualized organizational 
learning as the outcome of a process of organizations "en- 
coding inferences from history into routines that guide be- 
havior"; in contrast, Argyris and Sch6n (1978) defined learn- 
ing as a process of detecting and correcting error. In this 
paper I join the latter tradition in treating learning as a pro- 
cess and attempt to articulate the behaviors through which 
such outcomes as adaptation to change, greater understand- 
ing, or improved performance in teams can be achieved. For 
clarity, I use the term "learning behavior" to avoid confusion 
with the notion of learning outcomes. " (Edmondson 1999)

"The conceptualization of learning as a process has roots in 
the work of educational philosopher John Dewey, whose 
writing on inquiry and reflection (e.g., Dewey, 1956) has had 
considerable influence on subsequent learning theories (e.g., 
Kolb, 1984; Schbn, 1983). Dewey (1956) described learning 
as an iterative process of designing, carrying out, reflecting 
upon, and modifying actions, in contrast to what he saw as 
the human tendency to rely excessively on habitual or auto- 
matic behavior. Similarly, I conceptualize learning at the 
group level of analysis as an ongoing process of reflection 
and action, characterized by asking questions, seeking feed- 
back, experimenting, reflecting on results, and discussing 
errors or unexpected outcomes of actions. For a team to dis- 
cover gaps in its plans and make changes accordingly, team 
members must test assumptions and discuss differences of 
opinion openly rather than privately or outside the group. I 
refer to this set of activities as learning behavior, as it is 
through them that learning is enacted at the group level." (Edmonsdon 1999)

"This conceptualization is consistent with a definition of group 
learning proposed recently by Argote et al. (2001) as both processes and outcomes of group interaction 
activities through which individuals acquire, share, and com- 
bine knowledge, but it focuses on the processes and leaves 
outcomes of these processes to be investigated separately. " (Edmondson 1999)

"The management literature encompasses related discussions 
of learning, for example, learning as dependent on attention 
to feedback (Schon, 1983), experimentation (Henderson and 
Clark, 1990), and discussion of failure (Sitkin, 1992)" 

"Similarly, because errors provide a source of information about performance by revealing that something did not work as planned, the ability to discuss them productively has been associated with organizational effectiveness (Michael, 1976; Sitkin, 1992; Schein, 2010) (Edmondson)

"The construct has roots in early research on organiza- 
tional change, in which Schein and Bennis (1965) discussed 
the need to create psychological safety for individuals if they 
are to feel secure and capable of changing. Team psycho- 
logical safety is not the same as group cohesiveness, as re- 
search has shown that cohesiveness can reduce willingness 
to disagree and challenge others' views, such as in the phe- 
nomenon of groupthink (Janis, 1982), implying a lack of inter- 
personal risk taking. The term is meant to suggest neither a 
careless sense of permissiveness, nor an unrelentingly posi- 
tive affect but, rather, a sense of confidence that the team 
will not embarrass, reject, or punish someone for speaking 
up. This confidence stems from mutual respect and trust 
among team members. " (Edmondson 1999)

"The importance of trust in groups and organizations has long 
been noted by researchers (e.g., Golembiewski and Mc- 
Conkie, 1975; Kramer, 1999). Trust is defined as the expec- 
tation that others' future actions will be favorable to one's 
interests, such that one is willing to be vulnerable to those 
actions (Mayer, Davis, and Schoorman, 1995; Robinson, 
1996). Team psychological safety involves but goes beyond 
interpersonal trust; it describes a team climate characterized 
by interpersonal trust and mutual respect in which people 
are comfortable being themselves. "


From the data rose various factors affecting experimentation behaviour, and the literature review revealed how similar factors are related to learning in an organisation and organisational learning. 

Through the understanding of organisational learning real growth and support for learning can be offered. Learning process needs to be understood all from organisational, team and individual perspective. 

Organizational learning is approached from two different perspectives in literature. On the one hand, learning is considered as an outcome, and on the other it is considered as a process \cite{edmondson1999psychological}(Edmondson 1999). In the first perspective organizational learning is referred to be an outcome of a process of organisations "encoding interferences from history into routines that guide behaviour \cite{levitt1988organizational}(Levitt and March 1988). Whereas process perspective define learning as a process of continuous trial and error (Argyris and Sch�n 1978). Further definitions of learning are presented in chapter zz. In this thesis, learning is considered as the latter tradition of learning, which allows the growth and improved performance of individuals and organisations.

Organizational learning differs from individual or team learning. Organizational learning occurs through the shared knowledge, insights and approaches of the employees of an organisation. Secondly, organisational learning is based on prior knowledge and experience, the memory of organisation, which consists of the ways of working, processes and instructions of an organisation. Even though individual and team learning are highly related to organisational learning, it is not the sum of the previously mentioned. \cite{sydanmaanlakka2007}(Syd�nmaanlakka 2007)

Various definitions of learning organisations have been presented. One of the most famous definitions is from \citeasperson{senge1991} Peter Senge (1991), who describes learning organisation as follows: "Learning organisation is an organisation, where people are able to constantly develop and achieve intended results; where new ways of thinking are born and where people share goals and learn together."

"Learning, analysis, imitation, regeneration, and technological change are major
components of any effort to improve organizational performance and strengthen
competitive advantage. Each involves adaptation and a delicate trade-off between
exploration and exploitation. The present argument has been that these trade-offs are
affected by their contexts of distributed costs and benefits and ecological interaction.
The essence of exploitation is the refinement and extension of existing competences,
technologies, and paradigms. Its retums are positive, proximate, and predictable. The
essence of exploration is experimentation with new altematives. Its returns are
uncertain, distant, and often negative. Thus, the distance in time and space between
the locus of leaming and the locus for the realization of returns is generally greater in
the case of exploration than in the case of exploitation, as is the uncertainty." \cite{march1991exploration}(March, 1991)

Organizational structures are also likely to enhance or hinder creativity in organisational, team or individual levels (Shalley and Gilson 2004) \cite{shalley2004leaders}. 

According to Sternberg et al. (1997) a company can enhance its creative skills by focusing on six resources: knowledge, intellectual abilities, thinking styles, motivation, personality and environment. Sternberg et al. (1997) argues that too much information of may hinder change and be seen as rigidity in thinking. Thus, one should not over-weight the criticism of senior people in an organisation, and at least consider the chance for rigid thinking and intolerance for change. 
In addition, needs to be noted and understood that employees' thinking styles are shaped through what is rewarded meaning, that if organisational environment rewards well-behaving and instruction-following thinking style and action, employees tend to implement their style to that. We are urged to adapt to organisational style and fit in, and when this is not possible, people tend to leave. 

\subsection{Team performance and learning}

In order to create new value and competitive advantage in rapidly changing and uncertain organisational environments, new managerial imperative is growing focusing on teams. Thus, supporting teams in their work and understanding the aspects of team learning is required (Edmondson 1999) \cite{edmonsdon1999psychological}. Recent studies has moved the focus from individual learning to team learning. 

Work team refers to small group of people that exist within the context of a larger organisation, members share understanding of being member of the team and its tasks, responsibility for product or service team is working on \cite{hackman1987design}; \cite{alderfer1983intergroup} (Hackman 1987; Alderfer 1983) as well as its performance \cite{edmondson1999psychological}(Edmondson, 1999). Additionally, team members have supplementary knowledge and abilities compared to each other, and they share a goal, targets and way of working and approach \cite{edmondson1999psychological}(Edmondson, 1999). According to Katzenbach (1993)  \citeasperson{katzenbach1993wisdom} great team performance consists of continuos work of shaping a common purpose, agreeing on performance goals, defining a common working approach, developing high level complementary skills and being transparent on the results. He emphasises that through disciplined action groups transform to teams and argues how demanding schedules, long-standing habits and unwarranted assumptions tend to threaten team efficiency and performance.
 
In previous research, structural and design-related factors have been combined to have influence on work teams's effectiveness and team performance. Well-designed tasks and goals, suitable and functional team composition, as well as physical environment and practices ensuring transparent communication and information exchange, sufficient materials, resources and motivating rewards all affect team efficiency. (Hackman, 1987; Goodman et al. 1988; Campion et al. 1993 ) \cite{hackman1987design}; \cite{goodman1988groups}; \cite{campion1993relations}. Along with these factors, leader behaviour plays a major role in enhancing team effectiveness, and can be facilitated for instance through coaching and setting directions to employees (Hackman 1987) \cite{hackman1987design}. This perspective explains teams effectiveness through organisation and team structures, whereas organisational learning research puts emphasis on cognitive and interpersonal variables when explaining effectiveness in teams and individuals (Edmondson 1999) \cite{edmondson1999psychological}. For instance, Argyris (1993) \citeasperson{argyris1993knowledge} has argued how individual's negative beliefs about communication and interaction may inhibit learning behaviour and lead to ineffective working in an organisation. 

In addition, in order to function team needs a clear purpose and vision what makes it a team and why it exists. Teams get energy from significant performance challenges regardless of where they are in the organisation. Set of shared, demanding performance goals usually form a team, and personal chemistry or willingness to form a team may boost that. Thus, in order to receive great results teams should focus on performance regardless of the organizational hierarchy or what team does. Thus, team performance may exceed the results of what could be achieved if employees were acting alone as individuals without the team effort. \cite{katzenbach1993wisdom} (Katzenbach 1993)

Edmonsdon (1999) \citeasperson{edmondson1999psychological} have studied factors that affect and influence learning behaviour in teams by studying in which conditions and to what extent learning occurs naturally. Learning behaviour of teams refers to activities that team members carry out and through which team is able to obtain, adapt and reflect data and outcomes of actions which further shapes and improves team behaviour. Such activities consist of reflection and improvement-aiming factors such as asking for feedback, transparent information sharing, asking for help, admitting and discussing about failures and errors as well as experimenting. Through such activities teams may observe changes in environment, customer requirements and improve collective understanding. In addition, team's ability to discover and react to unexpected situations and consequences of their actions is likely to improve through learning behaviour.  Consequently, compared to low-learning teams that tend to get stuck and be unable to solve problems, teams who master in learning are greater in confronting difficult situation and improve their work. (Edmondson 1999) \cite{edmondson1999psychological}

The composition of the team matters. Studies have shown how team performance, especially related to innovation, is improved when team consists of individuals with various and different set of skills and characteristics (Buijs 2007) \cite{buijs2007innovation}. Homogeneity in teams easily leads to groupthink, routine work and repeating traditional daily practices, while even one or two different individuals can stimulate the innovativeness of a team, and actually, the outcasts and those who stand out from the group are required in order to think outside the box, challenge the status quo and show alternative solutions and ideas that would be missing without the participations of these individuals. \cite{sternberg1997creativity} (Stenberg et al. 1997)

Question of team composition comes relevant especially when forming teams for innovation. According to Buijs (2007) \citeasperson{buijs2007innovation} innovation teams are the heart and the engine of innovation process and essential for the rest of the organisation to accept changes and innovation results. As he suggests right people in the team is the premise for innovation, all the members should be chosen carefully starting from the leader. Furthermore, the leader should be allowed to affect on the formation of the rest of the team in order to ensure positive base for teamwork and innovation process. Accordingly, team membership should be based on voluntary. \cite{buijs2007innovation}

\paragraph{Psychological safety in teams}

As in changing and uncertain environments the importance of teams have been recognized, pressure on managers to understand and enhance team efficiency, work and learning has increased. Fast-pace environment requires organisations to enhance the ability of teams to learn and create environment where learning can occurs safely. While uncertainty affects all working life, change is faster and job security lower, psychological safety for individuals at work can be increased through great teams and teamwork. Employees should be engouraged to ask questions, seek for help and tolerate mistakes and uncertainty. (Edmondson 1999)\cite{edmondson1999psychological}

Team psychological safety refers to Amy Edmondson's concept of team members shared belief team being safe for interpersonal risk-taking. Together with team efficacy these have great affect on team performance and learning in an organisational team work. Edmondson's (1999) \cite{edmondson1999psychological} integrative perspective suggests that team performance and outcomes can be shaped through both team structure and shared beliefs, in contrast to previous studies that separate structural and interpersonal factors from each other. For instance, employee's willingness to take interpersonal risks depends highly on the experience of team safety and person's beliefs how others will respond in ideas or situations involving uncertainty. Team psychological safety refers to interpersonal trust among team but beyond that mutual respect and caring. \cite{edmondson1999psychological}(Endomndson 1999)

Psychological safety serves as a mechanism that assists in explaining how structural and interpersonal characteristics both have effects on learning and performance in teams \cite{edmondson1999psychological}. Psychological safety can be boosted for instance through structural factors such as context support and team leader coaching affecting behavioural and performance outcomes (Hackman 1987; Edmondson 1999) \cite{hackamn1987design};  \cite{edmondson1999psychological}. Furthermore, climate of safety and supportivenesss encourages employees to seek for feedback and ask for help in addition to admit and reflect mistakes. \cite{edmondson1999psychological} (Edmondson 1999)

Communication about ideas among team has been widely recognized being related to idea generation, creativity and innovation (e.g. Robinson and Stern 1997; Mumford et al. 2002; Monge et al. 1992; Amabile 1996) \cite{robinson1997coprorate};\cite{mumford2002social};\cite{monge1992communication}\cite{amabile1996assessing}. Organizational structures can influence in many ways creativity of a team and individual. For instance, by promoting open communication, idea and ongoing information exchange with internal and external team members as well as encouraging information seeking from different perspectives and sources is likely to enhance creativity (e.g. \cite{ancona1992demogpraphy}\cite{dougherty1996sustained}). Accorgind to Neweth and Staw (1989) \cite{staw1989tradeoff} social influence of others plays a major role for individuals' beliefs; attitudes towards job, for instance, rise from the social labelling of work by others.  Also Salancik and Pfeffer (1978) \cite{salancik1978social} argue the essential role opinions of others may have on individual: individual's perception of her work and organisation can be greatly influenced by opinions of others. Additionally, team member's collective view of support they get from their leader has been related to the team's creative endeavours and success in them (e.g., \cite{amabile1998kill} Amabile and Conti, 1999; \cite{amabile1996assessing} Amabile et al., 1996). 
 
Thus, as individuals oftentimes requires support and input from several invidivuals who help to challenge ideas in constructive ways, teams are essential in generating and implementing ideas  \cite{mumford2002social}. Stimulating those constructive individuals for creative actions may be valuable \cite{robinson1997corporate}. In addition, including team members in ideation assists in idea implementation and through participation new ideas are not that likely to be rejected or abandoned (Agrell and Gustafson 1994) \cite{agrell1994team}.

Leaders have a great role enabling creative behaviour in teams and individuals, yet team members also influence essentially in others. Thus, by utilising various human resource practices leaders should create an environment where creativity is encouraged and supported. (Shalley and Gilson 2004) \cite{shalley2004leaders} Study of Ancona and Caldwell (1992) \cite{ancona1992demograpgy} argue how changing the structure of teams is not sufficient and does not lead to improved performance, rather the leader and the team should find ways to foster positive effects of the team processes and reduce the negative ones. At team level this may mean focus on enhancing negotiation, problem-solving and conflict resolution skills while at organisational level leader should protect the team from external political pressures and reward the team from performance outcome instead of functional ones. \cite{ancona1992demography}

\subsection{Individual learning}

Various perspectives and definitions for learning has been studied, presented, analysed and utilised in order to understand individual's process of adapting new information and skills. Experiential learning theory refers to learning as a process of knowledge-creation through experiences while experiential learning process stands as a way to describe the central process of human adaptation to the social and physical environment - a holistic adaptation process that provides bridges across life situations and underlaying the lifelong process of learning. \cite{kolb1984experiential}. Also Jung (1923) \citeasperson{jung1923psychological} argues how learning involves concept of human being as a whole - from feeling and thinking to perceiving and behaving.

However, everyday problem-solving and immediate reactions to situations at hand are oftentimes related to performing instead of learning. Furthermore, long-term adaptations to our previous experiences and beliefs is mainly considered as developing, not learning. Yet, when talking about development and developing in individual, team or organisational level, the question highly concerns and is related to learning. \cite{kolb1984experiential}

Kolb's (1984) \citeasperson{kolb1984experiential} experiential learning theory consists of four elements: experience, perception, cognition and behaviour. Immediate experience forms a basis for reflection and observation, following assimilation to a theory from which new implications for action are deducted. In order to create new experiences, these implications serve as guides. Overall, experience of an individual is a focal point of learning, and giving personal meaning to abstract concepts, which can be afterwards shared with others. Furthermore, receiving feedback is considered essential in this approach for learning, as it serves a continuous process for goal-oriented action following evaluation of that action. Feedback can thus boost effective, goal-oriented learning process. \cite{kolb1984experiential}

Continuing with the model of \citeasperson{kolb1984experiential}, instead of conceiving learning in terms of outcomes, it should rather be conceived as a process. Ideas are not fixed and and immutable elements of thoughts, but can be formed and re-formed through experience. Furthermore, bringing the experiential learning into educational implications, all learning can be considered as relearning. Thus, all learning situations should take into account people arriving from all different experiential backgrounds to what they build their new experiences and knowledge on. This partly explains very likely resistance to new ideas, as when new information and experiences are in contradiction to old beliefs and experiences, new ideas and information is more difficult to adapt. In the education process learner's old beliefs and theories should be brought out, examined and tested, following integration of the new models and refined ideas into learner's belief systems. \cite{kolb1984experiential} (GAP BETWEEN IDEA AND EXPE?)

Kolb \citeasperson{kolb1984experiential} presents Piaget's interactive process approach to learning, according to which individual learning and adaptation of new ideas occurs through integration or substitution. Integration leads to stronger part of learner's conception of the world, whereas substitution requires real questioning of previous conceptions, and thus might take longer for the learner to adopt. Learning is a mutual process between accommodation of concepts or schemas to experiences around us and assimilation of events and experiences into existing concepts and schemas. This intelligent adaptation, learning, results from the tension between accommodation and assimilation. Through this tension growth and higher-level cognitive functioning occurs. According to Kolb \citeasperson{kolb1984experiential}, learning is a process filled with tension and conflict, and new knowledge, skills and attitudes are achieved through experiential learning, which consists of four modes and required abilities of learners: concrete experience abilities, reflective observation abilities, abstract conceptualisation abilities and active experimentation. First of all, individuals must openly involve themselves in new experiences, reflect and observe them from various perspectives, create concepts that can be integrated into more abstract theories as well as they need to be able to use these reflections and theories in active daily decision-making and problem-solving. 

Understanding of different learning styles and modes assists in supporting individuals in learning and problem-solving. 

Meaning of environment in learning should be emphasised. Learning concerns of transaction between an individual and the environment, learning does not happen only inside of individual's thoughts, experiences and processes but is dependent on the real world environment. 

\section{Innovation to the rescue}
Social innovation refers to the generation and implementation of novel ideas concerning people in demand to organise their interpersonal or social activities and interactions in new ways in order to achieve common goals. Results and products of social innovation, like other types of innovation, are likely to vary depending on the breadth and impact of the innovation. (Mumford and Gustafson 1988) \cite{mumford1988creativity} \citeasperson{mumford2002social}Mumford (2002) presents four factors affecting social innovation: active exchange of ideas and information in supportive climate, tangible and low-cost ideas that can be at the fewer guessed to be beneficial, support from upper level management, and effective communication through the innovation process in order to proceed from the idea to implementation.

Innovation can be contributed by encouraging idea generation, but also creating a climate of autonomy, offering intrinsic and extrinsic rewards and engaging employees with their work \cite{amabile1996assessing}; \cite{amabile1998kill}(Amabile et al, 1996; Amabile, 1998 Amabile ). Furthermore, characteristics associated with innovation are integration of work units, decentralisation of control and professionalisation are likely to effect innovation in a way that through these suitable environment for innovation, dynamic idea exchange and implementation is created \cite{mumford2002social}(Mumford 2002).

Managerial practices for technological innovations have been widely studied. According to Quinn (1985) \citeasperson{quinn1985managing} the essence lays in accepting the chaos of development. In addition, large and successful companies and their leaders listen carefully their users' needs, develop according this customer demand, define clear goals and framework for the work, encourage teams to challenge the status quo and find alternative solutions while avoiding detailed and technical or marketing plans in the beginning. Instead, they focus on early prototyping and iteration. 

According to studies creativity and innovation in an organisation requires integrated organisational approach, right climate, appropriate incentives for innovators, and a systematic way and resources to transform an idea into an innovation.  In individual level, creativity and innovation calls for various skills, such as teamwork, communication, coaching, project management, learning and learning to learn, visioning, change management and leadership, and ability to develop these skills. Oftentimes, even though the climate and practices are right for generate innovations, problems are faced when attempting to manage the change process. (Roffe 1999) \cite{roffe1999innovation}

Innovation and creativity are highly related, yet not the same thing: According to Hennessey and Amabile (1988) \citeasperson{hennessey19881}, individual creativity stands for an essential building block for organisational innovation and also Sethi et al. (2001) \citeasperson{sethi2001cross} argue creativity being essential in new idea generation and design processes that aim for innovative solutions. Other studies also emphasise the role of creativity a first step in creating something novel, whereas innovation refers to the implementation phase of the novel ideas in individual, team or organisational level (Shalley and Gilson 2004; Amabile et al. 1996; Mumford and Gustafson 1988) \cite{shalley2004leaders};\cite{amabile1996assessing};\cite{mumford1988creativity}. 

Social cohesion may inhibit innovativeness of the team and its individuals especially beyond a moderate level, while employees are more likely to settle on group think and traditional daily practices. However, according to the study of Sethi et al. (2001) \citeasperson{sethi2001cross} when a team shares superordinate identity, is encouraged to take risks, lets customer's requirements be heard, and actively lets senior management monitor the project, team is more likely to present innovative ideas and perform in innovative ways. According to this study, functional diversity does not effect on innovativeness, but team's superordinate identity can be strengthened by encouraging risk-taking and weakened by social cohesion.

\citeasperson{mumford2002social}Mumford (2002) argues in his study of Ben Franklin's social innovations, that key factor in successful social innovation lays in fast demonstrating, which he also refers as experimenting. Thus, in order to drive for social innovations, opportunism and showmanship of an individual or team may be required. \cite{mumford2002social}(Mumford 2002) Furthermore, according to Monge et al. (1992) \citeasperson{monge1992communication} group communication is likely to increase innovation under some circumstances,  and also Katzenbach (1993) \citeasperson{katzenbach1993wisdom} argues for culture of strong team performance. However, Amabile et al. (2004) \citeasperson{amabile2004leader} emphasise how, ultimately, truly novel ideas raise from individuals, making them the ultimate source of any new idea or solution to a problem \cite{amabile2004leader}(Amabile et al. 2004).

\subsection{Creativity, intrinsic motivation and everyday problem solving}

Divergent thinking refers to individual's ability to find multiple alternative solutions and ideas to problems at hand, and has been related to serve as a key capacity affecting creative thinking (Guilford 1967) \cite{guilford1967creativity}. Accordingly Mumford and Gustafson (1988) \citeasperson{mumford1988creativity} emphasis that creative people consistently and with confident tend to seek for alternative solutions, even under uncertain conditions. Even though expertise and  intelligence have been related to problem solving, series of causal analyses carried out by Vincent et al. (2002) \cite{vincent2002divergent} revealed unique effects divergent thinking had that were not attributed to intelligence and expertise. 

Shalley and Gilson (2004)\citeasperson{shalley2004leaders}, in turn, argue that through developing extensive set of skills, employees may learn to be more comfortable and confident in thinking from different perspectives, finding various alternative solutions, trying out novel things and seizing opportunities. According to Amabile and Hennessey (1988) \citeasperson{hennessey19881} individual creativity requires ability to think creatively, generate alternatives, engage in divergent thinking and tolerate or suspend judgment. Through this perspective creativity can be considered as a skill that can be learned and strengthened. Understanding of individual's creativity and ways to influence and improve it gives managers guidelines when creating an environment and leadership that support organisational innovation \cite{redmond1993putting} Redmond et al. (1993).

Several case studies has showed that creativity insights emerge gradually through the network and actions of an creative individual. Study of creativity is a combination of two different disciplines and research approach: sociological and historiometric lenses study the conditions in which creative actions and processes are likely to occur, whereas neurobiological approach presents neural structures and processes that are active and associated with creative outcomes. (Gardner 1988) \cite{gardner1988creativity}

Generation of novel, alternative solutions requires problem-finding skills \cite{runco1988problem}(Runco and Okuda 1988), which has been indicated to be one of the best predictors of creativity in 'real world' activities, when studied 91 elementary school students \cite{okuda1990}(Okuda et al. 1990). These findings suggest leaders, in order to enhance creativity of employees, to support learning of these skills for instance by facilitating problem-construction \cite{redmond1993putting}(Redmond et al. 1993).

Kasof (1997) \citeasperson{kasof1997creativity} argued in his study that breadth of attention affects on creative performance of an individual: wide spread of attention is usually related to creative ability. By breadth of attention Kasof refers to "number and range of stimuli attended to at any time." Breadth of attention being narrow, individuals are able to focus on narrow range of stimuli and are better at filtering redundant stimuli from awareness. However, those individuals with wide breadth of attention tend to be more aware of irrelevant or extraneous stimuli, these individuals are strongly affected by their environment and are highly arousable.\cite{kasof1997creativity}

Studies of creative characteristics of individuals has revealed factors such as wide interest in various fields, autonomy, belief of being creative and independence in decision-making (Shalley and Gilson (2004)) \cite{shalley2004leaders}. Indeed, broad interest stands for a sign of intrinsic motivation, which is also widely related to both creativity and well-being of an individual and innovation (e.g. Amabile 1988; Csikszentmihalyi 1999; Gardner 1988 \cite{hennessey19881}; \cite{csikszentmihalyi199916}; \cite{gardner1988creativity}). For instance in their study Tierney et al. (1999) \citeasperson{tierney1999examination}, positive correlation was found between employee's level of enjoyment while working on a creative task at hand and the level of creativity.  

For students of creativity, there is no surprise in attaching self-efficacy to creative actions \cite{mumford1988creativity}(Mumford and Gustafson, 1988), yet recently problem construction processes have been recognised and combined to everyday problem-solving and real-world creativity (Getzel and Csikszentmihalyi, 1975; Runco and Okuda 1988)\cite{getzels1975problem}; \cite{runco1988problem}. According to study of Gardner (1988) \cite{gardner1988creativity} correlation between creative problem solving and everyday problem solving exists: they seem to have the same roots in information processing skills. 

Without previous experience of the job routine and substance knowledge and expertise on the field creative endeavours are more rare. Even though has been argued how routine work and task familiarity is likely to lead very habitual performance (Ford, 1996) \cite{ford1996theory}, knowing the status quo may provide opportunities for creative actions through reflecting and practicing skills and activities requires in the field. (Shalley and Gilson 2004) \cite{shalley2004leaders} 

In their study Redmond et al. (1993)\citeasperson{redmond1993putting} found how leaders supporting employees problem-finding and problem construction led to more unique and novel solutions. Leaders encouraged employees to find alternative solutions, approach problems from different perspectives and overall supporting several alternative problem-solving strategies. In addition, study showed how through motivational mechanisms, such as self-set goals, involvement and commitment, problem construction may have positive influence on solution quality and originality. Thus, problem construction is likely to have its greatest impacts on performance when in the process employee is allowed to express his values, needs and interests \cite{redmond1993putting}(Redmond et al., 1993). 

Instead of managing creativity leaders should consider new approach: managing for creativity (Amabile and Khaire 2008) \cite{amabile2008creativity}. According to Isaksen (1983), in order to support employee's creativity, leaders should focus on creating and maintaining and environment of supportive empathy, respect, warmth, concreteness, genuiness, trust and flexibility. These factors have been combined to general and task-specific efficacy needs \cite{mumford1988creativity}(Mumford and Gustafson, 1988). Furthermore, through providing enough processing time for creating novel solutions is likely to enhance creative behaviour of employees \cite{isaksen1983toward}(Isaksen 1983). As creativity refers to finding novel solutions and generating understanding of problems at hand, leaders could facilitate the process of resource allocation, feedback and task management in order to support employee's creative process \cite{mumford1988creativity}(Mumford and Gustafson, 1988). 

Yet, leader alone is not able to boost creative solutions in employees: it is also a matter of personal characteristics, previous knowledge of the problem at hand and expertise in the field (Mumford and Gustafson 1988, Redmond et al. 1993)\cite{mumford1988creativity}; \cite{redmond1993putting}. Thus, in order to achieve novel solutions and fresh ideas, leaders may seek employees who have great knowledge and expertise of problem at hand or provide employees education and possibilities to develop their problem construction skills and furthermore encourage approaching problems from various perspectives.  \cite{redmond1993putting}(Redmond et al., 1993) 

Furthermore, supporting employee's feeling of self-efficacy is likely to improve creative skills of an employee \cite{redmond1993putting}(Redmond et al., 1993), and can be done through giving positive and realistic feedback, allowing adequate resources and physical support, clarifying task assignments, providing development support for employees, and assigning employees to appropriate tasks \cite{hennessey19881}(Hennessey and Amabile 1988). However, often acknowledging employee's skills, potential and accomplishments is likely to push an employee to the track of creativity \cite{redmond1993putting}(Redmond et al., 1993). 

Should also be noted that depending on the job, different level of creativity may be required. Certain jobs that are highly involved with novel solutions urges for creativity as major breakthrough and innovative ideas, whereas more routine and repetitive jobs such as assembly line work requires creativity in developing the job practicalities. (Shalley and Gilson, 2004) \cite{shalley2004leaders} 

Employees who consider and believe creativity as valued outcome are more willing to generate ideas, experiment, communicate openly with others about ideas and through this, overall, their behaviour will eventually lead to creative outcomes. (Shalley and Gilson 2004) \cite{shalley2004leaders} Accordingly, Csikszentmihalyi (1999) presents the belief and feeling an employee has on the capabilities, pressure, resources and sociotechnical system of work environment affects highly on the success of creativity. 

Thus, organizational environment plays a major role in employees' creative skills, and such stifling factors may be positive challenge at work, encouragement from organisational level, support from work group as well as supervisory encouragement. Furthermore, organisational impediments can lead to decreased level of creativity. (Amabile 1998) \cite{amabile1998kill} Hence, leadership has a great role in ensuring that the climate and culture, structure and practises of work and work environment together with human resource practices are supportive for creative endeavours to occur (Shalley and Gilson 2004; Oldham and Cummings 1996; Mumford et al. 2002) \cite{shalley2004leaders}; \cite{oldham1996employee}; \cite{mumford2002leading}. 

\section{Experimentation as a method for innovation and learning}

Normal business is considered as repetition, risk-avoidance and focusing on business outcomes (Buijs, 2007), while innovation requires novel solutions, thinking out of the box, risk-taking, breaking the rules, challenging the status quo and questioning the future \cite{burns1961management,kanter1984change,march1991exploration}(Burns and Stalker 1961; Kanter 1984, March 1991)

According to \citeasperson{edmondson1999psychological} Edmondson (1999), learning behaviour consists of seeking feedback, sharing information, asking for help, talking about errors and experimenting. Thus, experimentation behaviour seems to relate to learning of an individual, teams and organisations and can be supported by supporting these factors. 

According to \citeasperson{buijs2007innovation} Buijs (2007), innovation consists of coming up with novel ideas and implementing them. Ideating begins with exploring, developing and implementing the ideas, following introducing the ideas, which have turned into products or services, into the marketplace. 

Innovation process can be approached from several angles: first of all, content of the innovation has to be clear - whether the purpose is to innovate new products, manufacturing processes, ways of organising or ways of dealing with people. Secondly, psychological process of the innovation team has to be understood, essential being shared understanding, level of comfort with ambiguity and degree of trust between team members. Thirdly, creative process of the team, meaning idea producing process, needs to be understood and efficiently facilitated. Finally, the role of leading plays a major role, and together with playful attitude innovation process is likely to succeed. \cite{buijs2007innovation}(Buijs, 2007)

"Usually, the innovation process is described as a series of stages through which an idea is processed. During each stage, certain activities are executed in order to improve the quality of the idea and to let the idea grow. Each stage ends with a gate, in which the idea is checked and evaluated. It is then decided if the idea can be carried over into the next stage, if the idea needs to be further developed in the current stage, or if the idea needs to be abandoned altogether.---
Although stage-gate models are describing the process an idea is going through, the evaluation steps are useful not only for evaluating the quality of the ideas, but also for checking and evaluating the process itself. What went well? What can be improved? And what has been learned? Reflection occurs during evaluation and learning occurs during reflection. After all, every innovation process is an organizational learning process " \cite{buijs2007innovation}(Buijs, 2007)

Several factors have been recognized to affect on organisational innovation, yet many researchers have stated leadership behaviour being one of the most important. \cite{dong2003,amabile1998kill,jung2001transformational,mumford1988creativity} (Dong et al. 2003; Amabile, 1998; Jung, 2001; Mumford and Gustafson 1988) \citeasperson{dong2003} Dong et al. (2003) identify four hypotheses how top managers' leadership styles may affect both directly and indirectly their companies' ability to innovate. Indirectly here stands for instance for leader's possibility to empower employees and build organisational climate optimal for innovation. The study shows a positive relation between transformational leadership style and empowerment as well as innovation-supporting organisational climate. 

As in todays' business world it has been widely recognised how creativity and innovation are essential for business growth, researchers have studied factors affecting creativity and innovation in organisations. \citeasperson{amabile1998kill} Amabile (1998) has identified three factors being important for stimulating creative behaviour in individuals and organisations: individuals' intellectual capacity (creative thinking skills), expertise based on past experience and creativity-supporting work environment. Furthermore, Oldham and Cummings (1996) consider creativity skills and characteristics of and individual as important, yet they add the importance of characteristics of organisational context such as job complexity, supportive supervision or controlling supervision. 


Experimentation calls for creativity. 
"When individuals access a variety of alternatives, example solutions, or potentially related ideas, they are more likely to make connections that lead them to be creative (Amabile et al. 1996)"

"In addition, skills such as problem finding, problem construction, combination, and idea evaluation are important for creativity (Mumford et al. 2002, Vincent et al., 2002).

"In contrast to personality traits and creativity relevant skills, domain-specific knowledge reflects an individual's level of education, training, experience, and knowledge within a particular context (Gardner, 1993). "

Several factors form the basis of creativity skills of an individual, such as personality, technical knowledge, expertise, motives, and the supervisor's feedback style. In group level factors form of task structure, communication styles and task autonomy, and finally in organisational level strategy, structure, culture, climate and available resources all affect how creative actions are encountered. \cite{jung2003role}(Jung et al. 2003) 

Creativity is not restricted to artistic occupations only; it is required in various professions in which tasks presented involve complex, ill-designed problems where novel solutions are needed and status quo challenged \cite{mumford1988creativity}(Mumford and Gustafson 1988). 

According to \citeasperson{mumford2002leading_vai_mumford2002social} Mumford (2002), actually, idea implementation may require even more creativity than idea generation and \citeasperson{vincent2002divergent} Vincent et al (2002) states the two key sets of processes that are involved in creative work: : "(a) creative processes or the activities underlying initial idea generation and (b) innovation processes or the activities underlying the implementation of new ideas"

"Implementation, of course, depends on the evaluation of ideas \cite{runco1994problem}(Runco 1994)."

\citeasperson{amabile1996assessing,amabile1998kill} Amabile (1996,1998) defines creative thinking as a way how people approach problems at hand and come up with solutions. Creative thinking does not stand for intellectual capacity of an individual to create something new but rather as a combination of past experiences which creates expertise and the ability to apply creative thinking skills to these experiences and invent new solutions. 

Yet, intrinsic motivation of an individual has been recognised one of the most essential factors increasing creativity 

When dealing with novel solutions and challenging status quo, we are dealing with innovations. In order for company and its employees to be innovative, they need to take risks. Yet, at the same time usual management processes avoid risk-taking and focus on managing daily routine business. As \citeasperson{quinn1985managing} Quinn (1985) stated it in his Harvard Business Review article: we love innovation and we urge for innovation, but we can tolerate it only if it is controllable and results everything remaining the same. \cite{quinn1985managing}(Quinn 1985)

"Talking about innovation as a single process is misleading. Innovation is a multi-faceted process that is full of contradictions. It is simultaneously hard and soft, nice and nasty, fun and serious. Innovation involves technology and people, marketing thinking about manufacturing and manufacturing thinking about marketing." \cite{buijs2007innovation}(Buijs 2007)

"Within the broad framework I have described, major innovations are best managed as incremental, goal-oriented, interactive learning processes." \cite{quinn1985managing}(Quinn 1985)

\citeasperson{anriopoulos2000enhancing} Anriopoulos and Lowe (2000) define in their study for elements of perpetual challenging - a way to enhance creativity and innovation in an organisation. First of the elements is adventuring, which refers to "the process through which individuals are encouraged to explore uncertainty, so that they can generate innovative solutions. Adventuring is the only process, which occurs implicitly when employees' goal is the generation of new ideas." Furthermore, they define three categories of adventuring: introspecting, scenario making and experimenting. 
"Creative employees,
who aim to do original work,need tohave the
basic knowledge of their specific field. In
other words, they need to know what has
already been discovered, so they will be able
to go beyond that point." Jatkaa pointtia, ett� perustiet�myksen olisi hyv� olla kunnossa ty�ntekij�n ideoinnissa. 

"Scenario making refers to the development of
possible routes to tackle a particular
situation." In addition, scenario making attempts to clear what is already known about problem or situation at hand, and especially what is not known. --
Experimenting
occurs when employees utilise their full
potential by being involved in a personal
trial and error process. It refers to the
employees' ability to test the different
scenarios generated by different concepts in
terms of images or ideas, which can fit to the
parameters set by the client or the industry
within which their company operates, and
eventually decide the most appropriate one.--
the research has
shown that incremental risk is very often
cherished by creative organisations because
it stretches employees' capabilities and
consequently provides the basis upon which
employees can develop new knowledge and
skills to be used in other projects. By
stretching their creative capabilities creative
organisations try to enhance the quality of
services offered to their clients every time an
opportunity arises. --
Nevertheless, creative
employees need to know that they operate in
an environment where failure is not
penalised, as long as the creative process and
analytical methods are effectively used. This
process is labelled as safety netting since
from the interviews it was apparent that
creative employees should be in an
environment which tolerates failure. --
Conceptual confronting refers to the
employees' ability to question each other's
ideas so that their full potentials are utilised.
Conceptual confronting can be internal,
which refers to the questioning of each
other's ideas within the company, and
external, which occurs when informal and
casual debates with people from other
disciplines take place--
This research has revealed that a pre-set
obstacle like a deadline can stimulate
creativity, as employees' focus can be
narrowed to the issue in question, which
needs to be solved immediately. \cite{andriopoulos2000enhancing} (Andriopoulos and Lowe 2000)

Experimenting serve as a great method when testing and validating abstract concepts (\cite{kolb1984experiential}.

\section{Factors hindering experimentation and learning}

Several factors may affect on the gap between idea and action in employees of an organisation. 

One essential factor is the beliefs, emotions and actions of an employee. For instance, when feeling under risk or pressure, an employee is likely to inhibit learning behaviour as a result of feeling the fear of being rejected or feeling they are placing themselves at risk. This may occur in a situation where an employee should ask for help, yet is afraid of admitting he lacks abilities, skills or knowledge. \cite{edmondson1999psychological}(Endomndson 1999). In addition, admitting mistakes, asking for help and seeking feedback are all relevant abilities in the recent organisational world, yet threatening for an individual's image of himself and his skills \cite{brown1990politeness}(Brown 1990). Indeed,  learning is often inhibited as people tend to act in a way that inhibits learning when facing the potential for embarrassment of threat, even though their transparency and honesty would be highly important for the behaviour of the team. \cite{argyris1982reasoning}(Argyris 1982)

The phenomenon of threat of employees in organisations is widely studied and consensus is rising how threat effects on cognitive and behavioural flexibility and responsibility in reducing manner. 

Particularly when large new, complicated systems at hand, meaning of co-operation in production, development and communication rises exponentially. Especially in large organisations innovation can be inhibited by the errors increasing as a result of complexity of the system and inability to control, understand or make intelligent decisions. Challenging as it is for one department, faculty or company to survive on its own without communication and help of others in design, production and other business-related decisions, with management that takes the complex environment into account, the disastrous effects resulting from lack of communication can be lessen.  \cite{quinn1985managing}

"Of course, in a complex social situation, where many causes operate, not all of which are controllable \cite{katz1978social}(Katz and Kahn, 1978), innovation may ultimately depend on boiling things down to their essential and focusing on those essential elements of the situation you can do something about. \cite{mumford2002social}(Mumford 2002) Voisko t�m� toimia pointtina sille, ett� kokeileminen auttaa keskittym��n sellaiseen, ja vain sellaiseen, johon voi vaikuttaa itse. Eli discussioniin 

"Top management isolation. Many senior executives in big companies have little contact with conditions on the factory floor or with customers who might influence their thinking about technological innovation. Since risk perception is inversely related to familiarity and experience, financially oriented top managers are likely to perceive technological innovations as more problematic than acquisitions that may be just as risky but that will appear more familiar." \cite{hayes1982managing} (Hayes and Garvin 1982)

"Intolerance of fanatics. Big companies often view entrepreneurial fanatics as embarrassments or troublemakers. Many major cities are now ringed by companies founded by these "nonteam" players-often to the regret of their former employers.
Short time horizons. The perceived corporate need to report a continuous stream of quarterly profits conflicts with the long time spans that major innovations normally require. Such pressures often make publicly owned companies favor quick marketing fixes, cost cutting, and acquisition strategies over process, product, or quality innovations that would yield much more in the long run.
Accounting practices. By assessing all its direct, indirect, overhead, overtime, and service costs against a project, large corporations have much higher development expenses compared with entrepreneurs working in garages. A project in a big company can quickly become an exposed political target, its potential net present value may sink unacceptably, and an entry into small markets may not justify its sunk costs. An otherwise viable project may soon founder and disappear.
Excessive rationalism. Managers in big companies often seek orderly advance through early market research studies or PERT planning. Rather than managing the inevitable chaos of innovation productively, these managers soon drive out the very things that lead to innovation in order to prove their announced plans.
Excessive bureaucracy. In the name of efficiency, bureaucratic structures require many approvals and cause delays at every turn. Experiments that a small company can perform in hours may take days or weeks in large organizations. The interactive feedback that fosters innovation is lost, important time windows can be missed, and real costs and risks rise for the corporation.
Inappropriate incentives. Reward and control systems in most big companies are designed to minimize surprises. Yet innovation, by definition, is full of surprises. It often disrupts well-laid plans, accepted power patterns, and entrenched organizational behavior at high costs to many. Few large companies make millionaires of those who create such disruptions, however profitable the innovations may turn out to be. When control systems neither penalize opportunities missed nor reward risks taken, the results are predictable." \cite{quinn1982barriers} (Quinn 1982 on barriers to innovation) 

"Expertise,
intelligence, and divergent thinking did not exert
direct effects on performance. Instead, the effects on
performance were mediated by problem-solving
activities within the domain under consideration?
specifically, idea generation and idea implementation.
These observations about the role of expertise,
intelligence, and divergent thinking point to a broader
conclusion. The generation and implementation of
solutions to real-world problems of the sort examined
in this study, like performance on most significant
occupational tasks, are a complex phenomenon.
Both divergent thinking and expertise influence leader performance but only through the leader's ability to generate and implement solutions to significant organisational problems." (Vincent et al. 2002)

\section{Factors supporting experimentation and learning}

Creativity and innovation have gained wider acceptance as important factors creating value in organisational performance \cite{mumford2002leading_vai_mumford2002social}(Mumford et al. 2002). Creativity and innovation have for instance been studied to have enhancing impact to organisations profit and growth \cite{nystrom1990organizational}(Nystrom 1990). 

According to \citeasperson{garvin2008yours} Garvin et al. (2008) in learning organisation employees excel at creating, acquiring and transferring knowledge. They define building blocks for learning organisation: supportive learning environment, concrete learning processes and practices and leadership behaviour that reinforces learning. Building blocks can be considered and measured as independent components yet each of them vital to the whole. In order to improve long-term learning of an organisation, strengths and weaknesses of an organisation and its unit needs to be recognised. 

Likewise, \citeasperson{mumford1988creativity} Mumford and Gustafson (1988) emphasise the meaning of environmental variables as a means to support employee's creativity by providing resources to stimulate fresh ideas of employees. Furthermore, strong positive relations between organisational environmental variables have been found; organisational encouragement as well as support for innovation and creativity from team improve employee's creativity \cite{amabile1996assessing} (Amabile et al. 1996) 

In order experimenting to happen in an organisation safe and supportive environment has to be created. According to \citeasperson{garvin2008yours} Garvin et al. (2008) supportive learning environment consists of four characteristics: psychological safety, appreciation of differences, openness to new ideas and time for reflection. 

Futhermore, \citeasperson{amabile1998kill} Amabile (1998) suggests changes in organisational environment are likely to boost intrinsic motivation of an employee leading to increased creativity skills. Role of leaders and managers is essential; being a key person in organising group work and processes a leader may encourage employees to achieve shared goals. 

"Atmosphere and vision. Continuous innovation occurs largely because top executives appreciate innovation and manage their company's value system and atmosphere to support it. 
Because familiarity can foster understanding and psychological comfort, engineering and scientific leaders are often those who create atmospheres supportive of innovation, especially in a company's early life. Executive vision is more important than a particular management background.
Small, flat organizations. The most innovative large companies in my sample try to keep the total organization flat and project teams small. Development teams normally include only six or seven key people. This number seems to constitute a critical mass of skills while fostering maximum communication and commitment among members
Since it takes a chain of yesses and only one no to kill a project, jeopardy multiplies as management layers increase.
Recognizing the inadequacies of theory, innovative enterprises seem to move faster from paper studies to physical testing than do noninnovative enterprises. When possible, they encourage several prototype programs to proceed in parallel. Sony pursued 10 major options in developing its videotape recorder technology. Each option had two to three subsystem alternatives. Such redundancy helps the company cope with uncertainties in development, motivates people through competition, and improves the amount and quality of information available for making final choices on scale-ups or introductions.
Many experienced big companies are relying less on early market research and more on interactive development with lead customers. Hewlett-Packard, 3M, Sony, and Raychem frequently introduce radically new products through small teams that work closely with lead customers. These teams learn from their customers' needs and innovations, and rapidly modify designs and entry strategies based on this information.
Formal market analyses continue to be useful for extending product lines, but they are often misleading when applied to radical innovations. 
To allocate resources for innovation strategically, managers need to define the broad, long-term actions within and across divisions necessary to achieve their visions. They should determine which positions to hold at all costs, where to fall back, and where to expand initially and in the more distant future.
A company's strategy may often require investing most resources in current lines. But sufficient resources should also be invested in patterns that ensure intermediate and long-term growth; provide defenses against possible government, labor, competitive, or activist challenges; and generate needed organizational, technical, and external relations flexibilities to handle unforeseen opportunities or threats. Sophisticated portfolio planning within and among divisions can protect both current returns and future prospects-the two critical bases for that most cherished goal, high price-earnings ratios." \cite{quinn1985managing}(Quinn 1985)

\citeasperson{mumford1988creativity} Mumford and Gustafson (1988) have studied the gap between an idea an action, and came to conclusion it depending on various attributes relating to individual and circumstances. 

Psychological safety means that learning of employees occurs when employees do not fear being rejected, ask naive questions, make mistakes or present viewpoint of minority. Psychologically safe environment enables employees comfortably to express their thoughts at work. Appreciation of differences is important as opening minds for different ideas and world views increases both energy and motivation, brings out fresh thinking. Learning occurs when employees become aware of opposing ideas in a safe environment, and additionally openness to new ideas is required. Novel approaches are relevant for learning, thus employees should be encouraged in risk-taking and exploring and testing uncertain things. Lastly, through providing time for reflection learning in safe environment occurs. Instead of looking and judging by numbers of hours of work or results employees should be given enough time to reflect their work. Analytic and creative thinking will not occur under stress, heavy workload and too tight schedule. Under stress ability to recognise and react to problems and learn from experiences deteriorates. In supportive learning environment time for reflection is allowed. \cite{garvin2008yours} (Garvin et al. 2008)

Similar characteristics were found in data and are further presented in chapter \ref{chapter:datapresentation}. 

Edmondson in her research (1999) studied whether beliefs in interpersonal context varied between the same organisation and whether it affected in team outcomes

According to Garvin et al. (2008) second building block of organisational learning, consists of concrete learning processes and practices. It includes experimentation, information collection, analysis, education and training and information transfer. Organizational learning can be supported trough concrete steps and activities which are tested and further developed through experimentations. Furthermore, information and intelligence about customers as well as technological trends should be collected systematically and further analysed focusing on identifying problems and solving them. Training and education of new and established employees is an essential part of practices and processes. Finally, through transparent and meaningful knowledge sharing organisational learning can be enhanced, focus being on clear, well-defined and working communication systems that employees can easily relate and feel useful. Concrete processes together with efficient knowledge sharing methods ensures that essential information is available quickly and efficiently for employees who need it. 

 Thus, emphasis should be put on creating and defining concrete learning processes and practices.
 
Thirdly, leadership behaviour should reinforce learning. Behavior of leaders is highly related to the performance of employees (Kim and Mauborgne 2014) and organisational learning (Garvin et al. 2008). In order to encourage employees to learn, leaders should prompt dialogue and debate, ask questions and listen to employees. Example of the leaders was also recocgized from the data as an important factor affecting experimentation behaviour, and Garvin et al. also emphasises how through own example leaders can encourage employees to offer new ideas and options. 

These three building blocks overlap to some degree and reinforce one another. For instance, leadership behaviour helps in creating supportive learning environment and this supports managers and employees in creating and defining concrete learning processes and practices. Furthermore, concrete processes support leaders behaviour in a way that fosters learning and through own example cultivates that behaviour to others. 

Supportive leadership behaviour alone is not sufficient for guarantee organisational learning. Garvin et al. (2008) emphasise how organisations are not monolithic and managers should be sensitive to differences in culture, department and units. In addition to cultural differences, learning requires clear and targeted processes and practices. Furthermore, learning should be considered as multidimensional, thus organisational forces should not be solely focused on a single area but to consider presented building blocks as a whole.

Intrinsic motivation of an individual is one of the most powerful tools to creative action non-traditional thinking (Amabile 1996; Deci and Ryan 1985, Jung 2001). 

Leaders supporting new ideas and idea exchange has been related to enhancing creativity especially among those employees who showed disposition towards creativity (Oldham and Cummings 1996). 

Expertise 
- Expert performance and its affects on implementing ideas (Ericsson and Charness 1994)

Collins and Amabile (1999) 

Creative work is demanding and time-consuming (Mumford et al. 2002) as well as requires attention over long periods of time involving high level of ambiguity and stress (Kasof 1997)
Thus, creative work is resource intensive (Mumford et al. 2002) involving risk. 
Through setting goals and doing small prototypes uncertainty can be reduced (Mumford 2002). 

Suoraan Mumfordin et al (2002) artsusta seuraavat kolme kappaletta, ker�� n�ist� olennaiset point it, kolme kohtaa!: "This integrative style seems to involve three crucial elements. The first major element of this style is idea generation. Indeed, the bulk of the available research on creative leadership stresses the role of the leader in facilitating others' idea generation. Thus, intellectual stimulation, or the application of creative problem solving techniques in guiding others, seems required along with support for new ideas, involvement with the people in developing ideas, and granting these people the freedom to pursue the ideas thus generated. Not only must leaders help people generate ideas, they must construct an environment where such ideas and likely to emerge. Thus, leaders must insure diversity in the group, open communications, and through role modeling, crisis management, and policy decisions seek to create a climate and culture where people are likely to generate and pursue new ideas.

These idea structuring activities, to contrast to idea generation, tend to be indirect involving the creation of action, or project, frameworks so as to maximize the autonomy of the individuals doing the work. In other words, it may be more useful to set a deadline than to show someone how to meet this deadline.

The third aspect of this style of leadership is idea promotion. Idea promotion involves gathering support form the broader organization for the creative enterprise as a whole as well as implementation of a specific idea or project. For the leaders of creative people, these promotional activities are essential primarily because they insure that the resources needed to carryout the work will be available"

According to Amabile et al. (2004) and their componential theory on creativity, "the support provided by immediate supervisors exerts
an influence on subordinates? creativity through direct help with the project, the development of
subordinate expertise, and the enhancement of subordinate intrinsic motivation. The componential
theory proposes that positive behaviors of supervisors include serving as a good work model, planning and setting goals appropriately, supporting the work group within the organization, communicating and
interacting well with the work group, valuing individual contributions to the project, providing
constructive feedback, showing confidence in the work group, and being open to new ideas. Thus, leader support behaviours should include both instrumental (or task-oriented) and socio-emotional (relationship-oriented actions." 

"Designing the work environment so that the natural flow of traffic through the building brings different functional areas in contact with each other could help facilitate and increase informal conversations." (Shalley and Gilson 2004) 

\subsection{Psychological safety}

"Team psychological safety should facilitate learning behavior 
in work teams because it alleviates excessive concern about 
others' reactions to actions that have the potential for em- 
barrassment or threat, which learning behaviors often have. 
For example, team members may be unwilling to bring up 
errors that could help the team make subsequent changes 
because they are concerned about being seen as incompe- 
tent, which allows them to ignore or discount the negative 
consequences of their silence for team performance. In con- 
trast, if they respect and feel respected by other team mem- 
bers and feel confident that team members will not hold the 
error against them, the benefits of speaking up are likely to 
be given more weight." (Edmondson 1999)

Interestingly, studies show how nominal groups perform remarkably better in ideation and brainstorming processes by producing greater amount of ideas than real groups. This may be due to the learnt practices and norms of a real work group, fear of failure that prevents free idea exchange and fear of evaluation and others judgement when suggesting creative solutions. (Jung 2001) 

Accordingly, Amabile (1998) suggested that creative thinking can be encouraged by shaping organisational culture such that employees feel encouraged to tell their ideas out loud freely and without judging, increasing idea exchange and discussion about them. 

"According to Mumford et al. (2002), organizational climate and culture represent collective social construction, over which leaders have substantial control and influence. Jung (2001) also views managers as playing key roles in developing, transforming, and institutionalizing organizational culture."

"Schein (1992) argues that as organizational founders and leaders communicate what they believe to be right and wrong, these personal beliefs become part of the organization's climate and culture. "

"In summary, there is substantial theoretical support for expecting that leaders play a major role in establishing an innovative organizational culture and facilitating creativity in organizations. In addition, the role of creativity and innovation in determining organizational performance has been well established. However, as Mumford et al. (2002) have observed, conspicuously absent from the literature are empirical studies on the link between leadership and innovation at the organizational level while incorporating contextual variables. Considering that researchers have emphasized the transformational leader's role in creating an intellectually stimulating work environment, it is surprising to find that no study has yet examined how a transformational leadership style affects innovative organizational climate and how it further affects creativity and organizational innovation. " (Jung et al. 2003)

"Previous research presents some intriguing evidence that people?s perceptions of the work
environment created by their team leaders and, in particular, their perceptions of instrumental and
socioemotional support, relate to their creativity" (Oldham and Cummings 1996)

According to Mumford and Gustafson (1988) "environmental variables can affect creativity by structuring problem solving efforts and by facilitating development and application of the basic generating processes giving rise to novel problem solutions." 
Furthermore, environmental factors can stimulate individuals' willingness or motivation to pursue new ideas and social environment is the source for resources and support needed in implementing novel ideas. (Mumford and Gustafson 1988)

"The data also suggest 
that team psychological safety is something beyond interper- 
sonal trust; there was evidence of a coherent interpersonal 
climate within each group characterized by the absence or 
presence of a blend of trust, respect for each other's com- 
petence, and caring about each other as people. But building 
trust may be an important ingredient in creating a climate of 
psychological safety. -- Although building trust may not nec- 
essarily create a climate of mutual respect and caring, trust 
may provide a foundation for further development of the in- 
terpersonal beliefs that constitute team psychological safety. " (Edmondson 1999)

"Effective team leader coaching and con- 
text support, such as access to information and resources, 
appear to contribute to, but not to fully shape, an environ- 
ment in which team members can develop shared beliefs 
that well-intentioned interpersonal risks will not be punished, 
and these beliefs enable team members to take proactive 
learning-oriented action, which in turn fosters effective per- 
formance. " (Edmonsdon 1999)

\subsection{Leadership behaviour}

Leaders have a great influence on employees on several levels and through several mechanisms, such as role modelling, goal definition, reward allocation, resource distribution, defining norms and values of the company, showing the way to interact as a group, condition employees' perceptions of work environment and being the lead decision maker on organisational structure and procedures (Avolio and Bass 1988) T�m� aika suoraan Amabile et al. 2004, joten muokkaa v�h�n! As a result of these, leaders have a great impact on employees' behaviour, thus leaders may have a significant effect on employees' creativity. (Amabile 1988) 

Leadership style has great impact on organisational innovation and creativity. For instance, leaders shape and define the goals and working context (Amabile 1998; Redmond 1993). Through a long-term vision (separated from short-term business outcomes, which usually focuses on quarterly profit), leader's are able to direct employee's efforts towards creativity and innovative work processes leading to likeminded outcomes (Amabile 1996)
Furthermore, in shaping and influencing organisational culture, leaders are key actors (Schein, 2010) for instance by sustaining and nurturing organisational climate that supports and encourages creative efforts and learning (Yukl, 2002). 

According to Redmond (1993) a leader can have an affect on employee's level of creativity through leadership behaviours such as problem construction, learning goals and feelings of self-efficacy. 

In addition, creating and supporting a reward-system that values creative performance, provides both intrinsic and extrinsic rewards for employee's efforts to learn new skills and to challenge status quo by experimenting new approaches, employees are constantly willing to engage in creative endeavours (Jung, 2001; Mumford and Gustafson 1988)

As Mumford et al (2002) put it, organisations may now need jazz group leaders rather than orchestra directors. 

Leaders can affect employees' creativity and innovation skills both directly and indirectly (Jung et al. 2003). By stimulating employee's intrinsic motivation and higher level needs leaders are able to affect directly on employees' creativity (Tierney et al. 1999), where indirect way may be through establishing a work environment where new ways of doing are encouraged and failure is not being punished (1996). 

Leadership behaviour is only recently recognized as essential part of enhancing creativity and innovation skills of employees (Mumford et al. 2002). Mumford et al (2002) suggests this may be due to our romantic perception of creative act, which defines creativity as an heroic act of an individual and leaders only being a hindrance to the creativity of an individual. Furthermore, other reason to this may be found in the conventional models of leadership, which not encourage employees to challenge the status quo but to achieve required goals. 

Conventional leadership behaviour focuses on internal activities within the team, whereas innovative team leader needs various set of skills and approaches in order to encourage developing and growing of teams and individuals. For instance, according to Barckzak and Wilemon (1989) leaders of innovative teams utilise wide range of familiar and unfamiliar techniques in order to accomplish the team objectives, whereas leaders of operating teams use only a few familiar techniques. Even though in this study innovation teams were not studied, similar elements of developing by experimenting and encouragement for that may be recognised, when dealing with new tasks and developing something which result is uncertain. 

Also Buijs (2007) \citeasperson{buijs2007innovation}states how leaders dealing with uncertain and new innovations should stay certain about uncertainties and provide a safe environment and encourage employees to work on current task comfortably. Thus, high level of tolerance for dealing with different states of minds and various personal feelings is required from a leader. 

Leaders and their behaviour have great influence on the creativity and innovation ability of employees (eg. Mumford et al. 2002; Jung 2001; Amabile 1998).

In order to encourage creativity and experimenting in teams, leaders should lead by example and act as role models. Leaders should consider their own behaviour and actions in a way that stimulates employees to new and innovative, creative approaches to problems. In addition, they can even request creative and innovative solutions form the team, which may lead to better results in creativity of individuals (Amabile et al. 2002). Lack of time for creativity and developing is not usually considered as part of work, thus time allocated for that is needed, as was also found in the empirical part of the thesis.   (Waldman, mum for, amiable, khaire)

Big ideas do not hatch overnight and creative thinking requires time. Leaders should allow team members time to think creatively, as according to studies under pressure creativity actually falls into decline even though individuals may feel more creative, yet actually they are perhaps working more and getting things done. According to this study, employees were clearly less creative while time pressure increased. (Amabile et al 2002) 

Furthremore, leaders can assist their employees by recognising times with high pressure, and allowing employees to focus on certain thing at a time, leaving the expectations of creativity and new ideas into the future moment, when time pressure has decreased. On the other hand, if creativity is required under stress, leader should transparently explain the importance and reasons behind the strict schedule and required goals. Thus an employee may relate to the problem at hand and engage better at his work. Indeed, helping people to understand the importance of work is essential especially under high time pressure. (Amabile et al. 2002)

Through encouraging employees in risk-taking and making mistakes, leaders are likely to boost innovation. (Farson and Keyes, 2002)

Failure as a part of innovation and development process begins to be generally recognised and approved (Farson and Keyes, 2002). Succeeding companies even thrive for failure in order to learn fast and find the best practices and business models. For instance, credit company Capital One conducts continually large amount of market experiments. They now most of the tests will not pay off, yet they also know how much can be learned about customers and markets from failed tests in early phase of development.  (Farson and Keyes, 2002). Yet leaders fail in showing their employees the support and tools for failing fast and early enough. Failing in a personal matter remains a difficult subject, as failing never feels exceptionally great, and often employees still consider failed work as failing personally. (Farson and Keyes, 2002) 

Failure-tolerant leaders put effort on explaining to employees how important part failure is to the development process as a whole, and how failing actually refers to a point where surprising, failed outcomes are not reflected and further analysed in order to learn. Permorming accordingly, admitting own failures and not chasing anyone to blame, failure-tolerant leaders encourage failure, lower the threshold and ease the fear of failing of employees. (Farson and Keyes, 2002)

Naturally management need to take seriously issues about safety and health, yet most of the failures should be seen as opportunities for growth. Furthermore, failure-tolerant leaders treat success and failure similarly, analysing and reflecting the outcomes in order to grow the intellectual capital of the team, including experience, knowledge and creativity. Other characteristics of failure-tolerant leaders are being rather collaborative than controlling, listening carefully, seeing the bigger picture, asking questions and focusing on the development and future rather than blaming on mistakes. In addition, in order to gain empathy and trust among employees, leader should admit their own mistakes, as it shows self-confidence and honesty, assisting in forming closer ties with employees. Vulnerability and transparency play a major role in trustworthy relationship between leader and employees. (Farson and Keyes, 2002)

Through the green light given and their own example leaders can change the focus from success and failure into thinking in terms of learning and experience. (Farson and Keyes, 2002)

"Expertise, specifically in the domain of organisational leadership also had a moderate direct effect on the expression of divergent thinking skills." (Vincent et al. 2002)

As Amabile and Khaire (2008) draw a poetical picture how leader cannot manage creativity, but manages for creativity. Furthermore, they suggest that culture that fosters creativity includes leadership that enables collaboration, enhances diversity, encourages ideation, maps the stages of creativity to different needs, accepts inability and utility of failure and motivates employees with intellectual challenges. 

In experimentation process, employees need to contribute imagination, and this may require new kind of encouragement for creativity from the leaders. (Amabile and Khaire, 2008) Much success rises from employee's own initiatives, which results from wide amount of autonomy at work. 

Change from authority-based leadership to collaboration with employees has occurred in literature and in practice (Amabile and Khaire, 2008; Farson and Keyes, 2002). 

A culture of creativity can be fostered in an organisation through opening the organisation to diverse perspectives and openness to various ideas. This calls for safe environment for employees to share their thinking from different fields of expertise. Furthermore, engouraging passion and knowledge of an employee is likely to result in more creative action at work,  (Amabile and Khaire, 2008)


Role models 
Lead by example
Allow experimenting 

Empowering employees is an essential tasks of leaders, through which a work environment is created where employees desire to seek innovative approaches to per form their work tasks (Jung et al. 2003). Transformational leaders encourage employees to participate in developing by highlighting the importance of cooperation, providing the opportunity to learn from shared experience and allowing employees to perform necessary actions in order to be more effective (Bass, 1990). Furthermore, autonomy and freedom to perform essential tasks has major effects on organisational creativity, as individuals are more likely to produce creative work when having the feeling of personal control over how to approach given tasks (Amabile et al. 1996) 
Yet, in order to maintain organisational innovation and risk-taking, autonomy given to an employee can not be in contradiction with fear of failure or discouragement towards challenging status quo or trying out novel solutions (Yukl, 2002). Thus, organisational climate has to support and encourage innovation (Mumford and Gustafson 1988) by valuing initiative and innovative approaches that support employees in risk-taking, accepting challenging assignments and stimulate intrinsic motivation towards work (Jung et al. 2003).

Allow resources (material: Katz and Allen 1988)
Provide clear focus (Barckzak and Wilemon 1989)

Transformational leadership refers to leadership style and processes which emphasises longer-term and vision-based motivational processes (Bass and Avolio 1997). Furthermore, through offering an explanation of the importance and value of the work, leaders encourage their employees' to think beyond self-interest (Yukl, 2002) 

Jung (2001) has studied how leadership style affects group's creativity and performance by comparing transactional and transformational leadership styles. Transformational leader refers to a leader who encourages divergent thinking and looking at problems from unconventional perspectives, while providing and explaining clearly defined goals and facilitating the innovation process of employees (Bass and Avolio 1990). Furthermore, development of clear long-term vision and practises supporting the way to achieve it is essential characteristic of transformational leaders (Avolio and Bass 1988). The relationship between transformational leader and an employee is active and emotionally attached (Avolio and Bass 1988) and through the strong attachment resulting from tight relationship leaders can better support employees in using their personal values and self-concepts in the way that employees can pursue higher level performance and fulfil personal needs through the work. This focus of transformational leadership on value alignment is likely to lead to the root of intrinsic motivation of an employee (Gardner and Avolio 1998), which is considered as one of the key elements in creative thinking and innovation skills of an employee (Jung 2001; Amabile 1998, Deci and Ryan 1985)

Transformational leaders can build environments that support creative actions (Sosik and al 1998; Avolio and Bass 1988). According to Sosik and al (1998) one key characteristic of transformational leader is the intellectual stimulation, which is likely to encourage creativity and divergent thinking leading to unconventional solutions to problems at hand. 

According to Bass and Avolio (1994) transformational leadership consists of four unique yet interrelated behavioural components: inspirational motivation (articulating long-term vision), intellectual stimulation (promoting creativity and innovation), idealised influenced (meaning charismatic role modelling) and individualized consideration referring to coaching and mentoring leadership style. 

In contrast to transformational leadership, transactional leadership refers to focus on employees ability to fulfil and achieve clearly defined goals (Hollander 1978; House 1971) and successful goal achievement is rewarded (Waldman et al. 1990). This exchange relationship between leader and employee is based on a contract of specified goals and emphasises on the process of achievement of objectives (Avolio and Bass 1988) but does not encourage employee's to develop their creativity and innovation skills (Jung 2001). Instead, employees are rather motivated extrinsically to perform their job under transactional leader but not expected to question and change the status quo in creative ways (Amabile 1998). 

Intrinsic motivation has been attached to creative and innovation performance of an individual as intrinsically motivated individuals usually prefer novel solutions, challenging status quo and trying out new ways for solving a problem at hand. (Amabile et al. 1994)


"Another common conception of creative leadership holds that the leader of creative people must inspire providing followers with a meaningful, motivating vision of the work and its implications (Sosik et al., 1999)."

Because leaders define the context in which their followers interact and work toward a common goal, we believe that previous findings of a positive link between transformational leadership and individual creativity can be extrapolated to an organizational level. 

Few studies have been made linking the transformational leadership and positive outcomes of employees' creativity in organisational level and outcomes, even though several studies have been made suggesting the positive relation between these factors. in their study Jung et al (2003) draw this link clearer and suggest that while leaders define the context and goals of their employes, transformational leadership can be extrapolated to an organisational level. 

Jung et al (2003) set four hypothesis. First of all, they suggest that transformational leadership is positively related to organizational innovation. Secondly Transformational leadership is positively related to employees' perceptions of (a) empowerment and (b) support for innovation. Thirdly
Employees' perceptions of (a) empowerment and (b) support for innovation have a positive relationship with organizational innovation.
Lastly, Employees' perceptions of (a) empowerment and (b) support for innovation moderate the relationship between transformational leadership and organizational innovation such that the relationship will be stronger when perceived empowerment/support for innovation is high rather than low.

"We have integrated extant discussions of leadership to propose four sets of hypotheses about how transformation leadership shown by top executives directly and indirectly affects innovation at the collective level of the organization" (Jung et al. 2003)
"Findings based on 32 Taiwanese companies provide support for our expectation that a direct and positive relationship exists between transformational leadership and organizational innovation. We also find positive and significant relationships between transformational leadership and empowerment as well as support for innovation and a positive relationship between support for innovation and organizational innovation. These results support our proposition that transformational leadership by the top manager can enhance organizational innovation directly and also indirectly by creating an organizational culture in which employees are encouraged to freely discuss and try out innovative ideas and approaches."

"Because undertaking innovative approaches to work typically requires making risky decisions, empowerment per se, if not accompanied by guidance and some measure of structure, could lead to negative consequences in a high power distance culture." (jung et al. 2003)

"Because many aspects of leadership behavior can be learned or modified, our findings suggest that organizations can improve their innovativeness by helping managers to develop and display transformational leadership behaviors through training and mentoring processes." (Jung et al. 2003)

"Second, and not totally independent of the possible role of power distance, our finding may underline a general need for transformational leaders to maintain a balance between letting people feel empowered and providing guidance via defining goals and agenda. Creative work processes often involve a complicated process of pulling available resources together to recognize current market trends, focus on the core message and strategy, and develop people in line with the strategy (Schein, 1992). Unless the leader plays an active role in providing guidance, coordinating and supporting these activities, employees or organizational units might wind up working at cross-purposes. Indeed, several studies have found a positive relationship between leaders' initiation of work structure and performance of creative activities (i.e., Keller, 1992). At the same time, the importance of allowing room for individual experimentation and initiative is emphasized by Mumford et al. (2002), who state that ??planning by leaders should not focus on the conduct of a specific piece of work. Rather, leaders' planning should focus on progress, the general types of projects that should be pursued, and the consequences of pursuing project results into development? (p. 716). " (Jung et al. 2003)

Time (Gruber and Davis 1988)
"Employees' willingness to experiment and take risks also may depend on the tightness of the resource and time constraints that they face at work. In turn, these aspects of the work environment may be affected by the extent to which superiors permit subordinate participation in establishing budgets and performance standards and in the latter's performance evaluation." (Jung et al. 2003)

encourage risk-taking (Sethi et al. 2001) 

"Sethi et al. (2001) were able to demonstrate that when a multifunctional team is given a powerful stimulus to take risks, the members of the team are more motivated to integrate their particular perspectives. As a result, members tend to identify themselves with the team (more than with their respective functional areas) and tend to be more committed to the success of the team. In this research, the cited interaction had a positive effect on the radicalness of innovation. "

Waldman and Bass 1991
Hohn, 2000

Under some circumstances, according to Monge et al. (1992) \citeasperson{monge1992communication} group communication is likely to increase innovation. Thus, leaders should consider managing wide range of formal and informal meetings and facilitated discussions in order to create opportunities for ideation. Furthermore, innovation occurs over time and is a dynamic process. Leaders should be sensitive in which pace more managerial impact is needed, and in which pace of the process more freedom and autonomy should be allowed for employees. \cite{monge1992communication} 

Organizational leaders play a great role in establishing strong team performance culture. This can be achieved through addressing and demanding performance that meets the need of customers, employees and shareholders. Teams should not be fostered by the sake of the team only, rather should leaders clearly state how the team performance affects to customers and through that foster clearer performance ethics and cultures. In addition, even though people tend to have great sense of individualism, it does not have to bias the teamwork performance, as real teams find ways to support individual strengths and performance for shared goal. Furthermore, in order to team function properly and efficiently, discipline across the team and organisation is needed, focusing again on performance.  \cite{katzenbach1993wisdom}(Katzenbach 1993) 

Tolerance for ambiguity 
 
 Ambiguity is often perceived by individuals when lacking sufficient cues to structure a situation, and usually arises from novelty, complexity or unsolvability of situation at hand (Budner 1962). 
 
Role model
According to studies leaders have a strong direct impact on employee's behaviour and way of performing at workplace (Katz and Kahn 1978; Redmond et al. 1993)
Leadership plays a major role in defining group goals, controlling resources and providing rewards through interactive leadership process, making leadership behaviour an essential environmental variable in stimulating creative behaviour as a means for achieving goals. (tsekkaa t�� rakenne viel�, Redmond et al. 1993) Katz and Kahn (1978) even refer to role of the leader in a sense where leader defines by his example the reality of workplace; norms, practices and culture. 

By defining organisational culture, climate and group norms leaders shape the way of working of employees. Through such role-modelling and mentoring process leaders also show employees in practise how tasks are performed. Employees, in turn, follow the example of leader in order to achieve high level of performance. (Redmond et al. 1993) 
 
Role-modeling stands also as powerful tool for opening employee's eyes and attitudes to new perspectives, thinking 'out of the box' and to adopt generative and exploratory thinking processes (Jung et al. 2003). 
"Role modeling by supervisors also can influence employee creativity" (Shalley and Gilson 2004) 
"Best thing a manager can do to encourage an inventing style is to act as role model and use that style herself" (Sternberg et al. 1997)
 
 Although different leadership styles and their effect on employee's creativity behaviour has not yet been studied widely, some studies show, how transformational leadership behaviour encourages employees look problems from different perspectives and thus widen their intellectual and creativity skills (Jung 2001; Sosik et al. 1998). Jung (2001) has studied the relation between leadership style and group creativity finding that transformational leadership is most likely to stimulate creative effort of employees. 
 
 In his study, Jung (2001) emphasises that transformational leadership skills can be practised in order to foster creativity and intellectual skills of employees and shape organisational culture. His study showed how transformational leadership; encouraging divergent thinking and solving problems at hand from unconventional perspectives, is likely to increase intrinsic motivation of employees leading to more creative problem solving and behaviour.  Through brainstorming activities that focus on non-traditional thinking and fantasising intellectual skills of employees can be enhanced (Sosik et al. 1998). 
 
Sosik et al (1998) furthermore suggested that anonymous ideating through nominal groups leads to better results and greater amount of ideas than brainstorming activities in real groups. This may be due to fear of failure and measurement of performance (Jung 2001) tsekkaa t�h�n viel�, tai yhdist� aiempaan. jossain se lukee.

"In this paper, we will break the silence surrounding these issues and expose the implications of Quin's message. We will introduce yet another paradox: innovation leaders should be controlled schizophrenics. We will argue that in order to managing innovation (processes) the responsible leaders (managers and/or consultants) have to behave and act in different and conflicting roles and take on different attitudes at the same time. They must do this without losing contact and rapport with their innovation team members. They need to have a great tolerance for dealing with the different conflicting and competing aspects of innovation within the innovation team. Like the leader, the innovation team is also struggling with the same conflicting aspects but with a different time scale, different interest levels and with differences in knowledge and experience. The innovation team members expect their leader(s) to be in charge and to be in control. Yet, they also want support, enthusiasm and trust. They do not want to hear that their leader sometimes has doubts about the direction to be taken, the sequence of the process and about the success of the project. If the innovation leader shows his doubts too strongly, then he ruins the game." (Buijs, 2007)

As Buijs (2007) argues, leaders who are to lead employees and work handling innovations need to understand the paradox and natural conflicts between routine processes (exploitation) in order to earn money in the present and the innovation processes (exploration) in order to earn money in the future. Buijs' (2007) four aspects for innovation (innovation process, psychological process of innovation team, creativity process, leading and playing), which leader should be able to master all providing a secure environment for a team to perform in novel and creative ways. As Buijs states "While the leader is already thinking about the next uncertain step to be taken, the team has to be encouraged to execute the present step comfortably."

"Incrementalism helps deal with the psychological, political, and motivational factors that are crucial to project success. By keeping goals broad at first, a manager avoids creating undue opposition to a new idea. A few concrete goals may be projected as a challenge. To maintain flexibility, intermediate steps are not developed in detail. Alternate routes can be tried and failures hidden. As early problems are solved, momentum, confidence, and identity build around the new approach. Soon a project develops enough adherents and objective data to withstand its critics' opposition.
As it comes more clearly into competition for resources, its advocates strive to solve problems and maintain its viability. Finally, enough concrete information exists for nontechnical managers to compare the programs fairly with more familiar options. The project now has the legitimacy and political clout to survive-which might never have happened if its totality has been disclosed or planned in detail at the beginning. Many sound technical projects have died because their managers did not deal with the politics of survival." (Quinn 1985)

"Of all of the
forces that impinge on people?s daily experience of the work environment in these organizations, one of
the most immediate and potent is likely to be the leadership of these teams?those ??local leaders?? who
direct and evaluate their work, facilitate or impede their access to resources and information, and in a
myriad of other ways touch their engagement with tasks and with other people." (Amabile et al 2004)

"The classic behavioral approach is the two-factor theory of leadership, which specifies that all leader
behaviors can be characterized as either task oriented (??initiating structure??) or relationship oriented
(??consideration??)(Fleishman, 1953). Task-oriented behaviors focus on getting the job done, and include
such things as clarifying roles and responsibilities, planning projects, monitoring the work, and
managing time and resources. Relationship-oriented behaviors focus on the socioemotional: showing
consideration for subordinates? feelings, acting friendly and personally supportive to them, and being
concerned for their welfare. It is important to note that in the leader behavior literature, the term
??support?? typically refers to relationship-oriented behaviors only, while in the creativity literature,
??support?? typically refers to all leader behaviors that could enhance creativity?both task- and
relationship-oriented. We adopt the latter, broader usage." (Amabile et al. 2004)

"This study suggests that a leader who interacts daily with subordinates may, through certain
behaviors directed at those subordinates, influence their daily perceptions, feelings, and performance,
ultimately influencing the overall creativity of the work that they do.--
In particular, local leaders display support for subordinates and their work by monitoring progress
efficiently and fairly, consulting with them on important decisions, supporting them emotionally, and
recognizing them for good work. They display a lack of support by monitoring progress inefficiently
or unfairly, giving unclear or inappropriate task assignments, and failing to address important
problems.--
The componential theory of organizational creativity (Amabile, 1988, 1997) provides a general
framework in which to understand these effects. It states that the perceived work environment can
have a significant impact on individual and team creativity, it identifies local leader support as one
important aspect of the perceived work environment for creativity, and it suggests some behavioral
elements that may constitute such support.--
The componential theory of organizational creativity (Amabile, 1988, 1997) provides a general
framework in which to understand these effects. It states that the perceived work environment can
have a significant impact on individual and team creativity, it identifies local leader support as one
important aspect of the perceived work environment for creativity, and it suggests some behavioral
elements that may constitute such support." (Amabile et al. 2004)

"Several behaviors deserve particular emphasis in the leader?s repertoire, behaviors requiring the
following: skill in communication and other aspects of interpersonal interaction; an ability to obtain
useful ongoing information about the progress of projects; an openness to and appreciation of
subordinates? ideas; empathy for subordinates? feelings (including their need for recognition); and
facility for using interpersonal networks to both give and receive information relevant to the project.
Perhaps just as importantly, there are also several behaviors for leaders to avoid or reduce, including
giving assignments without sufficient regard to the capability or other responsibilities of the subordinate
receiving them; micromanaging the details of high-level subordinates? work; and dealing inadequately
with difficult technical or interpersonal problems (whether due to technical incompetence, interpersonal
incompetence, inattention, or sloth)." (Amabile et al. 2004)

In their study, Shalley and Gilson (2004) present how leaders should use human resource practices in order to develop work context which improves the creativity skills of employees. 

"that many of the traditional roles of leadership are being redefined in today's flatter organizational structures. With employees having more direct responsibility over their day-to-day work, the leaders' role is being redefined so that they are more involved in external resource acquisition and boundary spanning. " (Shalley and Gilson 2004) 

"Oldham and Cummings (1996) found that supportive, noncontrolling supervisors created a work environment that fostered creativity. Finally, Tierney, Farmer, and Graen (1999) found that open interactions with supervisors and the receipt of encouragement and support lead to enhanced employee creativity."

 "However, the research on supervisor support consistently finds that contextual factors interact with individual characteristics to affect creative performance. Therefore, leaders need to understand their employees to provide the right levels of support needed for creativity to occur." (Shalley and Gilson 2004) 
 
 As Garvin et al. (2008) bring out in their article, reasonable question to ask for this fresh leadership approach is "where is the glory in being a "facilitator" as a manager? How do you get a management level made up of real humans who aspire to that role and will do it?" Elizabeth Long Lingo (2010) has offered one perspective to this question in her study with production of music. She claims that producer is the one bringing it all together; it is actually hard leadership exercise, where people from different fields and teams need to work together for one production, where there are no clear rules for who is controlling the output nor yardstick how good or bad the production is. Through creating a shared purpose and common goal in production team, and while still letting "other apply their distinctive expertise", a producer actually operates at the centre of the storm without being at the focus of attention as well as aims for productivity without being over controlling. According to this example, glory comes from being able to help others to find and realise their unique talents at the same time with achieving a collective goal. 
 
 "Team leader coaching is also likely to be an im- 
portant influence on team psychological safety. A team lead- 
er's behavior is particularly salient; team members are likely 
to attend to each other's actions and responses but to be 
particularly aware of the behavior of the leader (Tyler and 
Lind, 1992). If the leader is supportive, coaching-oriented, 
and has non-defensive responses to questions and chal- 
lenges, members are likely to conclude that the team consti- 
tutes a safe environment. In contrast, if team leaders act in 
authoritarian or punitive ways, team members may be reluc- 
tant to engage in the interpersonal risk involved in learning 
behaviors such as discussing errors, as was the case in the 
study of hospital teams mentioned above (Edmondson, 
1996). Furthermore, team leaders themselves can engage in 
learning behaviors, demonstrating the appropriateness of and 
lack of punishment for such risks." (Edmondson 1999)

"Organizations use a variety of types of teams. Team type 
varies across several dimensions, including cross-functional 
versus single-function, time-limited versus enduring, and 
manager-led versus self-led. These dimensions combine to 
form different types of teams, such as a time-limited new 
product development team or an ongoing self-directed pro- 
duction team. The team learning model should be applicable 
across multiple types of teams, because the social psycho- 
logical mechanism at the core of the model concerns people 
taking action in the presence of others, and the salience of 
interpersonal threat should hold across settings." (Edmondson 1999) en oo karma onko olennaista
 

\subsection{Practices and structures for developing and experimenting}

According to Amabile et al. (2002) clear time should be allocated for developing especially when the aim is to flourish idea generation, creativity, learning and experimentation of new concepts. Time pressure should be minimal in order bright ideas to glow as cognitive processing requires time of an individual and team. Yet, no sense of urgency leads employees easily to auto-pilot mode, in which routine tasks are performed without further thinking and analysing. Thus, creative time for playing with ideas, brainstorming, learning and experimenting should be allocated in an organisation in order truly new things to develop. Shared goals are once more essential in engaging team members to play with ideas and feel more motivated in developing their work.

Creativity, exchanging ideas and turning them into action requires intrinsic motivation from employees (Jung 2001). Thus, in order to increase creativity and innovation at workplace, leaders should foster organisational culture in which individuals find their motivation in divergent thinking and trying out new ways of performing tasks (Amabile 1998). 

Mumford et al (2002) -artsusta
-  In the sense that leaders must acquire resources and encourage the generation of new ideas (e.g., McGourty et al., 1996, t�� lis�tty referensseihin), there is some truth to this view. However, leaders seem to serve a number of other roles when people are engaged in creative work: evaluating their ideas (Sharma, 1999), integrating their ideas with the needs of the organization Mumford, 2000a and Mumford, 2000b, and creating conditions where people can generate ideas in the first place (Andrews and Gordon, 1970, t�m� l�ytyy my�s referensseist�). 


 "Amabile (1998) also has suggested that by influencing the nature of the work environment and organizational culture, leaders can affect organizational members' work attitudes and motivation in their interactions, thereby affecting their collective organizational achievement."
 
 "The extent to which they
will produce creative?novel and useful?ideas during their everyday work depends not only on their
individual characteristics, but also on the work environment that they perceive around them" (Amabile et al 1996)

Prior studies show how creative efforts of employees require sufficient amount of time and energy. (Gardner 1988; Getzels and Csikszentmihalyi, 1975) Redmond et al. (1993) argue how leaders should allow enough time for problem solving and creative actions. 

"For instance, while job rotation has become popular, managers need to ensure that employees have enough experience in an area of work if they want them to be creative. Therefore, while individuals from different areas may bring a new perspective to the work, they also need to have sufficient experience and familiarity with the target area so that creativity can occur." (Shalley and Gilson 2004) 

"We recognize that managers are sometimes wary of giving employees too much autonomy, such as giving them full control over how their work is planned and conducted. However, giving appropriate levels of autonomy to employees may be useful." (Shalley and Gilson 2004) 

"Taken together, this body of research indicates that leaders need to set appropriate goals and requirements so that individuals will aspire to be creative. At the same time, they need to be aware that if job requirements or goals are set for behaviors that may not result in creative outcomes or may directly contradict engaging in creative activities, then employees may exhibit less creativity in their jobs." (Shalley and Gilson 2004) 

"when managing for creativity, time is a critical resource that managers need to ensure their employees have access to. Here, it is important to emphasize that, in general, it is far easier for most employees to stick to routine tried and tested methods that are typically more efficient rather than experimenting and trying to come up with creative approaches. For instance, it takes more cognitive effort to generate multiple alternatives, suspend judgment, and look at problems in a different and often divergent manner. However, by engaging in creative activities, the quality of decisions or judgments should be better." (Shalley and Gilson 2004) 

"Amabile and Gryskiewicz (1987) found that one frequently mentioned factor necessary for promoting creativity was sufficient time to think creatively, explore different perspectives, and play with ideas.  Likewise, Katz and Allen (1988) found that for engineers working on new technologies, uninterrupted time was considered to be critical."

"A recent study by Amabile et al. (2002) found that individuals under time pressure are significantly less likely to engage in creative cognitive processing."

"Therefore, managers need to ensure that employees have enough time to be creative, which can be especially difficult in today's fast-paced, rapidly changing world." (Shalley and Gilson 2004) 

"In addition to time, employees need access to material resources to be creative (Katz and Allen, 1985). However, with regards to material resources, managers are faced with an interesting dilemma. That is, while material resources have been described as important for creativity, it also has been suggested that their availability or abundance might negatively impact creativity (Csikszentmihalyi, 1997). For example, while resources are needed to perform one's job, not having everything that is needed readily at hand, in fact, may stretch employees to think of different ways of doing their work. In other words, a lack of material resources may actually help foster creativity. Taking this a step further, Csikszentmihalyi (1997) suggests that resources can make individuals too comfortable, which can have a ?deadening effect on creativity? (p. 321). "

"Providing employees with performance feedback is a key function that many managers struggle with. Giving feedback can be particularly important for creativity and yet particularly difficult in that creativity often involves trying new things and taking risks. " (Shalley and Gilson 2004) 

"Another important component of organizational structure is how levels of responsibility and formal reporting relationships are organized. For instance, a highly bureaucratic organization may not encourage employees to try new ways of doing their work, whereas a flatter structure with wider spans of control may be more conducive to employee creativity." (Shalley and Gilson 2004) 

"Cummings and Oldham (1997) found that individuals with creative personalities produced more creative outputs than those with less creative personalities only when they were surrounded by an organizational context that facilitated creativity."

"With regards to creativity, it should be important that employees perceive their work context as one where decisions are made and applied in a just manner. " (Shalley and Gilson 2004) -> Fariness

"While overall climate is often regarded as a hard thing to change, there are several components of climate that are reasonably manageable and should have an effect on creativity. For example, fostering a climate where risk taking and constructive task conflict are encouraged can be role modeled and actively encouraged and supported by management. Likewise, a review of a division's or organization's management hierarchy and reporting structure may highlight that employees are not encouraged to make decisions on their own and thus may be less likely to try new ways of doing their work. Finally, if the bureaucracy associated with changing anything is such that it takes a great deal of time and effort to get new ideas considered, employees also may be less likely to try new approaches to work." (Shalley and Garvin 2004) 

"The human resource practices used to select, train, appraise, and reward employees all need to be systematically linked together so employees know what is expected of them and when and how. This also ties back to the importance of procedural justice in that if employees understand how, when, and for what they will be rewarded, promoted, or even fired, then they should have a stronger sense of fairness and subsequent organizational commitment, loyalty, and increased levels of organizational citizenship behavior. In addition, it is specifically these types of attitudes that need to be fostered for creativity to occur. For instance, employees who are not loyal or committed to their organizations will not be willing to give more than is required by their job and therefore will be more likely to stick to the tried and true ways of performing their tasks rather than searching for alternative solutions." (Shalley and Garvin 2004) 

"The practical implications of our review for the day-to-day management of creative people should be highlighted. First, across the empirical studies reviewed, one common theme is that individuals need to feel they are working in a supportive work context. This applies to how leaders interact with employees, how coworkers, team members, and even others outside of work interact with employees, whether sufficient resources are available, how employees expect to be evaluated and rewarded, and whether the climate is perceived to be supportive (e.g., a perceived fair environment). Thus, managers should attempt to increase the supportiveness of the work context." (Shalley and Garvin 2004) 

Amabilen (1997) tekstiss� pdf:n sivulla 17 on hyv� kiteytys Amabilen osalta, mit� innovointi vaatii organisaatiolta. 

In addition, Amabile and Khaire (2008) argue how sufficient time and resources should be allowed for exploration. 
 
\subsection{Attitude towards failure and risk}
In order to decrease the fear of failure, Amabile and Khaire (2008) \citeasperson{amabile2008creativity} suggest leaders should put emphasis on creating an environment where an employee feels it is safe to fail and speaking out loud ideas nor making mistakes does not result in punishment or humiliation. Leaders should, instead, motivate and encourage employees to ideate and learn by stating how essential experimenting, iterating and failing is for learning and developing. 
"those in a 
position to initiate learning behavior may believe they are 
placing themselves at risk; for example, by admitting an er- 
ror or asking for help, an individual may appear incompetent 
and thus suffer a blow to his or her image. In addition, such 
individuals may incur more tangible costs if their actions cre- 
ate unfavorable impressions on people who influence deci- 
sions about promotions, raises, or project assignments" (Edmondson 1999)

"Asking for help, admitting errors, and 
seeking feedback exemplify the kinds of behaviors that pose 
a threat to face (Brown, 1990) even when 
doing so would provide benefits for the team or organization (edmondson 1999)"

"there is no innovation process without failures and mistakes. Organizations need to learn from them as quickly as possible. If the organization out-learns its competitors, they then take the lead. And taking the lead is what innovation is all about!" (Buijs 2007)

"The phenomenon of threat 
rigidity has been explored at multiple levels of analysis, 
showing that threat has the effect of reducing cognitive and 
behavioral flexibility and responsiveness, despite the implicit 
need for these to address the source of threat (Staw, Sand- 
elands, and Dutton, 1981)."

"In sum, people tend to act in 
ways that inhibit learning when they face the potential for 
threat or embarrassment (Argyris, 1982)." 

"Nonetheless, in some environments, people perceive the 
career and interpersonal threat as sufficiently low that they 
do ask for help, admit errors, and discuss problems. Some 
insight into this may be found in research showing that fa- 
miliarity among group members can reduce the tendency to 
conform and suppress unusual information (Sanna and Shot- 
land, 1990); however, this does not directly address the 
question of when group members will be comfortable with 
interpersonally threatening actions." (Edmondson 1999)

Failing is widely considered as essential part of learning (Farson and Keyes, 2002; )

Andriopoulos and Lowe (2000) :
"The risk attached to creative work implies both a need to experiment and a need to tolerate failure Andriopoulos and Lowe, 2000 and Quinn, 1989."(T�� andriopoulos l�ytyy PDF:n�, k�y l�pi, on varmaan hyv�� setti�!) 

Suoraa lainausta Mumford et alilta (2002) 
"Thus, creative work is contextualized with the success of creative ventures depending on an awareness of the capabilities of, and pressures on, extant socio-technical systems. In fact, it is this contextualization of creative activities that accounts for such well known phenomena as simultaneous invention and the tendency for innovation to occur in spurts within a given industry Csikszentmihalyi, 1999"

Feeling of self-efficacy may affect individual's willingness to provide unique and novel ideas even when some degree of risk is involved (Mumford and Gustafson 1988). Training, coaching, giving feedback and assigning tasks seem to be useful approaches for leaders, who pursue to contribute empoloyee's self-efficacy (Amabile 1983; 1988). 

Fear of failure can be decreased through transformational leaders who foster the culture of intrinsic motivation and rewards from creative endeavours, sea exchange and discussion. (Amabile 1998) 

According to Amabile et al.(1996) creative solutions in an organisation can be achieved by encouraging employees to reach and experiment new perspectives and ways of performing. Essential for this is not being punished for negative outcomes. Organizational environment that allows failing is likely to assist in employees acquiring diverse perspectives and questioning the status quo and habitual way of performing. 

"In addition, established theory recognises the need for developing opportunities where employees can exploit uncertainty. Sternberg et al. (1997) argue that managers should let messiness exist. In other words, they suggest that uncertainty associated with creative projects must not be controlled in order to establish some order."
"The importance of
challenging work has also been emphasised
by Amabile (1997), who states that matching
creative employees to assignments, based on
their skills and interests, enhances their
motivation toward work." 

"Motivate people to contribute ideas by 
making it safe to fail. Stress that the goal is to 
experiment constantly, fail early and often?
and learn as much as possible in the process. 
Convince people that they won?t be punished 
or humiliated if they speak up or make 
mistakes." (Amabile and Khaire 2008)

"The comparison of perpetual
challenging with established corporate
creativity theory reveals that the emergent
category of adventuring comes closer to the
five stages of Amabile's (1988) componential
framework of creativity. Specifically, the
properties of adventuring, such as
introspecting, scenario making and
experimenting can be associated with
Amabile's preparation, response generation
and validation stages of the creative process.
Moreover, her conceptual model emphasises
the fact that the process can have a negative
consequence, which can also be found also in
the emergent category of ``adventuring''
under the title ``mistake making'' within this
grounded theory study." \cite{andriopoulos2000enhancing}


"to develop new and useful products or processes, individuals have to be willing to try and to possibly fail. For many, this is not an easy thing to do and can, in part, depend on the individual's predisposition toward risk as well as the organizations culture, which will be discussed later in this article. " (Shalley and Gilson 2004) 

"Research has indicated that people tend to avoid risk and prefer more certain outcomes (Bazerman, 1994). However, because creativity does not just happen but rather evolves through a trial-and-error process that involves risk taking, failure will often occur along with success. If employees are risk averse, it is much easier for them to continue performing in more routine ways rather than take a chance with a new, and potentially better, approach. Therefore, a key in the motivation of employees toward creativity is to ensure that they feel encouraged to take risks and break out of routine, safe ways of doing things." (Shalley and Gilson 2004) 

"If leaders value and want employees to be creative, a critical contextual factor they need to attend to is fostering an environment where risk taking is encouraged and uncertainty is not avoided. This has been referred to as providing a culture where employees feel psychologically safe such that blame or punishment will not be assigned for new ideas or breaking with the status quo (e.g Edmondson, 1999). In support of these arguments, Nystrom (1990) found that organizational divisions were more innovative when their cultures reflected challenge and risk taking, " (Shalley and Gilson 2004) 

"participative safety, being able to give input without being judged or ridiculed, has been positively linked to creativity (De Dreu and West, 2001)"

Predicting the future being impossible, focus should be in managing risks involved in playing with creative ideas in both the company and individual level. As Stenberg et al. (1997) state, "as uncomfortable as it is, while not being able to predict and control uncertainty in creative projects, the messiness does have to let exist. Kanter (1983) continues that, actually, opportunities grow from uncertainty and creative endeavours rise when struggling with uncertainty and mess, as individuals impose order where it does not exist, and thus individuals are forced to form new connections. Furthermore, allowing employees freedom to act actually arouses desire to act. (Kanter 1983)

According to Amabile and Khaire (2008) essential part of creating a safe environment for creativity is managers to decrease the fear of failure. Instead, constant experimenting should be the goal of working, learning by doing and iterating until sufficiently is learnt from the process. 

Furthermore, when company grows, it usually leads to more conservative actions and increase in fear of failure. When fearing failure managers tend to deny it happened and erase it from the memory instead of learning from it. (Amabile and Khaire 2008) T�m� n�kyi jossain haastiksessa! 

According to Edmonsdon  (1999) "any business that experiments vigorously experience failure?which, when it happens, should be mined to improve creative problem solving, team learning, and organizational performance" 

"the premise that learning behavior in so- 
cial settings is risky but can be mitigated by a team's toler- 
ance of imperfection and error. This appeared to be a 
tolerance (or lack of tolerance) that was understood by all 
team members--
The implication of this result is that people's beliefs about how oth- 
ers will respond if they engage in behavior for which the out- 
come is uncertain affects their willingness to take 
interpersonal risks.. " (Edmondson 1999)

By creating an environment that serves psychological safety for employees, organisations may capitalise on failure. In a safe environment employees are convinced they are not humiliated or punished when failing or saying out loud their ideas, concerns or raise discussion. Furthermore, failure can be divided in three separate categories: unsuccessful trials, system break-downs and process deviations, which all need to be recognized and analysed and dealt with in order learning to happen. Especially unsuccessful trials are fruitful and essential for creative learning, yet overcoming "deep ingrained norms that stigmatise failure and thereby inhibit experimenting" is needed. (Garvin et al. 2008)


\section{Experimentation-driven innovation}

Failed experiments should not be considered as failing, instead they offer valuable learning points. 

According to Thomke (2003), in the beginning every product is an idea, that was being shaped through the process of experimentation, and the ability to do experimentations is actually a measurement of company?s ability to innovate.

Experimentation is essential in order to learn about the idea, concept and prototype and whether it actually addresses a new need or a problem or solves the one at hand. Prototyping is critical part of the process, as testing the prototype in a real environment gives instant and valuable feedback for further development. 

Thomke (2003) suggests four steps for organizations to be more innovative. First of all, organization should allow and manage the work for the employees so that fast experimentation is possible. This usually requires challenging routine ways of working and shaping the routines, yet fast experimenting is essential in order to get rapid feedback for shaping the ideas. In addition, team engagement is essential, as the whole team need to understand the meaning of experimenting and developing and it should be encouraged to sharing information and ideas in as early stage of development process as possible and throughout the process. Thomke suggests using small teams and parallel experiments especially when the time is the most critical factor. 

Secondly, failing early and often, yet avoiding mistakes is important for experimenting. Failure can disclose important information and reveal gaps in knowledge, and is thus important as early phase of the development as possible. However, according to Thomke (2003), this is not an usual way for an organisation to think about failure, thus building the capacity for rapid experimentation as well as tolerating and learning from failure is essential and often requires overcoming ingrained attitudes. Encouraging and creating a culture where failing is allowed and not being afraid of, brainstorming sessions where judgement is not allowed are important for 

However, Thomke (2003) does not suggest failing and making mistakes as a result of poorly planned experiments. Mistakes and failures produce most value, when the experiment is well planned and the goal or hypothesis that needs to be tested is clear.

Thirdly, anticipating and exploiting early information can save a lot of resources in the development process. If problems are shown in the late-stage of the process, they can be even 100 times more costly than the ones discovered in the early stage. According to IDEO, an innovation and design-firm, using human-centered design-based approach, the key elements in the design process and prototyping is it being rough, rapid and right. The right-element reminds that even though the prototype itself is likely to be incomplete, it has to show the right specific aspects of a product. This forces developers to decide the factors that can initially be rough and those that must be right. In addition, exploiting early information serves as a good method for developers reflecting changing customer preferences. Briefly, information in the early stage of the developing process should be listened and discovered carefully, as the problems are cheaper and easier to solve. 

Lastly, for enlightened experimentation Thomke (2003) puts emphasis on combining new and traditional technologies. For company


\subsection{Approach of Mind}
Dream and think big, but act small. 

%In the Mind approach, there are three types of ideas: (Voiko t�t� kirjottaa kun ei taida olla viel� ihan "tieteellist�"

Experimentation approach suggests only after one has tried out an idea and performed an experiment, has he enough information and experience to continue to the execution of the idea.  

\section{Research surroundings and methodology}
\subsection{Company description}
 

\subsection{Experimentation challenge description}

In order to study experimentation behaviour in an organisation, an experimentation challenge was designed and organised. The challenge was organised separately for two client organisations of MindExpe project: The K-Retailer's Association and Service Foundation for People with an Intellectual Disability (In Finnish, Kehitysvammaisten Palvelus��ti�, KVPS). The data analysed in this thesis is gathered from different service units of KVPS and KVPS Tukena Ltd, which is a part of KVPS, focusing on providing support services for people with an intellectual disability. 

The kick-off for the experimentation challenge for the management level was held in April 2013. As the approach for developing through experimenting is not yet widely studied nor recognised way of working in the client organisation, during the launching two researchers of Mind told briefly through examples about the approach. Furthermore, practicalities and frames for the competition were presented. The managers of the units were thus given the responsibility to bring the information of experimentation challenge to their units. Mind team offered posters where easy steps for ideating and experimenting were presented.

Time for experimenting was from 24th of April until 11th of June, meaning seven weeks in total. Participants were asked to perform quick and easy experiments, reflect the learnings of them and report the experimentations through either e-mail or paper formula. 

Each unit participating experimentation challenge was responsible for its own activity. After the kick-off for experimentation challenge project leader called once to immediate superior of some units in order to gain knowledge how the team is contributing to the challenge and whether experimentations are conducted or not. However, no additional support or advising was given to units, and teams were self-driven in their activity. 

During an experimentation challenge, participants, employees of the company units, were asked to ideate ways to improve the work life, from the perspective of an employee and especially from the perspective of a customer. In addition to only ideate, they were encouraged to plan as small and easy way to test the idea as possible, in order to perform it during the time frame. Intentionally, Mind team did not restrict the style, theme or ways of experimenting. This let participants participate in a way feasible for them, their unit and working pace.

Experimentations were then reported to the jury, which consisted of members from both the development and management team of KVPS and Mind researchers. Best experiments and best reflections (experiments that helped the team to reflect and learn more about the idea, whether or not the experiment itself was successful) were rewarded in the closing session of the experimentation challenge as well as the unit that performed most experiments. In the evaluation process, jury focused on how well the goal of an experiment was recognised and kept in mind, how useful the experiment was (for instance for work efficiency or customer satisfaction), and what was learnt from the experiment. 

During the experimentation challenge KVPS Tukena reported 33 experiments and 20 were reported by the foundation, so altogether 53 experiments were reported. Experiments themselves were not further analysed in this study, as in the focus and interest of this study is the experience of an individual of the experimentation process. 

Schedule of the experimentation challenge

Experimentation challenge was essential part of empirical study, which overall took place during the year 2013. The experimentation challenge was organised during the spring and summer, following the closing session with rewards and interviews of 14 employees during the autumn 2013. Below are the detailed dates of the challenge. 

23rd of April 2013: The experimentation challenge was launched to the whole KVPS and Tukena Group
24rd April to 11th June 2013: Experimentation challenge
20th of September 2013: Closing session of the challenge and rewarding winners

In order to better understand the practicalities and structure of the challenge, experimentation challenge was first pivoted with two units of KVPS and two stores of K-Retailer's Association, before the actual challenge for the whole organisation was launched. However, the data gathered for this study does not consist of the interviews made from the pilot challenge, yet they gave the direction and frames for the actual experimentation challenge and assisted in framing the structure for the interviews.


\subsection{Research methods}

*T��h�n on viel� ihan kesken, voi muna* 
According to Morgan and Smircich (1980) qualitative research should be seen as an approach rather than a set of techniques, and especially when exploring social phenomenon, as in this study, qualitative research serves as relevant approach. 

When in the field of qualitative research, case study method can be used both in theory building and theory testing. Furthermore, it can also serve as a method for interpretive research design, which allows the constructs of interest emerging from the data and not to be defined and known in advance. In interpretive research social reality is seen as embedded within their social settings, as well as it is impossible to abstract it from them. Researchers then focus on interpreting the reality using sense-making process in comparison to hypothesis testing process. (Bhattacherjee, 2012). In this study, case study together with action research is used. 

Action research together with case study allows researching in real-life setting and furthermore supports the use of interpretive data analysis. 

Thematic analysis was used to analyze the data collected. The focus was on recognising factors affecting experimentation behaviour of an individual. Analysis process followed the idea of Braun and Clarke?s (2006) step-by-step process, which is used to identify, analyze and report patterns within data without being tied to any pre-existing theoretical framework. Whereas closely related method grounded theory focuses on theory building about the social phenomenon being studied, thematic analysis can be used more flexibly as detailed knowledge of theoretical framework. 

However, action research has been critized on its similarities to consulting instead of proper scientific approach. Action research have been claimed of producing information unqualified for generalisation. To enhance the credibility of this research and address the concerns associated with interpretive and action research, systematic and transparent description of the research methodology, data gathering and analysis process is offered. In addition, quotations from the data are presented widely in the results and in thematic and interpretive analysis process two to three researchers and supervisor of this thesis have been involved. 

\subsection{Data gathering}
The empirical data comprises of 14 semi-structured interviews, meeting notes and altogether 53 reported experimentations. Interviewees were employees from KVPS Tukena housing service units and KVPS foundation. Writer of this thesis carried out all the interviews face-to-face with the interviewees. Interviews lasted from half an hour to an hour. The interviews were recorded and transcribed. All interviews were held in the interviewee?s mother tongue, Finnish, therefore all the quotes presented in the thesis have been translated into English. Interviews concentrated on finding advantages and challenges concerning experimentation behavior and experimentation-driven development. Brief summary of the interviews and roles of the interviewees are illustrated in table x. 

As this thesis is written as a part of a MINDexpe project, and the data collected will be used as a part of other MIND researchers doctoral studies, collecting data from the perspective of factors affecting experimenting was not the only topic of concern. Thus, all of the data in interviews were not straightly relevant to interest of this thesis.

Interviews were carried out after the experimentation challenge. Interviewees were chosen widely from the units of KVPS and Tukena Group so that both active and less active units were heard. Interviews focused on identifying factors affecting experimentation behaviour of an individual in organisational context. The structure of the interview can be found in the appendix x. Even though factors affecting experimentation behaviour were mainly on focus, during the analysis process another theme was recognised from the data: effects experimenting has on an individual. 

If the interviewee had not taken part in the experimentation competition the interview focused on finding whether the routine work of an interviewee consisted of characteristics of experimentation behavior. Discussions with immediate superiors of the interviewees as well as interview notes served mostly as a tool for gaining an overall understanding of the routine work and attitude towards experimentation behavior. 

\subsection{Analysis process}
After all interviews being transcribed they were read through in order to create a preliminary understanding of the data collected. Then, to further analyze the material, transcriptions were coded into themes arising from the interviews. Two to three researchers conducted this initial categorizing under themes that seemed most appropriate. Once this phase was finished the writer of this thesis examined the themes more closely aiming to create appropriate themes and refine categories by bringing together themes that were closely connected and eliminate those without many quotations. The supervisor of this thesis together with other researchers assisted in refining the categories and themes in order to enhance the credibility of the study. The analysis process resulted in two classes, which were further divided into categories and subcategories presented below: 

Class 1: Factors affecting experimentation 
Category 1: Role of the immediate superior (Leading by example, Supporting ideation and experimentation, Giving licenses to do experiments)
Category 2: Role of the team (Democracy and low hierarchy, Supportive climate and team practices, Attitude towards failing, Team engagement) 
Category 3: Structures and practices of developing (Resources allocated for ideation and development, Collecting of ideas, Implementing new ways of working)
Category 4: Characteristics and know-how of an employee (Substance know-how, Tolerance for uncertainty, Self-criticism and confidence, Attitude and motivation towards developing)
Category 5: The gap between an idea and experiment (Characteristics of an idea and experiment, Static friction, Stakeholder distance and customer involvement)

Class 2: Effects of experimenting on individual
Category 6: Emotional experience and engagement (Positive emotions: happiness, excitement, inspiration, boost to self-esteem, Negative emotions: frustration, disappointment, fear of failure and fatigue, Engagement and motivation towards work)
Category 7: Learning (Reflection of work, Process know-how, Reflection towards work) 

\section{Results and analysis}
Two classes were identified in the analysis process: factors that have an effect on experimentation and effects that experimentation has on individual. This chapter introduces the results of the study. The results are described in two categories which are further divided into several subcategories. 

\subsection{Factors affecting experimentation behavior}
A suitable context for experimenting was defined by the interviewees through interviews and several factors were identified that affect in a way or another on experimentation behavior of an employee in an organization. In the analysis process, five different categories were formed of factors affecting experimentation. Table x summarizes those categories and subcategories. 

The experimentation process consists of ideating, planning an experiment and conducting the experiment as well as reflecting and learning from it in order to start the iterative experimentation process. Factors affecting these phases were identified from the data. 

\subsubsection{Role of the immediate superior}
Different kind of leadership behavior that affects experimenting was recognized from the data. This category consists of actions an immediate superior can perform in order to encourage or discourage experimentation. Three main themes were recognized from the data, which are presented as subcategories Leading by example, Supporting ideation and experimentation and Giving license to do experiments.

\paragraph{Leading by example}

According to the study attitude and actions of an immediate superior towards developing are important factors for the organizational unit as a whole. Interviewees claimed that experiments rarely happen if the immediate superior is not involved in the experimentation process and his attitude towards new ideas and developing is passive or negative. 

As interviewee 5 noted, especially when the work environment is passive towards developing and experiments, it is important for immediate superiors to act as role models and by own example create a trustworthy environment where employees can ideate and conduct experiments without fearing failure. Especially in a situation where an employee is not getting support from colleagues for his idea and is feeling disappointed, immediate superior can lead by example and join in the experiment in order it to happen and encourage the whole team to conduct experiments.
   
\begin{quote}
``Often they get shut down, new ideas, which is very sad, then I do not feel like even trying anymore, and in this point the leader is required. That he joins and says that now we will try this ? then it will succeed, but if it is only among colleagues, they [experiments] usually do not happen.'' [Interviewee 5]
\end{quote}
\begin{quote}
 ``Our immediate superior is such a lovely person, real idea bank herself! -- Luckily she is very development-oriented.. I mean it is nice that she does not stick to routines either.'' [Interviewee 11]
\end{quote}
Altogether, experiments are more likely to happen in workplace when immediate superior leads by example in ideating and conducting experiments himself, as can be seen in the comment above from interviewee 11.

\paragraph{Supporting ideation and experimentation}

According to the study immediate superior?s support and encouragement towards experimenting was experienced highly important in order experiments and ideation to occur among employees. Interviewees reported how the support from the immediate superior gives freedom to try out new practices, be creative and ideate together. This can be recognised from the comment of interviewee 14. Support from the immediate superior also encourages an employee to test ones limits, utilize one?s working experience and abilities. 
\begin{quote}
 ``Both the immediate superior and the nurse in charge supported right away when they knew I am good at handwork, so they told me to use it as much as possible and experiment with customers.. and they told everyone the same.''[Interviewee 14]
\end{quote}
The immediate superior of an employee acts also as a bridge between the employees and upper level management bringing ideas from the organization unit to upper levels. Few interviewees reported their immediate superior being extremely supporting and fighting for employees? ideas. According to interviewees, these superiors received support from upper level management as well. 
\begin{quote}
 ``And if we talk about even bigger experiments, so that we have to ask from upper level management, she [immediate superior] usually conveys our ideas further. So we get quite well support from there as well and it is only rarely when some idea is being shut down right away.'' [Interviewee 10]
\end{quote}
In most occasions where interviewees experienced support and encouragement towards experimentation, an immediate superior was himself very keen to developing, trying out new practices and experimentation-driven approach in work.   
\begin{quote}
``It [feedback and appreciation from the upper level management] makes the experiment more viable, appropriate and bold.'' [Interviewee 3]
\end{quote}
According to the study some immediate superiors show appreciation and support by noticing and rewarding conducted experimentations or successful ideas. This gives employees feeling that the experimentation and their work is meaningful and is thus likely to encourage experimenting behavior. Interviewee 3 states above, how the experimentation becomes bigger through appreciation and support. 

\paragraph{Giving licenses to do experiments}
However, immediate superior can not only encourage and support his employees in ideation and experimenting with words, he also has to allocate and allow resources for experimentation to happen. Thus, immediate superior is responsible for creating both environment and tools where experimenting is possible and resources are allocated for it. Many similar comments like interviewee 14 states below were recognised from the data. 
\begin{quote}
``Experimenting is possible exactly because the management is positive towards things like that. It has direct influence.. and that I can buy equipment I need and I get a possibility to organize new kind of activities and they don?t resist it..'' [Interviewee 14]
\end{quote}
Important part of allowing experiments and giving license to conduct them is allowing them to fail. If failing when trying something new is considered punishable or the goals are exaggerated, it is likely to discourage employees to conduct experiments. In contrary, one interviewee described how his immediate superior gave license to do one experiment and promised to uphold an employee if some negative feedback or results occur from it. 

Furthermore, immediate superiors may even demand developing and doing experiments by explicitly requesting creative and innovative solutions, as can be seen in the comment of interviewee 1. 
\begin{quote}
``She [immediate superior] will give space to that [experimenting] and actually even requires that we start experimenting and ideating. So, that is the idea of all these projects, to be able to create new ways of doing things.'' [Interviewee 1]
\end{quote}
However, in some units there was a clear contradiction between the request of experiments and the resources allocated for ideation and experimentation. Again, in few units immediate superiors had taken this into account and provided time and resources for experimentation behavior. 

Furthermore, as interviewee 8 noted, an essential part of experimenting is it being voluntary.
\begin{quote}
``In a certain way I think that workplace should encourage [to do experiments], but it cannot be forced..''[Interviewee 8]
\end{quote}
Freedom to ideate and participate experimenting depending on own motivation and interests as well as freedom to not do so was experienced important among interviewees. According to the study there is a significant difference whether the leader or a team encourages an employee to do experiments or if he is forced to do so. When feeling free to try out new things, ideate and develop himself and his work without asking permission constantly from different parts, an employee is more likely to perform experiments and develop his work.

\subsubsection{Role of the team}
Several aspects on how different characteristics of the team affects on experimenting behavior were identified in the study. Subcategories Democracy and low hierarchy, Supportive climate and team practices towards ideating and experimenting, Attitude towards failure as well as Engagement of the team are described below.

\paragraph{Democracy and low hierarchy}
Few interviewees reported their organizational unit having a low hierarchy making it possible to perform spontaneous experiments and tasks without asking permission and opinion from many parts. As interviewee 13 summarises, this was seen to lower the threshold and encourage experimenting.
\begin{quote}
``The low hierarchy kind of.. when you are in a big institution you always have to sort out if you can get the car of the institution and many other things, so here we don?t have those kinds of things.''. [Interviewee 13]
\end{quote}
However, most of the interviewees described the democratic view being strong in addition to low hierarchy. Overly democratic environment and decision-making in a team can either courage employees to participate in ideating and in experimenting or make the environment too passive for actually performing experimentations, when one always has to have the majority on his side in order to try out new things. Interviewee 6 emphasises this in the quotation below. 
\begin{quote}
``There is a lot this kind of where you have to take the whole work group into account, big workgroup, as team work of course takes its own time so that everyone will then be, involved in developing.'' [Interviewee 6]
\end{quote}
\begin{quote}
``So there are once in a while divergent opinions. But then we discuss, and we decide together what will be done. And everyone has to kind of work like has been agreed. So if the majority [of employees] says we will do like this, then we will do like that.''[Interviewee 9]
\end{quote}
Furthermore, when ideas are turned into experiments and practice, interviewees reported team size being a factor affecting on how efficiently or well experimentation is executed. When a size of the team is large, meaning over five employees, it is more challenging to get all employees involved and hear everyone?s opinion. In turn, if there are no people to reflect one?s ideas with, ideas may remain in one?s head and never come alive. A compromise would be needed in between these aspects. 

In addition, ways of sharing information and ideas among team is essential for experimentation. These aspects are presented more deeply in the next subcategory Supportive climate and practices. 


\paragraph{Supportive climate and team practices}
Climate among the team seems to affect a lot on how easily ideas are said out loud in a team and experimentations performed. Factors such as open and creative atmosphere and support and positive feedback from colleagues have clearly a positive boost towards experimenting in an organization.

A following pattern was identified from the data: Employees tend to ask opinion and permission from immediate superiors while also having a need for support from the team. Emerging ideas are preferably discussed with closest and most trustworthy colleagues in order to receive support, feedback, deeper understanding and reflection for employee?s idea. Only after receiving other opinion and support are they brought to a team meeting under discussion. If the climate does not support ideating nor is safe for throwing ideas, employees are very unlikely to tell their ideas to others. 

However, interviewees, reported conversations and ideating sessions together with the whole team being important as they may encourage others to ideate and tell their suggestions out loud. In addition, one is likely to achieve better results with a team than only ideating alone. One person reported realizing it being important to say ideas out loud despite feeling insecure as it may lead to surprising outcomes and encourage more silent colleagues to participate in ideating. 

In addition, interviewees described ideas starting to grow the more they are thrown into discussion, and the heterogeneity of the team being mostly inspirational and beneficial in ideation phase. Interviewee 6 summarises the power of ideating in teams. 
\begin{quote}
``But then I realize that the more I say my ideas out loud, others also get excited and ideas keep coming. So it is worth speaking, even though sometimes one might think that I cannot be always talking, so it is good to keep on talking and others will follow..'' [Interviewee 6]
\end{quote}
\begin{quote}
``We also have few who are not that active in throwing ideas or performing and they like the routines, I guess the workplace could somehow support them in developing..'' [Interviewee 11]
\end{quote}
Furthermore, as interviewee 11 states above, team and climate should encourage and support employees to participate in ideating and experimenting in order an employee to overcome oneself and gently push towards new ways of doing things. 

\paragraph{Attitude towards failing}
While discussion and giving and receiving feedback are essential in ideating phase, license to fail plays a major role when an actual experimentation is performed. Team being judgemental towards ideas and experiments can prevent actual learning and reflection of experimentations.  

Thus, one part of the supportive climate towards ideating and experimenting is the attitude towards the results of experiments. Seems that if the climate in the team allows failing and does not take it too seriously, ideation and experimentation occur more often than if a team is afraid of failing. Quote from interviewee 1 describes attitude in their unit towards failing, several similar comments about trying again together and allowing failure were recognised from the data. 
\begin{quote}
``Well no one will get punished or be thrown tomatoes at [if an experimentation fails].. I think we go through the idea and experiment, and try another way.. We do not have here that kind of attitude that we would not be allowed to fail.'' [Interviewee 1]
\end{quote}

Most interviewees described the team being very supportive and attitude towards failure positive and constructive. They rather see failure as an opportunity or a learning point. Some interviewees described going through a failed idea or experiment among team in order to find an alternative way to test the idea. This attitude seemed to help the team in experimentation behavior. 
\begin{quote}
``I think the team reacts very well to it [failing], kind of laughing and saying that these things happen and are part of this work.''[Interviewee 6]
\end{quote}
As interviewee 6 described above, humor was also described as an important way to cope with failures and to support team members in their work. 

\paragraph{Team engagement}
While most of the interviewees reported the workplace environment being very democratic and discursive, they described a high possibility that experimentations are not likely to happen if everyone in the team is not involved and engaged to turning idea into an experiment. Comment from interviewee 9 describes this further.
\begin{quote}
 ``I think that the workplace has a major role [in experiments to happen], and especially that everyone are engaged. Thus we get things going and forward..'' [Interviewee 9]
\end{quote}
Interviewees reported feeling frustrated when realizing how all colleagues who are involved in the experiment are not engaging to it, thus preventing an experiment to happen and get relevant feedback from it. One way to motivate and engage the team was found from the data and is presented in the subcategory Characteristics of the idea and experiment under category Gap between idea and experiment. Furthermore, the team is more likely to engage to an experiment when the purpose of the experiment is clear and shared goal exists.  

In addition, as interviewee 14 mentioned below, in order to foster the engagement of employees and the team, a team could encourage new employees to utilize their own strengths and ideate courageously. Few interviewees described their colleagues being highly supportive towards one?s special abilities and skills, and encouraging everyone to use them freely at work. This was experienced as improving the level of engagement towards developing and ideating.
\begin{quote}
``Every employee is allowed to use those resources and creativity that one has in the job..'' [Interviewee 14]
\end{quote}
Shared and understandable goal for an experiment forms a strong basis for idea turning into an experiment. According to the interviewees employees and the whole team is more likely to engage to the experiment if they understand in a deeper level the reason and goal for experiment. Thus, as following quote from interviewee 11 suggests, employees are more easily involved if the goal for the experiment can be clearly justified. 
\begin{quote}
``There we had a clear goal, that somehow we just have to make it work. Then we just processed it and nothing special, it was that kind [of an experiment] with a clear goal, yep.'' [Interviewee 11]
\end{quote}

\subsubsection{Structures and practices of developing}
According to the study, various organizational structures and daily practices can support or prevent experimentation and ideation. First of all, the meaning of Resources allocated for ideation and development is presented. Secondly, seems that most of the units interviewed lack of systematic way for Collecting ideas, resulting to inefficient way of developing. In addition, in order the change actually happen Implementing new ways of working has to be considered. 

\paragraph{Resources allocated for ideation and development}
Only little or no solid structure for ideation and developing was found in organizational units throughout the analyzing process. Interviewees described ideas usually emerging when a problem is encountered and an alternative way of performing is needed. In some occasions, ideas emerge accidentally  in conversations with colleagues, team meetings and rarely in meetings where developing is a specific agenda. Rather than developing purposefully interviewees described their daily work as practice-driven. 
\begin{quote}
``It can be that some person suddenly brings a good idea or then we have a specific team meeting, where the idea is to develop something. Or then some new idea might come up in some bigger meeting by accident..''[Interviewee 4]
\end{quote}
No time allocated for ideation and development of ideas leads easily to a situation where routines are repeated. In every interview time came to prominence as a lacking factor preventing ideation and experimentation from happening. Interviewees described the usual way for telling ideas and planning experiments being weekly or monthly team meetings with colleagues. As interviewee 1 states, however, these meeting usually did not include specific time for ideating and developing, more did they concentrate on routine issues. Most of the interviewees wished to have more time for ideation, developing and implementing experimentation-driven development to daily routine. Only one interviewee described having enough time for ideating. 
\begin{quote}
``Well there is not that much time to ideate during them [weekly meetings].'' [Interviewee 1]
\end{quote}
At present, most interviewees described how developing is not seen as a routine part of work but as an additional part that needs time allocated for it. Few interviewees, like interviewee 2 above, experienced experimenting challenge as a refreshing way to remind the workplace of challenging conventions and the daily routine and even thought about it becoming an annual tradition. The interviewees described that reflection is more likely to happen in between different projects and in project-type work, and when the emphasis of the work is not project-like, developing is more likely to be put aside. 
\begin{quote}
``Once a year could be kind of more intensive period or so, maybe it would maintain that no one would be too routinized. And especially in these projects that last long, so long that they are actually no longer projects, it could be quite good..'' [Interviewee 2]
\end{quote}
One interviewee described peer resources being used only little in order to exchange ideas and best practices. He suggested more meetings and ideation sessions with peer colleagues throughout Finland or even abroad in order to exchange opinions, gain perspective and find fresh, new and valuable practices to daily routine work. 
\begin{quote}
``Well, I guess the workload of some people is already so huge, causing also that people start doing things in the same way, continuing the routine..'' [Interviewee 5]
\end{quote}
Quote below from interviewee 5 clarifies how heavy workload can lead to repeating routine way of working. Sick leaves, heavy workload and hectic pace of work all affect on motivation and possibilities for developing and ideating. Furthermore, resources such as money were mentioned during the interviews as a lacking factor preventing experimentation. 

In turn, interviewees who described successful experiments said an essential factor for the success was that all the resources such as people, equipment and time were at the right place at the same time and there where no hindrances preventing experiment. Thus seems that an aspiration for using resources efficiently is essential. 

\paragraph{Collecting of ideas}
In addition to rather usual communication problems in an organization, the interviewed units lacked a working system for collecting ideas or feedback from experiments. The need for collecting ideas systematically was however recognized, as interviewee 5 emphasises.. Interviewees report daily work and notions in a system called DomaCare, and all employees are responsible for reading both those notes as well as ones from team meetings. 
\begin{quote}
``So of course during this one year we have had all kinds of good and bad ideas, and they are not documented.. So.. It could have been a good idea to document them..'' [Interviewee 5]
\end{quote}
However, DomaCare is not a place for new ideas, and people not reading what has been written in DomaCare remain a problem. Information is exchanged when work shift ends and other begins. Interviewees described through conversation essential information is likely to come up, thus telling about ideas and experiments should be obvious. For instance a situation where something has been already experimented before, but no reporting was made; without conversation the same experiment may be performed again, resulting to waisting time and other resources.In turn, interviewee 4 describes how successful experimentations are shared with a team. 
\begin{quote}
``But yep, if someone has experimented some good thing in his own project, it will be informed to others as well. Like hey this is what we have and this is worth experimenting.We have regular team meetings where information is shared widely, so we now what others are doing.''[Interviewee 4]
\end{quote}
The interviewees regarding reporting of experiments brought up somewhat contradictory points of views. Some interviewees claimed positive experiments being more under discussion and reporting, whereas others emphasized how failed ones raise more conversation and opinions among colleagues. However, clear and systematic practice for this was not recognized, and seems the most popular mode to share knowledge remains face-to-face conversations.

\paragraph{Implementing new ways of working}
After experimentation being successfully executed, interviewees described the major difficulty lying behind the implementation process. Interviewees reported insightful experiments and solutions that would be important to implement in the daily routine. However, seems that structures and practices easily prevent implementing new ways of performing. For instance, new practice taking more resources in the implementation phase yet being more efficient and helpful in the long run, is more likely to be turned down and workplace sticking to old routines. 

Interviewee 5 describes one way how interviewees have tried to ease the difficulty of implementing new way of working; Deciding person or people who are in charge of the change to happen in the beginning. 
\begin{quote}
``And then, if it requires actions and processing we agree on who will start to do it. So that it will not remain only in speech, as happens so often.'' [Interviewee 5]
\end{quote}
Same phenomena and strong synergy to the difficulty of the implementation of a new routine is described in the subcategory Static Friction, which can be found under a category Gap between an idea and experiment. 

\subsubsection{Characteristics and know-how of an employee}
In the analysis process occurred that individual characteristics and knowledge of an employee could assist experimentation-driven approach in development as well as prevent experimentation from happening. In this category these factors are presented in three subcategories. Substance know-how explains the extent to which prior working experience can both encourage experimenting and in turn lead to repeating routines and resisting change. Individual characteristics consist of the factors how tolerance of uncertainty, employee?s self-criticism and confidence affect experimenting. Attitude towards development of an employee also has a major impact on experimenting behavior ? whether an employee considers developing part of the work or not and whether he is open for new ideas and breaking routines. 

\paragraph{Substance know-how}
According to the study prior work experience affects on the threshold for experimenting especially when experiments concern customers. Through substance knowledge an employee can gain wide understanding of the field, customer and ways of working, which can assist experimenting by adding the courage of the employee to perform an experiment. Knowing the customer, stakeholder and field of work are likely to add self-esteem and employees self-image as workers. 

Those interviewees who were recently graduated and had little previous working experience described it taking time to get to know the routines and the organizational culture as well as customers before actually being ready to suggest anything new or perform experiments. They felt easily insecure and described it important listening to more experienced colleagues and asking their opinion about new ideas before experimenting. When asked what is needed from an employee to begin with performing experiments the interviewees reported experience and knowing what to do being essential. Interviewee 6 describes the meaning of work experience to the ability to be creative and ideate. 
\begin{quote}
``In the beginning it of course took some time for me to learn the basics of the work, so maybe my own creativity and ability to ideate now grows with the experience of this work..'' [Interviewee 6]
\end{quote}
Furthermore, interviewees who had several years working experience, like interviewee 14 below, considered as a positive factor that they were able to combine previous experience to present work environment and customers. Working with same customer segment in different units gives perspective of what kind of ideas can be easily experimented and what may need more effort and resources. 
\begin{quote}
``I used my previous experience as an instructor.. It was useful for this experiment..'' [Interviewee 14]
\end{quote}
In addition, working many years with same customers leads to a high mutual trust between an employee and a customer. This again eases suggesting new ideas and performing experiments with customers. 
\begin{quote}
 ``I think there is the gained trust that has grown during this working journey. So that.. I think there is no resident [customer] who could not join doing these [experiments].''[Interviewee 7]
\end{quote}
In turn, many years of working experience from the same field and similar customer segments can also lead to repeating familiar routines and resisting change. Few interviewees described being essential that there are also newly graduated people or trainees with no prior experience in order to more clearly perceive unnecessary routines and bring new and fresh ideas into workplace. 

\paragraph{Tolerance for uncertainty, self-criticism and confidence}
According to the study, the way an employee tolerates uncertainty may have high impact on the experimentation behavior. When asking how interviewees feel experimentation-driven approach affecting on their work few of them described the difficulty lying in the feeling of uncertainty and incompleteness. Where some experienced uncertainty and incompleteness as threats and anxious factors in work, others emphasized those being factors that make working interesting. Interviewee 4 emphasises the ability to tolerate uncertainty and own failures.
\begin{quote}
``On the other hand it [uncertainty] is richness in work, so that one will not get too routinized. But of course one has to tolerate the uncertainty and has to tolerate your own failures, that ?ok, this time I chose wrong?..''[Interviewee 4]
\end{quote}
Furthermore, high level of self-criticism may prevent an employee from conducting experiments. Few interviewees described how they usually like to spend a lot of time in planning and refining a new idea before trying it in action. However, through the deadline and the pressure of experimentation challenge they were able to lower the level of self-criticism and try something incomplete. Interviewees learnt surprising facts already from the small experiment, and were overall satisfied experimenting even though they were not totally satisfied with the idea or experiment. As interviewee 4 states, a trifle of pressure can boost experimenting.
\begin{quote}
``I would have probably thought about this idea for ages and be like this is not good enough yet and it is not perfect''. [Interviewee 4]
\end{quote}
Self-criticism is also likely to prevent an employee saying ideas out loud. An employee may feel insecure and that his idea is actually poor and not worth sharing. An employee may even feel scared of team member shooting down his idea. As mentioned in the category Role of the team, a team can support its members to lower the level of self-criticism and encourage in ideating and experimenting. 

In turn, some interviewees described throwing also wild ideas among a team, as they may lead to something good and encourage others in ideating as well, as interviewee 6 points out. Their level of self-criticism was considerably lower and confidence higher than the ones who were not that enthusiastic in sharing ideas out loud. According to interviewee 11, personality affects on employee's opinion on failure. 
\begin{quote}
 ``I sometimes say out loud stupid ideas as well.. It?s that sometimes stupid ideas can lead to anything.'' [Interviewee 6]
\end{quote}
\begin{quote}
``I do think it depends on a personality. One has to be ready for failing and not fear it.'' [Interviewee 11]
\end{quote}
Even though saying ideas out loud requires courage and confidence, it can be learnt by experience. According to the study the employees who had more previous work experience were also the ones who were more confident in telling ideas out loud and conducting experiments without fearing failure.

\paragraph{Attitude and motivation towards developing}
According to the interviews attitude towards developing affects on experimentation behavior. If an employee considers developing and learning new things important and part of the work, he is more likely to reframe failures as opportunities, be resilient in developing and find alternative ways of doing things that needs change. 

If a team is not supporting employee?s idea and experimenting, an employee has to have motivation to try again and find another way. For instance, if the team is likely to be negative towards new ideas, interviewees described telling their ideas first to a trusted colleague. After the support from a close colleague an employee feels he has enough confidence to tell the idea to the whole team. Interviewee 9 describes this phenomena in the quote below. 
\begin{quote}
``I guess I find the certain people to whom I? [tell ideas out loud]. Then I dare to tell to others and as I have an urge to develop and learn, so through that I try..'' [Interviewee 9]
\end{quote}
Thus, the meaning of an employee?s motivation and attitude towards developing and learning is remarkable. In one experiment an employee ideated and prepared a prototype during his leisure time and during the process learnt new skills he had not known before. His motivation was so high it took him less than two weeks from the idea to actual prototype and first tests with customers. In turn, if an employee is not excited about learning new things, enjoys routines and prefers little change, he most likely will not be the first one to ideate or conduct experiments. As interviewee 14 states, developing requires action from an individual. 
\begin{quote}
 ``It [developing] requires activity from oneself and that an employee is willing to act and knows what he wants.'' [Interviewee 14]
\end{quote}
Furhtermore, in order ideation and experimentation to begin an employee has to be motivated and have the resilient attitude towards failing and learning from it. 

\subsubsection{The gap between an idea and experiment}
According to the study an idea that is said out loud in an organization is not always easily developed into an experiment or a new routine; there seems to be a gap between an idea and experiment.  In this category factors related to idea and people involved that are critical for experimentation to happen are presented.

Interviewees reported several factors that need to be taken into account when moving from an idea to experiment, and those factors are here divided into three subcategories. Characteristics of an idea and experiment focus on how the size, riskiness and relevance of the idea and experiment can prevent or support an experiment to happen. In addition, a phenomenon called Static friction was recognized in the study, meaning that even though employees are excited about ideating and experimenting in the beginning, for some reason experiments still do not take place. Stakeholder distance and customer involvement consists of the importance of stakeholder and customer opinion on the experiment as well as mutual trust between different parties that experiment concerns. 

\paragraph{Characteristics of an idea and experiment}
Characteristics of an idea seem to have an impact on whether or not it is experimented. For instance, the simpler and more concrete the idea is, the more likely it is to be experimented. Experimentation seems to help in making abstract ideas into concrete things and reflect the problem more clearly. However, even though the idea gains positive feedback among workplace, it still might not be experimented. The more resources, planning and opinions from different parts are needed in experiment and the more complex it feels among participants, the more likely it is to remain in ideation phase and not evolved into an experiment.  Interviewee 1 describes below why an idea actually turned into an actual experiment.
\begin{quote}
``It was as concrete thing as possible, that did not take too much.. or more 
negotiation with different parts..'' [Interviewee 1]
\end{quote}
The risk level of an idea affects on the bridge between throwing ideas and actually experimenting. When talking about performed experimentations interviewees, like interviewee 10 below, described the first ideas behind them being easy and simple, and especially possible to experiment with a low risk of anything bad to happen to customers or people involved in the experiment. In addition, interviewees reported that suitable experiments take into account the characteristics and possible limitations of the team or experiment. 
\begin{quote}
``But of course one has to think through that the experiment benefits everyone and it will not cause any harm to anyone.''[Interviewee 10]
\end{quote}
In addition, relevance and importance of an idea and the problem it attempts to solve are essential for the gap between an idea and experiment. Be the problem widely recognized among workplace, an attempt to experiment something new is rather likely to get support and engagement from colleagues and stakeholders. Likewise, according to the study if employees do not consider an idea important, and are not motivated to perform it, the idea will presumably be shut down by colleagues rather than considered from different perspectives or experimented nevertheless. 

\paragraph{Static friction}
Static friction in organizational environment came to prominence in the study.  Static friction here means a workplace, despite of eagerness towards new ideas and ideating, sticking to the routines and not being able to act in a different way and experimenting new ways of working. Employees are likely to get excited of ideas and even ideate eagerly together as a team, yet when it comes to implementation and actually performing experiments or do something differently, employees are no longer willing to take responsibility or be that excited about the idea. Interviewee 8 describes the phenomenon.
\begin{quote}
``I think we are always so excited about everything, but when we start going through details and who will actually take charge of this and who will be involved, then I think we are no longer that excited..'' [Interviewee 8]
\end{quote}
In turn, as interviewee 3 put it, few interviewees described a situation where little or no static friction is recognized. Common factor to these descriptions is that when coming up with an idea, employees begin right away going through different possibilities and actions on how to perform an experiment or implement the idea and actually proceed with them. Yet in the study descriptions about static friction were in majority. 
\begin{quote}
``I start quickly ideating where I can contact next [in order to make perform the experiment or gain a certain goal]..'' [Interviewee 3]
\end{quote}
\begin{quote}
``One really learns from that [experimenting and developing], definitely yes. One only needs to begin and get involved, which is usually the hard part..'' [Interviewee 5]
\end{quote}
Even though interviewees emphasized learning and experimenting new ideas being essential for work, lot of resistance and inactivity occurred when actual experimenting was supposed to happen. Interviewees described even being surprised how after all the enthusiasm towards experimenting, no one was willing to take the lead and the experiment was never performed. Interviewees supposed and admitted that at times it feels too exhausting and difficult to break routines and it is easier to continue performing tasks as is used to. 


\paragraph{Stakeholder distance and customer involvement}
According to the study the relevance and closeness of the idea to the customer enhances the engagement for experiment of employees. According to interviewee 7, the need for a change rising from a customer, improves the likelihood of the idea taken seriously and experimented. The same applies when a clear need to try something new is present. This is usually faced as a problem in present way of working, yet it can also be an attempt to improve the quality of customer?s life or the atmosphere at the workplace.  
\begin{quote}
 ``Especially when the idea rises from customer himself, we take every idea into account and consider everything that concerns a customer.'' [Interviewee 7]
\end{quote}
In this specific working field experimentation that concerns customers needs permission usually from both customer and relatives. According to the study this is occasionally a challenging network to deal with, as the requirements and wishes from customers and stakeholders may be highly contradictory. Yet, worthy ideas also rise from relatives as well as relevant information of what has already been experimented with the customers and what was learnt from that. Interviewee 10 describes this phenomenon. 
\begin{quote}
``So relatives play a very central role in our customers? lives, and almost everything is still discussed and checked with them and ask for support from them, like can we do this. And some very good ideas may also rise from them. Or then it can be like ?Oh no, this has been experimented for 15 years now, and it doesn?t work, so you should not start doing this.?.. So we have to remember that there is the network outside this workplace that is usually also involved in these experiments.'' [Interviewee 10]
\end{quote}
Mutual trust between people involved in experiments is needed. Interviewees described mutual trust being a relevant part of experimenting, and experienced the trust especially among customers and stakeholders as highly important factor in order experiments to happen. 

\subsection{The effects of experimenting on individual}
Even though the interview focused on the factors affecting experimentation in an organization, the study revealed that experimenting also has an impact on employees. 

Experimentation has an effect on employee on different levels. First of all, wide variety of emotions, such as excitement, fear of failure, disappointment and uncertainty is involved and rises up at different phases of the experimentation process. Secondly, experimenting helps an employee to learn and reflect on one?s work as well as to gain process know-how of experimenting. It seems that through experimenting an employee is likely to encounter surprising outcomes that would not have been realized and learnt through planning.  

\subsubsection{Emotional experience and engagement}
The study revealed that during the experimentation process an individual experiences wide range of emotions. Those factors are presented here in three subcategories, which are Positive emotions: Happiness, excitement, inspiration, boost to self-esteem, Negative emotions: Frustration, disappointment, fear of failure and fatigue and Engagement and motivation towards work. These subcategories are described and explained, in which part of the experimentation process they are faced. 

\paragraph{Positive emotions: happiness, excitement, inspiration, boost to self-esteem}
Experimentation usually begins with ideation, and in this phase interviewees described feeling creative, happy and excited. Ideating feels inspiring when colleagues support and join the ideating and plan together how the idea could be experimented. Interviewees described being excited and happy especially when the idea was their own or they were highly involved in ideating and planning the experiment.  Furthermore, ideating as a group also felt more empowering than ideating alone and not getting support for one?s ideas. 

As Interviewee 4 states, positive results of experimenting, meaning that new way of doing things works better than the previous way, is likely to raise positive emotions. Interviewees described that getting good feedback from the experiments and ideas as well as getting support from both customers and colleagues raise positive emotions and give boost to self-confidence. Furthermore, a successful experimentation encourages experimentation behavior and ideating and was described to nourish one?s creativity, as interviewee 6 emphasises.  
\begin{quote}
``If something works better than before [experimenting], you get a good feeling out of that.'' [Interviewee 4]
\end{quote}
\begin{quote}
``Experimenting nourishes creativity and ability to throw oneself to something new.'' [Interviewee 6]
\end{quote}
Among some interviewees the uncertain outcomes of experiments can also be seen as exciting and refreshing possibilities. As by experimenting new ways of performing work tasks are tested, experimenting brings exciting new aspects and challenges to routine work. Good ones tend to spread and may lead to something new and energetic and are likely to bring energy and stimulation to an employee. Interviewee 10 describes how experimentation brings stimulation to work. 
\begin{quote}
``So the experimentation kind of spreads. And I do also like routines and stuff: that things go in a certain way. But, it does bring stimulation to work, that we try out new routines.'' [Interviewee 10]
\end{quote}

\paragraph{Negative emotions: frustration, disappointment, fear of failure and fatigue}
Among positive emotions, interviewees described encountering various uncomfortable and complicated feelings throughout the experimentation process. 

In cases where an idea of an employee gets only little or no support from the team or a manager, feelings of frustration and disappointment may occur. This phenomenon is likely to discourage employees to say their ideas out loud in the future and thus makes the level of employee?s self-criticism higher. In the quote below of interviewee 14, is described the feeling of disappointment when encountering resistance from colleagues or team.
\begin{quote}
``When you are totally excited about something [idea].. for sure there comes the disappointment like what is it now, why this idea cannot go through, what is it that is so difficult in this.''[Interviewee 14]
\end{quote}
While the outcome of experimentation is usually difficult to forecast and cannot be planned in beforehand, this has an influence on emotions of an employee. Tight schedule of experimenting and little planning combined to uncertain outcomes can raise anxious emotions. Saying out loud one?s ideas might feel scary as the employees feel insecure about their idea. This forces them to encounter an uncomfortable fear of failure of their idea being shot down. 

In cases where the experimentation in where all the people involved are highly excited about, does not reach the goals set for the experiment, it is likely to cause frustration, disappointment and sadness among employees. In most cases, interviewees felt failing personally when they felt that experimentation failed. Interviewees usually described experimentation as failed if it did not reach the goals set for the experiment. 

In situations where there is no specific closure for the experiment or for some reason the experimentation is not finished, interviewees described feeling disappointed. This rather typical need for getting things done is described in the quote of interviewee 5.
 \begin{quote}
``Kind of disappointment, or kind of feeling of failure.. ? I always want to finish what I start.. So I do not like if things are not finished..''[Interviewee 5]
\end{quote}
\begin{quote}
``In addition, it [the experiment] cannot be too big, as it easily inflates.. so I think it is better to start, no matter how great idea there was, to slightly narrow it in some certain idea and then experiment that one. As I think many times it is a major hindrance that employees have no energy when the experiment inflates too much..'' [Interviewee 8]
\end{quote}
As described in quotation above from interviewee 8, the size of an experiment has an effect on the energy level of an employee. When planned experiment is too large or complicated and tasks needed to perform it too challenging or numerous, lots of resources are consumed and the energy level of an individual is lowered causing fatigue. 

\paragraph{Engagement and motivation towards work}
The experience of experimentation and seeing the result of experimentation is likely to encourage employees in their work.  Especially when the employees performing an experiment are satisfied with the outcomes of the experimentation, so that they consider it successful, experimenting improves the engagement and motivation of an employee towards his work. Furthermore, this encourages employees to be more creative and say out loud one?s ideas as well as gives a boost to energy level.

 As mentioned in the factors affecting experimenting, receiving feedback is important part of experimentation. In the field studied, employees work in a very close interface with customers and stakeholders. Thus, positive feedback from them raised positive emotions and encouraged to continue developing. Interviewee 10 states how important the feedback from work is for engagement.
\begin{quote}
``But every time there is a good idea and it works when we try it, and we notice that it helps, of course it improves my performing in the work also in mental level. And there comes moments, with customers, when something works with them and we get positive feedback --- of course it is very important. It then makes me happy and motivates, and encourages further on. Or feedback from relatives, if we get positive feedback from them, it again encourages.'' [Interviewee 10]
\end{quote}
Overall, interviewees described they felt more engaged to their work when they were able to perform ?quick and dirty? experiments from which they get instant feedback. Ideating and new ways of performing work-related tasks through experimenting increased the meaningfulness of work and made it more interesting. In addition, through ideating and experimenting interviewees felt they have more influence on their own work.

\subsubsection{Learning}
Experimenting seems to have an impact on learning skills of an individual. Learning occurs in various levels, and three subcategories where learning was especially noticed were formed from the data. Reflection of work means that experimenting helps an employee to reflect ways of working and the work overall. Secondly, through experimenting an employee gains deeper understanding of experimentation process, which is here called Process know-how.  In addition, it seems that an experimentation process helps and individual to overcome anxious and insecure emotions described above, thus improving the Resilience towards work of an employee. 

\paragraph{Reflection of work}
According to the study experimentations helped to question conventional ways of working and offered wider, more objective perspective that helped to improve the work. Some interviewees noticed the same as interviewee 2; that stopping to reflecting one's work is not and ordinary practice for an organization and its employees. Experimentation process helped employees to reflect and make the purpose of their own work clearer, as interviewee 8 emphasises. 
\begin{quote}
``Many things are done without actually stopping to think about them more, like would there be something to improve. It might be kind of quite typical way to act for an organization, to forget further evaluation.''[Interviewee 2]
\end{quote}
\begin{quote}
``Maybe it differs when.. you have to think, or you get to think, but let?s say that you have to think some issue in a deeper level and maybe make your thoughts more structural and that you have some understanding from what you view your own work and work of others ? So it does give in a certain way a deeper understanding on how one is doing his work.'' [Interviewee 8]
\end{quote}
As mentioned in the factors that affect experimentation, giving and receiving feedback is relevant in experimentation-driven approach. Receiving instant feedback improves learning process of an employee and the ability to iterate, meaning that an employee can reflect the outcomes of an experiment and improve his idea for another experiment and work overall. 

Furthermore, interviewees described surprising outcomes of small experimentations they performed and said talking about experimentations as well as performing them led to new information and ideas from different parts, including colleagues, customers and stakeholders. The interviewees described learning things they most likely would not have learnt without experimenting something new. Through experiments they also got deeper contact with stakeholders such as customer?s parents. Interviewee 14 describes below his relation to customers and relatives.
\begin{quote}
``The working becomes more interesting and I have clearer targets [through experimenting]. I get better in contact with customer and find new aspects as I told.. With customer?s relatives we talk in a different way when I tell that I?ve been planning of this kind of experiment ?-- and then the relatives begin to tell the history of a customer..'' [Interviewee 14]
\end{quote}
In addition to experimenting making the actual core of the work clearer, in this field of work where employees work very close to customers it helped the employees, like interviewee 6, to understand and listen better their needs and become more customer-oriented.  
\begin{quote}
``Through that [an experiment] we learnt to listen more and be even more customer-oriented ''[Interviewee 6]
\end{quote}
\begin{quote}
``Seems to me that my own prejudices are best repealed by just starting to do and act. Through that also abilities are found.'' [Interviewee 3]
\end{quote}
Furthermore, according to interviewee 3, employees learnt to overcome prejudices through rapid experimenting and found hidden capabilities.

\paragraph{Process know-how}
Factor called process know-how was recognized from the data. In this instance, process know-how means understanding of experimentation process, including the ability to ideate, plan and perform small experiments, the ability to reflect and learn from experiments and do iterations. 

Seems that process know-how of an individual improves through the experience of experimentation. Interviewees who were more familiar with vocabulary and the process of experimentation and who had done at least one experiment during the experimentation challenge were likely to reflect experiments in a deeper level than those who were not that familiar with the process and vocabulary. 

Those who had deeper understanding of experimenting process emphasized that one can learn from each experiment whether the actual goal set for the experiment is achieved or not, and those teachings are relatively important. As interviewee 12, some said the more the experiment fails, the more could be learned from it. 
\begin{quote}
``One always learns [from experiments]. It is kind of like the worse the experimentation is, the better one learns from it.''[Interviewee 12]
\end{quote}
\begin{quote}
``Failure is also a result, it leads to something. You can improve or try once more.'' [Interviewee 3]
\end{quote}
As interviewee 3 states, failure was rather seen as a result or learning point than totally failing. Those employees who had process know-how on experimenting were able to continue and learn better from each experiment, understand failures as learning objectives, process ideas and truly develop their work and challenge conventions. They reported feedback being an essential part of the work and developing, as it teaches what has to be done differently and what was successful. Interviewee 10 shows process know-how by stating the role of feedback, whether it is positive or negative. 
\begin{quote}
``Feedback is important in that essence so that one knows is it worth to continue to other experiments. So it [feedback] is always good, whether it was positive or negative, but it is always needed.''[Interviewee 10]
\end{quote}
When talking about feedback helping reflection of the experimentations, interviewees also reported discussion with colleagues being essential. Through discussion and feedback important information is exchanged and new aspects can be found and learnt. Yet, according to the study, receiving and listening to feedback requires humility and willingness to admit own faults and receive help from others.

Furthermore, essential part of process know-how is starting to perform with small experiments, prototyping with small group of customers, making a prototype as simple as possible and learning from the iterative process. Interviewees described being surprised by how fast a rough prototype can be done and how helpful it can be for work, compared to usual way of working meaning years of developing before some tool is launched for use. 
\begin{quote}
``This thing [a prototype], was welcomed so well, and actually it helped right away.. when usually these kind of tools are developed for many years before they are valid. So actually this kind of very simple system built this fast.. So it helped right away and that was a happy surprise..'' [Interviewee 1]
\end{quote}
Furthermore, a deeper perspective of process know-how affects bigger changes in an organization. One interviewee described experimentation as a way to manage and change complex systems while smaller experiments can assist bigger changes to happen. 
\begin{quote}
``But it can be, that this kind of small experiment can help bigger changes to happen.. ? And I guess that?s the trick in this whole thing and behind, that large things consist of several small ones and if those small ones can be fixed in several ways, it can have great impacts..'' [Interviewee 1]
\end{quote}

\paragraph{Resilience towards work}
Even though in experimentation process an employee goes through negative emotions described earlier such as frustration, fear of failure or disappointment, experimenting helps to overcome those anxious feelings and improves resilience of an employee.  Performing experiments forces employees to turn abstract ideas into concrete, smaller and lighter steps that are easier to approach, thus making the gap between planning and experimenting smaller.  

When experimentation is experienced as important among participants or there is a real problem to be solved, employees may turn all the disappointment and frustration rising from previous experiments or abatement of the idea from colleagues or a leader into passion of performing better. This improves the capability of resilience, as interviewee 13 comments. 
\begin{quote}
``So first comes the frustration, but after that like next year we?ll show them--. It may turn upside down when I get to process it in my head, like it?s a bummer how badly this went, but it can be we will try it again in a bit different way and we will do it better then. This also encourages to continue..''[Interviewee 13]
\end{quote}
Choosing the right terms may have a major impact on behaviour and resilience. Experiment as a word in a way consists of failing, and according to the study, compared to failing in daily routine work, failing in an experiment is experienced rather acceptable. This leads an employee feeling less pressure for succeeding with the first try. In addition, when the effort put on the first experiment is bearable, it is easier to persistently try another way. Interviewee 11 describes below his resistance. 
\begin{quote}
``Sometimes you really feel like giving up when you no longer come up with solutions how to make something work. But still you just.. You have a small break and then you get back to business.''[Interviewee 11]
\end{quote}
Even though during ideating and experimenting negative feelings are likely to occur, the characteristics, experience and know-how of experimenting helps to overcome these emotions and turn them to resilience. Iterative process of developing, dividing a task or a problem into smaller steps and learning by doing all support the emotional struggle with self-criticism, fear of failure, insecurity and uncertainty. 

\section{Discussion}
Even though almost every of 14 interviewed person described how the immediate superior they had have succeeded in establishing a climate where experimenting is possible to happen, all units did not perform experiments during the experimentation challenge. Thus leadership behavior is only one factor affecting experimentation behavior in an organization. 

Interviewees described that experimenting stands as an excellent method for learning, major factors being both the amount of experimentations done and the reflection on them. Few interviewees described using this method in their daily life and that ?experimentation? is a new word in their vocabulary while few told realizing through the experimentation challenge that their work is actually about trying out new ways of doing things and finding the best way to help customers in their daily lives.

Some of the interviewees described even liking the routine and not being that open for change and new ideas. So according to the study previous experience can encourage performing experiments and improve believing in their ideas, yet it can also lead to routine hard to change after many years. 


At present the informal structure for developing daily work can roughly be divided as follows: the idea emerges from work experience of an employee, a problem at hand, from customer or stakeholder need or by accident during a conversation with colleagues or friends. According to the interviewees, coincidence and problem-based approach play major roles at the moment. After the idea has emerged, support for the idea is asked from a trustworthy colleague and only after that it is democratically discussed in a team meeting. In order an idea to turn into experiment or a new routine, essential arguments and major part of employees ready to engage to the experiment are needed. When an idea is agreed to turn into an experiment, responsibility has to be divided and responsible employees chosen. After that a project team that continues with turning the idea into experimentation is collected. 

(an idea: draw a figure how the process idea to new way of working is done at the moment. Then in the discussion part, suggest a new one) 

\section{Conclusions}
Final part of the thesis concludes the findings and consists of x chapters. The results presented in previous part are further discussed in the first chapter. In addition, the theoretical framework presented in the theoretical part is supplemented with the findings of the empirical part. The research questions are revisited in order to determine how they were answered in the thesis. Both theoretical and managerial implications of the thesis are discussed in the second chapter. In the final chapter future research topics are discussed and the reliability of the thesis is evaluated.

\subsection{Reliability of the thesis}
Lincoln and cuba, l�hde taijan dipasta
Common criteria for evaluating the quality of a quantitative study are reliability and validity, where reliability refers to repeatability and validity to accuracy in means of measurement. In this thesis, however, quantitative approach is not used and in order to assess the reliability of the thesis, Lincoln and Guba?s (1985) approach on reliability is used. According to this approach, reliability is assessed through trustworthiness, which consists of four aspects: credibility, transferability, dependability and confirmability. 

Credibility refers to the interpretations made of the original data and their credibility (Lincoln and Guba, pp. 301-316). In this thesis? 

In addition, direct quotations are used in this thesis in order to reveal the data behind the interpretations. Co-researchers and professors have also been discussing about the interpretations thus adding credibility. 

Transferability means the possibilities to transfer the results and findings to another context. In the thesis?

Dependability refers to the consistency of the research process. Throughout the thesis the research design and process is described clearly. The research questions are presented in the beginning of the thesis and further revisited in the conclusions, and the results are evaluated through the research questions. 
Theory building process followed the principles of chosen research method, case study. 

Confirmability refers to objectivity and neutrality of the thesis. The writer of the thesis has never been working on studied industry field and was not involved in the empirical case other than in a role of interviewer and observer. In the data analysis process other researchers were involved and the results were discussed at least among three different researchers. In the theoretical research? 

\section{References}
%
%Bailey, C., A. 2007. A Guide to Qualitative Field Research. 2nd Ed. Sage Publications. Thousand Oaks, London.
Bhattacherjee, Anol (2012). Social Science Research: Principles, Methods, and Practices. Online: Open Access Textbooks. Book 3. %Available at: http:// scholarcommons.usf.edu/oa_textbooks/3 (Accessed 13 August 2014)
%Charmaz. K. (2006). Constructing grounded theory: A practical guide through qualitative analysis. Thousand Oaks, CA: Sage. 
%Dubois, A., & Araujo, L. 2004. Research methods in industrial marketing studies. 
%Rethinking marketing: Developing a new understanding of markets, 207-227
%
%Eisenhardt, K. 1989. Building Theories from Case Study Research. Academy of 
%Management Review. Vol 14, No. 4: 532-550
%
%Hyde, K. F. 2000. Recognising deductive processes in qualitative research. 
%Qualitative Market Research: An International Journal, 3(2), 82-90.
%Lichtman, M. (2013). Qualitative research in education: A User?s Guide. Chapter 12: Making meaning from your data (3rd ed.). Virginia Tech, CA: Sage.

%Morgan, G., & Smircich, L. (1980). The Case for Qualitative Research. Academy of Management Review, Vol. 5, No. 4, 491-500.

%Ragin, C. C. 1992 " Casing" and the process of social inquiry1. What Is a Case?: 
%Exploring the Foundations of Social Inquiry, 217. 
%
%Ragin, C. C. 1997 Turning the tables: How case-oriented research challenges 
%variable-oriented research. Comparative social research, 16, 27-42.
%Strauss, A., & Corbin, J. (1990). Basics of qualitative research: Grounded theory procedures and techniques (2nd ed.). Newbury Park, CA: Sage.
%Yin, R.K. 1989. Case study research: Design and Methods (reviewed edition). 
%Newbury Park, CA: Sage publications. 
%Yin, R. (1994). Case Study Research: Design and Methods, 2nd Edition, SAGE Publications.
%Yin, R.K. 1994. Case study research: Design and Methods (2nd edition). Newbury 
%Park, CA: Sage publications




%:Appendix

\end{document}  